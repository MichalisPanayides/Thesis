\section{Markov chain model}\label{sec:markov_model}

A Markov chain is a stochastic model that is the primary analytical tool to
study queues.
Under the assumption that all rates (arrival and service) are Markovian the
queueing system can be represented by a Markov chain
model~\cite{kemeny1976markov}.
The states of the Markov chain are denoted by \((u,v)\) where:

\begin{itemize}
    \item \(u\) is the number of individuals blocked in node 2
    \item \(v\) is the number of individuals either in node 1 or in the
    service centre
\end{itemize}

The set of all possible combination of pairs \((u, v)\) form all the possible
states that the system can visit.
The state space of the Markov chain is denoted as the set \(S=S(T)\) which can
be written as the disjoint union:

\begin{align}
    S(T) =& S_1(T) \cup S_2(T) \text{ where:} \nonumber \\
    S_1(T) =& \left\{(0, v)\in\mathbb{N}_0^2 \; | \; v < T \right\}
    \label{eq:definition_of_S_as_disjoint_union} \\
    S_2(T) =& \{(u, v)\in\mathbb{N}_0^2 \; | \; v \geq T \} \nonumber
\end{align}

\(S_1\) consists of the set of states where the number of individuals in node
\(1\) is less than \(T\) (i.e. \(v < T\)) and subsequently the number of
individuals in node \(2\) is zero (i.e. \(u = 0\)).
Similarly, \(S_2\) consists of the set of states where the number of individuals
in node 1 is greater than or equal to \(T\) (i.e. \(v \geq T\)) and
hence it is possible for individuals to be at node 2 (i.e.
\(u \geq 0\)).
This is illustrated diagrammatically in Figure~\ref{fig:general_markov_model}.

Having defined the set of states of the Markov chain model, the generator
matrix can also be obtained.
The generator matrix \(Q\) of the Markov chain consists of the
rates between the numerous states of the model.
Every entry \( Q_{ij} = Q_{(u_i, v_i),(u_j, v_j)} \) represents the rate from
state \( i = (u_i, v_i) \) to state \( j = (u_j , v_j) \) for all
\( (u_i, v_i), (u_j, v_j) \in S \).
The entries of \(Q\) can be calculated using the state-mapping function
described in equation~\eqref{eq:markov_transition_rate}.
Here \(\Lambda\) denotes the overall arrival rate in the model for
both types of individuals (i.e. \(\Lambda = \lambda_1 + \lambda_2\)).

\begin{equation} \label{eq:markov_transition_rate}
    Q_{ij} =
    \begin{cases}
        \Lambda, & \textbf{if } (u_i, v_i) - (u_j, v_j) = (0,-1) \textbf{ and }
        v_i < \text{T} \\
        \lambda_1, & \textbf{if } (u_i, v_i) - (u_j, v_j) = (0,-1)
        \textbf{ and } v_i \geq \text{T} \\
        \lambda_2, & \textbf{if } (u_i, v_i) - (u_j, v_j) = (-1,0) \\
        v_i \mu, & \textbf{if } (u_i, v_i) - (u_j, v_j) = (0,1) \textbf{ and }
        v_i \leq C \textbf{ or} \\ & \hspace{0.37cm}(u_i, v_i) - (u_j, v_j) =
        (1,0) \textbf{ and } v_i = T \leq C \\
        C \mu, & \textbf{if } (u_i, v_i) - (u_j, v_j) = (0,1) \textbf{ and }
        v_i > C
        \textbf{ or} \\ & \hspace{0.37cm}(u_i, v_i) - (u_j, v_j) = (1,0)
        \textbf{ and } v_i = T > C\\
        -\sum_{j=1}^{|Q|}{Q_{ij}} & \textbf{if } i = j \\
        0, & \textbf{otherwise}
    \end{cases}
\end{equation}

Note that for large values of \(N\) and \(M\) most of the entries of the
transition matrix will be zero.
In order to speed up the computation of the transition matrix, instead of
considering every possible pair of states in the state space a new function
that maps a state to every possible destination state can be used.
Function \(\mathcal{M}\) from
equation~\eqref{eq:state_map_to_destination_states}
takes a state \((u, v)\) and maps it to the set of
all possible destination states that the system can go to when on that state.

\begin{equation}\label{eq:state_map_to_destination_states}
    \mathcal{M}(u, v) =
    \begin{cases}
        \{(u, v + 1), (u, v - 1)\} & \textbf{if } v < T \\
        \{(u + 1, v), (u, v + 1), (u, v - 1)\} & \textbf{if } v = T
        \textbf{ and } u = 0 \\
        \{(u + 1, v), (u, v + 1), (u - 1, v)\} & \textbf{if } v = T
        \textbf{ and } u > 0 \\
        \{(u, v + 1), (u + 1, v), (u, v - 1)\} & \textbf{if } v > T \\
    \end{cases}
\end{equation}


A visualisation of how the transition rates relate to the states of the model
can be seen in the general Markov chain model shown in
Figure~\ref{fig:general_markov_model}.

\begin{figure}[H]
    \centering
    \scalebox{.8}
    {\begin{tikzpicture}[-, node distance = 0.9cm, auto, every node/.style={scale=0.7}]

    % Markov chain variables
    \tikzmath{
        let \initdist = 0.5cm;
        let \altdist = 1.2cm;
        let \minsz = 1.6cm;
    }

    % S_1 and S_2 rectangles
    \tikzmath{
        let \leftOne = -0.8;
        let \rightOne = 2.7;
        let \upOne = 0.8;
        let \downOne = -2.7;
        let \leftTwo = 2.8;
        let \rightTwo = 13;
        let \upTwo = -2.90;
        let \downTwo = -16.8;
    }

    % General case variables
    \tikzmath{
        let \GCsmallx = 8.3;
        let \GCsmally = -9.5;
        let \GCbigx = 4.2;
        let \GCbigy = -12;
    }

    % Rectangle for S1
    \draw[ultra thin, dashed] (\leftOne, \downOne) -- (\leftOne, \upOne);
    \draw[ultra thin, dashed] (\leftOne, \upOne) -- (\rightOne, \upOne);
    \draw[ultra thin, dashed] (\rightOne, \upOne) -- node 
    {\Huge{\( \quad S_1 \)}}(\rightOne, \downOne);
    \draw[ultra thin, dashed] (\rightOne, \downOne) -- (\leftOne, \downOne);

    % Rectangle for S2
    \draw[ultra thin, dashed] (\leftTwo, \downTwo) -- node 
    {\Huge{\( S_2 \quad \)}}(\leftTwo, \upTwo);
    \draw[ultra thin, dashed] (\leftTwo, \upTwo) -- (\rightTwo, \upTwo);
    \draw[ultra thin, dashed] (\rightTwo, \upTwo) -- (\rightTwo, \downTwo);
    \draw[ultra thin, dashed] (\rightTwo, \downTwo) -- (\leftTwo, \downTwo);

    % Small square of general case
    \draw [thick] (\GCsmallx, \GCsmally) -- node {} 
    (\GCsmallx + 0.4, \GCsmally);
    \draw [thick] (\GCsmallx + 0.4, \GCsmally) -- node {} 
    (\GCsmallx + 0.4, \GCsmally - 0.4);
    \draw [thick] (\GCsmallx + 0.4, \GCsmally - 0.4) -- node {} 
    (\GCsmallx, \GCsmally - 0.4);
    \draw [thick] (\GCsmallx, \GCsmally - 0.4) -- node {} 
    (\GCsmallx, \GCsmally);


    % Dashed lines to from small square to big one 
    \draw [ultra thin] (\GCsmallx, \GCsmally) -- node {} 
    (\GCbigx, \GCbigy);
    \draw [ultra thin] (\GCsmallx + 0.4, \GCsmally) -- node {} 
    (\GCbigx + 4, \GCbigy);
    \draw [ultra thin] (\GCsmallx, \GCsmally - 0.4) -- node {} (7, \GCbigy);
    \draw [ultra thin] (\GCsmallx + 0.4, \GCsmally - 0.4) -- node {} 
    (\GCbigx + 4, \GCbigy - 4);
    
    % Big Square of general case
    \draw [ultra thick] (\GCbigx, \GCbigy) -- node {} (\GCbigx + 4, \GCbigy);
    \draw [ultra thick] (\GCbigx + 4, \GCbigy) -- node {} 
    (\GCbigx + 4, \GCbigy - 4);
    \draw [ultra thick] (\GCbigx + 4, \GCbigy - 4) -- node {General Case} 
    (\GCbigx, \GCbigy - 4);
    \draw [ultra thick] (\GCbigx, \GCbigy - 4) -- node {} (\GCbigx, \GCbigy);

    % First Line
    \node[state, minimum size=\minsz] (zero) {(0,0)};
    \node[state, node distance = \initdist, minimum size=\minsz, below right=of zero] 
    (one) {(0,1)};
    \node[draw=none, node distance = \initdist, minimum size=\minsz, below right=of one] 
    (two) {\textbf{\( \ddots \)}};
    \node[state, node distance = \initdist, minimum size=\minsz, below right=of two] 
    (three) {(0,T)};
    \node[state, node distance = \altdist, minimum size=\minsz, right=of three] 
    (four) {\footnotesize(0,T+1)};
    \node[draw=none, node distance = \altdist, minimum size=\minsz, right=of four] 
    (five) {\textbf{\dots}};
    \node[state, minimum size=\minsz, right=of five] (six) {(0,C)};
    \node[draw=none, minimum size=\minsz, right=of six] (seven) {\textbf{\dots}};

    % Second Line
    \node[state, minimum size=\minsz, below=of three] (three_one) {(1,T)};
    \node[state, minimum size=\minsz, below=of four] (four_one) {\footnotesize(1,T+1)};
    \node[draw=none, minimum size=\minsz, below=of five] (five_one) {\textbf{\dots}};
    \node[state, minimum size=\minsz, right=of five_one] (six_one) {(1,C)};
    \node[draw=none, minimum size=\minsz, right=of six_one] (seven_one) {\textbf{\dots}};
    
    % Third Line
    \node[state, minimum size=\minsz, below=of three_one] (three_two) {(2,T)};
    \node[state, minimum size=\minsz, below=of four_one] (four_two) {\footnotesize(2,T+1)};
    \node[draw=none, minimum size=\minsz, below=of five_one] (five_two) 
    {\textbf{\dots}};
    \node[state, minimum size=\minsz, right=of five_two] (six_two) {(2,C)};
    \node[draw=none, minimum size=\minsz, right=of six_two] (seven_two) 
    {\textbf{\dots}};

    % Fourth line
    \node[draw=none, node distance = \altdist, minimum size=\minsz, below=of three_two] 
    (three_three) {\textbf{\vdots}};
    \node[draw=none, node distance = \altdist, minimum size=\minsz, below=of four_two] 
    (four_three) {\textbf{\vdots}};
    \node[draw=none, node distance = 2cm, minimum size=\minsz, below=of five_two] 
    (five_three) {};
    \node[draw=none, node distance = \altdist, minimum size=\minsz, below=of six_two] 
    (six_three) {\textbf{\vdots}};

    % Fifth line
    \node[draw=none, node distance = 0.3cm, minimum size=\minsz, below=of four_three] 
    (general_case_up) {};
    \node[state, node distance = \altdist, minimum size=\minsz, below=of general_case_up] 
    (general_case_mid) {\( (u_i, v_i) \)};

    \node[draw=none, node distance = \altdist, minimum size=\minsz, below=of general_case_mid] 
    (general_case_down) {};
    \node[draw=none, node distance = \altdist, minimum size=\minsz, left=of general_case_mid] 
    (general_case_left) {};
    \node[draw=none, node distance = \altdist, minimum size=\minsz, right=of general_case_mid] 
    (general_case_right) {};

    \draw[every loop]
        % First Horizontal Edges
        (zero) edge[bend left] node {\( \Lambda \)} (one)
        (one) edge[bend left] node {\( \mu \)} (zero)
        (one) edge[bend left] node {\( \Lambda \)} (two)
        (two) edge[bend left] node {\( 2 \mu \)} (one)
        (two) edge[bend left] node {\( \Lambda \)} (three)
        (three) edge[bend left] node {\( T \mu \)} (two)
        (three) edge[bend left] node {\( \lambda_1 \)} (four)
        (four) edge[bend left] node {\( (T+1) \mu \)} (three)
        (four) edge[bend left] node {\( \lambda_1 \)} (five)
        (five) edge[bend left] node {\( (T+2) \mu \)} (four)
        (five) edge[bend left] node {\( \lambda_1 \)} (six)
        (six) edge[bend left] node {\( C\mu \)} (five)
        (six) edge[bend left] node {\( \lambda_1 \)} (seven)
        (seven) edge[bend left] node {\( C\mu \)} (six)

        % Second Horizontal Edges
        (three_one) edge[bend left] node {\( \lambda_1 \)} (four_one)
        (four_one) edge[bend left] node {\( (T+1) \mu \)} (three_one)
        (four_one) edge[bend left] node {\( \lambda_1 \)} (five_one)
        (five_one) edge[bend left] node {\( (T+2) \mu \)} (four_one)
        (five_one) edge[bend left] node {\( \lambda_1 \)} (six_one)
        (six_one) edge[bend left] node {\( C\mu \)} (five_one)
        (six_one) edge[bend left] node {\( \lambda_1 \)} (seven_one)
        (seven_one) edge[bend left] node {\( C\mu \)} (six_one)

        % Third Horizontal Edges
        (three_two) edge[bend left] node {\( \lambda_1 \)} (four_two)
        (four_two) edge[bend left] node [below] {\( (T+1) \mu \)} (three_two)
        (four_two) edge[bend left] node {\( \lambda_1 \)} (five_two)
        (five_two) edge[bend left] node {\( (T+2) \mu \)} (four_two)
        (five_two) edge[bend left] node {\( \lambda_1 \)} (six_two)
        (six_two) edge[bend left] node {\( C\mu \)} (five_two)
        (six_two) edge[bend left] node {\( \lambda_1 \)} (seven_two)
        (seven_two) edge[bend left] node {\( C\mu \)} (six_two)

        % First Vertical Edges
        (three) edge[bend left] node {\( \lambda_2 \)} (three_one)
        (three_one) edge[bend left] node {\( T \mu \)} (three)
        (three_one) edge[bend left] node {\( \lambda_2 \)} (three_two)
        (three_two) edge[bend left] node {\( T\mu \)} (three_one)
        (three_two) edge[bend left] node {\( \lambda_2 \)} (three_three)
        (three_three) edge[bend left] node {\( T\mu \)} (three_two)

        % Second Vertical Edges
        (four) edge node {\( \lambda_2 \)} (four_one)
        (four_one) edge node {\( \lambda_2 \)} (four_two)
        (four_two) edge node {\( \lambda_2 \)} (four_three)

        % Fourth Vertical Edges
        (six) edge node {\( \lambda_2 \)} (six_one)
        (six_one) edge node {\( \lambda_2 \)} (six_two)
        (six_two) edge node {\( \lambda_2 \)} (six_three)

        % General Case
        (general_case_left) edge[bend left] node {\( \lambda_1 \)} (general_case_mid)
        (general_case_mid) edge[bend left] node {\( v_i \mu \)} (general_case_left)
        (general_case_right) edge[bend left] node {\( (v_i +1) \mu \)} (general_case_mid)
        (general_case_mid) edge[bend left] node {\( \lambda_1 \)} (general_case_right)
        % (five_three) edge node {\( \lambda_2 \)} (general_case_mid)
        (general_case_up) edge node {\( \lambda_2 \)} (general_case_mid)
        (general_case_mid) edge node {\( \lambda_2 \)} (general_case_down)
        ;
\end{tikzpicture}
}
    \caption{Generic case of Markov chain model. The diagram shows the
    two disjoint sets of states \(S_1\) and \(S_2\) and the transition rates
    between the states.} 
    \label{fig:general_markov_model}
\end{figure}


In order to consider this model numerically an adjustment needs to be made.
The problem defined above assumes no upper boundary to the number of individuals
that can wait for service or for the ones that are blocked in node 2.
Therefore, a different state space \( \tilde S \) is constructed where
\( \tilde S \subseteq S \) and there is a maximum allowed number of individuals
\(N\) that can be in node 1 and a maximum allowed number of individuals
\(M\) that can be blocked in node 2:

\begin{equation}\label{eq:truncated_state_space}
    \tilde S = \left\{ (u, v) \in S\;| u \leq M, v\leq N \right\}
\end{equation}

The adjusted Markov chain model with states \(\tilde S\) can be seen in
Figure~\ref{fig:adjusted_markov_model}.

\begin{figure}[H]
    \centering
    \scalebox{.8}
    {\begin{tikzpicture}[-, node distance = 0.9cm, auto, every node/.style={scale=0.7}]

    % Markov chain variables
    \tikzmath{
        let \initdist = 0.9cm;
        let \altdist = 0.9cm;
        let \minsz = 1.8cm;
    }

    % First Line
    \node[state, minimum size=\minsz] (one) {(0,0)};
    \node[draw=none, node distance = \initdist, minimum size=\minsz, right=of one] 
    (two) {\textbf{\dots}};
    \node[state, node distance = \initdist, minimum size=\minsz, right=of two] 
    (three) {(0,T)};
    \node[draw=none, node distance = \altdist, minimum size=\minsz, right=of three] 
    (four) {\textbf{\dots}};
    \node[state, node distance = \altdist, minimum size=\minsz, right=of four] 
    (five) {(0, C)};
    \node[draw=none, minimum size=\minsz, right=of five] (six) {\textbf{\dots}};
    \node[state, minimum size=\minsz, right=of six] (seven) {(0, N)};

    % Second Line
    \node[state, minimum size=\minsz, below=of three] (three_one) {(1,T)};
    \node[draw=none, minimum size=\minsz, below=of four] (four_one) {\textbf{\dots}};
    \node[state, minimum size=\minsz, below=of five] (five_one) {(1, C)};
    \node[draw=none, minimum size=\minsz, right=of five_one] (six_one) {\textbf{\dots}};
    \node[state, minimum size=\minsz, right=of six_one] (seven_one) {(1, N)};
    
    % Third Line
    \node[draw=none, minimum size=\minsz, below=of three_one] (three_two) {\textbf{\vdots}};
    \node[draw=none, minimum size=\minsz, below=of four_one] (four_two) {\textbf{\(\ddots\)}};
    \node[draw=none, minimum size=\minsz, below=of five_one] (five_two) 
    {\textbf{\vdots}};
    \node[draw=none, minimum size=\minsz, right=of five_two] (six_two) {\textbf{\(\ddots\)}};
    \node[draw=none, minimum size=\minsz, right=of six_two] (seven_two) 
    {\textbf{\vdots}};

    % Fourth line
    \node[state, node distance = \altdist, minimum size=\minsz, below=of three_two] 
    (three_three) {(M, T)};
    \node[draw=none, node distance = \altdist, minimum size=\minsz, below=of four_two] 
    (four_three) {\textbf{\dots}};
    \node[state, node distance = \altdist, minimum size=\minsz, below=of five_two] 
    (five_three) {(M, C)};
    \node[draw=none, node distance = \altdist, minimum size=\minsz, below=of six_two] 
    (six_three) {\textbf{\dots}};
    \node[state, node distance = \altdist, minimum size=\minsz, below=of seven_two] 
    (seven_three) {(M, N)};

    \draw[every loop]
        % First Horizontal Edges
        (one) edge[bend left] node {\( \Lambda \)} (two)
        (two) edge[bend left] node {\( 2 \mu \)} (one)
        (two) edge[bend left] node {\( \Lambda \)} (three)
        (three) edge[bend left] node {\( T \mu \)} (two)
        (three) edge[bend left] node {\( \lambda_1 \)} (four)
        (four) edge[bend left] node {\( (T+1) \mu \)} (three)
        (four) edge[bend left] node {\( \lambda_1 \)} (five)
        (five) edge[bend left] node {\( C\mu \)} (four)
        (five) edge[bend left] node {\( \lambda_1 \)} (six)
        (six) edge[bend left] node {\( C\mu \)} (five)
        (six) edge[bend left] node {\( \lambda_1 \)} (seven)
        (seven) edge[bend left] node {\( C\mu \)} (six)

        % Second Horizontal Edges
        (three_one) edge[bend left] node {\( \lambda_1 \)} (four_one)
        (four_one) edge[bend left] node {\( (T+1) \mu \)} (three_one)
        (four_one) edge[bend left] node {\( \lambda_1 \)} (five_one)
        (five_one) edge[bend left] node {\( C\mu \)} (four_one)
        (five_one) edge[bend left] node {\( \lambda_1 \)} (six_one)
        (six_one) edge[bend left] node {\( C\mu \)} (five_one)
        (six_one) edge[bend left] node {\( \lambda_1 \)} (seven_one)
        (seven_one) edge[bend left] node {\( C\mu \)} (six_one)

        % Third Horizontal Edges
        (three_three) edge[bend left] node {\( \lambda_1 \)} (four_three)
        (four_three) edge[bend left] node [below] {\( (T+1) \mu \)} (three_three)
        (four_three) edge[bend left] node {\( \lambda_1 \)} (five_three)
        (five_three) edge[bend left] node {\( C\mu \)} (four_three)
        (five_three) edge[bend left] node {\( \lambda_1 \)} (six_three)
        (six_three) edge[bend left] node {\( C\mu \)} (five_three)
        (six_three) edge[bend left] node {\( \lambda_1 \)} (seven_three)
        (seven_three) edge[bend left] node {\( C\mu \)} (six_three)

        % First Vertical Edges
        (three) edge[bend left] node {\( \lambda_2 \)} (three_one)
        (three_one) edge[bend left] node {\( T \mu \)} (three)
        (three_one) edge[bend left] node {\( \lambda_2 \)} (three_two)
        (three_two) edge[bend left] node {\( T\mu \)} (three_one)
        (three_two) edge[bend left] node {\( \lambda_2 \)} (three_three)
        (three_three) edge[bend left] node {\( T\mu \)} (three_two)

        % Second Vertical Edges
        (five) edge node {\( \lambda_2 \)} (five_one)
        (five_one) edge node {\( \lambda_2 \)} (five_two)
        (five_two) edge node {\( \lambda_2 \)} (five_three)

        % Fourth Vertical Edges
        (seven) edge node {\( \lambda_2 \)} (seven_one)
        (seven_one) edge node {\( \lambda_2 \)} (seven_two)
        (seven_two) edge node {\( \lambda_2 \)} (seven_three)
        ;
\end{tikzpicture}
}
    \caption{Adjusted case of the Markov chain model. The diagram makes use of
    the truncated state space \( \tilde S \) where the state space \(S\) is
    bounded by \(N\) and \(M\).}
    \label{fig:adjusted_markov_model}
\end{figure}


\subsection{Steady state probability vector}
\label{sec:steady_state_probabilities}

The generator matrix \( Q \) defined in~\eqref{eq:markov_transition_rate} can
be used to get the probability vector \( \pi \) that contains the steady state
probabilities of the Markov chain model.
The vector \( \pi \) is commonly used to study stochastic systems and it's main
purpose is to keep track of the probability of being at any given state of
the Markov chain model.
\(\pi_i\) is the steady state probability of being in state \((u_i, v_i) \in
\tilde S\) which is the \(i^{\text{th}}\) state of \(\tilde S\) for some
ordering of \(\tilde S\).
The term \textbf{steady state} refers to the instance of the vector \( \pi \)
where the probabilities of being at any state becomes stable over time.
Thus, by considering the steady state vector \( \pi \) the relationship between
it and \( Q \) is given by:

\begin{equation}\label{eq:steady_state_from_generator_matrix}
    \frac{d\pi}{dt} = \pi Q = \vec{0}
\end{equation}

The following parameters of the Markov model will act as a running example for
all approaches:

\begin{center}
    \begin{tabular}{|c|c|c|c|c|c|c|c|}
        \hline
        Parameter & \(\lambda_1\) & \(\lambda_2\) & \(\mu\) & \(C\) & \(T\)
        & \(N\) & \(M\) \\
        \hline
        Value & 1.0 & 2.0 & 2.0 & 2.0 & 3.0 & 4.0 & 2.0 \\
        \hline
    \end{tabular}
\end{center}


\subsection{Numerical integration approach}

A method that can be used to get the steady state probability vector is
to solve the differential equation~\eqref{eq:steady_state_from_generator_matrix}
numerically.
Two methods of solving the differential equation were considered.
Both methods observe the value of \(\pi\) over time until it
reaches the steady state based on some initial starting value \(\pi_0\):

\begin{gather}
    \frac{d\pi}{dt} = \pi Q \\
    \pi(t_0) = \pi_0 \nonumber \\
    \text{where } \pi_0 =
    [\frac{1}{|\pi|}, \frac{1}{|\pi|}, \dots, \frac{1}{|\pi|}] \nonumber
\end{gather}

Two types of methods were considered to solve the differential equation
numerically.
The first method uses a combination of Adams' method~\cite{adams_method} and the
backward differentiation formula (BDF)~\cite{backward_differentiation_formula}.
This method is generally used to solve systems of the form
\(\frac{dy}{dt} = f\) with a dense or banded Jacobian when the problem is stiff,
which then uses the BDF algorithm, while when the problem is non-stiff it uses
Adams' method.
This was implemented using \texttt{scipy.integrate.odeint} from the python
library \texttt{SciPy}~\cite{2020SciPy-NMeth} that uses the
\texttt{lsoda}~\cite{lsoda_algorithm} integration method.

\begin{lstlisting}[
    style=pystyle,
    caption={Steady state probabilities calculation using numeric integration
    using the \texttt{odeint} function.},
    label={lst:steady_state_probabilities_odeint},
]
>>> import ambulance_game as abg
>>> import scipy as sci
>>> Q = abg.markov.get_transition_matrix(
...     lambda_1=1,
...     lambda_2=2,
...     mu=2,
...     num_of_servers=2,
...     threshold=3,
...     system_capacity=4,
...     buffer_capacity=2
... )
>>> pi = abg.markov.get_steady_state_numerically(
...     Q, integration_function=sci.integrate.odeint
... )
>>> pi
array([0.17596013, 0.2639402 , 0.19795515, 0.14846636, 0.08660538,
       0.05464387, 0.02474439, 0.02268236, 0.02500215])

\end{lstlisting}



The second approach uses the explicit Runge-Kutta integration method of order 5
by controlling the error assuming accuracy of order
4~\cite{solve_ivp_rk45_method, runge_kutta_formulas}.
The general recursive formula for the explicit family of Runge-Kutta methods is
given by:

\begin{equation}
    y_{n+1} = y_n + h \sum_{i=1}^s b_i k_i
\end{equation}
\begin{align}
    k_1 & = f(t_n, y_n), \nonumber \\
    k_2 & = f(t_n+c_2h, y_n+h(a_{21}k_1)), \nonumber \\
    k_3 & = f(t_n+c_3h, y_n+h(a_{31}k_1+a_{32}k_2)), \nonumber \\
        & \ \ \vdots \nonumber \\
    k_s & = f(t_n+c_s h, y_n+h(a_{s1}k_1+a_{s2}k_2+\cdots+a_{s,s-1}k_{s-1}))
    \nonumber
\end{align}

where \(y_0\) is the given initial value, \(s\) is the number of stages and
\(h\) is the step size.
The coefficients \(b_i\), \(c_i\), and \(a_{ij}\) are usually arranged in a
mnemonic device known as the Butcher's tableau.
This was implemented using \texttt{scipy.integrate. solve\_ivp} from the python
library \texttt{SciPy}~\cite{2020SciPy-NMeth}.

\begin{lstlisting}[
    style=pystyle,
    caption={Steady state probabilities calculation using numeric integration
    using the \texttt{solve\_ivp} function.},
    label={lst:steady_state_probabilities_solve_ivp},
]
>>> pi = abg.markov.get_steady_state_numerically(
...     Q, integration_function=sci.integrate.solve_ivp
... )
>>> pi
array([0.17596012, 0.26394019, 0.19795515, 0.14846637, 0.08660539,
       0.05464388, 0.02474439, 0.02268236, 0.02500215])

\end{lstlisting}


\subsection{Linear algebraic approach}

The steady state probability vector \( \pi \) can be obtained by solving the
linear equation:

\begin{equation}\label{eq:numpy_linalg_solve_1}
    Q^T \pi = \vec{0} \hspace{0.5cm} \text{such that} \hspace{0.5cm}
    \sum_{i} \pi_i = 1
\end{equation}

The two equations can be combined into one by augmenting the matrix \( Q^T \)
in such a way that it includes the extra constraint \( \sum_i \pi_i = 1 \).
The new augmented matrix \(\tilde Q\) is defined as \(Q\) with the final
column replaced with a vector of ones and vector \(\vec{b}\) is defined
as a column vector of \(0\)s apart from the final element which is \(1\).
Note that, \(\tilde Q\) needs to be a square matrix in order to solve the
equation using linear algebra (i.e. the matrix needs to be invertible).
Thus, the steady state probability vector can be calculated by solving the
linear equation:

\begin{equation}
    \tilde Q^T \pi = \vec{b}
\end{equation}

Using LU decomposition with partial pivoting and row interchanges, matrix
\(Q^T\) can be expressed of the form \(P \times L \times U\), where \(P\) is
a permutation matrix, \(L\) is a unit lower triangular matrix, and \(U\) is
an upper triangular matrix~\cite{strang2006linear}.
The factored form of \(Q^T\) can then be used to solve the system.
This was implemented using \texttt{numpy.linalg.solve} from the
python library \texttt{numpy}~\cite{2020NumPy-Array, lapack99}.


\begin{lstlisting}[
    style=pystyle,
    caption={Steady state probabilities calculation using linear algebraic
    approach with \texttt{numpy.linalg.solve}.},
    label={lst:steady_state_probabilities_numpy_linalg_solve_1},
]
>>> import numpy as np
>>> pi = abg.markov.get_steady_state_algebraically(
...     Q, algebraic_function=np.linalg.solve
... )
>>> pi
array([0.17596013, 0.2639402 , 0.19795515, 0.14846636, 0.08660538,
       0.05464387, 0.02474439, 0.02268236, 0.02500215])

\end{lstlisting}


\subsection{Least squares approach}

Another approach that is considered is the least squares method.
As the problem becomes more complex (i.e. as the artificial parameters \(N\)
and \(M\) defined in equation~\eqref{eq:truncated_state_space} increase)
the computational time required to solve it increases by a lot.
Thus, one may obtain a good approximation of the steady state vector \( \pi \)
by solving the following equation:

\begin{equation}
    \pi = \text{argmin}_{\pi \in \mathbb{R}^{|\pi|}}\|\tilde Q^T \pi - b\|_2^2
\end{equation}

The above expression gets the vector \( \pi \) that approximately solves
equation \(\tilde Q^T \pi = b\).
This was implemented using \texttt{numpy.linalg.lstsq} from the python
library \texttt{numpy}~\cite{2020NumPy-Array}.

\begin{lstlisting}[
    style=pystyle,
    caption={Steady state probabilities calculation using least squares
    approach with \texttt{numpy.linalg.lstsq}.},
    label={lst:steady_state_probabilities_numpy_linalg_lstsq},
]
>>> pi = abg.markov.get_steady_state_algebraically(
...     Q, algebraic_function=np.linalg.lstsq
... )
>>> pi
array([0.17596013, 0.2639402 , 0.19795515, 0.14846636, 0.08660538,
       0.05464387, 0.02474439, 0.02268236, 0.02500215])

\end{lstlisting}


An additional approach that was considered but not completed was a
closed-form formula for the steady state probabilities.
The work that was done on this approach is described in
Appendix~\ref{app:steady_state_probs_closed_form}.
