\subsection{Numeric results and timings}\label{sec:truncation_effect}

\subsubsection{Markov chain waiting time approaches comparison}
\label{sec:waiting_time_approach_comparison}

In Section \ref{sec:waiting_time} three different approaches for calculating
the waiting time using the Markov chain model have been introduces.
The three approaches are the recursive approach, the direct approach and
the closed form approach.
In this section the three approaches are compared in terms of accuracy and
computation time.

In terms of accuracy the three approaches get close to identical results.
Figures~\ref{fig:waiting_time_accuracy_over_N} and
\ref{fig:waiting_time_accuracy_over_M} show the differences of the three
approaches for different values of \(N\) and \(M\).


\begin{figure}[H]
    \includegraphics[width=12cm]{chapters/03_queueing_model/img/numeric_results_and_timings/waiting_time_formulas_comparison/waiting_time_over_N.png}
    \caption{Waiting times of the three waiting time approaches for different
    values of \(N\).}
    \label{fig:waiting_time_accuracy_over_N}
\end{figure}
    
\begin{figure}[H]
    \includegraphics[width=12cm]{chapters/03_queueing_model/img/numeric_results_and_timings/waiting_time_formulas_comparison/waiting_time_over_M.png}
    \caption{Waiting times of the three waiting time approaches for different
    values of \(M\).}
    \label{fig:waiting_time_accuracy_over_M}
\end{figure}


Since the results of the three approaches are almost identical, the computation
time of the three approaches is the main factor that determines which
approach will be used.
Figures~\ref{fig:waiting_time_algorithm_duration_over_N}, 
\ref{fig:waiting_time_algorithm_duration_over_M} and 
\ref{fig:waiting_time_algorithm_duration_over_N_and_M} show the computation
time needed to calculate the waiting time for different values of \(N\) and
\(M\).

\begin{figure}[H]
    \includegraphics[width=12cm]{chapters/03_queueing_model/img/numeric_results_and_timings/waiting_time_formulas_comparison/algorithm_duration_over_NN.png}
    \caption{Duration of the three waiting time approaches for different values
    of \(N\).}
    \label{fig:waiting_time_algorithm_duration_over_N}
\end{figure}

\begin{figure}[H]
    \includegraphics[width=12cm]{chapters/03_queueing_model/img/numeric_results_and_timings/waiting_time_formulas_comparison/algorithm_duration_over_M.png}
    \caption{Duration of the three waiting time approaches for different values
    of \(M\).}
    \label{fig:waiting_time_algorithm_duration_over_M}
\end{figure}


\begin{figure}[H]
    \includegraphics[width=12cm]{chapters/03_queueing_model/img/numeric_results_and_timings/waiting_time_formulas_comparison/algorithm_duration_over_N_and_M.png}
    \caption{Duration of the three waiting time approaches for different values
    of \(N\) and \(M\).}
    \label{fig:waiting_time_algorithm_duration_over_N_and_M}
\end{figure}

The direct approach is the slowest approach, which makes sense since it requires
solving a system of linear equations.
The recursive approach and the closed form approach give similar durations,
although the recursive approach seems to be more linear in nature than the
closed form method. 


\subsubsection{Markov chain - Simulation comparison of steady state probabilities}

Another comparison that can be made is the comparison between the steady state
probabilities calculated using the Markov chain and the simulation.
The steady state probabilities, defined in
Section~\ref{sec:steady_state_probabilities}, are an essential measure since
they are necessary in the calculation of all performance measures using the
Markov chain model.
Although, there is no need to calculate the steady state probabilities for the
simulation, it is interesting to see how the two approaches compare.
Figures~\ref{fig:comparison_steady_state_probabilities_1}
to~\ref{fig:comparison_steady_state_probabilities_5} show a comparison between
the steady state probabilities between the Markov chain and the simulation
for different values of \(\mu\).
The value of \(\mu\) gradually increases from \(\mu = 0.03\) to \(\mu = 0.27\)
and for each one three plots are generated; the steady state probabilities
generated by the Markov chain, the steady state probabilities generated by the
simulation and the difference between the two.
The rest of the parameters have the following values:

\begin{multicols}{2}
    \begin{itemize}
        \item \(\lambda_1\) = 0.3
        \item \(\lambda_2\) = 0.3
        \item \(C\) = 5
        \item \(T\) = 5
        \item \(N\) = 20
        \item \(M\) = 20
    \end{itemize}    
\end{multicols}


\begin{figure}[H]
    \includegraphics[width=\textwidth, trim=200 60 170 60, clip]{chapters/03_queueing_model/img/numeric_results_and_timings/steady_state_probabilities/main_1.png}
    \caption{\(\mu = 0.03\)}
    \label{fig:comparison_steady_state_probabilities_1}
\end{figure}

\begin{figure}[H]
    \includegraphics[width=\textwidth, trim=200 60 170 60, clip]{chapters/03_queueing_model/img/numeric_results_and_timings/steady_state_probabilities/main_3.png}
    \caption{\(\mu = 0.09\)}
    \label{fig:comparison_steady_state_probabilities_2}
\end{figure}

\begin{figure}[H]
    \includegraphics[width=\textwidth, trim=200 60 170 60, clip]{chapters/03_queueing_model/img/numeric_results_and_timings/steady_state_probabilities/main_5.png}
    \caption{\(\mu = 0.15\)}
    \label{fig:comparison_steady_state_probabilities_3}
\end{figure}

\begin{figure}[H]
    \includegraphics[width=\textwidth, trim=200 60 170 60, clip]{chapters/03_queueing_model/img/numeric_results_and_timings/steady_state_probabilities/main_7.png}
    \caption{\(\mu = 0.21\)}
    \label{fig:comparison_steady_state_probabilities_4}
\end{figure}

\begin{figure}[H]
    \includegraphics[width=\textwidth, trim=200 60 170 60, clip]{chapters/03_queueing_model/img/numeric_results_and_timings/steady_state_probabilities/main_9.png}
    \caption{\(\mu = 0.27\)}
    \label{fig:comparison_steady_state_probabilities_5}
\end{figure}

Figure~\ref{fig:comparison_steady_state_probabilities_1} has the smallest
value of \(\mu = 0.03\).
It can be seen that for both the Markov chain and the simulation the steady
state probabilities are close to zero for most states apart from the
states close to when the system is full.
That is because the arrival rate of individuals is much lower than the service
rate of individuals even with 5 servers.
Note here that because the model is immediately flooded from the beginning, the
simulation has no time to explore the state space, so the heatmap looks like
it is missing some of its pieces.
Additionally, it is interesting to note that as we increase the value of \(\mu\)
in Figures~\ref{fig:comparison_steady_state_probabilities_2} -
\ref{fig:comparison_steady_state_probabilities_5} smaller states in the model
have a higher value since individuals now exit the system faster.
Also, note that for all values of \(\mu\) the difference between the Markov
chain approach and the simulation is small.


\subsubsection{Simulation VS Markov chain durations}

The choice of the artificial parameters \(N\) and \(M\) is an important 
decision of the model.
In the untruncated simulation these values can be omitted by setting them to
infinity.
This is not possible when obtaining the steady state probabilities of the
finite state Markov chain.
The value of \(N\) and \(M\) can be chosen arbitrarily large, but the
computation time increases as the size of the state space increases.
Table \ref{tab:truncation_effect_timings_old} shows the relative timings of the
different approaches used to get the performance measures for different values
of \(N\) and \(M\).
Note that \(N\) and \(M\) have the same value throughout the table.
Simulation is ran for a runtime of \(10^4\) time units and the displayed
durations are for a single run of the simulation and similarly for 100 runs of
the simulation.
For getting the performance measures using the finite state Markov chain each
timing recorded is for the computation of the steady state probabilities and
then the corresponding performance measure.

\tiny
\begin{table}[H]
    \centering
    \begin{tabular}{c|cc|ccc}
        & \multicolumn{2}{c}{\textbf{Simulation}} & 
        \multicolumn{3}{c}{\textbf{Markov chain}} \\
        \textbf{Value of} & \textbf{Single} & \textbf{100} & 
        \textbf{Waiting} & \textbf{Blocking} & 
        \textbf{Proportion} \\
        \textbf{\textit{N} and \textit{M}} & \textbf{trial} & \textbf{trials} & 
        \textbf{formula} & \textbf{formula} & \textbf{formula} \\
        \hline
        \(10\) & 1 & 144.3 & 0.015  & 0.014  & 0.014 \\
        \hline
        \(30\) & 1 & 143.4 & 3.73   & 3.83   & 3.65 \\
        \hline
        \(50\) & 1 & 139.8 & 31.57  & 38.39  & 31.98 \\
        \hline
        \(\infty\) & 1 & 142.1 & N/A & N/A & N/A \\
    \end{tabular}
    \caption{Relative timings of the simulation and Markov chain model}
    \label{tab:truncation_effect_timings_old}
\end{table}
\normalsize

After some investigation it was found that a huge proportion of the duration of
time needed to get the performance measures using the Markov chain approach is
due to the creation of the generator matrix defined in
equation~(\ref{eq:markov_transition_rate}).
For example, for \(N = M = 50\) the state space of the Markov chain consists
of approximately \(2500\) states (depending on the value of \(T\)).
Thus, the generator matrix, that consists of the rates from each state to every
other state, has approximately \(2500^2 = 6250000\) entries.
For larger values of \(N\) and \(M\), the creation of this matrix is the most
time consuming part of the Markov chain approach, even though most entries in
the matrix are zero.
By using equation~(\ref{eq:state_map_to_destination_states}), the set of states
with a non-zero rate can be used to fill out the generator matrix.
Thus, instead of iterating over the set states twice, the generator matrix can
be filled out by iterating over the states only once and using
\(\mathcal{M}(u,v)\) defined in
equation~(\ref{eq:state_map_to_destination_states}).
Table~\ref{tab:truncation_effect_timings_new} shows how the relative timings
of the different approaches change when using this smarter approach.


\tiny
\begin{table}[H]
    \centering
    \begin{tabular}{c|cc|ccc}
        & \multicolumn{2}{c}{\textbf{Simulation}} & 
        \multicolumn{3}{c}{\textbf{Markov chain}} \\
        \textbf{Value of} & \textbf{Single} & \textbf{100} & 
        \textbf{Waiting} & \textbf{Blocking} & 
        \textbf{Proportion} \\
        \textbf{\textit{N} and \textit{M}} & \textbf{trial} & \textbf{trials} & 
        \textbf{formula} & \textbf{formula} & \textbf{formula} \\
        \hline
        \(10\) & 1 & 119.2 & 0.000415 & 0.000146 & 0.000274 \\
        \hline
        \(30\) & 1 & 108.2 & 0.008040 & 0.035941 & 0.013451 \\
        \hline
        \(50\) & 1 & 109.3 & 0.147455 & 1.229303 & 0.179336 \\
        \hline
        \(\infty\) & 1 & 127.4 & N/A & N/A & N/A \\
    \end{tabular}
    \caption{Relative timings of the simulation and Markov chain model using
    the smarter approach}
    \label{tab:truncation_effect_timings_new}
\end{table}
\normalsize

Overall, it can be seen that using the Markov chain approach is much faster than
simulating the system.
Although, by choosing a larger value of \(N\) and \(M\) the computation time of
the Markov chain model increases while the simulation time stays relatively 
similar.

\subsubsection{Truncation effect on performance measures}

This section is used to demonstrate the accuracy of the performance measure
formulas of the constructed Markov model compared to the simulation as well as
the effect of truncating the model.
The simulation was run 100 times and the recorded mean waiting time at each
iteration is used to populate the violin plots that are shown in
Figures~\ref{fig:markov_vs_des_waiting_time_comparison_overall},
\ref{fig:markov_vs_des_blocking_time_comparison_overall} and
\ref{fig:markov_vs_des_proportion_within_time_comparison_overall}.

Figure \ref{fig:markov_vs_des_waiting_time_comparison_overall} shows a 
comparison between the calculated mean waiting time using Markov chains and the
simulated waiting time using discrete event simulation over a range of values of 
\(\lambda_2\) (details of the discrete event simulation model can be found in 
Section~\ref{sec:discrete_event_simulation}).

\begin{figure}[H]
    \includegraphics[width=\textwidth]{chapters/03_queueing_model/img/numeric_results_and_timings/truncation_effect/waiting/demo.pdf}
    \caption{
        Comparison of mean waiting time between values obtained from the Markov 
        chain formula, values obtained from the truncated simulation and values
        obtained from the untruncated simulation.
    }
    \label{fig:markov_vs_des_waiting_time_comparison_overall}
\end{figure}

In detail, Figure~\ref{fig:markov_vs_des_waiting_time_comparison_overall} shows
the calculated mean waiting time using the Markov chain, using a truncated 
simulation and using a simulation with infinite capacity (without the artificial
parameters \(N\) and \(M\)).
Each plot corresponds to different values of \(N\) and \(M\) and is run over 
different values of \(\lambda_2\).
The untruncated simulation values are the same at all three graphs since
the effect of truncation does not apply to it.
The waiting times generated by the truncated simulation match the ones generated 
by the Markov chains model.
Note that this comparison includes both type 1 and type 2 individuals.
% TODO: Include type 1 and type 2 plot?


Similarly Figure~\ref{fig:markov_vs_des_blocking_time_comparison_overall} shows
the mean blocking time equivalent comparison between the three approaches used
for the waiting time (Markov chain, truncated simulation and untruncated
simulation).
Similar to the waiting time, the blocking time among the different approaches
begin to get closer together as the value of \(N\) and \(M\) increases.
Note that the blocking time can only be calculated for type 2 individuals.

\begin{figure}[H]
    \includegraphics[width=\textwidth]{chapters/03_queueing_model/img/numeric_results_and_timings/truncation_effect/blocking/demo.pdf}
    \caption{
        Comparison of mean waiting time between values obtained from the Markov 
        chain formula, values obtained from the truncated simulation and values
        obtained from the untruncated simulation.
    }
    \label{fig:markov_vs_des_blocking_time_comparison_overall}
\end{figure}

Finally, Figure~\ref{fig:markov_vs_des_proportion_within_time_comparison_overall}
shows the overall proportion of individuals whose time in the system lies within
a time target for different values of \(N\) and \(M\).
Similar to the previous figures, as \(N\) and \(M\) increase the proportion of
individuals between the simulation and the Markov chain get closer.


\begin{figure}[H]
    \includegraphics[width=\textwidth]{chapters/03_queueing_model/img/numeric_results_and_timings/truncation_effect/proportion/demo.pdf}
    \caption{
        Comparison of mean waiting time between values obtained from the Markov 
        chain formula, values obtained from the truncated simulation and values
        obtained from the untruncated simulation.
    }
    \label{fig:markov_vs_des_proportion_within_time_comparison_overall}
\end{figure}

