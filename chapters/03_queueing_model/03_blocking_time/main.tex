\subsection{Blocking time}\label{sec:blocking_time}

Unlike the waiting time in Section~\ref{sec:waiting_time}, 
the blocking time is only calculated for type 2 individuals.
That is because type 1 individuals cannot be blocked.
Thus, one only needs to consider the pathway of type 2 individuals to get the
mean blocking time of the system.

For the waiting time formula described in
equation~\eqref{eq:recursive_waiting_time_for_type_i} in
Section~\ref{sec:recursive_waiting_time}
the expected waiting time for each state was considered by ignoring all
arrivals.
Here, the same approach is used but ignoring only arrivals of type 2
individuals.
That is because for the waiting time formula, once an individual enters
node 1 (i.e. starts waiting) any individual arriving after them will
not affect their pathway.
That is not the case for the blocking time.
When a type 2 individual is blocked, any type 1 individual that arrives will
cause the blocked individual to stay blocked for more time.
Therefore, unlike Figure~\ref{fig:markov_variation_no_arrivals}, type 1 arrivals
are considered here.
Once again a variation of the already existing Markov chain model described in
Figure~\ref{fig:adjusted_markov_model} can be seen in
Figure~\ref{fig:markov_variation_no_type_2_arrivals} where type 2 arrivals are
ignored.

\begin{figure}[ht]
    \centering
    
\begin{tikzpicture}[-, node distance = 1.4cm, auto, every node/.style={scale=0.85}]

    \tikzmath{
        let \minsize = 1.8cm;
        let \nosize = 1cm;
    }

    \node[draw=none, minimum size=\nosize] (one) {};
    \node[state, minimum size=\minsize, right=of one] (two) {(0,T-1)};
    \node[state, minimum size=\minsize, right=of two] (three) {(0,T)};
    \node[state, minimum size=\minsize, right=of three] (four) {(0,T+1)};
    \node[draw=none, minimum size=\nosize, right=of four] (five) {};

    \node[state, minimum size=\minsize, below=of three] (three_one) {(1,T)};
    \node[state, minimum size=\minsize, below=of three_one] (three_two) {(2,T)};
    \node[state, minimum size=\minsize, below=of four] (four_one) {(1,T+1)};
    \node[state, minimum size=\minsize, below=of four_one] (four_two) {(2,T+1)};
    \node[draw=none, minimum size=\nosize, right=of four_one] (five_one) {};
    \node[draw=none, minimum size=\nosize, right=of four_two] (five_two) {};
    \node[draw=none, minimum size=\nosize, below=of three_two] (three_three) {};

    \draw[every loop]
        (two) edge[bend left] node {\((T-1) \mu\)} (one)
        (one) edge[bend left] node {\(\lambda_1\)} (two)
        (three) edge[bend left] node {\(T \mu\)} (two)
        (two) edge[bend left] node {\(\lambda_1\)} (three)
        (four) edge[bend left] node {\((T+1) \mu\)} (three)
        (three) edge[bend left] node {\(\lambda_1\)} (four)
        (five) edge[bend left] node {\((T+2) \mu\)} (four)
        (four) edge[bend left] node {\(\lambda_1\)} (five)
        (three_one) edge node {\(T \mu\)} (three)
        (four_one) edge[bend left] node {\((T+1) \mu\)} (three_one)
        (three_one) edge[bend left] node {\(\lambda_1\)} (four_one)
        (five_one) edge[bend left] node {\((T+2) \mu\)} (four_one)
        (four_one) edge[bend left] node {\(\lambda_1\)} (five_one)
        (three_two) edge node {\(T \mu\)} (three_one)
        (three_three) edge node {\(T \mu\)} (three_two)
        (four_two) edge[bend left] node {\((T+1) \mu\)} (three_two)
        (three_two) edge[bend left] node {\(\lambda_1\)} (four_two)
        (five_two) edge[bend left] node {\((T+2) \mu\)} (four_two)
        (four_two) edge[bend left] node {\(\lambda_1\)} (five_two)
        ;
\end{tikzpicture}
    \caption{Variation of Markov chain model where type 2 arrivals are removed
    (i.e. all arrows pointing down with a rate of \(\lambda_1\) are removed).
    This diagram is used as a visualisation aid for the blocking time formula.}
    \label{fig:markov_variation_no_type_2_arrivals}
\end{figure}

By the nature of this new Markov chain variation a similar recursive approach
to the waiting time cannot be used here.
Since both service completions and new arrivals can occur, the path of an
individual from arrival to departure is not fixed.
For example, for a particular Markov chain model with a threshold of \(T=2\),
an individual arriving at state \((2, 3)\) may have multiple different pathways.
Both of these are valid paths:
\begin{itemize}
    \item \((2, 3) \rightarrow (2, 2) \rightarrow (1, 2) \rightarrow (0, 2)\)
    \item \((2, 3) \rightarrow (2, 4) \rightarrow (2, 3) \rightarrow (2, 2)
    \rightarrow (1, 2) \rightarrow (0, 2)\)
\end{itemize}

Similar to equations~\eqref{eq:waiting_time_state_type_1}
and~\eqref{eq:waiting_time_state_type_2} the expected time spent in each state
here is denoted as:

\begin{equation}\label{eq:blocking_time_state}
    c_b(u,v) =
    \begin{cases}
        \frac{1}{\min(v,C) \mu}, & \text{if } v \leq C\\
        \frac{1}{\min(v,C) \mu + \lambda_1}, & \text{otherwise}
    \end{cases}
\end{equation}

In equation~\eqref{eq:blocking_time_state}, both service completions and
type 1 arrivals are considered.
Thus, from a blocked individual's perspective whenever the system moves from one
state \((u,v)\) to another state it can either be:

\begin{itemize}
    \item because of a service being completed: we will denote the
    probability of this happening by \(p_s(u,v)\).
    \item because of an arrival of an individual of type 1: denoting such
    probability by \(p_o(u,v)\).
\end{itemize}

These probabilities are given by:

\begin{equation}\label{eq:blocking_time_probs}
    p_s(u,v) = \frac{\min(v,C)\mu}{\lambda_1 + \min(v,C)\mu}, \qquad
    p_o(u,v) = \frac{\lambda_1}{\lambda_1 + \min(v,C)\mu}
\end{equation}




The set of states where blocking can occur is defined as the \textit{blocking
states} and consists of all states \((u,v)\) where \(u\) is non-zero.
In essence, the set of blocking state \(S_b\) is defined as:

\begin{equation}\label{eq:blocking_states}
    S_b = \{(u,v) \in S \; | \; u > 0\}
\end{equation}

From Figure~\ref{fig:markov_variation_no_type_2_arrivals} the set \(S_b\)
consists of all states below the first line of Markov chain.
In addition, in order to not consider individuals that will be lost to the
system, the set of accepting states needs to be taken into account.
As defined in Section~\ref{sec:recursive_waiting_time}, the set of accepting
states \(S_A^{(2)}\) is given by equation~\eqref{eq:accepting_states_type_2}.

Having defined \(c_b(u,v)\) and \(S_b\) a formula for the expected blocking time
at each state can be given by:

\small
\begin{equation}\label{eq:expected_blocking_time_at_each_state}
    b(u,v) =
    \begin{cases}
        0, & \textbf{if } (u,v) \notin S_b \\
        c_b(u,v) + b(u - 1, v), & \textbf{if } v = N = T\\
        c_b(u,v) + b(u, v-1), & \textbf{if } v = N \neq T \\
        c_b(u,v) + p_s(u,v) b(u-1, v) + p_o(u,v) b(u, v+1), & \textbf{if } u > 0
        \textbf{ and } v = T \\
        c_b(u,v) + p_s(u,v) b(u, v-1) + p_o(u,v) b(u, v+1), & \textbf{otherwise} \\
    \end{cases}
\end{equation}
\normalsize

Unlike equation~\eqref{eq:recursive_waiting_time_for_type_i},
equation~\eqref{eq:expected_blocking_time_at_each_state} cannot be solved
recursively.
Only a direct approach will be used to solve this equation.
By enumerating all possible equations generated
by~\eqref{eq:expected_blocking_time_at_each_state} for all states \((u,v)\) that
belong in \(S_b\) a system of linear equations arises where the unknown
variables are all the \(b(u,v)\) terms.
For instance, let us consider a Markov model where \(C=2, T=3, N=6, M=2\).
The Markov model is shown in Figure~\ref{fig:example_algeb_blocking}
and the equivalent equations are~\eqref{eq:first_eq_of_blocking_example}
-~\eqref{eq:last_eq_of_blocking_example}.
The equations considered here are only the ones that correspond to the blocking
states.

\begin{minipage}{0.45\textwidth}
    \begin{figure}[H]
        \scalebox{0.5}{\begin{tikzpicture}[-, node distance = 1cm, auto]
\node[state] (u0v0) {(0,0)};
\node[state, right=of u0v0] (u0v1) {(0,1)};
\draw[->](u0v0) edge[bend left] node {\( \Lambda \)} (u0v1);
\draw[->](u0v1) edge[bend left] node {\(\mu \)} (u0v0);
\node[state, right=of u0v1] (u0v2) {(0,2)};
\draw[->](u0v1) edge[bend left] node {\( \Lambda \)} (u0v2);
\draw[->](u0v2) edge[bend left] node {\(2\mu \)} (u0v1);
\node[state, below=of u0v2] (u1v2) {(1,2)};
\draw[->](u0v2) edge[bend left] node {\( \lambda_2 \)} (u1v2);
\draw[->](u1v2) edge[bend left] node {\(2\mu \)} (u0v2);
\node[state, below=of u1v2] (u2v2) {(2,2)};
\draw[->](u1v2) edge[bend left] node {\( \lambda_2 \)} (u2v2);
\draw[->](u2v2) edge[bend left] node {\(2\mu \)} (u1v2);
\node[state, right=of u0v2] (u0v3) {(0,3)};
\draw[->](u0v2) edge[bend left] node {\( \lambda_1 \)} (u0v3);
\draw[->](u0v3) edge[bend left] node {\(2\mu \)} (u0v2);
\node[state, right=of u1v2] (u1v3) {(1,3)};
\draw[->](u1v2) edge[bend left] node {\( \lambda_1 \)} (u1v3);
\draw[->](u1v3) edge[bend left] node {\(2\mu \)} (u1v2);
\draw[->](u0v3) edge node {\( \lambda_2 \)} (u1v3);
\node[state, right=of u2v2] (u2v3) {(2,3)};
\draw[->](u2v2) edge[bend left] node {\( \lambda_1 \)} (u2v3);
\draw[->](u2v3) edge[bend left] node {\(2\mu \)} (u2v2);
\draw[->](u1v3) edge node {\( \lambda_2 \)} (u2v3);
\node[state, right=of u0v3] (u0v4) {(0,4)};
\draw[->](u0v3) edge[bend left] node {\( \lambda_1 \)} (u0v4);
\draw[->](u0v4) edge[bend left] node {\(2\mu \)} (u0v3);
\node[state, right=of u1v3] (u1v4) {(1,4)};
\draw[->](u1v3) edge[bend left] node {\( \lambda_1 \)} (u1v4);
\draw[->](u1v4) edge[bend left] node {\(2\mu \)} (u1v3);
\draw[->](u0v4) edge node {\( \lambda_2 \)} (u1v4);
\node[state, right=of u2v3] (u2v4) {(2,4)};
\draw[->](u2v3) edge[bend left] node {\( \lambda_1 \)} (u2v4);
\draw[->](u2v4) edge[bend left] node {\(2\mu \)} (u2v3);
\draw[->](u1v4) edge node {\( \lambda_2 \)} (u2v4);
\end{tikzpicture}}
        \caption{Example of Markov model with \(C=2, T=3, N=6, M=2\).}
        \label{fig:example_algeb_blocking}
    \end{figure}
\end{minipage}
\begin{minipage}{0.5\textwidth}
    \scriptsize
    \begin{align}
        b(1,2) &= c_b(1,2) + p_o b(1,3) \label{eq:first_eq_of_blocking_example} \\
        b(1,3) &= c_b(1,3) + p_s b(1,2) + p_o b(1,4) \\
        b(1,4) &= c_b(1,4) + b(1,3) \\
        b(2,2) &= c_b(2,2) + p_s b(1,2) + p_o b(2,3) \\
        b(2,3) &= c_b(2,3) + p_s b(2,2) + p_o b(1,4) \\
        b(2,4) &= c_b(2,4) + b(2,3)\label{eq:last_eq_of_blocking_example}
    \end{align}
    \normalsize
\end{minipage}
\vspace{0.5cm}

Additionally, the above equations can be transformed into a linear system of the
form \(Zx=y\) where:

\begin{equation}\label{eq:example_direct_approach_blocking_time}
    Z=
    \begin{pmatrix}
         -1 & p_o &   0 &   0 &   0 &   0 \\ %(1,2)
        p_s &  -1 & p_o &   0 &   0 &   0 \\ %(1,3)
          0 &   1 & - 1 &   0 &   0 &   0 \\ %(1,4)
        p_s &   0 &   0 &  -1 & p_o &   0 \\ %(2,2)
          0 &   0 &   0 & p_s &  -1 & p_o \\ %(2,3)
          0 &   0 &   0 &   0 &   1 &  -1 \\ %(2,4)
    \end{pmatrix},
    x=
    \begin{pmatrix}
        b(1,2) \\
        b(1,3) \\
        b(1,4) \\
        b(2,2) \\
        b(2,3) \\
        b(2,4) \\
    \end{pmatrix},
    y=
    \begin{pmatrix}
        -c_b(1,2) \\
        -c_b(1,3) \\
        -c_b(1,4) \\
        -c_b(2,2) \\
        -c_b(2,3) \\
        -c_b(2,4) \\
    \end{pmatrix}
\end{equation}

A more generalised form of the linear system
of~\eqref{eq:example_direct_approach_blocking_time} can thus be given for any
value of \(C,T,N,M\) by:

\begin{align}
    b(1,T) =& c_b(1, T) + p_o b(1, T + 1)
    \label{eq:first_eq_of_blocking_general}\\
    b(1,T + 1) =& c_b(1, T + 1) + p_s(1, T) + p_o b(1, T + 1) \\
    b(1,T + 2) =& c_b(1, T + 2) + p_s(1, T + 1) + p_o b(1, T + 3) \\
    & \vdots \\
    b(1, N) =& c_b(1, N) + b(1, N - 1) \\
    b(2, T) =& c_b(2, T) + p_s b(1, T) + p_o b(2, T + 1) \\
    b(2, T + 1) =& c_b(2, T + 1) + p_s b(2, T) + p_o b(2, T + 2) \\
    & \vdots \\
    b(M, T) =& c_b(M, T) + b(M, T-1) \label{eq:last_eq_of_blocking_general}
\end{align}

The equivalent matrix form of the linear system of
equations~\eqref{eq:first_eq_of_blocking_general}
-~\eqref{eq:last_eq_of_blocking_general}
is given by \(Zx=y\), where:

\newcommand{\secondallthedots}{\vdots & \vdots & \vdots & \ddots & \vdots & \vdots &
\vdots & \vdots & \vdots & \ddots & \vdots & \vdots}

\begin{equation}
    Z =
    \begin{pmatrix}
        -1 & p_o & 0 & \dots & 0 & 0 & 0 & 0 & 0 & \dots & 0 & 0 \\ %(1,T)
        p_s & -1 & p_o & \dots & 0 & 0 & 0 & 0 & 0 & \dots & 0 & 0 \\ %(1,T+1)
        0 & p_s & -1 & \dots & 0 & 0 & 0 & 0 & 0 & \dots & 0 & 0 \\ %(1,T+2)
        \secondallthedots \\
        0 & 0 & 0 & \dots & 1 & -1 & 0 & 0 & 0 & \dots & 0 & 0 \\ %(1,N)
        p_s & 0 & 0 & \dots & 0 & 0 & -1 & p_o & 0 & \dots & 0 & 0 \\ %(2,T)
        0 & 0 & 0 & \dots & 0 & 0 & p_s & -1 & p_o & \dots & 0 & 0 \\ %(2,T+1)
        \secondallthedots \\
        0 & 0 & 0 & \dots & 0 & 0 & 0 & 0 & 0 & \dots & 1 & -1 \\ %(M,T)
    \end{pmatrix},
\end{equation}
\begin{equation}\label{eq:general_direct_approach_blocking_time}
    x =
    \begin{pmatrix}
        b(1,T) \\
        b(1,T+1) \\
        b(1,T+2) \\
        \vdots \\
        b(1,N) \\
        b(2,T) \\
        b(2,T+1) \\
        \vdots \\
        b(M,T) \\
    \end{pmatrix},
    y=
    \begin{pmatrix}
        -c_b(1,T) \\
        -c_b(1,T+1) \\
        -c_b(1,T+2) \\
        \vdots \\
        -c_b(1,N) \\
        -c_b(2,T) \\
        -c_b(2,T+1) \\
        \vdots \\
        -c_b(M,T) \\
    \end{pmatrix}
\end{equation}

Thus, having calculated the mean blocking time \(b(u,v)\) for every blocking
state individually, a similar formula to
equation~\eqref{eq:recursive_waiting_time_for_type_i} can be derived.
The resultant blocking time formula is given by:

\begin{equation}\label{eq:blocking_time_formula}
    B = \frac{\sum_{(u,v) \in S_A} \pi_{(u,v)} \; b(\mathcal{A}_2(u,v))}{
        \sum_{(u,v) \in S_A}\pi_{(u,v)}}
\end{equation}

Note here that \(\pi_(u,v)\) is the steady state probability that the Markov
chain model is at state \((u,v)\) described in
Section~\ref{sec:steady_state_probabilities}.

\subsubsection{Implementation}\label{sec:implementation_blocking_time}

The mean blocking time is only calculated using a direct approach
similar to the one described in Section~\ref{sec:waiting_direct_implementation}.
Since this implementation is the same as the waiting time one, the following
codes shows only the usage of the function rather than the function itself.

\begin{lstlisting}[style=pystyle]
>>> import ambulance_game as abg
>>> import numpy as np
>>>
>>> all_states = abg.markov.build_states(
...     threshold=2,
...     system_capacity=4,
...     buffer_capacity=3,
... )
>>> Q = abg.markov.get_transition_matrix(
...     lambda_1=1,
...     lambda_2=1,
...     mu=4,
...     num_of_servers=1,
...     threshold=2,
...     system_capacity=4,
...     buffer_capacity=3
... )
>>> pi = abg.markov.get_markov_state_probabilities(
...     abg.markov.get_steady_state_algebraically(
...         Q, algebraic_function=np.linalg.solve
...     ),
...     all_states,
... )
>>> round(
...     abg.markov.mean_blocking_time_formula_using_direct_approach(
...         all_states=all_states,
...         pi=pi,
...         lambda_1=1,
...         mu=4,
...         num_of_servers=1,
...         threshold=2,
...         system_capacity=4,
...         buffer_capacity=3,
...     ), 10
... )
0.1287843179

\end{lstlisting}
