\subsection{Proportion of individuals within target}
\label{sec:proportion_of_individuals_within_time}

Another performance measure that needs to be taken into consideration is the
proportion of individuals whose waiting and service times are within a
specified time target.
In order to consider such a measure though one would need to obtain the
distribution of time in the system for all individuals.
The complexity of such a task comes from the fact that different individuals
arrive at different states of the Markov model.
Consider the case when an arrival occurs when the model is at a specific state.

\subsubsection{Distribution of time at a specific state (with 1 server)}

\begin{figure}[H]
    \centering
    \scalebox{0.75}{\begin{tikzpicture}[-, node distance = 1cm, auto]
\node[state] (u0v0) {(0,0)};
\node[state, right=of u0v0] (u0v1) {(0,1)};
\draw[->](u0v0) edge[bend left] node {\( \Lambda \)} (u0v1);
\draw[->](u0v1) edge[bend left] node {\(\mu \)} (u0v0);
\node[state, right=of u0v1] (u0v2) {(0,2)};
\draw[->](u0v1) edge[bend left] node {\( \Lambda \)} (u0v2);
\draw[->](u0v2) edge[bend left] node {\(\mu \)} (u0v1);
\node[state, below=of u0v2] (u1v2) {(1,2)};
\draw[->](u0v2) edge[bend left] node {\( \lambda_2 \)} (u1v2);
\draw[->](u1v2) edge[bend left] node {\(\mu \)} (u0v2);
\node[state, below=of u1v2] (u2v2) {(2,2)};
\draw[->](u1v2) edge[bend left] node {\( \lambda_2 \)} (u2v2);
\draw[->](u2v2) edge[bend left] node {\(\mu \)} (u1v2);
\node[state, right=of u0v2] (u0v3) {(0,3)};
\draw[->](u0v2) edge[bend left] node {\( \lambda_1 \)} (u0v3);
\draw[->](u0v3) edge[bend left] node {\(\mu \)} (u0v2);
\node[state, right=of u1v2] (u1v3) {(1,3)};
\draw[->](u1v2) edge[bend left] node {\( \lambda_1 \)} (u1v3);
\draw[->](u1v3) edge[bend left] node {\(\mu \)} (u1v2);
\draw[->](u0v3) edge node {\( \lambda_2 \)} (u1v3);
\node[state, right=of u2v2] (u2v3) {(2,3)};
\draw[->](u2v2) edge[bend left] node {\( \lambda_1 \)} (u2v3);
\draw[->](u2v3) edge[bend left] node {\(\mu \)} (u2v2);
\draw[->](u1v3) edge node {\( \lambda_2 \)} (u2v3);
\node[state, right=of u0v3] (u0v4) {(0,4)};
\draw[->](u0v3) edge[bend left] node {\( \lambda_1 \)} (u0v4);
\draw[->](u0v4) edge[bend left] node {\(\mu \)} (u0v3);
\node[state, right=of u1v3] (u1v4) {(1,4)};
\draw[->](u1v3) edge[bend left] node {\( \lambda_1 \)} (u1v4);
\draw[->](u1v4) edge[bend left] node {\(\mu \)} (u1v3);
\draw[->](u0v4) edge node {\( \lambda_2 \)} (u1v4);
\node[state, right=of u2v3] (u2v4) {(2,4)};
\draw[->](u2v3) edge[bend left] node {\( \lambda_1 \)} (u2v4);
\draw[->](u2v4) edge[bend left] node {\(\mu \)} (u2v3);
\draw[->](u1v4) edge node {\( \lambda_2 \)} (u2v4);
\end{tikzpicture}}
    \caption{Example of Markov model with \(C=1, T=2, N=4, M=2\)}
    \label{fig:distribution_of_time_at_specific_state_1_server}
\end{figure}

Consider the Markov model of
Figure~\ref{fig:distribution_of_time_at_specific_state_1_server}
with one server (i.e.
the rate of service completion is \(\mu\) throughout the Markov model)
and a threshold of two individuals.
Assume that a type 1 individual arrives when the model is at state
\((0,3)\), thus forcing the model to move to state \((0,4)\).
The distribution of the time needed for the specified individual to exit the
system from state \((0,4)\) is given by the sum of exponentially distributed
random variables with the same parameter \(\mu\).
The sum of such random variables form the Erlang distribution which is defined
by the number of random variables \(k\) that are added together and their
exponential parameter \(\mu\).

\begin{align}
    & X_i \sim \text{Exp}(\mu) \nonumber \\
    X_1 + X_2 + & \dots + X_k \sim \text{Erlang}(k,\mu)
    \label{eq:erlang_distribution_definition}
\end{align}

Note here that these random variables represent the individual's pathway from
the perspective of the individual.
Thus, \(X_i\) represents the random variable of the time that it takes for an
individual to move from the \(i^{\text{th}}\) position of the queue to the
\((i-1)^{\text{th}}\) position (i.e. for someone in front of them to finish
their service) and \(X_0\) is the time it takes that individual from
starting their service to exiting the system.


\begin{align}
    (0,4) \Rightarrow \quad & X_3 \sim Exp(\mu) \nonumber \\
    (0,3) \Rightarrow \quad & X_2 \sim Exp(\mu) \nonumber \\
    (0,2) \Rightarrow \quad & X_1 \sim Exp(\mu) \nonumber \\
    (0,1) \Rightarrow \quad & X_0 \sim Exp(\mu) \nonumber \\
    S = X_3 + X_2 + & X_1 + X_0 = Erlang(4, \mu)
\end{align}

Thus, the waiting and service time of an individual in the model of
Figure~\ref{fig:distribution_of_time_at_specific_state_1_server}
can be captured by an
Erlang distributed random variable.
The general CDF of the Erlang distribution \(Erlang(k, \mu)\) is given by:

\begin{equation} \label{eq:cdf_erlang}
    P(S < t) = 1 - \sum_{i=0}^{k-1} \frac{1}{i!} e^{-\mu t} (\mu t)^i
\end{equation}

Unfortunately, the Erlang distribution can only be used for the sum of
identically distributed random variables from the exponential distribution.
Therefore, this approach cannot be used when one of the random variables has a
different parameter than the others.
In fact the only case where this can be use is only when the number of servers 
are \(C=1\), similar to the explored example, or when an individual arrives
and goes straight to service (i.e. when there is no other individual waiting
and there is an empty server).


\subsubsection{Distribution of time at a specific state (with multiple servers)}

\begin{figure}[H]
    \centering
    \scalebox{0.75}{\begin{tikzpicture}[-, node distance = 1cm, auto]
\node[state] (u0v0) {(0,0)};
\node[state, right=of u0v0] (u0v1) {(0,1)};
\draw[->](u0v0) edge[bend left] node {\( \Lambda \)} (u0v1);
\draw[->](u0v1) edge[bend left] node {\(\mu \)} (u0v0);
\node[state, right=of u0v1] (u0v2) {(0,2)};
\draw[->](u0v1) edge[bend left] node {\( \Lambda \)} (u0v2);
\draw[->](u0v2) edge[bend left] node {\(2\mu \)} (u0v1);
\node[state, below=of u0v2] (u1v2) {(1,2)};
\draw[->](u0v2) edge[bend left] node {\( \lambda_2 \)} (u1v2);
\draw[->](u1v2) edge[bend left] node {\(2\mu \)} (u0v2);
\node[state, below=of u1v2] (u2v2) {(2,2)};
\draw[->](u1v2) edge[bend left] node {\( \lambda_2 \)} (u2v2);
\draw[->](u2v2) edge[bend left] node {\(2\mu \)} (u1v2);
\node[state, right=of u0v2] (u0v3) {(0,3)};
\draw[->](u0v2) edge[bend left] node {\( \lambda_1 \)} (u0v3);
\draw[->](u0v3) edge[bend left] node {\(2\mu \)} (u0v2);
\node[state, right=of u1v2] (u1v3) {(1,3)};
\draw[->](u1v2) edge[bend left] node {\( \lambda_1 \)} (u1v3);
\draw[->](u1v3) edge[bend left] node {\(2\mu \)} (u1v2);
\draw[->](u0v3) edge node {\( \lambda_2 \)} (u1v3);
\node[state, right=of u2v2] (u2v3) {(2,3)};
\draw[->](u2v2) edge[bend left] node {\( \lambda_1 \)} (u2v3);
\draw[->](u2v3) edge[bend left] node {\(2\mu \)} (u2v2);
\draw[->](u1v3) edge node {\( \lambda_2 \)} (u2v3);
\node[state, right=of u0v3] (u0v4) {(0,4)};
\draw[->](u0v3) edge[bend left] node {\( \lambda_1 \)} (u0v4);
\draw[->](u0v4) edge[bend left] node {\(2\mu \)} (u0v3);
\node[state, right=of u1v3] (u1v4) {(1,4)};
\draw[->](u1v3) edge[bend left] node {\( \lambda_1 \)} (u1v4);
\draw[->](u1v4) edge[bend left] node {\(2\mu \)} (u1v3);
\draw[->](u0v4) edge node {\( \lambda_2 \)} (u1v4);
\node[state, right=of u2v3] (u2v4) {(2,4)};
\draw[->](u2v3) edge[bend left] node {\( \lambda_1 \)} (u2v4);
\draw[->](u2v4) edge[bend left] node {\(2\mu \)} (u2v3);
\draw[->](u1v4) edge node {\( \lambda_2 \)} (u2v4);
\end{tikzpicture}}
    \caption{Example of Markov model with \(C=2, T=2, N=4, M=2\)}
    \label{fig:distribution_of_time_at_specific_state_2_servers}
\end{figure}

Figure~\ref{fig:distribution_of_time_at_specific_state_2_servers} represents the
same Markov model as
Figure~\ref{fig:distribution_of_time_at_specific_state_1_server} with the only
exception that there are 2 servers here.
By applying the same logic, assuming that an individual arrives at state
\((0,4)\), the sum of the following random variables arises.

\begin{align}
    (0,4) \Rightarrow \quad & X_2 \sim Exp(2\mu) \nonumber \\
    (0,3) \Rightarrow \quad & X_1 \sim Exp(2\mu) \\
    (0,2) \Rightarrow \quad & X_0 \sim Exp(\mu) \nonumber
\end{align}

Since these exponentially distributed random variables do not share the same
parameter, an Erlang distribution cannot be used.
In fact, the problem can now be viewed as the sum of exponentially
distributed random variables with different parameters, which is in turn the
sum of Erlang distributed random variables.
The sum of Erlang distributed random variables is said to follow the
hypoexponential distribution.
The hypoexponential distribution is defined with two vectors of size equal
to the number of Erlang random variables that are added
together~\cite{Akkouchi2008, Smaili2013}.
For this particular example:

\footnotesize
\begin{align} \label{eq:multiple_servers_distribution_example}
    \begin{rcases}
        \begin{rcases}
            X_2 \sim Exp(2\mu) \\
            X_1 \sim Exp(2\mu)
        \end{rcases}
        X_1 + X_2 = S_1 \sim Erlang(2, 2\mu) \\
        X_0 \sim Exp(\mu) \Rightarrow \hspace{1cm} X_0 = S_2 \sim Erlang(1, \mu)
    \end{rcases}
    S_1 + S_2 = H \sim Hypo((2,1), (2\mu, \mu))
\end{align}
\normalsize

The random variable \(H\) from
equation~\eqref{eq:multiple_servers_distribution_example} follows a
hypoexponential
distribution with two vector parameters (\((2,1)\) and \((2\mu, \mu)\)).
The CDF of this distribution can be therefore used to get the probability of the
time spent in the system being less than a given target.
The CDF of the general hypoexponential distribution \(Hypo(\vec{r},
\vec{\lambda})\), is given by the~\eqref{eq:general_cdf_hypoexponential}, where
vector \(\vec{r}\) contains all \(k\)-values of the Erlang distributions
defined in~\eqref{eq:erlang_distribution_definition} and \(\vec{\lambda}\)
is a vector of the distinct parameters~\cite{Favaro2010}.

\begin{align} \label{eq:general_cdf_hypoexponential}
    & P(H < t) = 1 - \left( \prod_{j=1}^{\mid \vec{r} \mid} \lambda_j^{r_j}
    \right) \sum_{k=1}^{\mid \vec{r} \mid} \sum_{l=1}^{r_k}
    \frac{\Psi_{k,l}(-\lambda_k)t^{r_k - l} e^{-\lambda_k t}}
    {(r_k - l)! (l - 1)!} \nonumber \\
    & \textbf{where} \qquad \Psi_{k,l}(t) = - \frac{\partial^{l - 1}}
    {\partial t ^{l - 1}} \left( \prod_{j = 0, j \neq k}^{\mid \vec{r} \mid}
    (\lambda_j + t)^{-r_j} \right) \nonumber \\
    & \textbf{and} \quad \qquad \lambda_0 = 0, r_0 = 1
\end{align}


The computation of the derivative makes
equation~\eqref{eq:general_cdf_hypoexponential} computationally expensive.
In~\cite{Legros2015} an alternative linear version of that CDF is explored via
matrix analysis, and is given by the following formula:

\small
\begin{equation} \label{eq:linear_general_cdf_hypoexponential}
    \begin{split}
        F(x) = &1 - \sum_{k=1}^{n} \sum_{l=0}^{k-1} (-1)^{k-1} \binom{n}{k}
            \binom{k-1}{l} \sum_{j=1}^{n} \sum_{s=1}^{j-1} e^{-x \lambda_s}
            \prod_{l=1}^{s-1} \left( \frac{\lambda_l}{\lambda_l - \lambda_s} \right)
            ^ {k_s} \\
        & \times \sum_{s < a_1 < \dots < a_{l-1} < j}
            \left( \frac{\lambda_s}{\lambda_s - \lambda_{a_1}} \right) ^ {k_s}
            \prod_{m=s+1}^{a_1-1} \left( \frac{\lambda_m}{\lambda_m -
            \lambda_{a_1}}\right) ^ {k_m}
            \prod_{n=a_1}^{a_2-1} \left( \frac{\lambda_n}{\lambda_n -
            \lambda_{a_2}}\right) ^ {k_n} \\
        & \dots \prod_{r=a_l-1}^{j-1} \left( \frac{\lambda_r}{\lambda_r -
            \lambda_{a_j}}\right) ^ {k_r}
            \sum_{q=0}^{k_s - 1} \frac{((\lambda_s - \lambda_{a_1})x)^q}{q!}, \\
        & \text{for } \geq 0
    \end{split}
\end{equation}
\normalsize

Although equation~\eqref{eq:linear_general_cdf_hypoexponential} is a simplified
version of equation~\eqref{eq:general_cdf_hypoexponential} it still has some
unnecessary complexity.
The described expressions are general expressions used to get the CDF of
the hypoexponential distribution, which is in turn the sum of multiple
Erlang distributed random variables.
However, the random variable \(H\), described
in~\eqref{eq:multiple_servers_distribution_example}, is the sum of only two
different erlang distributed random variables.
Thus, perhaps a more simplified version of the above expressions can be derived
that is specific to the case of two Erlang distributed random variables.


\subsubsection{Specific CDF of hypoexponential distribution}
Equations~\eqref{eq:general_cdf_hypoexponential}
and~\eqref{eq:linear_general_cdf_hypoexponential} refer to the general CDF
of the
hypoexponential distribution where the size of the vector parameters can be of
any size~\cite{Favaro2010}.
In the Markov chain models described in
Figures~\ref{fig:distribution_of_time_at_specific_state_1_server} and
\ref{fig:distribution_of_time_at_specific_state_2_servers} the parameter vectors
of the hypoexponential distribution are of size two, and in fact, for any
possible version of the investigated Markov chain model the vectors can only be
of size two.
This is true since for any dimensions of this Markov chain model there will
always be at most two distinct exponential parameters; the parameter for
finishing a service (\(\mu\)) and the parameter for moving forward in the queue
(\(C \mu\)).
For the unique case of \(C=1\) the hypoexponential distribution will not be
used as this is equivalent to an Erlang distribution.
Therefore, by fixing the sizes of \(\vec{r}\) and \(\vec{\lambda}\) to 2, the
following specific expression for the CDF of the hypoexponential distribution
arises, where the derivative is removed:


\begin{align} \label{eq:specific_cdf_hypoexponential}
    & P(H < t) = 1 - \left( \prod_{j=1}^{\mid \vec{r} \mid} \lambda_j^{r_j}
    \right) \sum_{k=1}^{\mid \vec{r} \mid} \sum_{l=1}^{r_k}
    \frac{\Psi_{k,l}(-\lambda_k)t^{r_k - l} e^{-\lambda_k t}}{(r_k - l)!
    (l - 1)!} \nonumber \\
    & \textbf{where} \qquad \Psi_{k,l}(t) =
    \begin{cases}
        \frac{(-1)^{l} (l-1)!}{\lambda_2} \left[\frac{1}{t^l} - \frac{1}
        {(t + \lambda_2)^l}\right] , & k=1 \\
        - \frac{1}{t (t + \lambda_1)^{r_1}}, & k=2
    \end{cases} \nonumber \\
    & \textbf{and} \quad \qquad \lambda_0 = 0, r_0 = 1
\end{align}

Note here that the only difference between
equation~\eqref{eq:general_cdf_hypoexponential}
and~\eqref{eq:specific_cdf_hypoexponential} is the \(\Psi\), where it is now
only computed for \(k=1,2\).
The following subsection proves the following expression:

\begin{equation} \label{eq:hypoexponential_expression_to_proof}
    - \frac{\partial^{l - 1}}{\partial t ^{l - 1}}
    \left(
        \prod_{j = 0, j \neq k}^{\mid \vec{r} \mid} (\lambda_j + t)^{-r_j}
    \right) =
    \begin{cases}
        \frac{(-1)^{l} (l-1)!}{\lambda_2} \left[\frac{1}{t^l} - \frac{1}
        {(t + \lambda_2)^l}\right] , & k=1 \\
        - \frac{1}{t (t + \lambda_1)^{r_1}}, & k=2
    \end{cases}
\end{equation}



\subsubsection{Proof of specific hypoexponential distribution
(eq.~\eqref{eq:hypoexponential_expression_to_proof})}

This section aims to show that there exists a simplified expression of
equation~\eqref{eq:general_cdf_hypoexponential} that is specific to the proposed
Markov model.
Function \(\Psi\) is defined using the parameter \(t\) and the variables \(k\)
and \(l\).
Given the Markov model, the range of values that \(k\) and \(l\) can take can be
bounded.
First, from the range of the double summation in
equation~\eqref{eq:general_cdf_hypoexponential}, it can be seen that
\(k = 1, 2, \dots, \mid \vec{r} \mid\).
Now, \(\mid \vec{r} \mid\) represents the size of the parameter vectors that,
for the Markov model, will always be 2.
That is because, for all the exponentially distributed random variables that are
added together to form the new distribution, there are only two distinct
parameters, thus forming two erlang distributions. Therefore:

\begin{equation*}
    k = 1, 2
\end{equation*}

By observing equation~\eqref{eq:general_cdf_hypoexponential} once more, the
range of values that \(l\) takes are \(l = 1, 2, \dots, r_k\), where \(r_1\) is
subject to the individual's position in the queue and \(r_2 = 1\).
In essence, the hypoexponential distribution will be used with these bounds:

\begin{align}
    k = 1 & \qquad \Rightarrow \qquad l = 1, 2, \dots, r_1 \nonumber \\
    k = 2 & \qquad \Rightarrow \qquad l = 1
\end{align}

Thus the left hand side of equation~\eqref{eq:hypoexponential_expression_to_proof}
needs only to be defined for these bounds.
The specific hypoexponential distribution investigated here is of the form:
\[
    Hypo((r_1, 1)(\lambda_1, \lambda_2))
\]
Note the initial conditions \(\lambda_0=0, r_0=1\) defined in
equation~\eqref{eq:general_cdf_hypoexponential} also hold here.
Thus the proof is split into two parts, for \(k=1\) and \(k=2\).



\begin{itemize}
    \item \(k = 2, l = 1\)
    \begin{equation*}
        \begin{split}
            LHS &= - \frac{\partial^{1-1}}{\partial t^{1-1}}
            \left( \prod_{j=0, j \neq 2}^{2} (\lambda_j + t)^{-r_j} \right) \\
            &=-\left( (\lambda_0 + t)^{-r_0} \times (\lambda_1 + t)^{-r_1}
            \right) \\
            &=-\left( t^{-1} \times (\lambda_1 + t)^{-r_1} \right) \\
            &= - \frac{1}{t(t + \lambda_1)^{r_1}} \\
            & \hspace{7cm} \square
        \end{split}
    \end{equation*}
    \item \(k = 1, l = 1, \dots, r_1\)
    \begin{equation*}
        \begin{split}
            LHS &= -\frac{\partial^{l-1}}{\partial t^{l-1}}
            \left( \prod_{j=0, j \neq 1}^{2} (\lambda_j + t)^{-r_j} \right) \\
            &= -\frac{\partial^{l-1}}{\partial t^{l-1}}
            \left( (\lambda_o + t)^{-r_0} \times (\lambda_2 + t)^{-r_2}
            \right) \\
            &= -\frac{\partial^{l-1}}{\partial t^{l-1}}
            \left( \frac{1}{t(t + \lambda_2)}\right)
        \end{split}
    \end{equation*}

    In essence, the final part of the proof is to show that:

    \[
        -\frac{\partial^{l-1}}{\partial t^{l-1}}
        \left( \frac{1}{t(t + \lambda_2)}\right) =
        \frac{(-1)^{l} (l-1)!}{\lambda_2}\left[\frac{1}{t^l} - \frac{1}{(t +
        \lambda_2)^l}\right]
    \]

    \textbf{Proof by Induction:}
    \begin{enumerate}
        \item Base case (\(l=1\)):
        \begin{equation*}
            \begin{split}
                LHS &= -\frac{\partial^{1-1}}{\partial t^{1-1}}
                \left( \frac{1}{t(t + \lambda_2)}\right) =
                - \frac{1}{t(t + \lambda_2)} \\
                RHS &= \frac{(-1)^{1} (1-1)!}{\lambda_2}
                \left[\frac{1}{t^1} - \frac{1}{(t + \lambda_2)^1}\right] \\
                &= - \frac{t + \lambda_2 - t}{\lambda_2 t (t + \lambda_2)} \\
                &= - \frac{1}{t (t + \lambda_2)} \\
                LHS &= RHS
            \end{split}
        \end{equation*}
        \item Assume true for \(l = x\):
        \begin{equation*}
            -\frac{\partial^{x-1}}{\partial t^{x-1}}
            \left( \frac{1}{t(t + \lambda_2)}\right) =
            \frac{(-1)^{x} (x-1)!}{\lambda_2}
            \left[\frac{1}{t^x} - \frac{1}{(t + \lambda_2)^x}\right]
        \end{equation*}
        \item Prove true for \(l = x + 1\):
        
        \small
        \[
            \Bigg( \text{Show that:} \hspace{0.5cm}
            \frac{\partial^x}{\partial t ^ x}
            \left( \frac{-1}{t (t + \lambda_2)} \right) =
            \frac{(-1)^{x + 1} (x)!}{\lambda_2}
            \left[\frac{1}{t^{x+1}}-\frac{1}{(t + \lambda_2)^{x+1}}\right]\Bigg)
        \]
        \normalsize

        \begin{equation*}
            \begin{split}
                LHS &= \frac{\partial}{\partial t}
                \left[ \frac{\partial^{x-1}}{\partial t ^ {x-1}}
                \left( \frac{-1}{t (t + \lambda_2)} \right) \right] \\
                &= \frac{\partial}{\partial t} \left[
                    \frac{(-1)^x (x-1)!}{\lambda_2} \left(
                        \frac{1}{t^x} - \frac{1}{(t + \lambda_2)^x}
                    \right)
                \right] \\
                &= \frac{(-1)^x (x-1)!}{\lambda_2} \left(
                    \frac{(-x)}{t^{x+1}} - \frac{(-x)}{(t + \lambda_2)^x}
                \right) \\
                &= \frac{(-1)^x (x-1)! (-x)}{\lambda_2} \left(
                    \frac{1}{t^{x+1}} - \frac{1}{(t + \lambda_2)^x}
                \right) \\
                &= \frac{(-1)^{x+1} (x)!}{\lambda_2} \left(
                    \frac{1}{t^{x+1}} - \frac{1}{(t + \lambda_2)^x}
                \right) \\
                & = RHS \\
                & \hspace{7cm} \square
            \end{split}
        \end{equation*}
    \end{enumerate}
\end{itemize}

\subsubsection{Proportion within target for type 1 and type 2 individuals}

Given the two CDFs of the Erlang and Hypoexponential distributions
(equations~\eqref{eq:erlang_distribution_definition}
and~\eqref{eq:specific_cdf_hypoexponential}) a new function has to be defined to
decide which one to use between the two.
Based on the state of the model, there can be three scenarios when an individual
arrives.
\begin{enumerate}
    \item There is a free server and the individual does not have to wait
    \begin{equation*}
        X_{(u,v)} \sim Erlang(1, \mu)
    \end{equation*}
    \item The individual arrives at the queue at the \(n^{th}\) position and the
    model has \(C > 1\) servers
    \begin{equation*}
        X_{(u,v)} \sim Hypo((n, 1), (C \mu, \mu))
    \end{equation*}
    \item The individual arrives at a queue at the \(n^{th}\) position and the
    model has \(C = 1\) servers
    \begin{equation*}
        X_{(u,v)} \sim Erlang(n + 1, \mu)
    \end{equation*}
\end{enumerate}

Note here that for the first case \(Erlang(1, \mu)\) is equivalent to
\(Exp(\mu)\).
Define \(X_{(u,v)}^{(1)}\) as the distribution of type 1 individuals and
\(X_{(u,v)}^{(2)}\) as the distribution of type 2 individuals, when arriving at
state \((u,v)\) of the model.

\small
\begin{equation}
    X_{(u,v)}^{(1)} \sim
    \begin{cases}
        \textbf{Erlang}(v, \mu), & \textbf{if } C = 1 \textbf{ and } v>1 \\
        \textbf{Hypo}\left(\vec{r}=(v - C, 1), \vec{\lambda}=(C \mu, \mu)\right),
            & \textbf{if } C > 1 \textbf{ and } v>C \\
        \textbf{Erlang}(1, \mu), & \textbf{if } v \leq C
    \end{cases}
\end{equation}

\begin{equation}
    X_{(u,v)}^{(2)} \sim
    \begin{cases}
        \textbf{Erlang}(\min(v, T), \mu), & \textbf{if } C = 1
            \textbf{ and } v, T > 1 \\
        \textbf{Hypo}\left(\vec{r}=(\min(v, T) - C, 1), \vec{\lambda}=(C \mu, \mu)\right), &
            \textbf{if } C > 1 \textbf{ and } v, T  > C \\
        \textbf{Erlang}(1, \mu), & \textbf{if } v \leq C \textbf{ or } T \leq C
    \end{cases}
\end{equation}
\normalsize


Equations~\eqref{eq:cdf_erlang} and~\eqref{eq:specific_cdf_hypoexponential} can
now be used.
Therefore, the probability that an individual arriving at a specific state is
within a given time target \(t\) is given by the following formulas:


\footnotesize
\begin{equation}
    P(X_{(u,v)}^{(1)} < t) =
    \begin{cases}
        1 - \sum_{i=0}^{v-1} \frac{1}{i!} e^{-\mu t} (\mu t)^i,
            & \textbf{if } C = 1 \\
        & \textbf{and } v > 1 \\
        & \\
        1 - (\mu C)^{v-C} \mu
            \sum_{k=1}^{\mid \vec{r} \mid} \sum_{l=1}^{r_k}
            \frac{\Psi_{k,l}(-\lambda_k)t^{r_k - l}
            e^{-\lambda_k t}}{(r_k - l)! (l - 1)!},
            & \textbf{if } C > 1 \\
        \textbf{where } \vec{r}=(v - C, 1) \textbf{ and }
            \vec{\lambda}=(C \mu, \mu) & \textbf{and } v > C \\
        & \\
        1 - e^{-\mu t},  & \textbf{if } v \leq C
    \end{cases}
\end{equation}

\begin{equation}
    P(X_{(u,v)}^{(2)} < t) =
    \begin{cases}
        1 - \sum_{i=0}^{\min(v,T)-1} \frac{1}{i!} e^{-\mu t} (\mu t)^i,
            & \textbf{if } C = 1 \\
        & \textbf{and } v, T > 1 \\
        & \\
        1 - (C \mu ) ^ {\min(v,T) - C} \mu
            \sum_{k=1}^{\mid \vec{r} \mid} \sum_{l=1}^{r_k}
            \frac{\Psi_{k,l}(-\lambda_k)t^{r_k - l}
            e^{-\lambda_k t}}{(r_k - l)! (l - 1)!},
            & \textbf{if } C > 1 \\
        \textbf{where } \vec{r}=(\min(v, T) - C, 1) \textbf{ and }
            \vec{\lambda}=(C \mu, \mu) & \textbf{and } v, T  > C \\
        & \\
        1 - e^{-\mu t}, & \textbf{if } v \leq C \\
        & \textbf{or } T \leq C \\
    \end{cases}
\end{equation}
\normalsize

In addition the set of accepting states for type 1 (\(S_A^{(1)}\)) and type 2
(\(S_A^{(2)}\)) individuals defined in
equations~\eqref{eq:accepting_states_type_1}
and~\eqref{eq:accepting_states_type_2} are also needed here.
Note here that, \(S\) denotes the set of all states of the Markov chain model.


Having defined everything, a formula similar to the ones
of equations~\eqref{eq:recursive_waiting_time_for_type_i}
and~\eqref{eq:example_direct_approach_blocking_time} can be generated.
The following formula uses the state probability vector \(\pi\) to get the
weighted average of the probability below target of all states in the Markov
model.

\begin{equation}\label{eq:proportion_of_inds_type_1}
    P(X^{(1)} < t) = \frac{\sum_{(u,v) \in S_A^{(1)}} P(X_{u,v}^{(1)} < t)
    \pi_{u,v} }{\sum_{(u,v) \in S_A^{(1)}} \pi_{u,v}}
\end{equation}

\begin{equation}\label{eq:proportion_of_inds_type_2}
    P(X^{(2)} < t) = \frac{\sum_{(u,v) \in S_A^{(2)}} P(X_{u,v}^{(2)} < t)
    \pi_{u,v} }{\sum_{(u,v) \in S_A^{(2)}} \pi_{u,v}}
\end{equation}


\subsubsection{Overall proportion within target}

The overall proportion of individuals for both type 1 and type 2 individuals
is given by the equivalent formula of
equations~\eqref{eq:overall_waiting_time_coeff}
and~\eqref{eq:overall_waiting_time}.
The following formula uses the probability of lost individuals from both types
to get the weighted sum of the two already existing probabilities.

\begin{equation*}
    P(L'_1) = \sum_{(u,v) \, \in S_A^{(1)}} \pi(u,v), \hspace{1.5cm}
    P(L'_2) = \sum_{(u,v) \, \in S_A^{(2)}} \pi(u,v)
\end{equation*}

\small
\begin{equation}\label{eq:overall_proportion_within_target}
    P(X < t)= \frac{\lambda_1 P(L'_1)}{\lambda_2 P(L'_2)+\lambda_1 P(L'_1)}
    P(X^{(1)} < t) + \frac{\lambda_2 P(L'_2)}{\lambda_2 P(L'_2) + \lambda_1
    P(L'_1)} P(X^{(2)} < t)
\end{equation}
\normalsize

\subsubsection{Implementation}\label{sec:implementation_proportion_individuals}

This section focuses on the implementation of all necessary equations to 
calculate the proportion of individuals within target as described
in Section~\ref{sec:proportion_of_individuals_within_time}.
The first equation to be considered is the simplified version of
\(\Psi_{k, \lambda}(t)\) described in
equation~\eqref{eq:specific_cdf_hypoexponential}.
Code snippet~\ref{lst:specific_psi_function} shows the implementation of this
equation in python.

\begin{lstlisting}[
    style=pystyle,
    caption={Function for the simplified version of \(\Psi_{k, \lambda}(t)\)},
    label={lst:specific_psi_function},
]
>>> def specific_psi_function(
...     arg, k, l, exp_rates, freq, a
... ):
...     """
...     The specific version of the Psi function that is used for the
...     purpose of this study. Due to the way the hypoexponential cdf
...     works the function is called only for values of k=1 and k=2. 
...     For these values the following hold:
...         - k = 1 -> l = 1, ..., n
...         - k = 2 -> l = 1
...     """
...     if k == 1:
...         psi_val = (1 / (arg**l)) - (1 / (arg + exp_rates[2]) ** l)
...         psi_val *= (-1) ** l * math.factorial(l - 1) / exp_rates[2]
...         return psi_val
...     if k == 2:
...         psi_val = -1 / (arg * (arg + exp_rates[1]) ** freq[1])
...         return psi_val
...     return 0

\end{lstlisting}

The piece of code shown in~\ref{lst:hypoexponential_cdf} returns the cumulative
distribution function of the hypoexponential distribution.
In essence this is the value of \(P(H<t)\) outlined in
equation~\eqref{eq:specific_cdf_hypoexponential}.

\begin{lstlisting}[
    style=pystyle,
    caption={Function for the cumulative distribution of the Hypoexponential
    distribution},
    label={lst:hypoexponential_cdf},
]
>>> def hypoexponential_cdf(
...     x, exp_rates, freq, psi_func=specific_psi_function
... ):
...     """
...     The function represents the cumulative distribution function of the
...     hypoexponential distribution. It calculates the probability that a
...     hypoexponentially distributed random variable has a value less than
...     x. In other words calculate P(S < x) where S ~ Hypo(lambda, r)
...     where: lambda is a vector with distinct exponential parameters and 
...     r is a vector with the frequency of each distinct parameter
...     Note that: a Hypoexponentially distributed random variable can be
...     described as the sum of Erlang distributed random variables
...     Parameters
...     ----------
...     x : float
...         The target we want to calculate the probability for
...     exp_rates : tuple
...         The distinct exponential parameters
...     freq : tuple
...         The frequency of the exponential parameters
...     psi_func : function, optional
...         The function to be used to get Psi, by default
...         specific_psi_function
...     Returns
...     -------
...     float
...         P(S < x) where S ~ Hypo(lambda, r)
...     """
...     a = len(exp_rates)
...     exp_rates = (0,) + exp_rates
...     freq = (1,) + freq
...     summation = 0
...     for k in range(1, a + 1):
...         for l in range(1, (freq[k] + 1)):
...             psi = psi_func(
...                 arg=-exp_rates[k],
...                 k=k,
...                 l=l,
...                 exp_rates=exp_rates,
...                 freq=freq,
...                 a=a,
...             )
...             iteration = (
...                 psi * (x ** (freq[k] - l)) * np.exp(-exp_rates[k] * x)
...             )
...             iteration /= (
...                 np.math.factorial(freq[k] - l)*np.math.factorial(l - 1))
...             summation += float(iteration)
...     output = 1 - (
...         product_of_all_elements(
...             [exp_rates[j] ** freq[j] for j in range(1, a + 1)]
...         ) * summation
...     )
...     return output

\end{lstlisting}

Similarly the cumulative distribution function of the erlang distribution is
also needed here.
The code snippet in~\ref{lst:erlang_cdf} shows the implementation of this
function as described in equation~\eqref{eq:cdf_erlang}.

\begin{lstlisting}[
    style=pystyle,
    caption={Function for the cumulative distribution of the Erlang
    distribution},
    label={lst:erlang_cdf},
] 
>>> def erlang_cdf(mu, n, x):
...     """
...     Cumulative distribution function of the erlang distribution.
...     P(X < x) where X ~ Erlang(mu, n)
...     Parameters
...     ----------
...     mu : float
...         The parameter of the Erlang distribution
...     n : int
...         The number of Exponential distributions that are added together
...     x : float
...         The argument of the function
...     Returns
...     -------
...     float
...         The probability that the erlang distributed r.v. is less than x
...     """
...     return 1 - np.sum([
...         np.math.exp(-mu * x) * (mu * x) ** i
...         * (1 / np.math.factorial(i))
...         for i in range(n)
...     ])

\end{lstlisting}

Having defined all functions necessary the code snippet
in~\ref{lst:distribution_decision_function} chooses which of the two 
distributions to use to calculate the probability of an individual being within
a given time target.

\begin{lstlisting}[
    style=pystyle,
    caption={Function for deciding which distribution to use to calculate
    the probability of an individual being within a given time target.},
    label={lst:distribution_decision_function},
]
>>> def get_probability_of_waiting_time_in_system_less_than_target_for_state(
...     state,
...     class_type,
...     mu,
...     num_of_servers,
...     threshold,
...     target,
...     psi_func=specific_psi_function,
... ):
...     """
...     The function decides what probability distribution to use based on
...     the state we are currently on and the class type given. The two
...     distributions that are used are the Erlang and the Hypoexponential
...     distribution. The time it takes the system to exit a state and enter
...     the next one is known to be exponentially distributed. The sum of
...     exponentially distributed random variables is known to result in
...     either an Erlang distribution or a Hypoexponential distribution
...     The function works as follows:
...     - Checks whether the arriving individual will have to wait
...     - Finds the total number of states an individual will have to visit
...     - Depending on whether the parameters of the distributions to sum are
...     the same or not, call the appropriate cdf function.
...     Parameters
...     ----------
...     state : tuple
...     class_type : int
...     mu : float
...     num_of_servers : int
...     threshold : int
...     target : int
...     psi_func : function, optional
...     Returns
...     -------
...     float
...         The probability of spending less time than the target in the
...         system when the individual has arrived at a given state
...     """
...     if class_type == 0:
...         arrive_on_waiting_space = state[1] > num_of_servers
...         rep = state[1] - num_of_servers
...     elif class_type == 1:
...         arrive_on_waiting_space = (
...             state[1] > num_of_servers and threshold > num_of_servers
...         )
...         rep = min(state[1], threshold) - num_of_servers
...     else:
...         raise ValueError("Class type bust be 0 or 1")
... 
...     if arrive_on_waiting_space:
...         if num_of_servers == 1:
...             prob = erlang_cdf(mu=mu, n=rep + 1, x=target)
...         else:
...             param = num_of_servers * mu
...             prob = hypoexponential_cdf(
...                 x=target,
...                 exp_rates=(param, mu),
...                 freq=(rep, 1),
...                 psi_func=psi_func,
...             )
...     else:
...         prob = erlang_cdf(mu=mu, n=1, x=target)
...     return prob

\end{lstlisting}

Finally, putting everything together, going through all the states of a given
Markov chain model, the function defined in~\ref{lst:proportion_within_target}
calculates the probability of
spending less time than the target in the system for a given individual type.
This corresponds to both equations~\eqref{eq:proportion_of_inds_type_1}
and~\eqref{eq:proportion_of_inds_type_2}.

\begin{lstlisting}[
    style=pystyle,
    caption={Function for calculating the probability of spending less time
    than the target in the system for a given individual type.},
    label={lst:proportion_within_target},
]
>>> def get_proportion_of_individuals_within_time_target(
...     all_states,
...     pi,
...     class_type,
...     mu,
...     num_of_servers,
...     threshold,
...     system_capacity,
...     buffer_capacity,
...     target,
...     psi_func=specific_psi_function,
...     **kwargs,
... ):
...     """
...     Gets the probability that a certain class of individuals is within a
...     given time target. This functions runs for every state the function
...     get_probability_of_waiting_time_in_system_less_than_target_for_state
...     and by using the state probabilities to get the average proportion of
...     individuals within target.
...     Parameters
...     ----------
...     all_states : list
...     pi : numpy.array
...     class_type : int
...     mu : float
...     num_of_servers : int
...     threshold : int
...     system_capacity : int
...     buffer_capacity : int
...     target : float
...     psi_func : function, optional
...     Returns
...     -------
...     float
...         The probability of spending less time than the target in the
...         system
...     """
...     proportion_within_limit, probability_of_accepting = 0, 0
...     for (u, v) in all_states:
...         if abg.markov.utils.is_accepting_state(
...             state=(u, v),
...             class_type=class_type,
...             threshold=threshold,
...             system_capacity=system_capacity,
...             buffer_capacity=buffer_capacity,
...         ):
...             arriving_state = (u, v + 1)
...             if class_type == 1 and v >= threshold:
...                 arriving_state = (u + 1, v)
... 
...             proportion_within_limit_at_state = (
...                 get_probability_of_waiting_time_in_system_less_than_target_for_state(
...                     state=arriving_state,
...                     class_type=class_type,
...                     mu=mu,
...                     num_of_servers=num_of_servers,
...                     threshold=threshold,
...                     target=target,
...                     psi_func=psi_func,
...                 )
...             )
...             proportion_within_limit += (
...                 pi[u, v] * proportion_within_limit_at_state
...             )
...             probability_of_accepting += pi[u, v]
...     return proportion_within_limit / probability_of_accepting

\end{lstlisting}

Using all functions created so far, the proportion of individuals within target
can be calculated for a given Markov chain model and a given individual type.

\begin{lstlisting}[
    style=pystyle,
    caption={Combining all functions to calculate the proportion of type 1 
    individuals within a time target of \(1\) time unit.},
    label={lst:proportion_within_target_example},
]
>>> import ambulance_game as abg
>>> import numpy as np
>>> all_states = abg.markov.build_states(
...     threshold=2,
...     system_capacity=4,
...     buffer_capacity=3,
... )
>>> Q = abg.markov.get_transition_matrix(
...     lambda_1=1,
...     lambda_2=1,
...     mu=4,
...     num_of_servers=1,
...     threshold=2,
...     system_capacity=4,
...     buffer_capacity=3
... )
>>> pi = abg.markov.get_markov_state_probabilities(
...     abg.markov.get_steady_state_algebraically(
...         Q, algebraic_function=np.linalg.solve
...     ), all_states
... )
>>> round(get_proportion_of_individuals_within_time_target(
...         all_states=all_states,
...         pi=pi,
...         class_type=0,
...         mu=4,
...         num_of_servers=1,
...         threshold=2,
...         system_capacity=3,
...         buffer_capacity=4,
...         target=1
...     ), 10)
0.9190401179

\end{lstlisting}

This shows that for the given set of parameters and for type 1 individuals
the probability of spending less than 1 unit of time in the system is 91.9\%.
