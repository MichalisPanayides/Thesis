\section{Performance measures}\label{sec:queueing_performance_measures}
Using vector \(\pi\) there are numerous performance measures of the model that
can be calculated.
The following equations utilise \(\pi\) to get performance measures on the
average number of individuals in node 1 and in node 2:

\begin{itemize}
    \item Mean number of individuals in the entire system:
        \begin{equation}
            L_S = \sum_{i=1}^{|\pi|} \pi_i (u_i + v_i)
        \end{equation}
    \item Mean number of individuals in node 1:
        \begin{equation}
            L_1 = \sum_{i=1}^{|\pi|} \pi_i v_i
        \end{equation}
    \item Mean number of individuals in node 2:
        \begin{equation}
            L_2 = \sum_{i=1}^{|\pi|} \pi_i u_i
        \end{equation}
\end{itemize}

The python code for these functions may be obtained quite quickly using the set
of all states and the steady state probability.

\begin{lstlisting}[
    style=pystyle,
    caption={Code snippet for getting the set of all states and the steady
    state probabilities.},
    label={lst:set_of_all_states_and_steady_state_probabilities},
]
>>> import ambulance_game as abg
>>> import numpy as np
>>> all_states = abg.markov.build_states(
...     threshold=3 ,
...     system_capacity=4,
...     buffer_capacity=2
... )
>>> Q = abg.markov.get_transition_matrix(
...     lambda_1=1,
...     lambda_2=2,
...     mu=2,
...     num_of_servers=2,
...     threshold=3,
...     system_capacity=4,
...     buffer_capacity=2
... )
>>> pi = abg.markov.get_steady_state_algebraically(
...     Q, algebraic_function=np.linalg.lstsq
... )

\end{lstlisting}

Having the set of all states and the steady state probabilities, the mean
number of individuals in the system, in node 1 and in node 2 can be calculated
as shown in code snippets~\ref{lst:mean_number_of_individuals_in_system},
\ref{lst:mean_number_of_individuals_in_node_1}
and~\ref{lst:mean_number_of_individuals_in_node_2}.

\begin{lstlisting}[
    style=pystyle,
    caption={Code snippet for calculating the mean number of individuals in
    the system.},
    label={lst:mean_number_of_individuals_in_system},
]
>>> def get_mean_number_of_individuals_in_system(pi, states):
...     """Gets the mean number of individuals in the system
...     Parameters
...     ----------
...     pi : numpy.ndarray
...         steady state vector
...     states : list
...         list of tuples that contains all states
...     Returns
...     -------
...     float
...         Mean number of individuals in the whole model
...     """
...     states = np.array(states)
...     mean_inds_in_system = np.sum((states[:, 0] + states[:, 1]) * pi)
...     return mean_inds_in_system
>>>
>>> round(get_mean_number_of_individuals_in_system(pi, all_states), 10)
2.0872927227

\end{lstlisting}

\begin{lstlisting}[
    style=pystyle,
    caption={Code snippet for calculating the mean number of individuals in
    Node 1.},
    label={lst:mean_number_of_individuals_in_node_1},
]
>>> def get_mean_number_of_individuals_in_node_1(pi, states):
...     """Mean number of individuals in node 1
...     Parameters
...     ----------
...     pi : numpy.ndarray
...         steady state vector
...     states : list
...         list of tuples that contains all states
...     Returns
...     -------
...     float
...         Mean number of individuals
...     """
...     states = np.array(states)
...     mean_inds_in_node_1 = np.sum(states[:, 1] * pi)
...     return mean_inds_in_node_1
>>>
>>> round(get_mean_number_of_individuals_in_node_1(pi, all_states), 10)
1.8187129478

\end{lstlisting}

\begin{lstlisting}[
    style=pystyle,
    caption={Code snippet for calculating the mean number of individuals in
    Node 2.},
    label={lst:mean_number_of_individuals_in_node_2},
]
>>> def get_mean_number_of_individuals_in_node_2(pi, states):
...     """Mean number of class 2 individuals blocked
...     Parameters
...     ----------
...     pi : numpy.ndarray
...         steady state vector
...     states : list
...         list of tuples that contains all states
...     Returns
...     -------
...     float
...         Mean number of blocked class 2 individuals
...     """
...     states = np.array(states)
...     mean_blocked = np.sum(states[:, 0] * pi)
...     return mean_blocked
>>>
>>> round(get_mean_number_of_individuals_in_node_2(pi, all_states), 10)
0.2685797749

\end{lstlisting}


Consequently, there are some additional performance measures of interest that
are more complex to calculate.
Such performance measures are the mean waiting time in the system (for both
type 1 and type 2 individuals), the mean time blocked in node 2 (only
valid for type 2 individuals) and the proportion of individuals that wait in
node 1 within a predefined time target (for both types).
Under the scope of this study three approaches have been considered to calculate
these performance measures; a recursive algorithm, a direct approach and
a closed-form equation. 
Furthermore, different formulas arise for type 1 individuals and different
ones for type 2 individuals.


\section{Performance measures}\label{sec:queueing_performance_measures}
Using vector \(\pi\) there are numerous performance measures of the model that
can be calculated.
The following equations utilise \(\pi\) to get performance measures on the
average number of individuals in node 1 and in node 2:

\begin{itemize}
    \item Mean number of individuals in the entire system:
        \begin{equation}
            L_S = \sum_{i=1}^{|\pi|} \pi_i (u_i + v_i)
        \end{equation}
    \item Mean number of individuals in node 1:
        \begin{equation}
            L_1 = \sum_{i=1}^{|\pi|} \pi_i v_i
        \end{equation}
    \item Mean number of individuals in node 2:
        \begin{equation}
            L_2 = \sum_{i=1}^{|\pi|} \pi_i u_i
        \end{equation}
\end{itemize}

The python code for these functions may be obtained quite quickly using the set
of all states and the steady state probability.

\begin{lstlisting}[
    style=pystyle,
    caption={Code snippet for getting the set of all states and the steady
    state probabilities.},
    label={lst:set_of_all_states_and_steady_state_probabilities},
]
>>> import ambulance_game as abg
>>> import numpy as np
>>> all_states = abg.markov.build_states(
...     threshold=3 ,
...     system_capacity=4,
...     buffer_capacity=2
... )
>>> Q = abg.markov.get_transition_matrix(
...     lambda_1=1,
...     lambda_2=2,
...     mu=2,
...     num_of_servers=2,
...     threshold=3,
...     system_capacity=4,
...     buffer_capacity=2
... )
>>> pi = abg.markov.get_steady_state_algebraically(
...     Q, algebraic_function=np.linalg.lstsq
... )

\end{lstlisting}

Having the set of all states and the steady state probabilities, the mean
number of individuals in the system, in node 1 and in node 2 can be calculated
as shown in code snippets~\ref{lst:mean_number_of_individuals_in_system},
\ref{lst:mean_number_of_individuals_in_node_1}
and~\ref{lst:mean_number_of_individuals_in_node_2}.

\begin{lstlisting}[
    style=pystyle,
    caption={Code snippet for calculating the mean number of individuals in
    the system.},
    label={lst:mean_number_of_individuals_in_system},
]
>>> def get_mean_number_of_individuals_in_system(pi, states):
...     """Gets the mean number of individuals in the system
...     Parameters
...     ----------
...     pi : numpy.ndarray
...         steady state vector
...     states : list
...         list of tuples that contains all states
...     Returns
...     -------
...     float
...         Mean number of individuals in the whole model
...     """
...     states = np.array(states)
...     mean_inds_in_system = np.sum((states[:, 0] + states[:, 1]) * pi)
...     return mean_inds_in_system
>>>
>>> round(get_mean_number_of_individuals_in_system(pi, all_states), 10)
2.0872927227

\end{lstlisting}

\begin{lstlisting}[
    style=pystyle,
    caption={Code snippet for calculating the mean number of individuals in
    Node 1.},
    label={lst:mean_number_of_individuals_in_node_1},
]
>>> def get_mean_number_of_individuals_in_node_1(pi, states):
...     """Mean number of individuals in node 1
...     Parameters
...     ----------
...     pi : numpy.ndarray
...         steady state vector
...     states : list
...         list of tuples that contains all states
...     Returns
...     -------
...     float
...         Mean number of individuals
...     """
...     states = np.array(states)
...     mean_inds_in_node_1 = np.sum(states[:, 1] * pi)
...     return mean_inds_in_node_1
>>>
>>> round(get_mean_number_of_individuals_in_node_1(pi, all_states), 10)
1.8187129478

\end{lstlisting}

\begin{lstlisting}[
    style=pystyle,
    caption={Code snippet for calculating the mean number of individuals in
    Node 2.},
    label={lst:mean_number_of_individuals_in_node_2},
]
>>> def get_mean_number_of_individuals_in_node_2(pi, states):
...     """Mean number of class 2 individuals blocked
...     Parameters
...     ----------
...     pi : numpy.ndarray
...         steady state vector
...     states : list
...         list of tuples that contains all states
...     Returns
...     -------
...     float
...         Mean number of blocked class 2 individuals
...     """
...     states = np.array(states)
...     mean_blocked = np.sum(states[:, 0] * pi)
...     return mean_blocked
>>>
>>> round(get_mean_number_of_individuals_in_node_2(pi, all_states), 10)
0.2685797749

\end{lstlisting}


Consequently, there are some additional performance measures of interest that
are more complex to calculate.
Such performance measures are the mean waiting time in the system (for both
type 1 and type 2 individuals), the mean time blocked in node 2 (only
valid for type 2 individuals) and the proportion of individuals that wait in
node 1 within a predefined time target (for both types).
Under the scope of this study three approaches have been considered to calculate
these performance measures; a recursive algorithm, a direct approach and
a closed-form equation. 
Furthermore, different formulas arise for type 1 individuals and different
ones for type 2 individuals.


\section{Performance measures}\label{sec:queueing_performance_measures}
Using vector \(\pi\) there are numerous performance measures of the model that
can be calculated.
The following equations utilise \(\pi\) to get performance measures on the
average number of individuals in node 1 and in node 2:

\begin{itemize}
    \item Mean number of individuals in the entire system:
        \begin{equation}
            L_S = \sum_{i=1}^{|\pi|} \pi_i (u_i + v_i)
        \end{equation}
    \item Mean number of individuals in node 1:
        \begin{equation}
            L_1 = \sum_{i=1}^{|\pi|} \pi_i v_i
        \end{equation}
    \item Mean number of individuals in node 2:
        \begin{equation}
            L_2 = \sum_{i=1}^{|\pi|} \pi_i u_i
        \end{equation}
\end{itemize}

The python code for these functions may be obtained quite quickly using the set
of all states and the steady state probability.

\begin{lstlisting}[
    style=pystyle,
    caption={Code snippet for getting the set of all states and the steady
    state probabilities.},
    label={lst:set_of_all_states_and_steady_state_probabilities},
]
>>> import ambulance_game as abg
>>> import numpy as np
>>> all_states = abg.markov.build_states(
...     threshold=3 ,
...     system_capacity=4,
...     buffer_capacity=2
... )
>>> Q = abg.markov.get_transition_matrix(
...     lambda_1=1,
...     lambda_2=2,
...     mu=2,
...     num_of_servers=2,
...     threshold=3,
...     system_capacity=4,
...     buffer_capacity=2
... )
>>> pi = abg.markov.get_steady_state_algebraically(
...     Q, algebraic_function=np.linalg.lstsq
... )

\end{lstlisting}

Having the set of all states and the steady state probabilities, the mean
number of individuals in the system, in node 1 and in node 2 can be calculated
as shown in code snippets~\ref{lst:mean_number_of_individuals_in_system},
\ref{lst:mean_number_of_individuals_in_node_1}
and~\ref{lst:mean_number_of_individuals_in_node_2}.

\begin{lstlisting}[
    style=pystyle,
    caption={Code snippet for calculating the mean number of individuals in
    the system.},
    label={lst:mean_number_of_individuals_in_system},
]
>>> def get_mean_number_of_individuals_in_system(pi, states):
...     """Gets the mean number of individuals in the system
...     Parameters
...     ----------
...     pi : numpy.ndarray
...         steady state vector
...     states : list
...         list of tuples that contains all states
...     Returns
...     -------
...     float
...         Mean number of individuals in the whole model
...     """
...     states = np.array(states)
...     mean_inds_in_system = np.sum((states[:, 0] + states[:, 1]) * pi)
...     return mean_inds_in_system
>>>
>>> round(get_mean_number_of_individuals_in_system(pi, all_states), 10)
2.0872927227

\end{lstlisting}

\begin{lstlisting}[
    style=pystyle,
    caption={Code snippet for calculating the mean number of individuals in
    Node 1.},
    label={lst:mean_number_of_individuals_in_node_1},
]
>>> def get_mean_number_of_individuals_in_node_1(pi, states):
...     """Mean number of individuals in node 1
...     Parameters
...     ----------
...     pi : numpy.ndarray
...         steady state vector
...     states : list
...         list of tuples that contains all states
...     Returns
...     -------
...     float
...         Mean number of individuals
...     """
...     states = np.array(states)
...     mean_inds_in_node_1 = np.sum(states[:, 1] * pi)
...     return mean_inds_in_node_1
>>>
>>> round(get_mean_number_of_individuals_in_node_1(pi, all_states), 10)
1.8187129478

\end{lstlisting}

\begin{lstlisting}[
    style=pystyle,
    caption={Code snippet for calculating the mean number of individuals in
    Node 2.},
    label={lst:mean_number_of_individuals_in_node_2},
]
>>> def get_mean_number_of_individuals_in_node_2(pi, states):
...     """Mean number of class 2 individuals blocked
...     Parameters
...     ----------
...     pi : numpy.ndarray
...         steady state vector
...     states : list
...         list of tuples that contains all states
...     Returns
...     -------
...     float
...         Mean number of blocked class 2 individuals
...     """
...     states = np.array(states)
...     mean_blocked = np.sum(states[:, 0] * pi)
...     return mean_blocked
>>>
>>> round(get_mean_number_of_individuals_in_node_2(pi, all_states), 10)
0.2685797749

\end{lstlisting}


Consequently, there are some additional performance measures of interest that
are more complex to calculate.
Such performance measures are the mean waiting time in the system (for both
type 1 and type 2 individuals), the mean time blocked in node 2 (only
valid for type 2 individuals) and the proportion of individuals that wait in
node 1 within a predefined time target (for both types).
Under the scope of this study three approaches have been considered to calculate
these performance measures; a recursive algorithm, a direct approach and
a closed-form equation. 
Furthermore, different formulas arise for type 1 individuals and different
ones for type 2 individuals.


\section{Performance measures}\label{sec:queueing_performance_measures}
Using vector \(\pi\) there are numerous performance measures of the model that
can be calculated.
The following equations utilise \(\pi\) to get performance measures on the
average number of individuals in node 1 and in node 2:

\begin{itemize}
    \item Mean number of individuals in the entire system:
        \begin{equation}
            L_S = \sum_{i=1}^{|\pi|} \pi_i (u_i + v_i)
        \end{equation}
    \item Mean number of individuals in node 1:
        \begin{equation}
            L_1 = \sum_{i=1}^{|\pi|} \pi_i v_i
        \end{equation}
    \item Mean number of individuals in node 2:
        \begin{equation}
            L_2 = \sum_{i=1}^{|\pi|} \pi_i u_i
        \end{equation}
\end{itemize}

The python code for these functions may be obtained quite quickly using the set
of all states and the steady state probability.

\begin{lstlisting}[
    style=pystyle,
    caption={Code snippet for getting the set of all states and the steady
    state probabilities.},
    label={lst:set_of_all_states_and_steady_state_probabilities},
]
>>> import ambulance_game as abg
>>> import numpy as np
>>> all_states = abg.markov.build_states(
...     threshold=3 ,
...     system_capacity=4,
...     buffer_capacity=2
... )
>>> Q = abg.markov.get_transition_matrix(
...     lambda_1=1,
...     lambda_2=2,
...     mu=2,
...     num_of_servers=2,
...     threshold=3,
...     system_capacity=4,
...     buffer_capacity=2
... )
>>> pi = abg.markov.get_steady_state_algebraically(
...     Q, algebraic_function=np.linalg.lstsq
... )

\end{lstlisting}

Having the set of all states and the steady state probabilities, the mean
number of individuals in the system, in node 1 and in node 2 can be calculated
as shown in code snippets~\ref{lst:mean_number_of_individuals_in_system},
\ref{lst:mean_number_of_individuals_in_node_1}
and~\ref{lst:mean_number_of_individuals_in_node_2}.

\begin{lstlisting}[
    style=pystyle,
    caption={Code snippet for calculating the mean number of individuals in
    the system.},
    label={lst:mean_number_of_individuals_in_system},
]
>>> def get_mean_number_of_individuals_in_system(pi, states):
...     """Gets the mean number of individuals in the system
...     Parameters
...     ----------
...     pi : numpy.ndarray
...         steady state vector
...     states : list
...         list of tuples that contains all states
...     Returns
...     -------
...     float
...         Mean number of individuals in the whole model
...     """
...     states = np.array(states)
...     mean_inds_in_system = np.sum((states[:, 0] + states[:, 1]) * pi)
...     return mean_inds_in_system
>>>
>>> round(get_mean_number_of_individuals_in_system(pi, all_states), 10)
2.0872927227

\end{lstlisting}

\begin{lstlisting}[
    style=pystyle,
    caption={Code snippet for calculating the mean number of individuals in
    Node 1.},
    label={lst:mean_number_of_individuals_in_node_1},
]
>>> def get_mean_number_of_individuals_in_node_1(pi, states):
...     """Mean number of individuals in node 1
...     Parameters
...     ----------
...     pi : numpy.ndarray
...         steady state vector
...     states : list
...         list of tuples that contains all states
...     Returns
...     -------
...     float
...         Mean number of individuals
...     """
...     states = np.array(states)
...     mean_inds_in_node_1 = np.sum(states[:, 1] * pi)
...     return mean_inds_in_node_1
>>>
>>> round(get_mean_number_of_individuals_in_node_1(pi, all_states), 10)
1.8187129478

\end{lstlisting}

\begin{lstlisting}[
    style=pystyle,
    caption={Code snippet for calculating the mean number of individuals in
    Node 2.},
    label={lst:mean_number_of_individuals_in_node_2},
]
>>> def get_mean_number_of_individuals_in_node_2(pi, states):
...     """Mean number of class 2 individuals blocked
...     Parameters
...     ----------
...     pi : numpy.ndarray
...         steady state vector
...     states : list
...         list of tuples that contains all states
...     Returns
...     -------
...     float
...         Mean number of blocked class 2 individuals
...     """
...     states = np.array(states)
...     mean_blocked = np.sum(states[:, 0] * pi)
...     return mean_blocked
>>>
>>> round(get_mean_number_of_individuals_in_node_2(pi, all_states), 10)
0.2685797749

\end{lstlisting}


Consequently, there are some additional performance measures of interest that
are more complex to calculate.
Such performance measures are the mean waiting time in the system (for both
type 1 and type 2 individuals), the mean time blocked in node 2 (only
valid for type 2 individuals) and the proportion of individuals that wait in
node 1 within a predefined time target (for both types).
Under the scope of this study three approaches have been considered to calculate
these performance measures; a recursive algorithm, a direct approach and
a closed-form equation. 
Furthermore, different formulas arise for type 1 individuals and different
ones for type 2 individuals.


\input{chapters/03_queueing_model/03_performance_measures/01_waiting_time/main.tex}

\input{chapters/03_queueing_model/03_performance_measures/02_blocking_time/main.tex}

\input{chapters/03_queueing_model/03_performance_measures/03_proportion_of_individuals_within_time/main.tex}

\section{Performance measures}\label{sec:queueing_performance_measures}
Using vector \(\pi\) there are numerous performance measures of the model that
can be calculated.
The following equations utilise \(\pi\) to get performance measures on the
average number of individuals in node 1 and in node 2:

\begin{itemize}
    \item Mean number of individuals in the entire system:
        \begin{equation}
            L_S = \sum_{i=1}^{|\pi|} \pi_i (u_i + v_i)
        \end{equation}
    \item Mean number of individuals in node 1:
        \begin{equation}
            L_1 = \sum_{i=1}^{|\pi|} \pi_i v_i
        \end{equation}
    \item Mean number of individuals in node 2:
        \begin{equation}
            L_2 = \sum_{i=1}^{|\pi|} \pi_i u_i
        \end{equation}
\end{itemize}

The python code for these functions may be obtained quite quickly using the set
of all states and the steady state probability.

\begin{lstlisting}[
    style=pystyle,
    caption={Code snippet for getting the set of all states and the steady
    state probabilities.},
    label={lst:set_of_all_states_and_steady_state_probabilities},
]
>>> import ambulance_game as abg
>>> import numpy as np
>>> all_states = abg.markov.build_states(
...     threshold=3 ,
...     system_capacity=4,
...     buffer_capacity=2
... )
>>> Q = abg.markov.get_transition_matrix(
...     lambda_1=1,
...     lambda_2=2,
...     mu=2,
...     num_of_servers=2,
...     threshold=3,
...     system_capacity=4,
...     buffer_capacity=2
... )
>>> pi = abg.markov.get_steady_state_algebraically(
...     Q, algebraic_function=np.linalg.lstsq
... )

\end{lstlisting}

Having the set of all states and the steady state probabilities, the mean
number of individuals in the system, in node 1 and in node 2 can be calculated
as shown in code snippets~\ref{lst:mean_number_of_individuals_in_system},
\ref{lst:mean_number_of_individuals_in_node_1}
and~\ref{lst:mean_number_of_individuals_in_node_2}.

\begin{lstlisting}[
    style=pystyle,
    caption={Code snippet for calculating the mean number of individuals in
    the system.},
    label={lst:mean_number_of_individuals_in_system},
]
>>> def get_mean_number_of_individuals_in_system(pi, states):
...     """Gets the mean number of individuals in the system
...     Parameters
...     ----------
...     pi : numpy.ndarray
...         steady state vector
...     states : list
...         list of tuples that contains all states
...     Returns
...     -------
...     float
...         Mean number of individuals in the whole model
...     """
...     states = np.array(states)
...     mean_inds_in_system = np.sum((states[:, 0] + states[:, 1]) * pi)
...     return mean_inds_in_system
>>>
>>> round(get_mean_number_of_individuals_in_system(pi, all_states), 10)
2.0872927227

\end{lstlisting}

\begin{lstlisting}[
    style=pystyle,
    caption={Code snippet for calculating the mean number of individuals in
    Node 1.},
    label={lst:mean_number_of_individuals_in_node_1},
]
>>> def get_mean_number_of_individuals_in_node_1(pi, states):
...     """Mean number of individuals in node 1
...     Parameters
...     ----------
...     pi : numpy.ndarray
...         steady state vector
...     states : list
...         list of tuples that contains all states
...     Returns
...     -------
...     float
...         Mean number of individuals
...     """
...     states = np.array(states)
...     mean_inds_in_node_1 = np.sum(states[:, 1] * pi)
...     return mean_inds_in_node_1
>>>
>>> round(get_mean_number_of_individuals_in_node_1(pi, all_states), 10)
1.8187129478

\end{lstlisting}

\begin{lstlisting}[
    style=pystyle,
    caption={Code snippet for calculating the mean number of individuals in
    Node 2.},
    label={lst:mean_number_of_individuals_in_node_2},
]
>>> def get_mean_number_of_individuals_in_node_2(pi, states):
...     """Mean number of class 2 individuals blocked
...     Parameters
...     ----------
...     pi : numpy.ndarray
...         steady state vector
...     states : list
...         list of tuples that contains all states
...     Returns
...     -------
...     float
...         Mean number of blocked class 2 individuals
...     """
...     states = np.array(states)
...     mean_blocked = np.sum(states[:, 0] * pi)
...     return mean_blocked
>>>
>>> round(get_mean_number_of_individuals_in_node_2(pi, all_states), 10)
0.2685797749

\end{lstlisting}


Consequently, there are some additional performance measures of interest that
are more complex to calculate.
Such performance measures are the mean waiting time in the system (for both
type 1 and type 2 individuals), the mean time blocked in node 2 (only
valid for type 2 individuals) and the proportion of individuals that wait in
node 1 within a predefined time target (for both types).
Under the scope of this study three approaches have been considered to calculate
these performance measures; a recursive algorithm, a direct approach and
a closed-form equation. 
Furthermore, different formulas arise for type 1 individuals and different
ones for type 2 individuals.


\input{chapters/03_queueing_model/03_performance_measures/01_waiting_time/main.tex}

\input{chapters/03_queueing_model/03_performance_measures/02_blocking_time/main.tex}

\input{chapters/03_queueing_model/03_performance_measures/03_proportion_of_individuals_within_time/main.tex}

\section{Performance measures}\label{sec:queueing_performance_measures}
Using vector \(\pi\) there are numerous performance measures of the model that
can be calculated.
The following equations utilise \(\pi\) to get performance measures on the
average number of individuals in node 1 and in node 2:

\begin{itemize}
    \item Mean number of individuals in the entire system:
        \begin{equation}
            L_S = \sum_{i=1}^{|\pi|} \pi_i (u_i + v_i)
        \end{equation}
    \item Mean number of individuals in node 1:
        \begin{equation}
            L_1 = \sum_{i=1}^{|\pi|} \pi_i v_i
        \end{equation}
    \item Mean number of individuals in node 2:
        \begin{equation}
            L_2 = \sum_{i=1}^{|\pi|} \pi_i u_i
        \end{equation}
\end{itemize}

The python code for these functions may be obtained quite quickly using the set
of all states and the steady state probability.

\begin{lstlisting}[
    style=pystyle,
    caption={Code snippet for getting the set of all states and the steady
    state probabilities.},
    label={lst:set_of_all_states_and_steady_state_probabilities},
]
>>> import ambulance_game as abg
>>> import numpy as np
>>> all_states = abg.markov.build_states(
...     threshold=3 ,
...     system_capacity=4,
...     buffer_capacity=2
... )
>>> Q = abg.markov.get_transition_matrix(
...     lambda_1=1,
...     lambda_2=2,
...     mu=2,
...     num_of_servers=2,
...     threshold=3,
...     system_capacity=4,
...     buffer_capacity=2
... )
>>> pi = abg.markov.get_steady_state_algebraically(
...     Q, algebraic_function=np.linalg.lstsq
... )

\end{lstlisting}

Having the set of all states and the steady state probabilities, the mean
number of individuals in the system, in node 1 and in node 2 can be calculated
as shown in code snippets~\ref{lst:mean_number_of_individuals_in_system},
\ref{lst:mean_number_of_individuals_in_node_1}
and~\ref{lst:mean_number_of_individuals_in_node_2}.

\begin{lstlisting}[
    style=pystyle,
    caption={Code snippet for calculating the mean number of individuals in
    the system.},
    label={lst:mean_number_of_individuals_in_system},
]
>>> def get_mean_number_of_individuals_in_system(pi, states):
...     """Gets the mean number of individuals in the system
...     Parameters
...     ----------
...     pi : numpy.ndarray
...         steady state vector
...     states : list
...         list of tuples that contains all states
...     Returns
...     -------
...     float
...         Mean number of individuals in the whole model
...     """
...     states = np.array(states)
...     mean_inds_in_system = np.sum((states[:, 0] + states[:, 1]) * pi)
...     return mean_inds_in_system
>>>
>>> round(get_mean_number_of_individuals_in_system(pi, all_states), 10)
2.0872927227

\end{lstlisting}

\begin{lstlisting}[
    style=pystyle,
    caption={Code snippet for calculating the mean number of individuals in
    Node 1.},
    label={lst:mean_number_of_individuals_in_node_1},
]
>>> def get_mean_number_of_individuals_in_node_1(pi, states):
...     """Mean number of individuals in node 1
...     Parameters
...     ----------
...     pi : numpy.ndarray
...         steady state vector
...     states : list
...         list of tuples that contains all states
...     Returns
...     -------
...     float
...         Mean number of individuals
...     """
...     states = np.array(states)
...     mean_inds_in_node_1 = np.sum(states[:, 1] * pi)
...     return mean_inds_in_node_1
>>>
>>> round(get_mean_number_of_individuals_in_node_1(pi, all_states), 10)
1.8187129478

\end{lstlisting}

\begin{lstlisting}[
    style=pystyle,
    caption={Code snippet for calculating the mean number of individuals in
    Node 2.},
    label={lst:mean_number_of_individuals_in_node_2},
]
>>> def get_mean_number_of_individuals_in_node_2(pi, states):
...     """Mean number of class 2 individuals blocked
...     Parameters
...     ----------
...     pi : numpy.ndarray
...         steady state vector
...     states : list
...         list of tuples that contains all states
...     Returns
...     -------
...     float
...         Mean number of blocked class 2 individuals
...     """
...     states = np.array(states)
...     mean_blocked = np.sum(states[:, 0] * pi)
...     return mean_blocked
>>>
>>> round(get_mean_number_of_individuals_in_node_2(pi, all_states), 10)
0.2685797749

\end{lstlisting}


Consequently, there are some additional performance measures of interest that
are more complex to calculate.
Such performance measures are the mean waiting time in the system (for both
type 1 and type 2 individuals), the mean time blocked in node 2 (only
valid for type 2 individuals) and the proportion of individuals that wait in
node 1 within a predefined time target (for both types).
Under the scope of this study three approaches have been considered to calculate
these performance measures; a recursive algorithm, a direct approach and
a closed-form equation. 
Furthermore, different formulas arise for type 1 individuals and different
ones for type 2 individuals.


\input{chapters/03_queueing_model/03_performance_measures/01_waiting_time/main.tex}

\input{chapters/03_queueing_model/03_performance_measures/02_blocking_time/main.tex}

\input{chapters/03_queueing_model/03_performance_measures/03_proportion_of_individuals_within_time/main.tex}

\section{Performance measures}\label{sec:queueing_performance_measures}
Using vector \(\pi\) there are numerous performance measures of the model that
can be calculated.
The following equations utilise \(\pi\) to get performance measures on the
average number of individuals in node 1 and in node 2:

\begin{itemize}
    \item Mean number of individuals in the entire system:
        \begin{equation}
            L_S = \sum_{i=1}^{|\pi|} \pi_i (u_i + v_i)
        \end{equation}
    \item Mean number of individuals in node 1:
        \begin{equation}
            L_1 = \sum_{i=1}^{|\pi|} \pi_i v_i
        \end{equation}
    \item Mean number of individuals in node 2:
        \begin{equation}
            L_2 = \sum_{i=1}^{|\pi|} \pi_i u_i
        \end{equation}
\end{itemize}

The python code for these functions may be obtained quite quickly using the set
of all states and the steady state probability.

\begin{lstlisting}[
    style=pystyle,
    caption={Code snippet for getting the set of all states and the steady
    state probabilities.},
    label={lst:set_of_all_states_and_steady_state_probabilities},
]
>>> import ambulance_game as abg
>>> import numpy as np
>>> all_states = abg.markov.build_states(
...     threshold=3 ,
...     system_capacity=4,
...     buffer_capacity=2
... )
>>> Q = abg.markov.get_transition_matrix(
...     lambda_1=1,
...     lambda_2=2,
...     mu=2,
...     num_of_servers=2,
...     threshold=3,
...     system_capacity=4,
...     buffer_capacity=2
... )
>>> pi = abg.markov.get_steady_state_algebraically(
...     Q, algebraic_function=np.linalg.lstsq
... )

\end{lstlisting}

Having the set of all states and the steady state probabilities, the mean
number of individuals in the system, in node 1 and in node 2 can be calculated
as shown in code snippets~\ref{lst:mean_number_of_individuals_in_system},
\ref{lst:mean_number_of_individuals_in_node_1}
and~\ref{lst:mean_number_of_individuals_in_node_2}.

\begin{lstlisting}[
    style=pystyle,
    caption={Code snippet for calculating the mean number of individuals in
    the system.},
    label={lst:mean_number_of_individuals_in_system},
]
>>> def get_mean_number_of_individuals_in_system(pi, states):
...     """Gets the mean number of individuals in the system
...     Parameters
...     ----------
...     pi : numpy.ndarray
...         steady state vector
...     states : list
...         list of tuples that contains all states
...     Returns
...     -------
...     float
...         Mean number of individuals in the whole model
...     """
...     states = np.array(states)
...     mean_inds_in_system = np.sum((states[:, 0] + states[:, 1]) * pi)
...     return mean_inds_in_system
>>>
>>> round(get_mean_number_of_individuals_in_system(pi, all_states), 10)
2.0872927227

\end{lstlisting}

\begin{lstlisting}[
    style=pystyle,
    caption={Code snippet for calculating the mean number of individuals in
    Node 1.},
    label={lst:mean_number_of_individuals_in_node_1},
]
>>> def get_mean_number_of_individuals_in_node_1(pi, states):
...     """Mean number of individuals in node 1
...     Parameters
...     ----------
...     pi : numpy.ndarray
...         steady state vector
...     states : list
...         list of tuples that contains all states
...     Returns
...     -------
...     float
...         Mean number of individuals
...     """
...     states = np.array(states)
...     mean_inds_in_node_1 = np.sum(states[:, 1] * pi)
...     return mean_inds_in_node_1
>>>
>>> round(get_mean_number_of_individuals_in_node_1(pi, all_states), 10)
1.8187129478

\end{lstlisting}

\begin{lstlisting}[
    style=pystyle,
    caption={Code snippet for calculating the mean number of individuals in
    Node 2.},
    label={lst:mean_number_of_individuals_in_node_2},
]
>>> def get_mean_number_of_individuals_in_node_2(pi, states):
...     """Mean number of class 2 individuals blocked
...     Parameters
...     ----------
...     pi : numpy.ndarray
...         steady state vector
...     states : list
...         list of tuples that contains all states
...     Returns
...     -------
...     float
...         Mean number of blocked class 2 individuals
...     """
...     states = np.array(states)
...     mean_blocked = np.sum(states[:, 0] * pi)
...     return mean_blocked
>>>
>>> round(get_mean_number_of_individuals_in_node_2(pi, all_states), 10)
0.2685797749

\end{lstlisting}


Consequently, there are some additional performance measures of interest that
are more complex to calculate.
Such performance measures are the mean waiting time in the system (for both
type 1 and type 2 individuals), the mean time blocked in node 2 (only
valid for type 2 individuals) and the proportion of individuals that wait in
node 1 within a predefined time target (for both types).
Under the scope of this study three approaches have been considered to calculate
these performance measures; a recursive algorithm, a direct approach and
a closed-form equation. 
Furthermore, different formulas arise for type 1 individuals and different
ones for type 2 individuals.


\section{Performance measures}\label{sec:queueing_performance_measures}
Using vector \(\pi\) there are numerous performance measures of the model that
can be calculated.
The following equations utilise \(\pi\) to get performance measures on the
average number of individuals in node 1 and in node 2:

\begin{itemize}
    \item Mean number of individuals in the entire system:
        \begin{equation}
            L_S = \sum_{i=1}^{|\pi|} \pi_i (u_i + v_i)
        \end{equation}
    \item Mean number of individuals in node 1:
        \begin{equation}
            L_1 = \sum_{i=1}^{|\pi|} \pi_i v_i
        \end{equation}
    \item Mean number of individuals in node 2:
        \begin{equation}
            L_2 = \sum_{i=1}^{|\pi|} \pi_i u_i
        \end{equation}
\end{itemize}

The python code for these functions may be obtained quite quickly using the set
of all states and the steady state probability.

\begin{lstlisting}[
    style=pystyle,
    caption={Code snippet for getting the set of all states and the steady
    state probabilities.},
    label={lst:set_of_all_states_and_steady_state_probabilities},
]
>>> import ambulance_game as abg
>>> import numpy as np
>>> all_states = abg.markov.build_states(
...     threshold=3 ,
...     system_capacity=4,
...     buffer_capacity=2
... )
>>> Q = abg.markov.get_transition_matrix(
...     lambda_1=1,
...     lambda_2=2,
...     mu=2,
...     num_of_servers=2,
...     threshold=3,
...     system_capacity=4,
...     buffer_capacity=2
... )
>>> pi = abg.markov.get_steady_state_algebraically(
...     Q, algebraic_function=np.linalg.lstsq
... )

\end{lstlisting}

Having the set of all states and the steady state probabilities, the mean
number of individuals in the system, in node 1 and in node 2 can be calculated
as shown in code snippets~\ref{lst:mean_number_of_individuals_in_system},
\ref{lst:mean_number_of_individuals_in_node_1}
and~\ref{lst:mean_number_of_individuals_in_node_2}.

\begin{lstlisting}[
    style=pystyle,
    caption={Code snippet for calculating the mean number of individuals in
    the system.},
    label={lst:mean_number_of_individuals_in_system},
]
>>> def get_mean_number_of_individuals_in_system(pi, states):
...     """Gets the mean number of individuals in the system
...     Parameters
...     ----------
...     pi : numpy.ndarray
...         steady state vector
...     states : list
...         list of tuples that contains all states
...     Returns
...     -------
...     float
...         Mean number of individuals in the whole model
...     """
...     states = np.array(states)
...     mean_inds_in_system = np.sum((states[:, 0] + states[:, 1]) * pi)
...     return mean_inds_in_system
>>>
>>> round(get_mean_number_of_individuals_in_system(pi, all_states), 10)
2.0872927227

\end{lstlisting}

\begin{lstlisting}[
    style=pystyle,
    caption={Code snippet for calculating the mean number of individuals in
    Node 1.},
    label={lst:mean_number_of_individuals_in_node_1},
]
>>> def get_mean_number_of_individuals_in_node_1(pi, states):
...     """Mean number of individuals in node 1
...     Parameters
...     ----------
...     pi : numpy.ndarray
...         steady state vector
...     states : list
...         list of tuples that contains all states
...     Returns
...     -------
...     float
...         Mean number of individuals
...     """
...     states = np.array(states)
...     mean_inds_in_node_1 = np.sum(states[:, 1] * pi)
...     return mean_inds_in_node_1
>>>
>>> round(get_mean_number_of_individuals_in_node_1(pi, all_states), 10)
1.8187129478

\end{lstlisting}

\begin{lstlisting}[
    style=pystyle,
    caption={Code snippet for calculating the mean number of individuals in
    Node 2.},
    label={lst:mean_number_of_individuals_in_node_2},
]
>>> def get_mean_number_of_individuals_in_node_2(pi, states):
...     """Mean number of class 2 individuals blocked
...     Parameters
...     ----------
...     pi : numpy.ndarray
...         steady state vector
...     states : list
...         list of tuples that contains all states
...     Returns
...     -------
...     float
...         Mean number of blocked class 2 individuals
...     """
...     states = np.array(states)
...     mean_blocked = np.sum(states[:, 0] * pi)
...     return mean_blocked
>>>
>>> round(get_mean_number_of_individuals_in_node_2(pi, all_states), 10)
0.2685797749

\end{lstlisting}


Consequently, there are some additional performance measures of interest that
are more complex to calculate.
Such performance measures are the mean waiting time in the system (for both
type 1 and type 2 individuals), the mean time blocked in node 2 (only
valid for type 2 individuals) and the proportion of individuals that wait in
node 1 within a predefined time target (for both types).
Under the scope of this study three approaches have been considered to calculate
these performance measures; a recursive algorithm, a direct approach and
a closed-form equation. 
Furthermore, different formulas arise for type 1 individuals and different
ones for type 2 individuals.


\input{chapters/03_queueing_model/03_performance_measures/01_waiting_time/main.tex}

\input{chapters/03_queueing_model/03_performance_measures/02_blocking_time/main.tex}

\input{chapters/03_queueing_model/03_performance_measures/03_proportion_of_individuals_within_time/main.tex}

\section{Performance measures}\label{sec:queueing_performance_measures}
Using vector \(\pi\) there are numerous performance measures of the model that
can be calculated.
The following equations utilise \(\pi\) to get performance measures on the
average number of individuals in node 1 and in node 2:

\begin{itemize}
    \item Mean number of individuals in the entire system:
        \begin{equation}
            L_S = \sum_{i=1}^{|\pi|} \pi_i (u_i + v_i)
        \end{equation}
    \item Mean number of individuals in node 1:
        \begin{equation}
            L_1 = \sum_{i=1}^{|\pi|} \pi_i v_i
        \end{equation}
    \item Mean number of individuals in node 2:
        \begin{equation}
            L_2 = \sum_{i=1}^{|\pi|} \pi_i u_i
        \end{equation}
\end{itemize}

The python code for these functions may be obtained quite quickly using the set
of all states and the steady state probability.

\begin{lstlisting}[
    style=pystyle,
    caption={Code snippet for getting the set of all states and the steady
    state probabilities.},
    label={lst:set_of_all_states_and_steady_state_probabilities},
]
>>> import ambulance_game as abg
>>> import numpy as np
>>> all_states = abg.markov.build_states(
...     threshold=3 ,
...     system_capacity=4,
...     buffer_capacity=2
... )
>>> Q = abg.markov.get_transition_matrix(
...     lambda_1=1,
...     lambda_2=2,
...     mu=2,
...     num_of_servers=2,
...     threshold=3,
...     system_capacity=4,
...     buffer_capacity=2
... )
>>> pi = abg.markov.get_steady_state_algebraically(
...     Q, algebraic_function=np.linalg.lstsq
... )

\end{lstlisting}

Having the set of all states and the steady state probabilities, the mean
number of individuals in the system, in node 1 and in node 2 can be calculated
as shown in code snippets~\ref{lst:mean_number_of_individuals_in_system},
\ref{lst:mean_number_of_individuals_in_node_1}
and~\ref{lst:mean_number_of_individuals_in_node_2}.

\begin{lstlisting}[
    style=pystyle,
    caption={Code snippet for calculating the mean number of individuals in
    the system.},
    label={lst:mean_number_of_individuals_in_system},
]
>>> def get_mean_number_of_individuals_in_system(pi, states):
...     """Gets the mean number of individuals in the system
...     Parameters
...     ----------
...     pi : numpy.ndarray
...         steady state vector
...     states : list
...         list of tuples that contains all states
...     Returns
...     -------
...     float
...         Mean number of individuals in the whole model
...     """
...     states = np.array(states)
...     mean_inds_in_system = np.sum((states[:, 0] + states[:, 1]) * pi)
...     return mean_inds_in_system
>>>
>>> round(get_mean_number_of_individuals_in_system(pi, all_states), 10)
2.0872927227

\end{lstlisting}

\begin{lstlisting}[
    style=pystyle,
    caption={Code snippet for calculating the mean number of individuals in
    Node 1.},
    label={lst:mean_number_of_individuals_in_node_1},
]
>>> def get_mean_number_of_individuals_in_node_1(pi, states):
...     """Mean number of individuals in node 1
...     Parameters
...     ----------
...     pi : numpy.ndarray
...         steady state vector
...     states : list
...         list of tuples that contains all states
...     Returns
...     -------
...     float
...         Mean number of individuals
...     """
...     states = np.array(states)
...     mean_inds_in_node_1 = np.sum(states[:, 1] * pi)
...     return mean_inds_in_node_1
>>>
>>> round(get_mean_number_of_individuals_in_node_1(pi, all_states), 10)
1.8187129478

\end{lstlisting}

\begin{lstlisting}[
    style=pystyle,
    caption={Code snippet for calculating the mean number of individuals in
    Node 2.},
    label={lst:mean_number_of_individuals_in_node_2},
]
>>> def get_mean_number_of_individuals_in_node_2(pi, states):
...     """Mean number of class 2 individuals blocked
...     Parameters
...     ----------
...     pi : numpy.ndarray
...         steady state vector
...     states : list
...         list of tuples that contains all states
...     Returns
...     -------
...     float
...         Mean number of blocked class 2 individuals
...     """
...     states = np.array(states)
...     mean_blocked = np.sum(states[:, 0] * pi)
...     return mean_blocked
>>>
>>> round(get_mean_number_of_individuals_in_node_2(pi, all_states), 10)
0.2685797749

\end{lstlisting}


Consequently, there are some additional performance measures of interest that
are more complex to calculate.
Such performance measures are the mean waiting time in the system (for both
type 1 and type 2 individuals), the mean time blocked in node 2 (only
valid for type 2 individuals) and the proportion of individuals that wait in
node 1 within a predefined time target (for both types).
Under the scope of this study three approaches have been considered to calculate
these performance measures; a recursive algorithm, a direct approach and
a closed-form equation. 
Furthermore, different formulas arise for type 1 individuals and different
ones for type 2 individuals.


\input{chapters/03_queueing_model/03_performance_measures/01_waiting_time/main.tex}

\input{chapters/03_queueing_model/03_performance_measures/02_blocking_time/main.tex}

\input{chapters/03_queueing_model/03_performance_measures/03_proportion_of_individuals_within_time/main.tex}

\section{Performance measures}\label{sec:queueing_performance_measures}
Using vector \(\pi\) there are numerous performance measures of the model that
can be calculated.
The following equations utilise \(\pi\) to get performance measures on the
average number of individuals in node 1 and in node 2:

\begin{itemize}
    \item Mean number of individuals in the entire system:
        \begin{equation}
            L_S = \sum_{i=1}^{|\pi|} \pi_i (u_i + v_i)
        \end{equation}
    \item Mean number of individuals in node 1:
        \begin{equation}
            L_1 = \sum_{i=1}^{|\pi|} \pi_i v_i
        \end{equation}
    \item Mean number of individuals in node 2:
        \begin{equation}
            L_2 = \sum_{i=1}^{|\pi|} \pi_i u_i
        \end{equation}
\end{itemize}

The python code for these functions may be obtained quite quickly using the set
of all states and the steady state probability.

\begin{lstlisting}[
    style=pystyle,
    caption={Code snippet for getting the set of all states and the steady
    state probabilities.},
    label={lst:set_of_all_states_and_steady_state_probabilities},
]
>>> import ambulance_game as abg
>>> import numpy as np
>>> all_states = abg.markov.build_states(
...     threshold=3 ,
...     system_capacity=4,
...     buffer_capacity=2
... )
>>> Q = abg.markov.get_transition_matrix(
...     lambda_1=1,
...     lambda_2=2,
...     mu=2,
...     num_of_servers=2,
...     threshold=3,
...     system_capacity=4,
...     buffer_capacity=2
... )
>>> pi = abg.markov.get_steady_state_algebraically(
...     Q, algebraic_function=np.linalg.lstsq
... )

\end{lstlisting}

Having the set of all states and the steady state probabilities, the mean
number of individuals in the system, in node 1 and in node 2 can be calculated
as shown in code snippets~\ref{lst:mean_number_of_individuals_in_system},
\ref{lst:mean_number_of_individuals_in_node_1}
and~\ref{lst:mean_number_of_individuals_in_node_2}.

\begin{lstlisting}[
    style=pystyle,
    caption={Code snippet for calculating the mean number of individuals in
    the system.},
    label={lst:mean_number_of_individuals_in_system},
]
>>> def get_mean_number_of_individuals_in_system(pi, states):
...     """Gets the mean number of individuals in the system
...     Parameters
...     ----------
...     pi : numpy.ndarray
...         steady state vector
...     states : list
...         list of tuples that contains all states
...     Returns
...     -------
...     float
...         Mean number of individuals in the whole model
...     """
...     states = np.array(states)
...     mean_inds_in_system = np.sum((states[:, 0] + states[:, 1]) * pi)
...     return mean_inds_in_system
>>>
>>> round(get_mean_number_of_individuals_in_system(pi, all_states), 10)
2.0872927227

\end{lstlisting}

\begin{lstlisting}[
    style=pystyle,
    caption={Code snippet for calculating the mean number of individuals in
    Node 1.},
    label={lst:mean_number_of_individuals_in_node_1},
]
>>> def get_mean_number_of_individuals_in_node_1(pi, states):
...     """Mean number of individuals in node 1
...     Parameters
...     ----------
...     pi : numpy.ndarray
...         steady state vector
...     states : list
...         list of tuples that contains all states
...     Returns
...     -------
...     float
...         Mean number of individuals
...     """
...     states = np.array(states)
...     mean_inds_in_node_1 = np.sum(states[:, 1] * pi)
...     return mean_inds_in_node_1
>>>
>>> round(get_mean_number_of_individuals_in_node_1(pi, all_states), 10)
1.8187129478

\end{lstlisting}

\begin{lstlisting}[
    style=pystyle,
    caption={Code snippet for calculating the mean number of individuals in
    Node 2.},
    label={lst:mean_number_of_individuals_in_node_2},
]
>>> def get_mean_number_of_individuals_in_node_2(pi, states):
...     """Mean number of class 2 individuals blocked
...     Parameters
...     ----------
...     pi : numpy.ndarray
...         steady state vector
...     states : list
...         list of tuples that contains all states
...     Returns
...     -------
...     float
...         Mean number of blocked class 2 individuals
...     """
...     states = np.array(states)
...     mean_blocked = np.sum(states[:, 0] * pi)
...     return mean_blocked
>>>
>>> round(get_mean_number_of_individuals_in_node_2(pi, all_states), 10)
0.2685797749

\end{lstlisting}


Consequently, there are some additional performance measures of interest that
are more complex to calculate.
Such performance measures are the mean waiting time in the system (for both
type 1 and type 2 individuals), the mean time blocked in node 2 (only
valid for type 2 individuals) and the proportion of individuals that wait in
node 1 within a predefined time target (for both types).
Under the scope of this study three approaches have been considered to calculate
these performance measures; a recursive algorithm, a direct approach and
a closed-form equation. 
Furthermore, different formulas arise for type 1 individuals and different
ones for type 2 individuals.


\input{chapters/03_queueing_model/03_performance_measures/01_waiting_time/main.tex}

\input{chapters/03_queueing_model/03_performance_measures/02_blocking_time/main.tex}

\input{chapters/03_queueing_model/03_performance_measures/03_proportion_of_individuals_within_time/main.tex}

\section{Performance measures}\label{sec:queueing_performance_measures}
Using vector \(\pi\) there are numerous performance measures of the model that
can be calculated.
The following equations utilise \(\pi\) to get performance measures on the
average number of individuals in node 1 and in node 2:

\begin{itemize}
    \item Mean number of individuals in the entire system:
        \begin{equation}
            L_S = \sum_{i=1}^{|\pi|} \pi_i (u_i + v_i)
        \end{equation}
    \item Mean number of individuals in node 1:
        \begin{equation}
            L_1 = \sum_{i=1}^{|\pi|} \pi_i v_i
        \end{equation}
    \item Mean number of individuals in node 2:
        \begin{equation}
            L_2 = \sum_{i=1}^{|\pi|} \pi_i u_i
        \end{equation}
\end{itemize}

The python code for these functions may be obtained quite quickly using the set
of all states and the steady state probability.

\begin{lstlisting}[
    style=pystyle,
    caption={Code snippet for getting the set of all states and the steady
    state probabilities.},
    label={lst:set_of_all_states_and_steady_state_probabilities},
]
>>> import ambulance_game as abg
>>> import numpy as np
>>> all_states = abg.markov.build_states(
...     threshold=3 ,
...     system_capacity=4,
...     buffer_capacity=2
... )
>>> Q = abg.markov.get_transition_matrix(
...     lambda_1=1,
...     lambda_2=2,
...     mu=2,
...     num_of_servers=2,
...     threshold=3,
...     system_capacity=4,
...     buffer_capacity=2
... )
>>> pi = abg.markov.get_steady_state_algebraically(
...     Q, algebraic_function=np.linalg.lstsq
... )

\end{lstlisting}

Having the set of all states and the steady state probabilities, the mean
number of individuals in the system, in node 1 and in node 2 can be calculated
as shown in code snippets~\ref{lst:mean_number_of_individuals_in_system},
\ref{lst:mean_number_of_individuals_in_node_1}
and~\ref{lst:mean_number_of_individuals_in_node_2}.

\begin{lstlisting}[
    style=pystyle,
    caption={Code snippet for calculating the mean number of individuals in
    the system.},
    label={lst:mean_number_of_individuals_in_system},
]
>>> def get_mean_number_of_individuals_in_system(pi, states):
...     """Gets the mean number of individuals in the system
...     Parameters
...     ----------
...     pi : numpy.ndarray
...         steady state vector
...     states : list
...         list of tuples that contains all states
...     Returns
...     -------
...     float
...         Mean number of individuals in the whole model
...     """
...     states = np.array(states)
...     mean_inds_in_system = np.sum((states[:, 0] + states[:, 1]) * pi)
...     return mean_inds_in_system
>>>
>>> round(get_mean_number_of_individuals_in_system(pi, all_states), 10)
2.0872927227

\end{lstlisting}

\begin{lstlisting}[
    style=pystyle,
    caption={Code snippet for calculating the mean number of individuals in
    Node 1.},
    label={lst:mean_number_of_individuals_in_node_1},
]
>>> def get_mean_number_of_individuals_in_node_1(pi, states):
...     """Mean number of individuals in node 1
...     Parameters
...     ----------
...     pi : numpy.ndarray
...         steady state vector
...     states : list
...         list of tuples that contains all states
...     Returns
...     -------
...     float
...         Mean number of individuals
...     """
...     states = np.array(states)
...     mean_inds_in_node_1 = np.sum(states[:, 1] * pi)
...     return mean_inds_in_node_1
>>>
>>> round(get_mean_number_of_individuals_in_node_1(pi, all_states), 10)
1.8187129478

\end{lstlisting}

\begin{lstlisting}[
    style=pystyle,
    caption={Code snippet for calculating the mean number of individuals in
    Node 2.},
    label={lst:mean_number_of_individuals_in_node_2},
]
>>> def get_mean_number_of_individuals_in_node_2(pi, states):
...     """Mean number of class 2 individuals blocked
...     Parameters
...     ----------
...     pi : numpy.ndarray
...         steady state vector
...     states : list
...         list of tuples that contains all states
...     Returns
...     -------
...     float
...         Mean number of blocked class 2 individuals
...     """
...     states = np.array(states)
...     mean_blocked = np.sum(states[:, 0] * pi)
...     return mean_blocked
>>>
>>> round(get_mean_number_of_individuals_in_node_2(pi, all_states), 10)
0.2685797749

\end{lstlisting}


Consequently, there are some additional performance measures of interest that
are more complex to calculate.
Such performance measures are the mean waiting time in the system (for both
type 1 and type 2 individuals), the mean time blocked in node 2 (only
valid for type 2 individuals) and the proportion of individuals that wait in
node 1 within a predefined time target (for both types).
Under the scope of this study three approaches have been considered to calculate
these performance measures; a recursive algorithm, a direct approach and
a closed-form equation. 
Furthermore, different formulas arise for type 1 individuals and different
ones for type 2 individuals.


\section{Performance measures}\label{sec:queueing_performance_measures}
Using vector \(\pi\) there are numerous performance measures of the model that
can be calculated.
The following equations utilise \(\pi\) to get performance measures on the
average number of individuals in node 1 and in node 2:

\begin{itemize}
    \item Mean number of individuals in the entire system:
        \begin{equation}
            L_S = \sum_{i=1}^{|\pi|} \pi_i (u_i + v_i)
        \end{equation}
    \item Mean number of individuals in node 1:
        \begin{equation}
            L_1 = \sum_{i=1}^{|\pi|} \pi_i v_i
        \end{equation}
    \item Mean number of individuals in node 2:
        \begin{equation}
            L_2 = \sum_{i=1}^{|\pi|} \pi_i u_i
        \end{equation}
\end{itemize}

The python code for these functions may be obtained quite quickly using the set
of all states and the steady state probability.

\begin{lstlisting}[
    style=pystyle,
    caption={Code snippet for getting the set of all states and the steady
    state probabilities.},
    label={lst:set_of_all_states_and_steady_state_probabilities},
]
>>> import ambulance_game as abg
>>> import numpy as np
>>> all_states = abg.markov.build_states(
...     threshold=3 ,
...     system_capacity=4,
...     buffer_capacity=2
... )
>>> Q = abg.markov.get_transition_matrix(
...     lambda_1=1,
...     lambda_2=2,
...     mu=2,
...     num_of_servers=2,
...     threshold=3,
...     system_capacity=4,
...     buffer_capacity=2
... )
>>> pi = abg.markov.get_steady_state_algebraically(
...     Q, algebraic_function=np.linalg.lstsq
... )

\end{lstlisting}

Having the set of all states and the steady state probabilities, the mean
number of individuals in the system, in node 1 and in node 2 can be calculated
as shown in code snippets~\ref{lst:mean_number_of_individuals_in_system},
\ref{lst:mean_number_of_individuals_in_node_1}
and~\ref{lst:mean_number_of_individuals_in_node_2}.

\begin{lstlisting}[
    style=pystyle,
    caption={Code snippet for calculating the mean number of individuals in
    the system.},
    label={lst:mean_number_of_individuals_in_system},
]
>>> def get_mean_number_of_individuals_in_system(pi, states):
...     """Gets the mean number of individuals in the system
...     Parameters
...     ----------
...     pi : numpy.ndarray
...         steady state vector
...     states : list
...         list of tuples that contains all states
...     Returns
...     -------
...     float
...         Mean number of individuals in the whole model
...     """
...     states = np.array(states)
...     mean_inds_in_system = np.sum((states[:, 0] + states[:, 1]) * pi)
...     return mean_inds_in_system
>>>
>>> round(get_mean_number_of_individuals_in_system(pi, all_states), 10)
2.0872927227

\end{lstlisting}

\begin{lstlisting}[
    style=pystyle,
    caption={Code snippet for calculating the mean number of individuals in
    Node 1.},
    label={lst:mean_number_of_individuals_in_node_1},
]
>>> def get_mean_number_of_individuals_in_node_1(pi, states):
...     """Mean number of individuals in node 1
...     Parameters
...     ----------
...     pi : numpy.ndarray
...         steady state vector
...     states : list
...         list of tuples that contains all states
...     Returns
...     -------
...     float
...         Mean number of individuals
...     """
...     states = np.array(states)
...     mean_inds_in_node_1 = np.sum(states[:, 1] * pi)
...     return mean_inds_in_node_1
>>>
>>> round(get_mean_number_of_individuals_in_node_1(pi, all_states), 10)
1.8187129478

\end{lstlisting}

\begin{lstlisting}[
    style=pystyle,
    caption={Code snippet for calculating the mean number of individuals in
    Node 2.},
    label={lst:mean_number_of_individuals_in_node_2},
]
>>> def get_mean_number_of_individuals_in_node_2(pi, states):
...     """Mean number of class 2 individuals blocked
...     Parameters
...     ----------
...     pi : numpy.ndarray
...         steady state vector
...     states : list
...         list of tuples that contains all states
...     Returns
...     -------
...     float
...         Mean number of blocked class 2 individuals
...     """
...     states = np.array(states)
...     mean_blocked = np.sum(states[:, 0] * pi)
...     return mean_blocked
>>>
>>> round(get_mean_number_of_individuals_in_node_2(pi, all_states), 10)
0.2685797749

\end{lstlisting}


Consequently, there are some additional performance measures of interest that
are more complex to calculate.
Such performance measures are the mean waiting time in the system (for both
type 1 and type 2 individuals), the mean time blocked in node 2 (only
valid for type 2 individuals) and the proportion of individuals that wait in
node 1 within a predefined time target (for both types).
Under the scope of this study three approaches have been considered to calculate
these performance measures; a recursive algorithm, a direct approach and
a closed-form equation. 
Furthermore, different formulas arise for type 1 individuals and different
ones for type 2 individuals.


\input{chapters/03_queueing_model/03_performance_measures/01_waiting_time/main.tex}

\input{chapters/03_queueing_model/03_performance_measures/02_blocking_time/main.tex}

\input{chapters/03_queueing_model/03_performance_measures/03_proportion_of_individuals_within_time/main.tex}

\section{Performance measures}\label{sec:queueing_performance_measures}
Using vector \(\pi\) there are numerous performance measures of the model that
can be calculated.
The following equations utilise \(\pi\) to get performance measures on the
average number of individuals in node 1 and in node 2:

\begin{itemize}
    \item Mean number of individuals in the entire system:
        \begin{equation}
            L_S = \sum_{i=1}^{|\pi|} \pi_i (u_i + v_i)
        \end{equation}
    \item Mean number of individuals in node 1:
        \begin{equation}
            L_1 = \sum_{i=1}^{|\pi|} \pi_i v_i
        \end{equation}
    \item Mean number of individuals in node 2:
        \begin{equation}
            L_2 = \sum_{i=1}^{|\pi|} \pi_i u_i
        \end{equation}
\end{itemize}

The python code for these functions may be obtained quite quickly using the set
of all states and the steady state probability.

\begin{lstlisting}[
    style=pystyle,
    caption={Code snippet for getting the set of all states and the steady
    state probabilities.},
    label={lst:set_of_all_states_and_steady_state_probabilities},
]
>>> import ambulance_game as abg
>>> import numpy as np
>>> all_states = abg.markov.build_states(
...     threshold=3 ,
...     system_capacity=4,
...     buffer_capacity=2
... )
>>> Q = abg.markov.get_transition_matrix(
...     lambda_1=1,
...     lambda_2=2,
...     mu=2,
...     num_of_servers=2,
...     threshold=3,
...     system_capacity=4,
...     buffer_capacity=2
... )
>>> pi = abg.markov.get_steady_state_algebraically(
...     Q, algebraic_function=np.linalg.lstsq
... )

\end{lstlisting}

Having the set of all states and the steady state probabilities, the mean
number of individuals in the system, in node 1 and in node 2 can be calculated
as shown in code snippets~\ref{lst:mean_number_of_individuals_in_system},
\ref{lst:mean_number_of_individuals_in_node_1}
and~\ref{lst:mean_number_of_individuals_in_node_2}.

\begin{lstlisting}[
    style=pystyle,
    caption={Code snippet for calculating the mean number of individuals in
    the system.},
    label={lst:mean_number_of_individuals_in_system},
]
>>> def get_mean_number_of_individuals_in_system(pi, states):
...     """Gets the mean number of individuals in the system
...     Parameters
...     ----------
...     pi : numpy.ndarray
...         steady state vector
...     states : list
...         list of tuples that contains all states
...     Returns
...     -------
...     float
...         Mean number of individuals in the whole model
...     """
...     states = np.array(states)
...     mean_inds_in_system = np.sum((states[:, 0] + states[:, 1]) * pi)
...     return mean_inds_in_system
>>>
>>> round(get_mean_number_of_individuals_in_system(pi, all_states), 10)
2.0872927227

\end{lstlisting}

\begin{lstlisting}[
    style=pystyle,
    caption={Code snippet for calculating the mean number of individuals in
    Node 1.},
    label={lst:mean_number_of_individuals_in_node_1},
]
>>> def get_mean_number_of_individuals_in_node_1(pi, states):
...     """Mean number of individuals in node 1
...     Parameters
...     ----------
...     pi : numpy.ndarray
...         steady state vector
...     states : list
...         list of tuples that contains all states
...     Returns
...     -------
...     float
...         Mean number of individuals
...     """
...     states = np.array(states)
...     mean_inds_in_node_1 = np.sum(states[:, 1] * pi)
...     return mean_inds_in_node_1
>>>
>>> round(get_mean_number_of_individuals_in_node_1(pi, all_states), 10)
1.8187129478

\end{lstlisting}

\begin{lstlisting}[
    style=pystyle,
    caption={Code snippet for calculating the mean number of individuals in
    Node 2.},
    label={lst:mean_number_of_individuals_in_node_2},
]
>>> def get_mean_number_of_individuals_in_node_2(pi, states):
...     """Mean number of class 2 individuals blocked
...     Parameters
...     ----------
...     pi : numpy.ndarray
...         steady state vector
...     states : list
...         list of tuples that contains all states
...     Returns
...     -------
...     float
...         Mean number of blocked class 2 individuals
...     """
...     states = np.array(states)
...     mean_blocked = np.sum(states[:, 0] * pi)
...     return mean_blocked
>>>
>>> round(get_mean_number_of_individuals_in_node_2(pi, all_states), 10)
0.2685797749

\end{lstlisting}


Consequently, there are some additional performance measures of interest that
are more complex to calculate.
Such performance measures are the mean waiting time in the system (for both
type 1 and type 2 individuals), the mean time blocked in node 2 (only
valid for type 2 individuals) and the proportion of individuals that wait in
node 1 within a predefined time target (for both types).
Under the scope of this study three approaches have been considered to calculate
these performance measures; a recursive algorithm, a direct approach and
a closed-form equation. 
Furthermore, different formulas arise for type 1 individuals and different
ones for type 2 individuals.


\input{chapters/03_queueing_model/03_performance_measures/01_waiting_time/main.tex}

\input{chapters/03_queueing_model/03_performance_measures/02_blocking_time/main.tex}

\input{chapters/03_queueing_model/03_performance_measures/03_proportion_of_individuals_within_time/main.tex}

\section{Performance measures}\label{sec:queueing_performance_measures}
Using vector \(\pi\) there are numerous performance measures of the model that
can be calculated.
The following equations utilise \(\pi\) to get performance measures on the
average number of individuals in node 1 and in node 2:

\begin{itemize}
    \item Mean number of individuals in the entire system:
        \begin{equation}
            L_S = \sum_{i=1}^{|\pi|} \pi_i (u_i + v_i)
        \end{equation}
    \item Mean number of individuals in node 1:
        \begin{equation}
            L_1 = \sum_{i=1}^{|\pi|} \pi_i v_i
        \end{equation}
    \item Mean number of individuals in node 2:
        \begin{equation}
            L_2 = \sum_{i=1}^{|\pi|} \pi_i u_i
        \end{equation}
\end{itemize}

The python code for these functions may be obtained quite quickly using the set
of all states and the steady state probability.

\begin{lstlisting}[
    style=pystyle,
    caption={Code snippet for getting the set of all states and the steady
    state probabilities.},
    label={lst:set_of_all_states_and_steady_state_probabilities},
]
>>> import ambulance_game as abg
>>> import numpy as np
>>> all_states = abg.markov.build_states(
...     threshold=3 ,
...     system_capacity=4,
...     buffer_capacity=2
... )
>>> Q = abg.markov.get_transition_matrix(
...     lambda_1=1,
...     lambda_2=2,
...     mu=2,
...     num_of_servers=2,
...     threshold=3,
...     system_capacity=4,
...     buffer_capacity=2
... )
>>> pi = abg.markov.get_steady_state_algebraically(
...     Q, algebraic_function=np.linalg.lstsq
... )

\end{lstlisting}

Having the set of all states and the steady state probabilities, the mean
number of individuals in the system, in node 1 and in node 2 can be calculated
as shown in code snippets~\ref{lst:mean_number_of_individuals_in_system},
\ref{lst:mean_number_of_individuals_in_node_1}
and~\ref{lst:mean_number_of_individuals_in_node_2}.

\begin{lstlisting}[
    style=pystyle,
    caption={Code snippet for calculating the mean number of individuals in
    the system.},
    label={lst:mean_number_of_individuals_in_system},
]
>>> def get_mean_number_of_individuals_in_system(pi, states):
...     """Gets the mean number of individuals in the system
...     Parameters
...     ----------
...     pi : numpy.ndarray
...         steady state vector
...     states : list
...         list of tuples that contains all states
...     Returns
...     -------
...     float
...         Mean number of individuals in the whole model
...     """
...     states = np.array(states)
...     mean_inds_in_system = np.sum((states[:, 0] + states[:, 1]) * pi)
...     return mean_inds_in_system
>>>
>>> round(get_mean_number_of_individuals_in_system(pi, all_states), 10)
2.0872927227

\end{lstlisting}

\begin{lstlisting}[
    style=pystyle,
    caption={Code snippet for calculating the mean number of individuals in
    Node 1.},
    label={lst:mean_number_of_individuals_in_node_1},
]
>>> def get_mean_number_of_individuals_in_node_1(pi, states):
...     """Mean number of individuals in node 1
...     Parameters
...     ----------
...     pi : numpy.ndarray
...         steady state vector
...     states : list
...         list of tuples that contains all states
...     Returns
...     -------
...     float
...         Mean number of individuals
...     """
...     states = np.array(states)
...     mean_inds_in_node_1 = np.sum(states[:, 1] * pi)
...     return mean_inds_in_node_1
>>>
>>> round(get_mean_number_of_individuals_in_node_1(pi, all_states), 10)
1.8187129478

\end{lstlisting}

\begin{lstlisting}[
    style=pystyle,
    caption={Code snippet for calculating the mean number of individuals in
    Node 2.},
    label={lst:mean_number_of_individuals_in_node_2},
]
>>> def get_mean_number_of_individuals_in_node_2(pi, states):
...     """Mean number of class 2 individuals blocked
...     Parameters
...     ----------
...     pi : numpy.ndarray
...         steady state vector
...     states : list
...         list of tuples that contains all states
...     Returns
...     -------
...     float
...         Mean number of blocked class 2 individuals
...     """
...     states = np.array(states)
...     mean_blocked = np.sum(states[:, 0] * pi)
...     return mean_blocked
>>>
>>> round(get_mean_number_of_individuals_in_node_2(pi, all_states), 10)
0.2685797749

\end{lstlisting}


Consequently, there are some additional performance measures of interest that
are more complex to calculate.
Such performance measures are the mean waiting time in the system (for both
type 1 and type 2 individuals), the mean time blocked in node 2 (only
valid for type 2 individuals) and the proportion of individuals that wait in
node 1 within a predefined time target (for both types).
Under the scope of this study three approaches have been considered to calculate
these performance measures; a recursive algorithm, a direct approach and
a closed-form equation. 
Furthermore, different formulas arise for type 1 individuals and different
ones for type 2 individuals.


\input{chapters/03_queueing_model/03_performance_measures/01_waiting_time/main.tex}

\input{chapters/03_queueing_model/03_performance_measures/02_blocking_time/main.tex}

\input{chapters/03_queueing_model/03_performance_measures/03_proportion_of_individuals_within_time/main.tex}

\section{Performance measures}\label{sec:queueing_performance_measures}
Using vector \(\pi\) there are numerous performance measures of the model that
can be calculated.
The following equations utilise \(\pi\) to get performance measures on the
average number of individuals in node 1 and in node 2:

\begin{itemize}
    \item Mean number of individuals in the entire system:
        \begin{equation}
            L_S = \sum_{i=1}^{|\pi|} \pi_i (u_i + v_i)
        \end{equation}
    \item Mean number of individuals in node 1:
        \begin{equation}
            L_1 = \sum_{i=1}^{|\pi|} \pi_i v_i
        \end{equation}
    \item Mean number of individuals in node 2:
        \begin{equation}
            L_2 = \sum_{i=1}^{|\pi|} \pi_i u_i
        \end{equation}
\end{itemize}

The python code for these functions may be obtained quite quickly using the set
of all states and the steady state probability.

\begin{lstlisting}[
    style=pystyle,
    caption={Code snippet for getting the set of all states and the steady
    state probabilities.},
    label={lst:set_of_all_states_and_steady_state_probabilities},
]
>>> import ambulance_game as abg
>>> import numpy as np
>>> all_states = abg.markov.build_states(
...     threshold=3 ,
...     system_capacity=4,
...     buffer_capacity=2
... )
>>> Q = abg.markov.get_transition_matrix(
...     lambda_1=1,
...     lambda_2=2,
...     mu=2,
...     num_of_servers=2,
...     threshold=3,
...     system_capacity=4,
...     buffer_capacity=2
... )
>>> pi = abg.markov.get_steady_state_algebraically(
...     Q, algebraic_function=np.linalg.lstsq
... )

\end{lstlisting}

Having the set of all states and the steady state probabilities, the mean
number of individuals in the system, in node 1 and in node 2 can be calculated
as shown in code snippets~\ref{lst:mean_number_of_individuals_in_system},
\ref{lst:mean_number_of_individuals_in_node_1}
and~\ref{lst:mean_number_of_individuals_in_node_2}.

\begin{lstlisting}[
    style=pystyle,
    caption={Code snippet for calculating the mean number of individuals in
    the system.},
    label={lst:mean_number_of_individuals_in_system},
]
>>> def get_mean_number_of_individuals_in_system(pi, states):
...     """Gets the mean number of individuals in the system
...     Parameters
...     ----------
...     pi : numpy.ndarray
...         steady state vector
...     states : list
...         list of tuples that contains all states
...     Returns
...     -------
...     float
...         Mean number of individuals in the whole model
...     """
...     states = np.array(states)
...     mean_inds_in_system = np.sum((states[:, 0] + states[:, 1]) * pi)
...     return mean_inds_in_system
>>>
>>> round(get_mean_number_of_individuals_in_system(pi, all_states), 10)
2.0872927227

\end{lstlisting}

\begin{lstlisting}[
    style=pystyle,
    caption={Code snippet for calculating the mean number of individuals in
    Node 1.},
    label={lst:mean_number_of_individuals_in_node_1},
]
>>> def get_mean_number_of_individuals_in_node_1(pi, states):
...     """Mean number of individuals in node 1
...     Parameters
...     ----------
...     pi : numpy.ndarray
...         steady state vector
...     states : list
...         list of tuples that contains all states
...     Returns
...     -------
...     float
...         Mean number of individuals
...     """
...     states = np.array(states)
...     mean_inds_in_node_1 = np.sum(states[:, 1] * pi)
...     return mean_inds_in_node_1
>>>
>>> round(get_mean_number_of_individuals_in_node_1(pi, all_states), 10)
1.8187129478

\end{lstlisting}

\begin{lstlisting}[
    style=pystyle,
    caption={Code snippet for calculating the mean number of individuals in
    Node 2.},
    label={lst:mean_number_of_individuals_in_node_2},
]
>>> def get_mean_number_of_individuals_in_node_2(pi, states):
...     """Mean number of class 2 individuals blocked
...     Parameters
...     ----------
...     pi : numpy.ndarray
...         steady state vector
...     states : list
...         list of tuples that contains all states
...     Returns
...     -------
...     float
...         Mean number of blocked class 2 individuals
...     """
...     states = np.array(states)
...     mean_blocked = np.sum(states[:, 0] * pi)
...     return mean_blocked
>>>
>>> round(get_mean_number_of_individuals_in_node_2(pi, all_states), 10)
0.2685797749

\end{lstlisting}


Consequently, there are some additional performance measures of interest that
are more complex to calculate.
Such performance measures are the mean waiting time in the system (for both
type 1 and type 2 individuals), the mean time blocked in node 2 (only
valid for type 2 individuals) and the proportion of individuals that wait in
node 1 within a predefined time target (for both types).
Under the scope of this study three approaches have been considered to calculate
these performance measures; a recursive algorithm, a direct approach and
a closed-form equation. 
Furthermore, different formulas arise for type 1 individuals and different
ones for type 2 individuals.


\section{Performance measures}\label{sec:queueing_performance_measures}
Using vector \(\pi\) there are numerous performance measures of the model that
can be calculated.
The following equations utilise \(\pi\) to get performance measures on the
average number of individuals in node 1 and in node 2:

\begin{itemize}
    \item Mean number of individuals in the entire system:
        \begin{equation}
            L_S = \sum_{i=1}^{|\pi|} \pi_i (u_i + v_i)
        \end{equation}
    \item Mean number of individuals in node 1:
        \begin{equation}
            L_1 = \sum_{i=1}^{|\pi|} \pi_i v_i
        \end{equation}
    \item Mean number of individuals in node 2:
        \begin{equation}
            L_2 = \sum_{i=1}^{|\pi|} \pi_i u_i
        \end{equation}
\end{itemize}

The python code for these functions may be obtained quite quickly using the set
of all states and the steady state probability.

\begin{lstlisting}[
    style=pystyle,
    caption={Code snippet for getting the set of all states and the steady
    state probabilities.},
    label={lst:set_of_all_states_and_steady_state_probabilities},
]
>>> import ambulance_game as abg
>>> import numpy as np
>>> all_states = abg.markov.build_states(
...     threshold=3 ,
...     system_capacity=4,
...     buffer_capacity=2
... )
>>> Q = abg.markov.get_transition_matrix(
...     lambda_1=1,
...     lambda_2=2,
...     mu=2,
...     num_of_servers=2,
...     threshold=3,
...     system_capacity=4,
...     buffer_capacity=2
... )
>>> pi = abg.markov.get_steady_state_algebraically(
...     Q, algebraic_function=np.linalg.lstsq
... )

\end{lstlisting}

Having the set of all states and the steady state probabilities, the mean
number of individuals in the system, in node 1 and in node 2 can be calculated
as shown in code snippets~\ref{lst:mean_number_of_individuals_in_system},
\ref{lst:mean_number_of_individuals_in_node_1}
and~\ref{lst:mean_number_of_individuals_in_node_2}.

\begin{lstlisting}[
    style=pystyle,
    caption={Code snippet for calculating the mean number of individuals in
    the system.},
    label={lst:mean_number_of_individuals_in_system},
]
>>> def get_mean_number_of_individuals_in_system(pi, states):
...     """Gets the mean number of individuals in the system
...     Parameters
...     ----------
...     pi : numpy.ndarray
...         steady state vector
...     states : list
...         list of tuples that contains all states
...     Returns
...     -------
...     float
...         Mean number of individuals in the whole model
...     """
...     states = np.array(states)
...     mean_inds_in_system = np.sum((states[:, 0] + states[:, 1]) * pi)
...     return mean_inds_in_system
>>>
>>> round(get_mean_number_of_individuals_in_system(pi, all_states), 10)
2.0872927227

\end{lstlisting}

\begin{lstlisting}[
    style=pystyle,
    caption={Code snippet for calculating the mean number of individuals in
    Node 1.},
    label={lst:mean_number_of_individuals_in_node_1},
]
>>> def get_mean_number_of_individuals_in_node_1(pi, states):
...     """Mean number of individuals in node 1
...     Parameters
...     ----------
...     pi : numpy.ndarray
...         steady state vector
...     states : list
...         list of tuples that contains all states
...     Returns
...     -------
...     float
...         Mean number of individuals
...     """
...     states = np.array(states)
...     mean_inds_in_node_1 = np.sum(states[:, 1] * pi)
...     return mean_inds_in_node_1
>>>
>>> round(get_mean_number_of_individuals_in_node_1(pi, all_states), 10)
1.8187129478

\end{lstlisting}

\begin{lstlisting}[
    style=pystyle,
    caption={Code snippet for calculating the mean number of individuals in
    Node 2.},
    label={lst:mean_number_of_individuals_in_node_2},
]
>>> def get_mean_number_of_individuals_in_node_2(pi, states):
...     """Mean number of class 2 individuals blocked
...     Parameters
...     ----------
...     pi : numpy.ndarray
...         steady state vector
...     states : list
...         list of tuples that contains all states
...     Returns
...     -------
...     float
...         Mean number of blocked class 2 individuals
...     """
...     states = np.array(states)
...     mean_blocked = np.sum(states[:, 0] * pi)
...     return mean_blocked
>>>
>>> round(get_mean_number_of_individuals_in_node_2(pi, all_states), 10)
0.2685797749

\end{lstlisting}


Consequently, there are some additional performance measures of interest that
are more complex to calculate.
Such performance measures are the mean waiting time in the system (for both
type 1 and type 2 individuals), the mean time blocked in node 2 (only
valid for type 2 individuals) and the proportion of individuals that wait in
node 1 within a predefined time target (for both types).
Under the scope of this study three approaches have been considered to calculate
these performance measures; a recursive algorithm, a direct approach and
a closed-form equation. 
Furthermore, different formulas arise for type 1 individuals and different
ones for type 2 individuals.


\section{Performance measures}\label{sec:queueing_performance_measures}
Using vector \(\pi\) there are numerous performance measures of the model that
can be calculated.
The following equations utilise \(\pi\) to get performance measures on the
average number of individuals in node 1 and in node 2:

\begin{itemize}
    \item Mean number of individuals in the entire system:
        \begin{equation}
            L_S = \sum_{i=1}^{|\pi|} \pi_i (u_i + v_i)
        \end{equation}
    \item Mean number of individuals in node 1:
        \begin{equation}
            L_1 = \sum_{i=1}^{|\pi|} \pi_i v_i
        \end{equation}
    \item Mean number of individuals in node 2:
        \begin{equation}
            L_2 = \sum_{i=1}^{|\pi|} \pi_i u_i
        \end{equation}
\end{itemize}

The python code for these functions may be obtained quite quickly using the set
of all states and the steady state probability.

\begin{lstlisting}[
    style=pystyle,
    caption={Code snippet for getting the set of all states and the steady
    state probabilities.},
    label={lst:set_of_all_states_and_steady_state_probabilities},
]
>>> import ambulance_game as abg
>>> import numpy as np
>>> all_states = abg.markov.build_states(
...     threshold=3 ,
...     system_capacity=4,
...     buffer_capacity=2
... )
>>> Q = abg.markov.get_transition_matrix(
...     lambda_1=1,
...     lambda_2=2,
...     mu=2,
...     num_of_servers=2,
...     threshold=3,
...     system_capacity=4,
...     buffer_capacity=2
... )
>>> pi = abg.markov.get_steady_state_algebraically(
...     Q, algebraic_function=np.linalg.lstsq
... )

\end{lstlisting}

Having the set of all states and the steady state probabilities, the mean
number of individuals in the system, in node 1 and in node 2 can be calculated
as shown in code snippets~\ref{lst:mean_number_of_individuals_in_system},
\ref{lst:mean_number_of_individuals_in_node_1}
and~\ref{lst:mean_number_of_individuals_in_node_2}.

\begin{lstlisting}[
    style=pystyle,
    caption={Code snippet for calculating the mean number of individuals in
    the system.},
    label={lst:mean_number_of_individuals_in_system},
]
>>> def get_mean_number_of_individuals_in_system(pi, states):
...     """Gets the mean number of individuals in the system
...     Parameters
...     ----------
...     pi : numpy.ndarray
...         steady state vector
...     states : list
...         list of tuples that contains all states
...     Returns
...     -------
...     float
...         Mean number of individuals in the whole model
...     """
...     states = np.array(states)
...     mean_inds_in_system = np.sum((states[:, 0] + states[:, 1]) * pi)
...     return mean_inds_in_system
>>>
>>> round(get_mean_number_of_individuals_in_system(pi, all_states), 10)
2.0872927227

\end{lstlisting}

\begin{lstlisting}[
    style=pystyle,
    caption={Code snippet for calculating the mean number of individuals in
    Node 1.},
    label={lst:mean_number_of_individuals_in_node_1},
]
>>> def get_mean_number_of_individuals_in_node_1(pi, states):
...     """Mean number of individuals in node 1
...     Parameters
...     ----------
...     pi : numpy.ndarray
...         steady state vector
...     states : list
...         list of tuples that contains all states
...     Returns
...     -------
...     float
...         Mean number of individuals
...     """
...     states = np.array(states)
...     mean_inds_in_node_1 = np.sum(states[:, 1] * pi)
...     return mean_inds_in_node_1
>>>
>>> round(get_mean_number_of_individuals_in_node_1(pi, all_states), 10)
1.8187129478

\end{lstlisting}

\begin{lstlisting}[
    style=pystyle,
    caption={Code snippet for calculating the mean number of individuals in
    Node 2.},
    label={lst:mean_number_of_individuals_in_node_2},
]
>>> def get_mean_number_of_individuals_in_node_2(pi, states):
...     """Mean number of class 2 individuals blocked
...     Parameters
...     ----------
...     pi : numpy.ndarray
...         steady state vector
...     states : list
...         list of tuples that contains all states
...     Returns
...     -------
...     float
...         Mean number of blocked class 2 individuals
...     """
...     states = np.array(states)
...     mean_blocked = np.sum(states[:, 0] * pi)
...     return mean_blocked
>>>
>>> round(get_mean_number_of_individuals_in_node_2(pi, all_states), 10)
0.2685797749

\end{lstlisting}


Consequently, there are some additional performance measures of interest that
are more complex to calculate.
Such performance measures are the mean waiting time in the system (for both
type 1 and type 2 individuals), the mean time blocked in node 2 (only
valid for type 2 individuals) and the proportion of individuals that wait in
node 1 within a predefined time target (for both types).
Under the scope of this study three approaches have been considered to calculate
these performance measures; a recursive algorithm, a direct approach and
a closed-form equation. 
Furthermore, different formulas arise for type 1 individuals and different
ones for type 2 individuals.


\input{chapters/03_queueing_model/03_performance_measures/01_waiting_time/main.tex}

\input{chapters/03_queueing_model/03_performance_measures/02_blocking_time/main.tex}

\input{chapters/03_queueing_model/03_performance_measures/03_proportion_of_individuals_within_time/main.tex}

\section{Performance measures}\label{sec:queueing_performance_measures}
Using vector \(\pi\) there are numerous performance measures of the model that
can be calculated.
The following equations utilise \(\pi\) to get performance measures on the
average number of individuals in node 1 and in node 2:

\begin{itemize}
    \item Mean number of individuals in the entire system:
        \begin{equation}
            L_S = \sum_{i=1}^{|\pi|} \pi_i (u_i + v_i)
        \end{equation}
    \item Mean number of individuals in node 1:
        \begin{equation}
            L_1 = \sum_{i=1}^{|\pi|} \pi_i v_i
        \end{equation}
    \item Mean number of individuals in node 2:
        \begin{equation}
            L_2 = \sum_{i=1}^{|\pi|} \pi_i u_i
        \end{equation}
\end{itemize}

The python code for these functions may be obtained quite quickly using the set
of all states and the steady state probability.

\begin{lstlisting}[
    style=pystyle,
    caption={Code snippet for getting the set of all states and the steady
    state probabilities.},
    label={lst:set_of_all_states_and_steady_state_probabilities},
]
>>> import ambulance_game as abg
>>> import numpy as np
>>> all_states = abg.markov.build_states(
...     threshold=3 ,
...     system_capacity=4,
...     buffer_capacity=2
... )
>>> Q = abg.markov.get_transition_matrix(
...     lambda_1=1,
...     lambda_2=2,
...     mu=2,
...     num_of_servers=2,
...     threshold=3,
...     system_capacity=4,
...     buffer_capacity=2
... )
>>> pi = abg.markov.get_steady_state_algebraically(
...     Q, algebraic_function=np.linalg.lstsq
... )

\end{lstlisting}

Having the set of all states and the steady state probabilities, the mean
number of individuals in the system, in node 1 and in node 2 can be calculated
as shown in code snippets~\ref{lst:mean_number_of_individuals_in_system},
\ref{lst:mean_number_of_individuals_in_node_1}
and~\ref{lst:mean_number_of_individuals_in_node_2}.

\begin{lstlisting}[
    style=pystyle,
    caption={Code snippet for calculating the mean number of individuals in
    the system.},
    label={lst:mean_number_of_individuals_in_system},
]
>>> def get_mean_number_of_individuals_in_system(pi, states):
...     """Gets the mean number of individuals in the system
...     Parameters
...     ----------
...     pi : numpy.ndarray
...         steady state vector
...     states : list
...         list of tuples that contains all states
...     Returns
...     -------
...     float
...         Mean number of individuals in the whole model
...     """
...     states = np.array(states)
...     mean_inds_in_system = np.sum((states[:, 0] + states[:, 1]) * pi)
...     return mean_inds_in_system
>>>
>>> round(get_mean_number_of_individuals_in_system(pi, all_states), 10)
2.0872927227

\end{lstlisting}

\begin{lstlisting}[
    style=pystyle,
    caption={Code snippet for calculating the mean number of individuals in
    Node 1.},
    label={lst:mean_number_of_individuals_in_node_1},
]
>>> def get_mean_number_of_individuals_in_node_1(pi, states):
...     """Mean number of individuals in node 1
...     Parameters
...     ----------
...     pi : numpy.ndarray
...         steady state vector
...     states : list
...         list of tuples that contains all states
...     Returns
...     -------
...     float
...         Mean number of individuals
...     """
...     states = np.array(states)
...     mean_inds_in_node_1 = np.sum(states[:, 1] * pi)
...     return mean_inds_in_node_1
>>>
>>> round(get_mean_number_of_individuals_in_node_1(pi, all_states), 10)
1.8187129478

\end{lstlisting}

\begin{lstlisting}[
    style=pystyle,
    caption={Code snippet for calculating the mean number of individuals in
    Node 2.},
    label={lst:mean_number_of_individuals_in_node_2},
]
>>> def get_mean_number_of_individuals_in_node_2(pi, states):
...     """Mean number of class 2 individuals blocked
...     Parameters
...     ----------
...     pi : numpy.ndarray
...         steady state vector
...     states : list
...         list of tuples that contains all states
...     Returns
...     -------
...     float
...         Mean number of blocked class 2 individuals
...     """
...     states = np.array(states)
...     mean_blocked = np.sum(states[:, 0] * pi)
...     return mean_blocked
>>>
>>> round(get_mean_number_of_individuals_in_node_2(pi, all_states), 10)
0.2685797749

\end{lstlisting}


Consequently, there are some additional performance measures of interest that
are more complex to calculate.
Such performance measures are the mean waiting time in the system (for both
type 1 and type 2 individuals), the mean time blocked in node 2 (only
valid for type 2 individuals) and the proportion of individuals that wait in
node 1 within a predefined time target (for both types).
Under the scope of this study three approaches have been considered to calculate
these performance measures; a recursive algorithm, a direct approach and
a closed-form equation. 
Furthermore, different formulas arise for type 1 individuals and different
ones for type 2 individuals.


\input{chapters/03_queueing_model/03_performance_measures/01_waiting_time/main.tex}

\input{chapters/03_queueing_model/03_performance_measures/02_blocking_time/main.tex}

\input{chapters/03_queueing_model/03_performance_measures/03_proportion_of_individuals_within_time/main.tex}

\section{Performance measures}\label{sec:queueing_performance_measures}
Using vector \(\pi\) there are numerous performance measures of the model that
can be calculated.
The following equations utilise \(\pi\) to get performance measures on the
average number of individuals in node 1 and in node 2:

\begin{itemize}
    \item Mean number of individuals in the entire system:
        \begin{equation}
            L_S = \sum_{i=1}^{|\pi|} \pi_i (u_i + v_i)
        \end{equation}
    \item Mean number of individuals in node 1:
        \begin{equation}
            L_1 = \sum_{i=1}^{|\pi|} \pi_i v_i
        \end{equation}
    \item Mean number of individuals in node 2:
        \begin{equation}
            L_2 = \sum_{i=1}^{|\pi|} \pi_i u_i
        \end{equation}
\end{itemize}

The python code for these functions may be obtained quite quickly using the set
of all states and the steady state probability.

\begin{lstlisting}[
    style=pystyle,
    caption={Code snippet for getting the set of all states and the steady
    state probabilities.},
    label={lst:set_of_all_states_and_steady_state_probabilities},
]
>>> import ambulance_game as abg
>>> import numpy as np
>>> all_states = abg.markov.build_states(
...     threshold=3 ,
...     system_capacity=4,
...     buffer_capacity=2
... )
>>> Q = abg.markov.get_transition_matrix(
...     lambda_1=1,
...     lambda_2=2,
...     mu=2,
...     num_of_servers=2,
...     threshold=3,
...     system_capacity=4,
...     buffer_capacity=2
... )
>>> pi = abg.markov.get_steady_state_algebraically(
...     Q, algebraic_function=np.linalg.lstsq
... )

\end{lstlisting}

Having the set of all states and the steady state probabilities, the mean
number of individuals in the system, in node 1 and in node 2 can be calculated
as shown in code snippets~\ref{lst:mean_number_of_individuals_in_system},
\ref{lst:mean_number_of_individuals_in_node_1}
and~\ref{lst:mean_number_of_individuals_in_node_2}.

\begin{lstlisting}[
    style=pystyle,
    caption={Code snippet for calculating the mean number of individuals in
    the system.},
    label={lst:mean_number_of_individuals_in_system},
]
>>> def get_mean_number_of_individuals_in_system(pi, states):
...     """Gets the mean number of individuals in the system
...     Parameters
...     ----------
...     pi : numpy.ndarray
...         steady state vector
...     states : list
...         list of tuples that contains all states
...     Returns
...     -------
...     float
...         Mean number of individuals in the whole model
...     """
...     states = np.array(states)
...     mean_inds_in_system = np.sum((states[:, 0] + states[:, 1]) * pi)
...     return mean_inds_in_system
>>>
>>> round(get_mean_number_of_individuals_in_system(pi, all_states), 10)
2.0872927227

\end{lstlisting}

\begin{lstlisting}[
    style=pystyle,
    caption={Code snippet for calculating the mean number of individuals in
    Node 1.},
    label={lst:mean_number_of_individuals_in_node_1},
]
>>> def get_mean_number_of_individuals_in_node_1(pi, states):
...     """Mean number of individuals in node 1
...     Parameters
...     ----------
...     pi : numpy.ndarray
...         steady state vector
...     states : list
...         list of tuples that contains all states
...     Returns
...     -------
...     float
...         Mean number of individuals
...     """
...     states = np.array(states)
...     mean_inds_in_node_1 = np.sum(states[:, 1] * pi)
...     return mean_inds_in_node_1
>>>
>>> round(get_mean_number_of_individuals_in_node_1(pi, all_states), 10)
1.8187129478

\end{lstlisting}

\begin{lstlisting}[
    style=pystyle,
    caption={Code snippet for calculating the mean number of individuals in
    Node 2.},
    label={lst:mean_number_of_individuals_in_node_2},
]
>>> def get_mean_number_of_individuals_in_node_2(pi, states):
...     """Mean number of class 2 individuals blocked
...     Parameters
...     ----------
...     pi : numpy.ndarray
...         steady state vector
...     states : list
...         list of tuples that contains all states
...     Returns
...     -------
...     float
...         Mean number of blocked class 2 individuals
...     """
...     states = np.array(states)
...     mean_blocked = np.sum(states[:, 0] * pi)
...     return mean_blocked
>>>
>>> round(get_mean_number_of_individuals_in_node_2(pi, all_states), 10)
0.2685797749

\end{lstlisting}


Consequently, there are some additional performance measures of interest that
are more complex to calculate.
Such performance measures are the mean waiting time in the system (for both
type 1 and type 2 individuals), the mean time blocked in node 2 (only
valid for type 2 individuals) and the proportion of individuals that wait in
node 1 within a predefined time target (for both types).
Under the scope of this study three approaches have been considered to calculate
these performance measures; a recursive algorithm, a direct approach and
a closed-form equation. 
Furthermore, different formulas arise for type 1 individuals and different
ones for type 2 individuals.


\input{chapters/03_queueing_model/03_performance_measures/01_waiting_time/main.tex}

\input{chapters/03_queueing_model/03_performance_measures/02_blocking_time/main.tex}

\input{chapters/03_queueing_model/03_performance_measures/03_proportion_of_individuals_within_time/main.tex}

\section{Performance measures}\label{sec:queueing_performance_measures}
Using vector \(\pi\) there are numerous performance measures of the model that
can be calculated.
The following equations utilise \(\pi\) to get performance measures on the
average number of individuals in node 1 and in node 2:

\begin{itemize}
    \item Mean number of individuals in the entire system:
        \begin{equation}
            L_S = \sum_{i=1}^{|\pi|} \pi_i (u_i + v_i)
        \end{equation}
    \item Mean number of individuals in node 1:
        \begin{equation}
            L_1 = \sum_{i=1}^{|\pi|} \pi_i v_i
        \end{equation}
    \item Mean number of individuals in node 2:
        \begin{equation}
            L_2 = \sum_{i=1}^{|\pi|} \pi_i u_i
        \end{equation}
\end{itemize}

The python code for these functions may be obtained quite quickly using the set
of all states and the steady state probability.

\begin{lstlisting}[
    style=pystyle,
    caption={Code snippet for getting the set of all states and the steady
    state probabilities.},
    label={lst:set_of_all_states_and_steady_state_probabilities},
]
>>> import ambulance_game as abg
>>> import numpy as np
>>> all_states = abg.markov.build_states(
...     threshold=3 ,
...     system_capacity=4,
...     buffer_capacity=2
... )
>>> Q = abg.markov.get_transition_matrix(
...     lambda_1=1,
...     lambda_2=2,
...     mu=2,
...     num_of_servers=2,
...     threshold=3,
...     system_capacity=4,
...     buffer_capacity=2
... )
>>> pi = abg.markov.get_steady_state_algebraically(
...     Q, algebraic_function=np.linalg.lstsq
... )

\end{lstlisting}

Having the set of all states and the steady state probabilities, the mean
number of individuals in the system, in node 1 and in node 2 can be calculated
as shown in code snippets~\ref{lst:mean_number_of_individuals_in_system},
\ref{lst:mean_number_of_individuals_in_node_1}
and~\ref{lst:mean_number_of_individuals_in_node_2}.

\begin{lstlisting}[
    style=pystyle,
    caption={Code snippet for calculating the mean number of individuals in
    the system.},
    label={lst:mean_number_of_individuals_in_system},
]
>>> def get_mean_number_of_individuals_in_system(pi, states):
...     """Gets the mean number of individuals in the system
...     Parameters
...     ----------
...     pi : numpy.ndarray
...         steady state vector
...     states : list
...         list of tuples that contains all states
...     Returns
...     -------
...     float
...         Mean number of individuals in the whole model
...     """
...     states = np.array(states)
...     mean_inds_in_system = np.sum((states[:, 0] + states[:, 1]) * pi)
...     return mean_inds_in_system
>>>
>>> round(get_mean_number_of_individuals_in_system(pi, all_states), 10)
2.0872927227

\end{lstlisting}

\begin{lstlisting}[
    style=pystyle,
    caption={Code snippet for calculating the mean number of individuals in
    Node 1.},
    label={lst:mean_number_of_individuals_in_node_1},
]
>>> def get_mean_number_of_individuals_in_node_1(pi, states):
...     """Mean number of individuals in node 1
...     Parameters
...     ----------
...     pi : numpy.ndarray
...         steady state vector
...     states : list
...         list of tuples that contains all states
...     Returns
...     -------
...     float
...         Mean number of individuals
...     """
...     states = np.array(states)
...     mean_inds_in_node_1 = np.sum(states[:, 1] * pi)
...     return mean_inds_in_node_1
>>>
>>> round(get_mean_number_of_individuals_in_node_1(pi, all_states), 10)
1.8187129478

\end{lstlisting}

\begin{lstlisting}[
    style=pystyle,
    caption={Code snippet for calculating the mean number of individuals in
    Node 2.},
    label={lst:mean_number_of_individuals_in_node_2},
]
>>> def get_mean_number_of_individuals_in_node_2(pi, states):
...     """Mean number of class 2 individuals blocked
...     Parameters
...     ----------
...     pi : numpy.ndarray
...         steady state vector
...     states : list
...         list of tuples that contains all states
...     Returns
...     -------
...     float
...         Mean number of blocked class 2 individuals
...     """
...     states = np.array(states)
...     mean_blocked = np.sum(states[:, 0] * pi)
...     return mean_blocked
>>>
>>> round(get_mean_number_of_individuals_in_node_2(pi, all_states), 10)
0.2685797749

\end{lstlisting}


Consequently, there are some additional performance measures of interest that
are more complex to calculate.
Such performance measures are the mean waiting time in the system (for both
type 1 and type 2 individuals), the mean time blocked in node 2 (only
valid for type 2 individuals) and the proportion of individuals that wait in
node 1 within a predefined time target (for both types).
Under the scope of this study three approaches have been considered to calculate
these performance measures; a recursive algorithm, a direct approach and
a closed-form equation. 
Furthermore, different formulas arise for type 1 individuals and different
ones for type 2 individuals.


\section{Performance measures}\label{sec:queueing_performance_measures}
Using vector \(\pi\) there are numerous performance measures of the model that
can be calculated.
The following equations utilise \(\pi\) to get performance measures on the
average number of individuals in node 1 and in node 2:

\begin{itemize}
    \item Mean number of individuals in the entire system:
        \begin{equation}
            L_S = \sum_{i=1}^{|\pi|} \pi_i (u_i + v_i)
        \end{equation}
    \item Mean number of individuals in node 1:
        \begin{equation}
            L_1 = \sum_{i=1}^{|\pi|} \pi_i v_i
        \end{equation}
    \item Mean number of individuals in node 2:
        \begin{equation}
            L_2 = \sum_{i=1}^{|\pi|} \pi_i u_i
        \end{equation}
\end{itemize}

The python code for these functions may be obtained quite quickly using the set
of all states and the steady state probability.

\begin{lstlisting}[
    style=pystyle,
    caption={Code snippet for getting the set of all states and the steady
    state probabilities.},
    label={lst:set_of_all_states_and_steady_state_probabilities},
]
>>> import ambulance_game as abg
>>> import numpy as np
>>> all_states = abg.markov.build_states(
...     threshold=3 ,
...     system_capacity=4,
...     buffer_capacity=2
... )
>>> Q = abg.markov.get_transition_matrix(
...     lambda_1=1,
...     lambda_2=2,
...     mu=2,
...     num_of_servers=2,
...     threshold=3,
...     system_capacity=4,
...     buffer_capacity=2
... )
>>> pi = abg.markov.get_steady_state_algebraically(
...     Q, algebraic_function=np.linalg.lstsq
... )

\end{lstlisting}

Having the set of all states and the steady state probabilities, the mean
number of individuals in the system, in node 1 and in node 2 can be calculated
as shown in code snippets~\ref{lst:mean_number_of_individuals_in_system},
\ref{lst:mean_number_of_individuals_in_node_1}
and~\ref{lst:mean_number_of_individuals_in_node_2}.

\begin{lstlisting}[
    style=pystyle,
    caption={Code snippet for calculating the mean number of individuals in
    the system.},
    label={lst:mean_number_of_individuals_in_system},
]
>>> def get_mean_number_of_individuals_in_system(pi, states):
...     """Gets the mean number of individuals in the system
...     Parameters
...     ----------
...     pi : numpy.ndarray
...         steady state vector
...     states : list
...         list of tuples that contains all states
...     Returns
...     -------
...     float
...         Mean number of individuals in the whole model
...     """
...     states = np.array(states)
...     mean_inds_in_system = np.sum((states[:, 0] + states[:, 1]) * pi)
...     return mean_inds_in_system
>>>
>>> round(get_mean_number_of_individuals_in_system(pi, all_states), 10)
2.0872927227

\end{lstlisting}

\begin{lstlisting}[
    style=pystyle,
    caption={Code snippet for calculating the mean number of individuals in
    Node 1.},
    label={lst:mean_number_of_individuals_in_node_1},
]
>>> def get_mean_number_of_individuals_in_node_1(pi, states):
...     """Mean number of individuals in node 1
...     Parameters
...     ----------
...     pi : numpy.ndarray
...         steady state vector
...     states : list
...         list of tuples that contains all states
...     Returns
...     -------
...     float
...         Mean number of individuals
...     """
...     states = np.array(states)
...     mean_inds_in_node_1 = np.sum(states[:, 1] * pi)
...     return mean_inds_in_node_1
>>>
>>> round(get_mean_number_of_individuals_in_node_1(pi, all_states), 10)
1.8187129478

\end{lstlisting}

\begin{lstlisting}[
    style=pystyle,
    caption={Code snippet for calculating the mean number of individuals in
    Node 2.},
    label={lst:mean_number_of_individuals_in_node_2},
]
>>> def get_mean_number_of_individuals_in_node_2(pi, states):
...     """Mean number of class 2 individuals blocked
...     Parameters
...     ----------
...     pi : numpy.ndarray
...         steady state vector
...     states : list
...         list of tuples that contains all states
...     Returns
...     -------
...     float
...         Mean number of blocked class 2 individuals
...     """
...     states = np.array(states)
...     mean_blocked = np.sum(states[:, 0] * pi)
...     return mean_blocked
>>>
>>> round(get_mean_number_of_individuals_in_node_2(pi, all_states), 10)
0.2685797749

\end{lstlisting}


Consequently, there are some additional performance measures of interest that
are more complex to calculate.
Such performance measures are the mean waiting time in the system (for both
type 1 and type 2 individuals), the mean time blocked in node 2 (only
valid for type 2 individuals) and the proportion of individuals that wait in
node 1 within a predefined time target (for both types).
Under the scope of this study three approaches have been considered to calculate
these performance measures; a recursive algorithm, a direct approach and
a closed-form equation. 
Furthermore, different formulas arise for type 1 individuals and different
ones for type 2 individuals.


\input{chapters/03_queueing_model/03_performance_measures/01_waiting_time/main.tex}

\input{chapters/03_queueing_model/03_performance_measures/02_blocking_time/main.tex}

\input{chapters/03_queueing_model/03_performance_measures/03_proportion_of_individuals_within_time/main.tex}

\section{Performance measures}\label{sec:queueing_performance_measures}
Using vector \(\pi\) there are numerous performance measures of the model that
can be calculated.
The following equations utilise \(\pi\) to get performance measures on the
average number of individuals in node 1 and in node 2:

\begin{itemize}
    \item Mean number of individuals in the entire system:
        \begin{equation}
            L_S = \sum_{i=1}^{|\pi|} \pi_i (u_i + v_i)
        \end{equation}
    \item Mean number of individuals in node 1:
        \begin{equation}
            L_1 = \sum_{i=1}^{|\pi|} \pi_i v_i
        \end{equation}
    \item Mean number of individuals in node 2:
        \begin{equation}
            L_2 = \sum_{i=1}^{|\pi|} \pi_i u_i
        \end{equation}
\end{itemize}

The python code for these functions may be obtained quite quickly using the set
of all states and the steady state probability.

\begin{lstlisting}[
    style=pystyle,
    caption={Code snippet for getting the set of all states and the steady
    state probabilities.},
    label={lst:set_of_all_states_and_steady_state_probabilities},
]
>>> import ambulance_game as abg
>>> import numpy as np
>>> all_states = abg.markov.build_states(
...     threshold=3 ,
...     system_capacity=4,
...     buffer_capacity=2
... )
>>> Q = abg.markov.get_transition_matrix(
...     lambda_1=1,
...     lambda_2=2,
...     mu=2,
...     num_of_servers=2,
...     threshold=3,
...     system_capacity=4,
...     buffer_capacity=2
... )
>>> pi = abg.markov.get_steady_state_algebraically(
...     Q, algebraic_function=np.linalg.lstsq
... )

\end{lstlisting}

Having the set of all states and the steady state probabilities, the mean
number of individuals in the system, in node 1 and in node 2 can be calculated
as shown in code snippets~\ref{lst:mean_number_of_individuals_in_system},
\ref{lst:mean_number_of_individuals_in_node_1}
and~\ref{lst:mean_number_of_individuals_in_node_2}.

\begin{lstlisting}[
    style=pystyle,
    caption={Code snippet for calculating the mean number of individuals in
    the system.},
    label={lst:mean_number_of_individuals_in_system},
]
>>> def get_mean_number_of_individuals_in_system(pi, states):
...     """Gets the mean number of individuals in the system
...     Parameters
...     ----------
...     pi : numpy.ndarray
...         steady state vector
...     states : list
...         list of tuples that contains all states
...     Returns
...     -------
...     float
...         Mean number of individuals in the whole model
...     """
...     states = np.array(states)
...     mean_inds_in_system = np.sum((states[:, 0] + states[:, 1]) * pi)
...     return mean_inds_in_system
>>>
>>> round(get_mean_number_of_individuals_in_system(pi, all_states), 10)
2.0872927227

\end{lstlisting}

\begin{lstlisting}[
    style=pystyle,
    caption={Code snippet for calculating the mean number of individuals in
    Node 1.},
    label={lst:mean_number_of_individuals_in_node_1},
]
>>> def get_mean_number_of_individuals_in_node_1(pi, states):
...     """Mean number of individuals in node 1
...     Parameters
...     ----------
...     pi : numpy.ndarray
...         steady state vector
...     states : list
...         list of tuples that contains all states
...     Returns
...     -------
...     float
...         Mean number of individuals
...     """
...     states = np.array(states)
...     mean_inds_in_node_1 = np.sum(states[:, 1] * pi)
...     return mean_inds_in_node_1
>>>
>>> round(get_mean_number_of_individuals_in_node_1(pi, all_states), 10)
1.8187129478

\end{lstlisting}

\begin{lstlisting}[
    style=pystyle,
    caption={Code snippet for calculating the mean number of individuals in
    Node 2.},
    label={lst:mean_number_of_individuals_in_node_2},
]
>>> def get_mean_number_of_individuals_in_node_2(pi, states):
...     """Mean number of class 2 individuals blocked
...     Parameters
...     ----------
...     pi : numpy.ndarray
...         steady state vector
...     states : list
...         list of tuples that contains all states
...     Returns
...     -------
...     float
...         Mean number of blocked class 2 individuals
...     """
...     states = np.array(states)
...     mean_blocked = np.sum(states[:, 0] * pi)
...     return mean_blocked
>>>
>>> round(get_mean_number_of_individuals_in_node_2(pi, all_states), 10)
0.2685797749

\end{lstlisting}


Consequently, there are some additional performance measures of interest that
are more complex to calculate.
Such performance measures are the mean waiting time in the system (for both
type 1 and type 2 individuals), the mean time blocked in node 2 (only
valid for type 2 individuals) and the proportion of individuals that wait in
node 1 within a predefined time target (for both types).
Under the scope of this study three approaches have been considered to calculate
these performance measures; a recursive algorithm, a direct approach and
a closed-form equation. 
Furthermore, different formulas arise for type 1 individuals and different
ones for type 2 individuals.


\input{chapters/03_queueing_model/03_performance_measures/01_waiting_time/main.tex}

\input{chapters/03_queueing_model/03_performance_measures/02_blocking_time/main.tex}

\input{chapters/03_queueing_model/03_performance_measures/03_proportion_of_individuals_within_time/main.tex}

\section{Performance measures}\label{sec:queueing_performance_measures}
Using vector \(\pi\) there are numerous performance measures of the model that
can be calculated.
The following equations utilise \(\pi\) to get performance measures on the
average number of individuals in node 1 and in node 2:

\begin{itemize}
    \item Mean number of individuals in the entire system:
        \begin{equation}
            L_S = \sum_{i=1}^{|\pi|} \pi_i (u_i + v_i)
        \end{equation}
    \item Mean number of individuals in node 1:
        \begin{equation}
            L_1 = \sum_{i=1}^{|\pi|} \pi_i v_i
        \end{equation}
    \item Mean number of individuals in node 2:
        \begin{equation}
            L_2 = \sum_{i=1}^{|\pi|} \pi_i u_i
        \end{equation}
\end{itemize}

The python code for these functions may be obtained quite quickly using the set
of all states and the steady state probability.

\begin{lstlisting}[
    style=pystyle,
    caption={Code snippet for getting the set of all states and the steady
    state probabilities.},
    label={lst:set_of_all_states_and_steady_state_probabilities},
]
>>> import ambulance_game as abg
>>> import numpy as np
>>> all_states = abg.markov.build_states(
...     threshold=3 ,
...     system_capacity=4,
...     buffer_capacity=2
... )
>>> Q = abg.markov.get_transition_matrix(
...     lambda_1=1,
...     lambda_2=2,
...     mu=2,
...     num_of_servers=2,
...     threshold=3,
...     system_capacity=4,
...     buffer_capacity=2
... )
>>> pi = abg.markov.get_steady_state_algebraically(
...     Q, algebraic_function=np.linalg.lstsq
... )

\end{lstlisting}

Having the set of all states and the steady state probabilities, the mean
number of individuals in the system, in node 1 and in node 2 can be calculated
as shown in code snippets~\ref{lst:mean_number_of_individuals_in_system},
\ref{lst:mean_number_of_individuals_in_node_1}
and~\ref{lst:mean_number_of_individuals_in_node_2}.

\begin{lstlisting}[
    style=pystyle,
    caption={Code snippet for calculating the mean number of individuals in
    the system.},
    label={lst:mean_number_of_individuals_in_system},
]
>>> def get_mean_number_of_individuals_in_system(pi, states):
...     """Gets the mean number of individuals in the system
...     Parameters
...     ----------
...     pi : numpy.ndarray
...         steady state vector
...     states : list
...         list of tuples that contains all states
...     Returns
...     -------
...     float
...         Mean number of individuals in the whole model
...     """
...     states = np.array(states)
...     mean_inds_in_system = np.sum((states[:, 0] + states[:, 1]) * pi)
...     return mean_inds_in_system
>>>
>>> round(get_mean_number_of_individuals_in_system(pi, all_states), 10)
2.0872927227

\end{lstlisting}

\begin{lstlisting}[
    style=pystyle,
    caption={Code snippet for calculating the mean number of individuals in
    Node 1.},
    label={lst:mean_number_of_individuals_in_node_1},
]
>>> def get_mean_number_of_individuals_in_node_1(pi, states):
...     """Mean number of individuals in node 1
...     Parameters
...     ----------
...     pi : numpy.ndarray
...         steady state vector
...     states : list
...         list of tuples that contains all states
...     Returns
...     -------
...     float
...         Mean number of individuals
...     """
...     states = np.array(states)
...     mean_inds_in_node_1 = np.sum(states[:, 1] * pi)
...     return mean_inds_in_node_1
>>>
>>> round(get_mean_number_of_individuals_in_node_1(pi, all_states), 10)
1.8187129478

\end{lstlisting}

\begin{lstlisting}[
    style=pystyle,
    caption={Code snippet for calculating the mean number of individuals in
    Node 2.},
    label={lst:mean_number_of_individuals_in_node_2},
]
>>> def get_mean_number_of_individuals_in_node_2(pi, states):
...     """Mean number of class 2 individuals blocked
...     Parameters
...     ----------
...     pi : numpy.ndarray
...         steady state vector
...     states : list
...         list of tuples that contains all states
...     Returns
...     -------
...     float
...         Mean number of blocked class 2 individuals
...     """
...     states = np.array(states)
...     mean_blocked = np.sum(states[:, 0] * pi)
...     return mean_blocked
>>>
>>> round(get_mean_number_of_individuals_in_node_2(pi, all_states), 10)
0.2685797749

\end{lstlisting}


Consequently, there are some additional performance measures of interest that
are more complex to calculate.
Such performance measures are the mean waiting time in the system (for both
type 1 and type 2 individuals), the mean time blocked in node 2 (only
valid for type 2 individuals) and the proportion of individuals that wait in
node 1 within a predefined time target (for both types).
Under the scope of this study three approaches have been considered to calculate
these performance measures; a recursive algorithm, a direct approach and
a closed-form equation. 
Furthermore, different formulas arise for type 1 individuals and different
ones for type 2 individuals.


\input{chapters/03_queueing_model/03_performance_measures/01_waiting_time/main.tex}

\input{chapters/03_queueing_model/03_performance_measures/02_blocking_time/main.tex}

\input{chapters/03_queueing_model/03_performance_measures/03_proportion_of_individuals_within_time/main.tex}

\section{Performance measures}\label{sec:queueing_performance_measures}
Using vector \(\pi\) there are numerous performance measures of the model that
can be calculated.
The following equations utilise \(\pi\) to get performance measures on the
average number of individuals in node 1 and in node 2:

\begin{itemize}
    \item Mean number of individuals in the entire system:
        \begin{equation}
            L_S = \sum_{i=1}^{|\pi|} \pi_i (u_i + v_i)
        \end{equation}
    \item Mean number of individuals in node 1:
        \begin{equation}
            L_1 = \sum_{i=1}^{|\pi|} \pi_i v_i
        \end{equation}
    \item Mean number of individuals in node 2:
        \begin{equation}
            L_2 = \sum_{i=1}^{|\pi|} \pi_i u_i
        \end{equation}
\end{itemize}

The python code for these functions may be obtained quite quickly using the set
of all states and the steady state probability.

\begin{lstlisting}[
    style=pystyle,
    caption={Code snippet for getting the set of all states and the steady
    state probabilities.},
    label={lst:set_of_all_states_and_steady_state_probabilities},
]
>>> import ambulance_game as abg
>>> import numpy as np
>>> all_states = abg.markov.build_states(
...     threshold=3 ,
...     system_capacity=4,
...     buffer_capacity=2
... )
>>> Q = abg.markov.get_transition_matrix(
...     lambda_1=1,
...     lambda_2=2,
...     mu=2,
...     num_of_servers=2,
...     threshold=3,
...     system_capacity=4,
...     buffer_capacity=2
... )
>>> pi = abg.markov.get_steady_state_algebraically(
...     Q, algebraic_function=np.linalg.lstsq
... )

\end{lstlisting}

Having the set of all states and the steady state probabilities, the mean
number of individuals in the system, in node 1 and in node 2 can be calculated
as shown in code snippets~\ref{lst:mean_number_of_individuals_in_system},
\ref{lst:mean_number_of_individuals_in_node_1}
and~\ref{lst:mean_number_of_individuals_in_node_2}.

\begin{lstlisting}[
    style=pystyle,
    caption={Code snippet for calculating the mean number of individuals in
    the system.},
    label={lst:mean_number_of_individuals_in_system},
]
>>> def get_mean_number_of_individuals_in_system(pi, states):
...     """Gets the mean number of individuals in the system
...     Parameters
...     ----------
...     pi : numpy.ndarray
...         steady state vector
...     states : list
...         list of tuples that contains all states
...     Returns
...     -------
...     float
...         Mean number of individuals in the whole model
...     """
...     states = np.array(states)
...     mean_inds_in_system = np.sum((states[:, 0] + states[:, 1]) * pi)
...     return mean_inds_in_system
>>>
>>> round(get_mean_number_of_individuals_in_system(pi, all_states), 10)
2.0872927227

\end{lstlisting}

\begin{lstlisting}[
    style=pystyle,
    caption={Code snippet for calculating the mean number of individuals in
    Node 1.},
    label={lst:mean_number_of_individuals_in_node_1},
]
>>> def get_mean_number_of_individuals_in_node_1(pi, states):
...     """Mean number of individuals in node 1
...     Parameters
...     ----------
...     pi : numpy.ndarray
...         steady state vector
...     states : list
...         list of tuples that contains all states
...     Returns
...     -------
...     float
...         Mean number of individuals
...     """
...     states = np.array(states)
...     mean_inds_in_node_1 = np.sum(states[:, 1] * pi)
...     return mean_inds_in_node_1
>>>
>>> round(get_mean_number_of_individuals_in_node_1(pi, all_states), 10)
1.8187129478

\end{lstlisting}

\begin{lstlisting}[
    style=pystyle,
    caption={Code snippet for calculating the mean number of individuals in
    Node 2.},
    label={lst:mean_number_of_individuals_in_node_2},
]
>>> def get_mean_number_of_individuals_in_node_2(pi, states):
...     """Mean number of class 2 individuals blocked
...     Parameters
...     ----------
...     pi : numpy.ndarray
...         steady state vector
...     states : list
...         list of tuples that contains all states
...     Returns
...     -------
...     float
...         Mean number of blocked class 2 individuals
...     """
...     states = np.array(states)
...     mean_blocked = np.sum(states[:, 0] * pi)
...     return mean_blocked
>>>
>>> round(get_mean_number_of_individuals_in_node_2(pi, all_states), 10)
0.2685797749

\end{lstlisting}


Consequently, there are some additional performance measures of interest that
are more complex to calculate.
Such performance measures are the mean waiting time in the system (for both
type 1 and type 2 individuals), the mean time blocked in node 2 (only
valid for type 2 individuals) and the proportion of individuals that wait in
node 1 within a predefined time target (for both types).
Under the scope of this study three approaches have been considered to calculate
these performance measures; a recursive algorithm, a direct approach and
a closed-form equation. 
Furthermore, different formulas arise for type 1 individuals and different
ones for type 2 individuals.


\section{Performance measures}\label{sec:queueing_performance_measures}
Using vector \(\pi\) there are numerous performance measures of the model that
can be calculated.
The following equations utilise \(\pi\) to get performance measures on the
average number of individuals in node 1 and in node 2:

\begin{itemize}
    \item Mean number of individuals in the entire system:
        \begin{equation}
            L_S = \sum_{i=1}^{|\pi|} \pi_i (u_i + v_i)
        \end{equation}
    \item Mean number of individuals in node 1:
        \begin{equation}
            L_1 = \sum_{i=1}^{|\pi|} \pi_i v_i
        \end{equation}
    \item Mean number of individuals in node 2:
        \begin{equation}
            L_2 = \sum_{i=1}^{|\pi|} \pi_i u_i
        \end{equation}
\end{itemize}

The python code for these functions may be obtained quite quickly using the set
of all states and the steady state probability.

\begin{lstlisting}[
    style=pystyle,
    caption={Code snippet for getting the set of all states and the steady
    state probabilities.},
    label={lst:set_of_all_states_and_steady_state_probabilities},
]
>>> import ambulance_game as abg
>>> import numpy as np
>>> all_states = abg.markov.build_states(
...     threshold=3 ,
...     system_capacity=4,
...     buffer_capacity=2
... )
>>> Q = abg.markov.get_transition_matrix(
...     lambda_1=1,
...     lambda_2=2,
...     mu=2,
...     num_of_servers=2,
...     threshold=3,
...     system_capacity=4,
...     buffer_capacity=2
... )
>>> pi = abg.markov.get_steady_state_algebraically(
...     Q, algebraic_function=np.linalg.lstsq
... )

\end{lstlisting}

Having the set of all states and the steady state probabilities, the mean
number of individuals in the system, in node 1 and in node 2 can be calculated
as shown in code snippets~\ref{lst:mean_number_of_individuals_in_system},
\ref{lst:mean_number_of_individuals_in_node_1}
and~\ref{lst:mean_number_of_individuals_in_node_2}.

\begin{lstlisting}[
    style=pystyle,
    caption={Code snippet for calculating the mean number of individuals in
    the system.},
    label={lst:mean_number_of_individuals_in_system},
]
>>> def get_mean_number_of_individuals_in_system(pi, states):
...     """Gets the mean number of individuals in the system
...     Parameters
...     ----------
...     pi : numpy.ndarray
...         steady state vector
...     states : list
...         list of tuples that contains all states
...     Returns
...     -------
...     float
...         Mean number of individuals in the whole model
...     """
...     states = np.array(states)
...     mean_inds_in_system = np.sum((states[:, 0] + states[:, 1]) * pi)
...     return mean_inds_in_system
>>>
>>> round(get_mean_number_of_individuals_in_system(pi, all_states), 10)
2.0872927227

\end{lstlisting}

\begin{lstlisting}[
    style=pystyle,
    caption={Code snippet for calculating the mean number of individuals in
    Node 1.},
    label={lst:mean_number_of_individuals_in_node_1},
]
>>> def get_mean_number_of_individuals_in_node_1(pi, states):
...     """Mean number of individuals in node 1
...     Parameters
...     ----------
...     pi : numpy.ndarray
...         steady state vector
...     states : list
...         list of tuples that contains all states
...     Returns
...     -------
...     float
...         Mean number of individuals
...     """
...     states = np.array(states)
...     mean_inds_in_node_1 = np.sum(states[:, 1] * pi)
...     return mean_inds_in_node_1
>>>
>>> round(get_mean_number_of_individuals_in_node_1(pi, all_states), 10)
1.8187129478

\end{lstlisting}

\begin{lstlisting}[
    style=pystyle,
    caption={Code snippet for calculating the mean number of individuals in
    Node 2.},
    label={lst:mean_number_of_individuals_in_node_2},
]
>>> def get_mean_number_of_individuals_in_node_2(pi, states):
...     """Mean number of class 2 individuals blocked
...     Parameters
...     ----------
...     pi : numpy.ndarray
...         steady state vector
...     states : list
...         list of tuples that contains all states
...     Returns
...     -------
...     float
...         Mean number of blocked class 2 individuals
...     """
...     states = np.array(states)
...     mean_blocked = np.sum(states[:, 0] * pi)
...     return mean_blocked
>>>
>>> round(get_mean_number_of_individuals_in_node_2(pi, all_states), 10)
0.2685797749

\end{lstlisting}


Consequently, there are some additional performance measures of interest that
are more complex to calculate.
Such performance measures are the mean waiting time in the system (for both
type 1 and type 2 individuals), the mean time blocked in node 2 (only
valid for type 2 individuals) and the proportion of individuals that wait in
node 1 within a predefined time target (for both types).
Under the scope of this study three approaches have been considered to calculate
these performance measures; a recursive algorithm, a direct approach and
a closed-form equation. 
Furthermore, different formulas arise for type 1 individuals and different
ones for type 2 individuals.


\input{chapters/03_queueing_model/03_performance_measures/01_waiting_time/main.tex}

\input{chapters/03_queueing_model/03_performance_measures/02_blocking_time/main.tex}

\input{chapters/03_queueing_model/03_performance_measures/03_proportion_of_individuals_within_time/main.tex}

\section{Performance measures}\label{sec:queueing_performance_measures}
Using vector \(\pi\) there are numerous performance measures of the model that
can be calculated.
The following equations utilise \(\pi\) to get performance measures on the
average number of individuals in node 1 and in node 2:

\begin{itemize}
    \item Mean number of individuals in the entire system:
        \begin{equation}
            L_S = \sum_{i=1}^{|\pi|} \pi_i (u_i + v_i)
        \end{equation}
    \item Mean number of individuals in node 1:
        \begin{equation}
            L_1 = \sum_{i=1}^{|\pi|} \pi_i v_i
        \end{equation}
    \item Mean number of individuals in node 2:
        \begin{equation}
            L_2 = \sum_{i=1}^{|\pi|} \pi_i u_i
        \end{equation}
\end{itemize}

The python code for these functions may be obtained quite quickly using the set
of all states and the steady state probability.

\begin{lstlisting}[
    style=pystyle,
    caption={Code snippet for getting the set of all states and the steady
    state probabilities.},
    label={lst:set_of_all_states_and_steady_state_probabilities},
]
>>> import ambulance_game as abg
>>> import numpy as np
>>> all_states = abg.markov.build_states(
...     threshold=3 ,
...     system_capacity=4,
...     buffer_capacity=2
... )
>>> Q = abg.markov.get_transition_matrix(
...     lambda_1=1,
...     lambda_2=2,
...     mu=2,
...     num_of_servers=2,
...     threshold=3,
...     system_capacity=4,
...     buffer_capacity=2
... )
>>> pi = abg.markov.get_steady_state_algebraically(
...     Q, algebraic_function=np.linalg.lstsq
... )

\end{lstlisting}

Having the set of all states and the steady state probabilities, the mean
number of individuals in the system, in node 1 and in node 2 can be calculated
as shown in code snippets~\ref{lst:mean_number_of_individuals_in_system},
\ref{lst:mean_number_of_individuals_in_node_1}
and~\ref{lst:mean_number_of_individuals_in_node_2}.

\begin{lstlisting}[
    style=pystyle,
    caption={Code snippet for calculating the mean number of individuals in
    the system.},
    label={lst:mean_number_of_individuals_in_system},
]
>>> def get_mean_number_of_individuals_in_system(pi, states):
...     """Gets the mean number of individuals in the system
...     Parameters
...     ----------
...     pi : numpy.ndarray
...         steady state vector
...     states : list
...         list of tuples that contains all states
...     Returns
...     -------
...     float
...         Mean number of individuals in the whole model
...     """
...     states = np.array(states)
...     mean_inds_in_system = np.sum((states[:, 0] + states[:, 1]) * pi)
...     return mean_inds_in_system
>>>
>>> round(get_mean_number_of_individuals_in_system(pi, all_states), 10)
2.0872927227

\end{lstlisting}

\begin{lstlisting}[
    style=pystyle,
    caption={Code snippet for calculating the mean number of individuals in
    Node 1.},
    label={lst:mean_number_of_individuals_in_node_1},
]
>>> def get_mean_number_of_individuals_in_node_1(pi, states):
...     """Mean number of individuals in node 1
...     Parameters
...     ----------
...     pi : numpy.ndarray
...         steady state vector
...     states : list
...         list of tuples that contains all states
...     Returns
...     -------
...     float
...         Mean number of individuals
...     """
...     states = np.array(states)
...     mean_inds_in_node_1 = np.sum(states[:, 1] * pi)
...     return mean_inds_in_node_1
>>>
>>> round(get_mean_number_of_individuals_in_node_1(pi, all_states), 10)
1.8187129478

\end{lstlisting}

\begin{lstlisting}[
    style=pystyle,
    caption={Code snippet for calculating the mean number of individuals in
    Node 2.},
    label={lst:mean_number_of_individuals_in_node_2},
]
>>> def get_mean_number_of_individuals_in_node_2(pi, states):
...     """Mean number of class 2 individuals blocked
...     Parameters
...     ----------
...     pi : numpy.ndarray
...         steady state vector
...     states : list
...         list of tuples that contains all states
...     Returns
...     -------
...     float
...         Mean number of blocked class 2 individuals
...     """
...     states = np.array(states)
...     mean_blocked = np.sum(states[:, 0] * pi)
...     return mean_blocked
>>>
>>> round(get_mean_number_of_individuals_in_node_2(pi, all_states), 10)
0.2685797749

\end{lstlisting}


Consequently, there are some additional performance measures of interest that
are more complex to calculate.
Such performance measures are the mean waiting time in the system (for both
type 1 and type 2 individuals), the mean time blocked in node 2 (only
valid for type 2 individuals) and the proportion of individuals that wait in
node 1 within a predefined time target (for both types).
Under the scope of this study three approaches have been considered to calculate
these performance measures; a recursive algorithm, a direct approach and
a closed-form equation. 
Furthermore, different formulas arise for type 1 individuals and different
ones for type 2 individuals.


\input{chapters/03_queueing_model/03_performance_measures/01_waiting_time/main.tex}

\input{chapters/03_queueing_model/03_performance_measures/02_blocking_time/main.tex}

\input{chapters/03_queueing_model/03_performance_measures/03_proportion_of_individuals_within_time/main.tex}

\section{Performance measures}\label{sec:queueing_performance_measures}
Using vector \(\pi\) there are numerous performance measures of the model that
can be calculated.
The following equations utilise \(\pi\) to get performance measures on the
average number of individuals in node 1 and in node 2:

\begin{itemize}
    \item Mean number of individuals in the entire system:
        \begin{equation}
            L_S = \sum_{i=1}^{|\pi|} \pi_i (u_i + v_i)
        \end{equation}
    \item Mean number of individuals in node 1:
        \begin{equation}
            L_1 = \sum_{i=1}^{|\pi|} \pi_i v_i
        \end{equation}
    \item Mean number of individuals in node 2:
        \begin{equation}
            L_2 = \sum_{i=1}^{|\pi|} \pi_i u_i
        \end{equation}
\end{itemize}

The python code for these functions may be obtained quite quickly using the set
of all states and the steady state probability.

\begin{lstlisting}[
    style=pystyle,
    caption={Code snippet for getting the set of all states and the steady
    state probabilities.},
    label={lst:set_of_all_states_and_steady_state_probabilities},
]
>>> import ambulance_game as abg
>>> import numpy as np
>>> all_states = abg.markov.build_states(
...     threshold=3 ,
...     system_capacity=4,
...     buffer_capacity=2
... )
>>> Q = abg.markov.get_transition_matrix(
...     lambda_1=1,
...     lambda_2=2,
...     mu=2,
...     num_of_servers=2,
...     threshold=3,
...     system_capacity=4,
...     buffer_capacity=2
... )
>>> pi = abg.markov.get_steady_state_algebraically(
...     Q, algebraic_function=np.linalg.lstsq
... )

\end{lstlisting}

Having the set of all states and the steady state probabilities, the mean
number of individuals in the system, in node 1 and in node 2 can be calculated
as shown in code snippets~\ref{lst:mean_number_of_individuals_in_system},
\ref{lst:mean_number_of_individuals_in_node_1}
and~\ref{lst:mean_number_of_individuals_in_node_2}.

\begin{lstlisting}[
    style=pystyle,
    caption={Code snippet for calculating the mean number of individuals in
    the system.},
    label={lst:mean_number_of_individuals_in_system},
]
>>> def get_mean_number_of_individuals_in_system(pi, states):
...     """Gets the mean number of individuals in the system
...     Parameters
...     ----------
...     pi : numpy.ndarray
...         steady state vector
...     states : list
...         list of tuples that contains all states
...     Returns
...     -------
...     float
...         Mean number of individuals in the whole model
...     """
...     states = np.array(states)
...     mean_inds_in_system = np.sum((states[:, 0] + states[:, 1]) * pi)
...     return mean_inds_in_system
>>>
>>> round(get_mean_number_of_individuals_in_system(pi, all_states), 10)
2.0872927227

\end{lstlisting}

\begin{lstlisting}[
    style=pystyle,
    caption={Code snippet for calculating the mean number of individuals in
    Node 1.},
    label={lst:mean_number_of_individuals_in_node_1},
]
>>> def get_mean_number_of_individuals_in_node_1(pi, states):
...     """Mean number of individuals in node 1
...     Parameters
...     ----------
...     pi : numpy.ndarray
...         steady state vector
...     states : list
...         list of tuples that contains all states
...     Returns
...     -------
...     float
...         Mean number of individuals
...     """
...     states = np.array(states)
...     mean_inds_in_node_1 = np.sum(states[:, 1] * pi)
...     return mean_inds_in_node_1
>>>
>>> round(get_mean_number_of_individuals_in_node_1(pi, all_states), 10)
1.8187129478

\end{lstlisting}

\begin{lstlisting}[
    style=pystyle,
    caption={Code snippet for calculating the mean number of individuals in
    Node 2.},
    label={lst:mean_number_of_individuals_in_node_2},
]
>>> def get_mean_number_of_individuals_in_node_2(pi, states):
...     """Mean number of class 2 individuals blocked
...     Parameters
...     ----------
...     pi : numpy.ndarray
...         steady state vector
...     states : list
...         list of tuples that contains all states
...     Returns
...     -------
...     float
...         Mean number of blocked class 2 individuals
...     """
...     states = np.array(states)
...     mean_blocked = np.sum(states[:, 0] * pi)
...     return mean_blocked
>>>
>>> round(get_mean_number_of_individuals_in_node_2(pi, all_states), 10)
0.2685797749

\end{lstlisting}


Consequently, there are some additional performance measures of interest that
are more complex to calculate.
Such performance measures are the mean waiting time in the system (for both
type 1 and type 2 individuals), the mean time blocked in node 2 (only
valid for type 2 individuals) and the proportion of individuals that wait in
node 1 within a predefined time target (for both types).
Under the scope of this study three approaches have been considered to calculate
these performance measures; a recursive algorithm, a direct approach and
a closed-form equation. 
Furthermore, different formulas arise for type 1 individuals and different
ones for type 2 individuals.


\input{chapters/03_queueing_model/03_performance_measures/01_waiting_time/main.tex}

\input{chapters/03_queueing_model/03_performance_measures/02_blocking_time/main.tex}

\input{chapters/03_queueing_model/03_performance_measures/03_proportion_of_individuals_within_time/main.tex}

\section{Performance measures}\label{sec:queueing_performance_measures}
Using vector \(\pi\) there are numerous performance measures of the model that
can be calculated.
The following equations utilise \(\pi\) to get performance measures on the
average number of individuals in node 1 and in node 2:

\begin{itemize}
    \item Mean number of individuals in the entire system:
        \begin{equation}
            L_S = \sum_{i=1}^{|\pi|} \pi_i (u_i + v_i)
        \end{equation}
    \item Mean number of individuals in node 1:
        \begin{equation}
            L_1 = \sum_{i=1}^{|\pi|} \pi_i v_i
        \end{equation}
    \item Mean number of individuals in node 2:
        \begin{equation}
            L_2 = \sum_{i=1}^{|\pi|} \pi_i u_i
        \end{equation}
\end{itemize}

The python code for these functions may be obtained quite quickly using the set
of all states and the steady state probability.

\begin{lstlisting}[
    style=pystyle,
    caption={Code snippet for getting the set of all states and the steady
    state probabilities.},
    label={lst:set_of_all_states_and_steady_state_probabilities},
]
>>> import ambulance_game as abg
>>> import numpy as np
>>> all_states = abg.markov.build_states(
...     threshold=3 ,
...     system_capacity=4,
...     buffer_capacity=2
... )
>>> Q = abg.markov.get_transition_matrix(
...     lambda_1=1,
...     lambda_2=2,
...     mu=2,
...     num_of_servers=2,
...     threshold=3,
...     system_capacity=4,
...     buffer_capacity=2
... )
>>> pi = abg.markov.get_steady_state_algebraically(
...     Q, algebraic_function=np.linalg.lstsq
... )

\end{lstlisting}

Having the set of all states and the steady state probabilities, the mean
number of individuals in the system, in node 1 and in node 2 can be calculated
as shown in code snippets~\ref{lst:mean_number_of_individuals_in_system},
\ref{lst:mean_number_of_individuals_in_node_1}
and~\ref{lst:mean_number_of_individuals_in_node_2}.

\begin{lstlisting}[
    style=pystyle,
    caption={Code snippet for calculating the mean number of individuals in
    the system.},
    label={lst:mean_number_of_individuals_in_system},
]
>>> def get_mean_number_of_individuals_in_system(pi, states):
...     """Gets the mean number of individuals in the system
...     Parameters
...     ----------
...     pi : numpy.ndarray
...         steady state vector
...     states : list
...         list of tuples that contains all states
...     Returns
...     -------
...     float
...         Mean number of individuals in the whole model
...     """
...     states = np.array(states)
...     mean_inds_in_system = np.sum((states[:, 0] + states[:, 1]) * pi)
...     return mean_inds_in_system
>>>
>>> round(get_mean_number_of_individuals_in_system(pi, all_states), 10)
2.0872927227

\end{lstlisting}

\begin{lstlisting}[
    style=pystyle,
    caption={Code snippet for calculating the mean number of individuals in
    Node 1.},
    label={lst:mean_number_of_individuals_in_node_1},
]
>>> def get_mean_number_of_individuals_in_node_1(pi, states):
...     """Mean number of individuals in node 1
...     Parameters
...     ----------
...     pi : numpy.ndarray
...         steady state vector
...     states : list
...         list of tuples that contains all states
...     Returns
...     -------
...     float
...         Mean number of individuals
...     """
...     states = np.array(states)
...     mean_inds_in_node_1 = np.sum(states[:, 1] * pi)
...     return mean_inds_in_node_1
>>>
>>> round(get_mean_number_of_individuals_in_node_1(pi, all_states), 10)
1.8187129478

\end{lstlisting}

\begin{lstlisting}[
    style=pystyle,
    caption={Code snippet for calculating the mean number of individuals in
    Node 2.},
    label={lst:mean_number_of_individuals_in_node_2},
]
>>> def get_mean_number_of_individuals_in_node_2(pi, states):
...     """Mean number of class 2 individuals blocked
...     Parameters
...     ----------
...     pi : numpy.ndarray
...         steady state vector
...     states : list
...         list of tuples that contains all states
...     Returns
...     -------
...     float
...         Mean number of blocked class 2 individuals
...     """
...     states = np.array(states)
...     mean_blocked = np.sum(states[:, 0] * pi)
...     return mean_blocked
>>>
>>> round(get_mean_number_of_individuals_in_node_2(pi, all_states), 10)
0.2685797749

\end{lstlisting}


Consequently, there are some additional performance measures of interest that
are more complex to calculate.
Such performance measures are the mean waiting time in the system (for both
type 1 and type 2 individuals), the mean time blocked in node 2 (only
valid for type 2 individuals) and the proportion of individuals that wait in
node 1 within a predefined time target (for both types).
Under the scope of this study three approaches have been considered to calculate
these performance measures; a recursive algorithm, a direct approach and
a closed-form equation. 
Furthermore, different formulas arise for type 1 individuals and different
ones for type 2 individuals.


\section{Performance measures}\label{sec:queueing_performance_measures}
Using vector \(\pi\) there are numerous performance measures of the model that
can be calculated.
The following equations utilise \(\pi\) to get performance measures on the
average number of individuals in node 1 and in node 2:

\begin{itemize}
    \item Mean number of individuals in the entire system:
        \begin{equation}
            L_S = \sum_{i=1}^{|\pi|} \pi_i (u_i + v_i)
        \end{equation}
    \item Mean number of individuals in node 1:
        \begin{equation}
            L_1 = \sum_{i=1}^{|\pi|} \pi_i v_i
        \end{equation}
    \item Mean number of individuals in node 2:
        \begin{equation}
            L_2 = \sum_{i=1}^{|\pi|} \pi_i u_i
        \end{equation}
\end{itemize}

The python code for these functions may be obtained quite quickly using the set
of all states and the steady state probability.

\begin{lstlisting}[
    style=pystyle,
    caption={Code snippet for getting the set of all states and the steady
    state probabilities.},
    label={lst:set_of_all_states_and_steady_state_probabilities},
]
>>> import ambulance_game as abg
>>> import numpy as np
>>> all_states = abg.markov.build_states(
...     threshold=3 ,
...     system_capacity=4,
...     buffer_capacity=2
... )
>>> Q = abg.markov.get_transition_matrix(
...     lambda_1=1,
...     lambda_2=2,
...     mu=2,
...     num_of_servers=2,
...     threshold=3,
...     system_capacity=4,
...     buffer_capacity=2
... )
>>> pi = abg.markov.get_steady_state_algebraically(
...     Q, algebraic_function=np.linalg.lstsq
... )

\end{lstlisting}

Having the set of all states and the steady state probabilities, the mean
number of individuals in the system, in node 1 and in node 2 can be calculated
as shown in code snippets~\ref{lst:mean_number_of_individuals_in_system},
\ref{lst:mean_number_of_individuals_in_node_1}
and~\ref{lst:mean_number_of_individuals_in_node_2}.

\begin{lstlisting}[
    style=pystyle,
    caption={Code snippet for calculating the mean number of individuals in
    the system.},
    label={lst:mean_number_of_individuals_in_system},
]
>>> def get_mean_number_of_individuals_in_system(pi, states):
...     """Gets the mean number of individuals in the system
...     Parameters
...     ----------
...     pi : numpy.ndarray
...         steady state vector
...     states : list
...         list of tuples that contains all states
...     Returns
...     -------
...     float
...         Mean number of individuals in the whole model
...     """
...     states = np.array(states)
...     mean_inds_in_system = np.sum((states[:, 0] + states[:, 1]) * pi)
...     return mean_inds_in_system
>>>
>>> round(get_mean_number_of_individuals_in_system(pi, all_states), 10)
2.0872927227

\end{lstlisting}

\begin{lstlisting}[
    style=pystyle,
    caption={Code snippet for calculating the mean number of individuals in
    Node 1.},
    label={lst:mean_number_of_individuals_in_node_1},
]
>>> def get_mean_number_of_individuals_in_node_1(pi, states):
...     """Mean number of individuals in node 1
...     Parameters
...     ----------
...     pi : numpy.ndarray
...         steady state vector
...     states : list
...         list of tuples that contains all states
...     Returns
...     -------
...     float
...         Mean number of individuals
...     """
...     states = np.array(states)
...     mean_inds_in_node_1 = np.sum(states[:, 1] * pi)
...     return mean_inds_in_node_1
>>>
>>> round(get_mean_number_of_individuals_in_node_1(pi, all_states), 10)
1.8187129478

\end{lstlisting}

\begin{lstlisting}[
    style=pystyle,
    caption={Code snippet for calculating the mean number of individuals in
    Node 2.},
    label={lst:mean_number_of_individuals_in_node_2},
]
>>> def get_mean_number_of_individuals_in_node_2(pi, states):
...     """Mean number of class 2 individuals blocked
...     Parameters
...     ----------
...     pi : numpy.ndarray
...         steady state vector
...     states : list
...         list of tuples that contains all states
...     Returns
...     -------
...     float
...         Mean number of blocked class 2 individuals
...     """
...     states = np.array(states)
...     mean_blocked = np.sum(states[:, 0] * pi)
...     return mean_blocked
>>>
>>> round(get_mean_number_of_individuals_in_node_2(pi, all_states), 10)
0.2685797749

\end{lstlisting}


Consequently, there are some additional performance measures of interest that
are more complex to calculate.
Such performance measures are the mean waiting time in the system (for both
type 1 and type 2 individuals), the mean time blocked in node 2 (only
valid for type 2 individuals) and the proportion of individuals that wait in
node 1 within a predefined time target (for both types).
Under the scope of this study three approaches have been considered to calculate
these performance measures; a recursive algorithm, a direct approach and
a closed-form equation. 
Furthermore, different formulas arise for type 1 individuals and different
ones for type 2 individuals.


\section{Performance measures}\label{sec:queueing_performance_measures}
Using vector \(\pi\) there are numerous performance measures of the model that
can be calculated.
The following equations utilise \(\pi\) to get performance measures on the
average number of individuals in node 1 and in node 2:

\begin{itemize}
    \item Mean number of individuals in the entire system:
        \begin{equation}
            L_S = \sum_{i=1}^{|\pi|} \pi_i (u_i + v_i)
        \end{equation}
    \item Mean number of individuals in node 1:
        \begin{equation}
            L_1 = \sum_{i=1}^{|\pi|} \pi_i v_i
        \end{equation}
    \item Mean number of individuals in node 2:
        \begin{equation}
            L_2 = \sum_{i=1}^{|\pi|} \pi_i u_i
        \end{equation}
\end{itemize}

The python code for these functions may be obtained quite quickly using the set
of all states and the steady state probability.

\begin{lstlisting}[
    style=pystyle,
    caption={Code snippet for getting the set of all states and the steady
    state probabilities.},
    label={lst:set_of_all_states_and_steady_state_probabilities},
]
>>> import ambulance_game as abg
>>> import numpy as np
>>> all_states = abg.markov.build_states(
...     threshold=3 ,
...     system_capacity=4,
...     buffer_capacity=2
... )
>>> Q = abg.markov.get_transition_matrix(
...     lambda_1=1,
...     lambda_2=2,
...     mu=2,
...     num_of_servers=2,
...     threshold=3,
...     system_capacity=4,
...     buffer_capacity=2
... )
>>> pi = abg.markov.get_steady_state_algebraically(
...     Q, algebraic_function=np.linalg.lstsq
... )

\end{lstlisting}

Having the set of all states and the steady state probabilities, the mean
number of individuals in the system, in node 1 and in node 2 can be calculated
as shown in code snippets~\ref{lst:mean_number_of_individuals_in_system},
\ref{lst:mean_number_of_individuals_in_node_1}
and~\ref{lst:mean_number_of_individuals_in_node_2}.

\begin{lstlisting}[
    style=pystyle,
    caption={Code snippet for calculating the mean number of individuals in
    the system.},
    label={lst:mean_number_of_individuals_in_system},
]
>>> def get_mean_number_of_individuals_in_system(pi, states):
...     """Gets the mean number of individuals in the system
...     Parameters
...     ----------
...     pi : numpy.ndarray
...         steady state vector
...     states : list
...         list of tuples that contains all states
...     Returns
...     -------
...     float
...         Mean number of individuals in the whole model
...     """
...     states = np.array(states)
...     mean_inds_in_system = np.sum((states[:, 0] + states[:, 1]) * pi)
...     return mean_inds_in_system
>>>
>>> round(get_mean_number_of_individuals_in_system(pi, all_states), 10)
2.0872927227

\end{lstlisting}

\begin{lstlisting}[
    style=pystyle,
    caption={Code snippet for calculating the mean number of individuals in
    Node 1.},
    label={lst:mean_number_of_individuals_in_node_1},
]
>>> def get_mean_number_of_individuals_in_node_1(pi, states):
...     """Mean number of individuals in node 1
...     Parameters
...     ----------
...     pi : numpy.ndarray
...         steady state vector
...     states : list
...         list of tuples that contains all states
...     Returns
...     -------
...     float
...         Mean number of individuals
...     """
...     states = np.array(states)
...     mean_inds_in_node_1 = np.sum(states[:, 1] * pi)
...     return mean_inds_in_node_1
>>>
>>> round(get_mean_number_of_individuals_in_node_1(pi, all_states), 10)
1.8187129478

\end{lstlisting}

\begin{lstlisting}[
    style=pystyle,
    caption={Code snippet for calculating the mean number of individuals in
    Node 2.},
    label={lst:mean_number_of_individuals_in_node_2},
]
>>> def get_mean_number_of_individuals_in_node_2(pi, states):
...     """Mean number of class 2 individuals blocked
...     Parameters
...     ----------
...     pi : numpy.ndarray
...         steady state vector
...     states : list
...         list of tuples that contains all states
...     Returns
...     -------
...     float
...         Mean number of blocked class 2 individuals
...     """
...     states = np.array(states)
...     mean_blocked = np.sum(states[:, 0] * pi)
...     return mean_blocked
>>>
>>> round(get_mean_number_of_individuals_in_node_2(pi, all_states), 10)
0.2685797749

\end{lstlisting}


Consequently, there are some additional performance measures of interest that
are more complex to calculate.
Such performance measures are the mean waiting time in the system (for both
type 1 and type 2 individuals), the mean time blocked in node 2 (only
valid for type 2 individuals) and the proportion of individuals that wait in
node 1 within a predefined time target (for both types).
Under the scope of this study three approaches have been considered to calculate
these performance measures; a recursive algorithm, a direct approach and
a closed-form equation. 
Furthermore, different formulas arise for type 1 individuals and different
ones for type 2 individuals.


\input{chapters/03_queueing_model/03_performance_measures/01_waiting_time/main.tex}

\input{chapters/03_queueing_model/03_performance_measures/02_blocking_time/main.tex}

\input{chapters/03_queueing_model/03_performance_measures/03_proportion_of_individuals_within_time/main.tex}

\section{Performance measures}\label{sec:queueing_performance_measures}
Using vector \(\pi\) there are numerous performance measures of the model that
can be calculated.
The following equations utilise \(\pi\) to get performance measures on the
average number of individuals in node 1 and in node 2:

\begin{itemize}
    \item Mean number of individuals in the entire system:
        \begin{equation}
            L_S = \sum_{i=1}^{|\pi|} \pi_i (u_i + v_i)
        \end{equation}
    \item Mean number of individuals in node 1:
        \begin{equation}
            L_1 = \sum_{i=1}^{|\pi|} \pi_i v_i
        \end{equation}
    \item Mean number of individuals in node 2:
        \begin{equation}
            L_2 = \sum_{i=1}^{|\pi|} \pi_i u_i
        \end{equation}
\end{itemize}

The python code for these functions may be obtained quite quickly using the set
of all states and the steady state probability.

\begin{lstlisting}[
    style=pystyle,
    caption={Code snippet for getting the set of all states and the steady
    state probabilities.},
    label={lst:set_of_all_states_and_steady_state_probabilities},
]
>>> import ambulance_game as abg
>>> import numpy as np
>>> all_states = abg.markov.build_states(
...     threshold=3 ,
...     system_capacity=4,
...     buffer_capacity=2
... )
>>> Q = abg.markov.get_transition_matrix(
...     lambda_1=1,
...     lambda_2=2,
...     mu=2,
...     num_of_servers=2,
...     threshold=3,
...     system_capacity=4,
...     buffer_capacity=2
... )
>>> pi = abg.markov.get_steady_state_algebraically(
...     Q, algebraic_function=np.linalg.lstsq
... )

\end{lstlisting}

Having the set of all states and the steady state probabilities, the mean
number of individuals in the system, in node 1 and in node 2 can be calculated
as shown in code snippets~\ref{lst:mean_number_of_individuals_in_system},
\ref{lst:mean_number_of_individuals_in_node_1}
and~\ref{lst:mean_number_of_individuals_in_node_2}.

\begin{lstlisting}[
    style=pystyle,
    caption={Code snippet for calculating the mean number of individuals in
    the system.},
    label={lst:mean_number_of_individuals_in_system},
]
>>> def get_mean_number_of_individuals_in_system(pi, states):
...     """Gets the mean number of individuals in the system
...     Parameters
...     ----------
...     pi : numpy.ndarray
...         steady state vector
...     states : list
...         list of tuples that contains all states
...     Returns
...     -------
...     float
...         Mean number of individuals in the whole model
...     """
...     states = np.array(states)
...     mean_inds_in_system = np.sum((states[:, 0] + states[:, 1]) * pi)
...     return mean_inds_in_system
>>>
>>> round(get_mean_number_of_individuals_in_system(pi, all_states), 10)
2.0872927227

\end{lstlisting}

\begin{lstlisting}[
    style=pystyle,
    caption={Code snippet for calculating the mean number of individuals in
    Node 1.},
    label={lst:mean_number_of_individuals_in_node_1},
]
>>> def get_mean_number_of_individuals_in_node_1(pi, states):
...     """Mean number of individuals in node 1
...     Parameters
...     ----------
...     pi : numpy.ndarray
...         steady state vector
...     states : list
...         list of tuples that contains all states
...     Returns
...     -------
...     float
...         Mean number of individuals
...     """
...     states = np.array(states)
...     mean_inds_in_node_1 = np.sum(states[:, 1] * pi)
...     return mean_inds_in_node_1
>>>
>>> round(get_mean_number_of_individuals_in_node_1(pi, all_states), 10)
1.8187129478

\end{lstlisting}

\begin{lstlisting}[
    style=pystyle,
    caption={Code snippet for calculating the mean number of individuals in
    Node 2.},
    label={lst:mean_number_of_individuals_in_node_2},
]
>>> def get_mean_number_of_individuals_in_node_2(pi, states):
...     """Mean number of class 2 individuals blocked
...     Parameters
...     ----------
...     pi : numpy.ndarray
...         steady state vector
...     states : list
...         list of tuples that contains all states
...     Returns
...     -------
...     float
...         Mean number of blocked class 2 individuals
...     """
...     states = np.array(states)
...     mean_blocked = np.sum(states[:, 0] * pi)
...     return mean_blocked
>>>
>>> round(get_mean_number_of_individuals_in_node_2(pi, all_states), 10)
0.2685797749

\end{lstlisting}


Consequently, there are some additional performance measures of interest that
are more complex to calculate.
Such performance measures are the mean waiting time in the system (for both
type 1 and type 2 individuals), the mean time blocked in node 2 (only
valid for type 2 individuals) and the proportion of individuals that wait in
node 1 within a predefined time target (for both types).
Under the scope of this study three approaches have been considered to calculate
these performance measures; a recursive algorithm, a direct approach and
a closed-form equation. 
Furthermore, different formulas arise for type 1 individuals and different
ones for type 2 individuals.


\input{chapters/03_queueing_model/03_performance_measures/01_waiting_time/main.tex}

\input{chapters/03_queueing_model/03_performance_measures/02_blocking_time/main.tex}

\input{chapters/03_queueing_model/03_performance_measures/03_proportion_of_individuals_within_time/main.tex}

\section{Performance measures}\label{sec:queueing_performance_measures}
Using vector \(\pi\) there are numerous performance measures of the model that
can be calculated.
The following equations utilise \(\pi\) to get performance measures on the
average number of individuals in node 1 and in node 2:

\begin{itemize}
    \item Mean number of individuals in the entire system:
        \begin{equation}
            L_S = \sum_{i=1}^{|\pi|} \pi_i (u_i + v_i)
        \end{equation}
    \item Mean number of individuals in node 1:
        \begin{equation}
            L_1 = \sum_{i=1}^{|\pi|} \pi_i v_i
        \end{equation}
    \item Mean number of individuals in node 2:
        \begin{equation}
            L_2 = \sum_{i=1}^{|\pi|} \pi_i u_i
        \end{equation}
\end{itemize}

The python code for these functions may be obtained quite quickly using the set
of all states and the steady state probability.

\begin{lstlisting}[
    style=pystyle,
    caption={Code snippet for getting the set of all states and the steady
    state probabilities.},
    label={lst:set_of_all_states_and_steady_state_probabilities},
]
>>> import ambulance_game as abg
>>> import numpy as np
>>> all_states = abg.markov.build_states(
...     threshold=3 ,
...     system_capacity=4,
...     buffer_capacity=2
... )
>>> Q = abg.markov.get_transition_matrix(
...     lambda_1=1,
...     lambda_2=2,
...     mu=2,
...     num_of_servers=2,
...     threshold=3,
...     system_capacity=4,
...     buffer_capacity=2
... )
>>> pi = abg.markov.get_steady_state_algebraically(
...     Q, algebraic_function=np.linalg.lstsq
... )

\end{lstlisting}

Having the set of all states and the steady state probabilities, the mean
number of individuals in the system, in node 1 and in node 2 can be calculated
as shown in code snippets~\ref{lst:mean_number_of_individuals_in_system},
\ref{lst:mean_number_of_individuals_in_node_1}
and~\ref{lst:mean_number_of_individuals_in_node_2}.

\begin{lstlisting}[
    style=pystyle,
    caption={Code snippet for calculating the mean number of individuals in
    the system.},
    label={lst:mean_number_of_individuals_in_system},
]
>>> def get_mean_number_of_individuals_in_system(pi, states):
...     """Gets the mean number of individuals in the system
...     Parameters
...     ----------
...     pi : numpy.ndarray
...         steady state vector
...     states : list
...         list of tuples that contains all states
...     Returns
...     -------
...     float
...         Mean number of individuals in the whole model
...     """
...     states = np.array(states)
...     mean_inds_in_system = np.sum((states[:, 0] + states[:, 1]) * pi)
...     return mean_inds_in_system
>>>
>>> round(get_mean_number_of_individuals_in_system(pi, all_states), 10)
2.0872927227

\end{lstlisting}

\begin{lstlisting}[
    style=pystyle,
    caption={Code snippet for calculating the mean number of individuals in
    Node 1.},
    label={lst:mean_number_of_individuals_in_node_1},
]
>>> def get_mean_number_of_individuals_in_node_1(pi, states):
...     """Mean number of individuals in node 1
...     Parameters
...     ----------
...     pi : numpy.ndarray
...         steady state vector
...     states : list
...         list of tuples that contains all states
...     Returns
...     -------
...     float
...         Mean number of individuals
...     """
...     states = np.array(states)
...     mean_inds_in_node_1 = np.sum(states[:, 1] * pi)
...     return mean_inds_in_node_1
>>>
>>> round(get_mean_number_of_individuals_in_node_1(pi, all_states), 10)
1.8187129478

\end{lstlisting}

\begin{lstlisting}[
    style=pystyle,
    caption={Code snippet for calculating the mean number of individuals in
    Node 2.},
    label={lst:mean_number_of_individuals_in_node_2},
]
>>> def get_mean_number_of_individuals_in_node_2(pi, states):
...     """Mean number of class 2 individuals blocked
...     Parameters
...     ----------
...     pi : numpy.ndarray
...         steady state vector
...     states : list
...         list of tuples that contains all states
...     Returns
...     -------
...     float
...         Mean number of blocked class 2 individuals
...     """
...     states = np.array(states)
...     mean_blocked = np.sum(states[:, 0] * pi)
...     return mean_blocked
>>>
>>> round(get_mean_number_of_individuals_in_node_2(pi, all_states), 10)
0.2685797749

\end{lstlisting}


Consequently, there are some additional performance measures of interest that
are more complex to calculate.
Such performance measures are the mean waiting time in the system (for both
type 1 and type 2 individuals), the mean time blocked in node 2 (only
valid for type 2 individuals) and the proportion of individuals that wait in
node 1 within a predefined time target (for both types).
Under the scope of this study three approaches have been considered to calculate
these performance measures; a recursive algorithm, a direct approach and
a closed-form equation. 
Furthermore, different formulas arise for type 1 individuals and different
ones for type 2 individuals.


\input{chapters/03_queueing_model/03_performance_measures/01_waiting_time/main.tex}

\input{chapters/03_queueing_model/03_performance_measures/02_blocking_time/main.tex}

\input{chapters/03_queueing_model/03_performance_measures/03_proportion_of_individuals_within_time/main.tex}

\section{Performance measures}\label{sec:queueing_performance_measures}
Using vector \(\pi\) there are numerous performance measures of the model that
can be calculated.
The following equations utilise \(\pi\) to get performance measures on the
average number of individuals in node 1 and in node 2:

\begin{itemize}
    \item Mean number of individuals in the entire system:
        \begin{equation}
            L_S = \sum_{i=1}^{|\pi|} \pi_i (u_i + v_i)
        \end{equation}
    \item Mean number of individuals in node 1:
        \begin{equation}
            L_1 = \sum_{i=1}^{|\pi|} \pi_i v_i
        \end{equation}
    \item Mean number of individuals in node 2:
        \begin{equation}
            L_2 = \sum_{i=1}^{|\pi|} \pi_i u_i
        \end{equation}
\end{itemize}

The python code for these functions may be obtained quite quickly using the set
of all states and the steady state probability.

\begin{lstlisting}[
    style=pystyle,
    caption={Code snippet for getting the set of all states and the steady
    state probabilities.},
    label={lst:set_of_all_states_and_steady_state_probabilities},
]
>>> import ambulance_game as abg
>>> import numpy as np
>>> all_states = abg.markov.build_states(
...     threshold=3 ,
...     system_capacity=4,
...     buffer_capacity=2
... )
>>> Q = abg.markov.get_transition_matrix(
...     lambda_1=1,
...     lambda_2=2,
...     mu=2,
...     num_of_servers=2,
...     threshold=3,
...     system_capacity=4,
...     buffer_capacity=2
... )
>>> pi = abg.markov.get_steady_state_algebraically(
...     Q, algebraic_function=np.linalg.lstsq
... )

\end{lstlisting}

Having the set of all states and the steady state probabilities, the mean
number of individuals in the system, in node 1 and in node 2 can be calculated
as shown in code snippets~\ref{lst:mean_number_of_individuals_in_system},
\ref{lst:mean_number_of_individuals_in_node_1}
and~\ref{lst:mean_number_of_individuals_in_node_2}.

\begin{lstlisting}[
    style=pystyle,
    caption={Code snippet for calculating the mean number of individuals in
    the system.},
    label={lst:mean_number_of_individuals_in_system},
]
>>> def get_mean_number_of_individuals_in_system(pi, states):
...     """Gets the mean number of individuals in the system
...     Parameters
...     ----------
...     pi : numpy.ndarray
...         steady state vector
...     states : list
...         list of tuples that contains all states
...     Returns
...     -------
...     float
...         Mean number of individuals in the whole model
...     """
...     states = np.array(states)
...     mean_inds_in_system = np.sum((states[:, 0] + states[:, 1]) * pi)
...     return mean_inds_in_system
>>>
>>> round(get_mean_number_of_individuals_in_system(pi, all_states), 10)
2.0872927227

\end{lstlisting}

\begin{lstlisting}[
    style=pystyle,
    caption={Code snippet for calculating the mean number of individuals in
    Node 1.},
    label={lst:mean_number_of_individuals_in_node_1},
]
>>> def get_mean_number_of_individuals_in_node_1(pi, states):
...     """Mean number of individuals in node 1
...     Parameters
...     ----------
...     pi : numpy.ndarray
...         steady state vector
...     states : list
...         list of tuples that contains all states
...     Returns
...     -------
...     float
...         Mean number of individuals
...     """
...     states = np.array(states)
...     mean_inds_in_node_1 = np.sum(states[:, 1] * pi)
...     return mean_inds_in_node_1
>>>
>>> round(get_mean_number_of_individuals_in_node_1(pi, all_states), 10)
1.8187129478

\end{lstlisting}

\begin{lstlisting}[
    style=pystyle,
    caption={Code snippet for calculating the mean number of individuals in
    Node 2.},
    label={lst:mean_number_of_individuals_in_node_2},
]
>>> def get_mean_number_of_individuals_in_node_2(pi, states):
...     """Mean number of class 2 individuals blocked
...     Parameters
...     ----------
...     pi : numpy.ndarray
...         steady state vector
...     states : list
...         list of tuples that contains all states
...     Returns
...     -------
...     float
...         Mean number of blocked class 2 individuals
...     """
...     states = np.array(states)
...     mean_blocked = np.sum(states[:, 0] * pi)
...     return mean_blocked
>>>
>>> round(get_mean_number_of_individuals_in_node_2(pi, all_states), 10)
0.2685797749

\end{lstlisting}


Consequently, there are some additional performance measures of interest that
are more complex to calculate.
Such performance measures are the mean waiting time in the system (for both
type 1 and type 2 individuals), the mean time blocked in node 2 (only
valid for type 2 individuals) and the proportion of individuals that wait in
node 1 within a predefined time target (for both types).
Under the scope of this study three approaches have been considered to calculate
these performance measures; a recursive algorithm, a direct approach and
a closed-form equation. 
Furthermore, different formulas arise for type 1 individuals and different
ones for type 2 individuals.


\section{Performance measures}\label{sec:queueing_performance_measures}
Using vector \(\pi\) there are numerous performance measures of the model that
can be calculated.
The following equations utilise \(\pi\) to get performance measures on the
average number of individuals in node 1 and in node 2:

\begin{itemize}
    \item Mean number of individuals in the entire system:
        \begin{equation}
            L_S = \sum_{i=1}^{|\pi|} \pi_i (u_i + v_i)
        \end{equation}
    \item Mean number of individuals in node 1:
        \begin{equation}
            L_1 = \sum_{i=1}^{|\pi|} \pi_i v_i
        \end{equation}
    \item Mean number of individuals in node 2:
        \begin{equation}
            L_2 = \sum_{i=1}^{|\pi|} \pi_i u_i
        \end{equation}
\end{itemize}

The python code for these functions may be obtained quite quickly using the set
of all states and the steady state probability.

\begin{lstlisting}[
    style=pystyle,
    caption={Code snippet for getting the set of all states and the steady
    state probabilities.},
    label={lst:set_of_all_states_and_steady_state_probabilities},
]
>>> import ambulance_game as abg
>>> import numpy as np
>>> all_states = abg.markov.build_states(
...     threshold=3 ,
...     system_capacity=4,
...     buffer_capacity=2
... )
>>> Q = abg.markov.get_transition_matrix(
...     lambda_1=1,
...     lambda_2=2,
...     mu=2,
...     num_of_servers=2,
...     threshold=3,
...     system_capacity=4,
...     buffer_capacity=2
... )
>>> pi = abg.markov.get_steady_state_algebraically(
...     Q, algebraic_function=np.linalg.lstsq
... )

\end{lstlisting}

Having the set of all states and the steady state probabilities, the mean
number of individuals in the system, in node 1 and in node 2 can be calculated
as shown in code snippets~\ref{lst:mean_number_of_individuals_in_system},
\ref{lst:mean_number_of_individuals_in_node_1}
and~\ref{lst:mean_number_of_individuals_in_node_2}.

\begin{lstlisting}[
    style=pystyle,
    caption={Code snippet for calculating the mean number of individuals in
    the system.},
    label={lst:mean_number_of_individuals_in_system},
]
>>> def get_mean_number_of_individuals_in_system(pi, states):
...     """Gets the mean number of individuals in the system
...     Parameters
...     ----------
...     pi : numpy.ndarray
...         steady state vector
...     states : list
...         list of tuples that contains all states
...     Returns
...     -------
...     float
...         Mean number of individuals in the whole model
...     """
...     states = np.array(states)
...     mean_inds_in_system = np.sum((states[:, 0] + states[:, 1]) * pi)
...     return mean_inds_in_system
>>>
>>> round(get_mean_number_of_individuals_in_system(pi, all_states), 10)
2.0872927227

\end{lstlisting}

\begin{lstlisting}[
    style=pystyle,
    caption={Code snippet for calculating the mean number of individuals in
    Node 1.},
    label={lst:mean_number_of_individuals_in_node_1},
]
>>> def get_mean_number_of_individuals_in_node_1(pi, states):
...     """Mean number of individuals in node 1
...     Parameters
...     ----------
...     pi : numpy.ndarray
...         steady state vector
...     states : list
...         list of tuples that contains all states
...     Returns
...     -------
...     float
...         Mean number of individuals
...     """
...     states = np.array(states)
...     mean_inds_in_node_1 = np.sum(states[:, 1] * pi)
...     return mean_inds_in_node_1
>>>
>>> round(get_mean_number_of_individuals_in_node_1(pi, all_states), 10)
1.8187129478

\end{lstlisting}

\begin{lstlisting}[
    style=pystyle,
    caption={Code snippet for calculating the mean number of individuals in
    Node 2.},
    label={lst:mean_number_of_individuals_in_node_2},
]
>>> def get_mean_number_of_individuals_in_node_2(pi, states):
...     """Mean number of class 2 individuals blocked
...     Parameters
...     ----------
...     pi : numpy.ndarray
...         steady state vector
...     states : list
...         list of tuples that contains all states
...     Returns
...     -------
...     float
...         Mean number of blocked class 2 individuals
...     """
...     states = np.array(states)
...     mean_blocked = np.sum(states[:, 0] * pi)
...     return mean_blocked
>>>
>>> round(get_mean_number_of_individuals_in_node_2(pi, all_states), 10)
0.2685797749

\end{lstlisting}


Consequently, there are some additional performance measures of interest that
are more complex to calculate.
Such performance measures are the mean waiting time in the system (for both
type 1 and type 2 individuals), the mean time blocked in node 2 (only
valid for type 2 individuals) and the proportion of individuals that wait in
node 1 within a predefined time target (for both types).
Under the scope of this study three approaches have been considered to calculate
these performance measures; a recursive algorithm, a direct approach and
a closed-form equation. 
Furthermore, different formulas arise for type 1 individuals and different
ones for type 2 individuals.


\input{chapters/03_queueing_model/03_performance_measures/01_waiting_time/main.tex}

\input{chapters/03_queueing_model/03_performance_measures/02_blocking_time/main.tex}

\input{chapters/03_queueing_model/03_performance_measures/03_proportion_of_individuals_within_time/main.tex}

\section{Performance measures}\label{sec:queueing_performance_measures}
Using vector \(\pi\) there are numerous performance measures of the model that
can be calculated.
The following equations utilise \(\pi\) to get performance measures on the
average number of individuals in node 1 and in node 2:

\begin{itemize}
    \item Mean number of individuals in the entire system:
        \begin{equation}
            L_S = \sum_{i=1}^{|\pi|} \pi_i (u_i + v_i)
        \end{equation}
    \item Mean number of individuals in node 1:
        \begin{equation}
            L_1 = \sum_{i=1}^{|\pi|} \pi_i v_i
        \end{equation}
    \item Mean number of individuals in node 2:
        \begin{equation}
            L_2 = \sum_{i=1}^{|\pi|} \pi_i u_i
        \end{equation}
\end{itemize}

The python code for these functions may be obtained quite quickly using the set
of all states and the steady state probability.

\begin{lstlisting}[
    style=pystyle,
    caption={Code snippet for getting the set of all states and the steady
    state probabilities.},
    label={lst:set_of_all_states_and_steady_state_probabilities},
]
>>> import ambulance_game as abg
>>> import numpy as np
>>> all_states = abg.markov.build_states(
...     threshold=3 ,
...     system_capacity=4,
...     buffer_capacity=2
... )
>>> Q = abg.markov.get_transition_matrix(
...     lambda_1=1,
...     lambda_2=2,
...     mu=2,
...     num_of_servers=2,
...     threshold=3,
...     system_capacity=4,
...     buffer_capacity=2
... )
>>> pi = abg.markov.get_steady_state_algebraically(
...     Q, algebraic_function=np.linalg.lstsq
... )

\end{lstlisting}

Having the set of all states and the steady state probabilities, the mean
number of individuals in the system, in node 1 and in node 2 can be calculated
as shown in code snippets~\ref{lst:mean_number_of_individuals_in_system},
\ref{lst:mean_number_of_individuals_in_node_1}
and~\ref{lst:mean_number_of_individuals_in_node_2}.

\begin{lstlisting}[
    style=pystyle,
    caption={Code snippet for calculating the mean number of individuals in
    the system.},
    label={lst:mean_number_of_individuals_in_system},
]
>>> def get_mean_number_of_individuals_in_system(pi, states):
...     """Gets the mean number of individuals in the system
...     Parameters
...     ----------
...     pi : numpy.ndarray
...         steady state vector
...     states : list
...         list of tuples that contains all states
...     Returns
...     -------
...     float
...         Mean number of individuals in the whole model
...     """
...     states = np.array(states)
...     mean_inds_in_system = np.sum((states[:, 0] + states[:, 1]) * pi)
...     return mean_inds_in_system
>>>
>>> round(get_mean_number_of_individuals_in_system(pi, all_states), 10)
2.0872927227

\end{lstlisting}

\begin{lstlisting}[
    style=pystyle,
    caption={Code snippet for calculating the mean number of individuals in
    Node 1.},
    label={lst:mean_number_of_individuals_in_node_1},
]
>>> def get_mean_number_of_individuals_in_node_1(pi, states):
...     """Mean number of individuals in node 1
...     Parameters
...     ----------
...     pi : numpy.ndarray
...         steady state vector
...     states : list
...         list of tuples that contains all states
...     Returns
...     -------
...     float
...         Mean number of individuals
...     """
...     states = np.array(states)
...     mean_inds_in_node_1 = np.sum(states[:, 1] * pi)
...     return mean_inds_in_node_1
>>>
>>> round(get_mean_number_of_individuals_in_node_1(pi, all_states), 10)
1.8187129478

\end{lstlisting}

\begin{lstlisting}[
    style=pystyle,
    caption={Code snippet for calculating the mean number of individuals in
    Node 2.},
    label={lst:mean_number_of_individuals_in_node_2},
]
>>> def get_mean_number_of_individuals_in_node_2(pi, states):
...     """Mean number of class 2 individuals blocked
...     Parameters
...     ----------
...     pi : numpy.ndarray
...         steady state vector
...     states : list
...         list of tuples that contains all states
...     Returns
...     -------
...     float
...         Mean number of blocked class 2 individuals
...     """
...     states = np.array(states)
...     mean_blocked = np.sum(states[:, 0] * pi)
...     return mean_blocked
>>>
>>> round(get_mean_number_of_individuals_in_node_2(pi, all_states), 10)
0.2685797749

\end{lstlisting}


Consequently, there are some additional performance measures of interest that
are more complex to calculate.
Such performance measures are the mean waiting time in the system (for both
type 1 and type 2 individuals), the mean time blocked in node 2 (only
valid for type 2 individuals) and the proportion of individuals that wait in
node 1 within a predefined time target (for both types).
Under the scope of this study three approaches have been considered to calculate
these performance measures; a recursive algorithm, a direct approach and
a closed-form equation. 
Furthermore, different formulas arise for type 1 individuals and different
ones for type 2 individuals.


\input{chapters/03_queueing_model/03_performance_measures/01_waiting_time/main.tex}

\input{chapters/03_queueing_model/03_performance_measures/02_blocking_time/main.tex}

\input{chapters/03_queueing_model/03_performance_measures/03_proportion_of_individuals_within_time/main.tex}

\section{Performance measures}\label{sec:queueing_performance_measures}
Using vector \(\pi\) there are numerous performance measures of the model that
can be calculated.
The following equations utilise \(\pi\) to get performance measures on the
average number of individuals in node 1 and in node 2:

\begin{itemize}
    \item Mean number of individuals in the entire system:
        \begin{equation}
            L_S = \sum_{i=1}^{|\pi|} \pi_i (u_i + v_i)
        \end{equation}
    \item Mean number of individuals in node 1:
        \begin{equation}
            L_1 = \sum_{i=1}^{|\pi|} \pi_i v_i
        \end{equation}
    \item Mean number of individuals in node 2:
        \begin{equation}
            L_2 = \sum_{i=1}^{|\pi|} \pi_i u_i
        \end{equation}
\end{itemize}

The python code for these functions may be obtained quite quickly using the set
of all states and the steady state probability.

\begin{lstlisting}[
    style=pystyle,
    caption={Code snippet for getting the set of all states and the steady
    state probabilities.},
    label={lst:set_of_all_states_and_steady_state_probabilities},
]
>>> import ambulance_game as abg
>>> import numpy as np
>>> all_states = abg.markov.build_states(
...     threshold=3 ,
...     system_capacity=4,
...     buffer_capacity=2
... )
>>> Q = abg.markov.get_transition_matrix(
...     lambda_1=1,
...     lambda_2=2,
...     mu=2,
...     num_of_servers=2,
...     threshold=3,
...     system_capacity=4,
...     buffer_capacity=2
... )
>>> pi = abg.markov.get_steady_state_algebraically(
...     Q, algebraic_function=np.linalg.lstsq
... )

\end{lstlisting}

Having the set of all states and the steady state probabilities, the mean
number of individuals in the system, in node 1 and in node 2 can be calculated
as shown in code snippets~\ref{lst:mean_number_of_individuals_in_system},
\ref{lst:mean_number_of_individuals_in_node_1}
and~\ref{lst:mean_number_of_individuals_in_node_2}.

\begin{lstlisting}[
    style=pystyle,
    caption={Code snippet for calculating the mean number of individuals in
    the system.},
    label={lst:mean_number_of_individuals_in_system},
]
>>> def get_mean_number_of_individuals_in_system(pi, states):
...     """Gets the mean number of individuals in the system
...     Parameters
...     ----------
...     pi : numpy.ndarray
...         steady state vector
...     states : list
...         list of tuples that contains all states
...     Returns
...     -------
...     float
...         Mean number of individuals in the whole model
...     """
...     states = np.array(states)
...     mean_inds_in_system = np.sum((states[:, 0] + states[:, 1]) * pi)
...     return mean_inds_in_system
>>>
>>> round(get_mean_number_of_individuals_in_system(pi, all_states), 10)
2.0872927227

\end{lstlisting}

\begin{lstlisting}[
    style=pystyle,
    caption={Code snippet for calculating the mean number of individuals in
    Node 1.},
    label={lst:mean_number_of_individuals_in_node_1},
]
>>> def get_mean_number_of_individuals_in_node_1(pi, states):
...     """Mean number of individuals in node 1
...     Parameters
...     ----------
...     pi : numpy.ndarray
...         steady state vector
...     states : list
...         list of tuples that contains all states
...     Returns
...     -------
...     float
...         Mean number of individuals
...     """
...     states = np.array(states)
...     mean_inds_in_node_1 = np.sum(states[:, 1] * pi)
...     return mean_inds_in_node_1
>>>
>>> round(get_mean_number_of_individuals_in_node_1(pi, all_states), 10)
1.8187129478

\end{lstlisting}

\begin{lstlisting}[
    style=pystyle,
    caption={Code snippet for calculating the mean number of individuals in
    Node 2.},
    label={lst:mean_number_of_individuals_in_node_2},
]
>>> def get_mean_number_of_individuals_in_node_2(pi, states):
...     """Mean number of class 2 individuals blocked
...     Parameters
...     ----------
...     pi : numpy.ndarray
...         steady state vector
...     states : list
...         list of tuples that contains all states
...     Returns
...     -------
...     float
...         Mean number of blocked class 2 individuals
...     """
...     states = np.array(states)
...     mean_blocked = np.sum(states[:, 0] * pi)
...     return mean_blocked
>>>
>>> round(get_mean_number_of_individuals_in_node_2(pi, all_states), 10)
0.2685797749

\end{lstlisting}


Consequently, there are some additional performance measures of interest that
are more complex to calculate.
Such performance measures are the mean waiting time in the system (for both
type 1 and type 2 individuals), the mean time blocked in node 2 (only
valid for type 2 individuals) and the proportion of individuals that wait in
node 1 within a predefined time target (for both types).
Under the scope of this study three approaches have been considered to calculate
these performance measures; a recursive algorithm, a direct approach and
a closed-form equation. 
Furthermore, different formulas arise for type 1 individuals and different
ones for type 2 individuals.


\input{chapters/03_queueing_model/03_performance_measures/01_waiting_time/main.tex}

\input{chapters/03_queueing_model/03_performance_measures/02_blocking_time/main.tex}

\input{chapters/03_queueing_model/03_performance_measures/03_proportion_of_individuals_within_time/main.tex}

\section{Performance measures}\label{sec:queueing_performance_measures}
Using vector \(\pi\) there are numerous performance measures of the model that
can be calculated.
The following equations utilise \(\pi\) to get performance measures on the
average number of individuals in node 1 and in node 2:

\begin{itemize}
    \item Mean number of individuals in the entire system:
        \begin{equation}
            L_S = \sum_{i=1}^{|\pi|} \pi_i (u_i + v_i)
        \end{equation}
    \item Mean number of individuals in node 1:
        \begin{equation}
            L_1 = \sum_{i=1}^{|\pi|} \pi_i v_i
        \end{equation}
    \item Mean number of individuals in node 2:
        \begin{equation}
            L_2 = \sum_{i=1}^{|\pi|} \pi_i u_i
        \end{equation}
\end{itemize}

The python code for these functions may be obtained quite quickly using the set
of all states and the steady state probability.

\begin{lstlisting}[
    style=pystyle,
    caption={Code snippet for getting the set of all states and the steady
    state probabilities.},
    label={lst:set_of_all_states_and_steady_state_probabilities},
]
>>> import ambulance_game as abg
>>> import numpy as np
>>> all_states = abg.markov.build_states(
...     threshold=3 ,
...     system_capacity=4,
...     buffer_capacity=2
... )
>>> Q = abg.markov.get_transition_matrix(
...     lambda_1=1,
...     lambda_2=2,
...     mu=2,
...     num_of_servers=2,
...     threshold=3,
...     system_capacity=4,
...     buffer_capacity=2
... )
>>> pi = abg.markov.get_steady_state_algebraically(
...     Q, algebraic_function=np.linalg.lstsq
... )

\end{lstlisting}

Having the set of all states and the steady state probabilities, the mean
number of individuals in the system, in node 1 and in node 2 can be calculated
as shown in code snippets~\ref{lst:mean_number_of_individuals_in_system},
\ref{lst:mean_number_of_individuals_in_node_1}
and~\ref{lst:mean_number_of_individuals_in_node_2}.

\begin{lstlisting}[
    style=pystyle,
    caption={Code snippet for calculating the mean number of individuals in
    the system.},
    label={lst:mean_number_of_individuals_in_system},
]
>>> def get_mean_number_of_individuals_in_system(pi, states):
...     """Gets the mean number of individuals in the system
...     Parameters
...     ----------
...     pi : numpy.ndarray
...         steady state vector
...     states : list
...         list of tuples that contains all states
...     Returns
...     -------
...     float
...         Mean number of individuals in the whole model
...     """
...     states = np.array(states)
...     mean_inds_in_system = np.sum((states[:, 0] + states[:, 1]) * pi)
...     return mean_inds_in_system
>>>
>>> round(get_mean_number_of_individuals_in_system(pi, all_states), 10)
2.0872927227

\end{lstlisting}

\begin{lstlisting}[
    style=pystyle,
    caption={Code snippet for calculating the mean number of individuals in
    Node 1.},
    label={lst:mean_number_of_individuals_in_node_1},
]
>>> def get_mean_number_of_individuals_in_node_1(pi, states):
...     """Mean number of individuals in node 1
...     Parameters
...     ----------
...     pi : numpy.ndarray
...         steady state vector
...     states : list
...         list of tuples that contains all states
...     Returns
...     -------
...     float
...         Mean number of individuals
...     """
...     states = np.array(states)
...     mean_inds_in_node_1 = np.sum(states[:, 1] * pi)
...     return mean_inds_in_node_1
>>>
>>> round(get_mean_number_of_individuals_in_node_1(pi, all_states), 10)
1.8187129478

\end{lstlisting}

\begin{lstlisting}[
    style=pystyle,
    caption={Code snippet for calculating the mean number of individuals in
    Node 2.},
    label={lst:mean_number_of_individuals_in_node_2},
]
>>> def get_mean_number_of_individuals_in_node_2(pi, states):
...     """Mean number of class 2 individuals blocked
...     Parameters
...     ----------
...     pi : numpy.ndarray
...         steady state vector
...     states : list
...         list of tuples that contains all states
...     Returns
...     -------
...     float
...         Mean number of blocked class 2 individuals
...     """
...     states = np.array(states)
...     mean_blocked = np.sum(states[:, 0] * pi)
...     return mean_blocked
>>>
>>> round(get_mean_number_of_individuals_in_node_2(pi, all_states), 10)
0.2685797749

\end{lstlisting}


Consequently, there are some additional performance measures of interest that
are more complex to calculate.
Such performance measures are the mean waiting time in the system (for both
type 1 and type 2 individuals), the mean time blocked in node 2 (only
valid for type 2 individuals) and the proportion of individuals that wait in
node 1 within a predefined time target (for both types).
Under the scope of this study three approaches have been considered to calculate
these performance measures; a recursive algorithm, a direct approach and
a closed-form equation. 
Furthermore, different formulas arise for type 1 individuals and different
ones for type 2 individuals.


\section{Performance measures}\label{sec:queueing_performance_measures}
Using vector \(\pi\) there are numerous performance measures of the model that
can be calculated.
The following equations utilise \(\pi\) to get performance measures on the
average number of individuals in node 1 and in node 2:

\begin{itemize}
    \item Mean number of individuals in the entire system:
        \begin{equation}
            L_S = \sum_{i=1}^{|\pi|} \pi_i (u_i + v_i)
        \end{equation}
    \item Mean number of individuals in node 1:
        \begin{equation}
            L_1 = \sum_{i=1}^{|\pi|} \pi_i v_i
        \end{equation}
    \item Mean number of individuals in node 2:
        \begin{equation}
            L_2 = \sum_{i=1}^{|\pi|} \pi_i u_i
        \end{equation}
\end{itemize}

The python code for these functions may be obtained quite quickly using the set
of all states and the steady state probability.

\begin{lstlisting}[
    style=pystyle,
    caption={Code snippet for getting the set of all states and the steady
    state probabilities.},
    label={lst:set_of_all_states_and_steady_state_probabilities},
]
>>> import ambulance_game as abg
>>> import numpy as np
>>> all_states = abg.markov.build_states(
...     threshold=3 ,
...     system_capacity=4,
...     buffer_capacity=2
... )
>>> Q = abg.markov.get_transition_matrix(
...     lambda_1=1,
...     lambda_2=2,
...     mu=2,
...     num_of_servers=2,
...     threshold=3,
...     system_capacity=4,
...     buffer_capacity=2
... )
>>> pi = abg.markov.get_steady_state_algebraically(
...     Q, algebraic_function=np.linalg.lstsq
... )

\end{lstlisting}

Having the set of all states and the steady state probabilities, the mean
number of individuals in the system, in node 1 and in node 2 can be calculated
as shown in code snippets~\ref{lst:mean_number_of_individuals_in_system},
\ref{lst:mean_number_of_individuals_in_node_1}
and~\ref{lst:mean_number_of_individuals_in_node_2}.

\begin{lstlisting}[
    style=pystyle,
    caption={Code snippet for calculating the mean number of individuals in
    the system.},
    label={lst:mean_number_of_individuals_in_system},
]
>>> def get_mean_number_of_individuals_in_system(pi, states):
...     """Gets the mean number of individuals in the system
...     Parameters
...     ----------
...     pi : numpy.ndarray
...         steady state vector
...     states : list
...         list of tuples that contains all states
...     Returns
...     -------
...     float
...         Mean number of individuals in the whole model
...     """
...     states = np.array(states)
...     mean_inds_in_system = np.sum((states[:, 0] + states[:, 1]) * pi)
...     return mean_inds_in_system
>>>
>>> round(get_mean_number_of_individuals_in_system(pi, all_states), 10)
2.0872927227

\end{lstlisting}

\begin{lstlisting}[
    style=pystyle,
    caption={Code snippet for calculating the mean number of individuals in
    Node 1.},
    label={lst:mean_number_of_individuals_in_node_1},
]
>>> def get_mean_number_of_individuals_in_node_1(pi, states):
...     """Mean number of individuals in node 1
...     Parameters
...     ----------
...     pi : numpy.ndarray
...         steady state vector
...     states : list
...         list of tuples that contains all states
...     Returns
...     -------
...     float
...         Mean number of individuals
...     """
...     states = np.array(states)
...     mean_inds_in_node_1 = np.sum(states[:, 1] * pi)
...     return mean_inds_in_node_1
>>>
>>> round(get_mean_number_of_individuals_in_node_1(pi, all_states), 10)
1.8187129478

\end{lstlisting}

\begin{lstlisting}[
    style=pystyle,
    caption={Code snippet for calculating the mean number of individuals in
    Node 2.},
    label={lst:mean_number_of_individuals_in_node_2},
]
>>> def get_mean_number_of_individuals_in_node_2(pi, states):
...     """Mean number of class 2 individuals blocked
...     Parameters
...     ----------
...     pi : numpy.ndarray
...         steady state vector
...     states : list
...         list of tuples that contains all states
...     Returns
...     -------
...     float
...         Mean number of blocked class 2 individuals
...     """
...     states = np.array(states)
...     mean_blocked = np.sum(states[:, 0] * pi)
...     return mean_blocked
>>>
>>> round(get_mean_number_of_individuals_in_node_2(pi, all_states), 10)
0.2685797749

\end{lstlisting}


Consequently, there are some additional performance measures of interest that
are more complex to calculate.
Such performance measures are the mean waiting time in the system (for both
type 1 and type 2 individuals), the mean time blocked in node 2 (only
valid for type 2 individuals) and the proportion of individuals that wait in
node 1 within a predefined time target (for both types).
Under the scope of this study three approaches have been considered to calculate
these performance measures; a recursive algorithm, a direct approach and
a closed-form equation. 
Furthermore, different formulas arise for type 1 individuals and different
ones for type 2 individuals.


\input{chapters/03_queueing_model/03_performance_measures/01_waiting_time/main.tex}

\input{chapters/03_queueing_model/03_performance_measures/02_blocking_time/main.tex}

\input{chapters/03_queueing_model/03_performance_measures/03_proportion_of_individuals_within_time/main.tex}

\section{Performance measures}\label{sec:queueing_performance_measures}
Using vector \(\pi\) there are numerous performance measures of the model that
can be calculated.
The following equations utilise \(\pi\) to get performance measures on the
average number of individuals in node 1 and in node 2:

\begin{itemize}
    \item Mean number of individuals in the entire system:
        \begin{equation}
            L_S = \sum_{i=1}^{|\pi|} \pi_i (u_i + v_i)
        \end{equation}
    \item Mean number of individuals in node 1:
        \begin{equation}
            L_1 = \sum_{i=1}^{|\pi|} \pi_i v_i
        \end{equation}
    \item Mean number of individuals in node 2:
        \begin{equation}
            L_2 = \sum_{i=1}^{|\pi|} \pi_i u_i
        \end{equation}
\end{itemize}

The python code for these functions may be obtained quite quickly using the set
of all states and the steady state probability.

\begin{lstlisting}[
    style=pystyle,
    caption={Code snippet for getting the set of all states and the steady
    state probabilities.},
    label={lst:set_of_all_states_and_steady_state_probabilities},
]
>>> import ambulance_game as abg
>>> import numpy as np
>>> all_states = abg.markov.build_states(
...     threshold=3 ,
...     system_capacity=4,
...     buffer_capacity=2
... )
>>> Q = abg.markov.get_transition_matrix(
...     lambda_1=1,
...     lambda_2=2,
...     mu=2,
...     num_of_servers=2,
...     threshold=3,
...     system_capacity=4,
...     buffer_capacity=2
... )
>>> pi = abg.markov.get_steady_state_algebraically(
...     Q, algebraic_function=np.linalg.lstsq
... )

\end{lstlisting}

Having the set of all states and the steady state probabilities, the mean
number of individuals in the system, in node 1 and in node 2 can be calculated
as shown in code snippets~\ref{lst:mean_number_of_individuals_in_system},
\ref{lst:mean_number_of_individuals_in_node_1}
and~\ref{lst:mean_number_of_individuals_in_node_2}.

\begin{lstlisting}[
    style=pystyle,
    caption={Code snippet for calculating the mean number of individuals in
    the system.},
    label={lst:mean_number_of_individuals_in_system},
]
>>> def get_mean_number_of_individuals_in_system(pi, states):
...     """Gets the mean number of individuals in the system
...     Parameters
...     ----------
...     pi : numpy.ndarray
...         steady state vector
...     states : list
...         list of tuples that contains all states
...     Returns
...     -------
...     float
...         Mean number of individuals in the whole model
...     """
...     states = np.array(states)
...     mean_inds_in_system = np.sum((states[:, 0] + states[:, 1]) * pi)
...     return mean_inds_in_system
>>>
>>> round(get_mean_number_of_individuals_in_system(pi, all_states), 10)
2.0872927227

\end{lstlisting}

\begin{lstlisting}[
    style=pystyle,
    caption={Code snippet for calculating the mean number of individuals in
    Node 1.},
    label={lst:mean_number_of_individuals_in_node_1},
]
>>> def get_mean_number_of_individuals_in_node_1(pi, states):
...     """Mean number of individuals in node 1
...     Parameters
...     ----------
...     pi : numpy.ndarray
...         steady state vector
...     states : list
...         list of tuples that contains all states
...     Returns
...     -------
...     float
...         Mean number of individuals
...     """
...     states = np.array(states)
...     mean_inds_in_node_1 = np.sum(states[:, 1] * pi)
...     return mean_inds_in_node_1
>>>
>>> round(get_mean_number_of_individuals_in_node_1(pi, all_states), 10)
1.8187129478

\end{lstlisting}

\begin{lstlisting}[
    style=pystyle,
    caption={Code snippet for calculating the mean number of individuals in
    Node 2.},
    label={lst:mean_number_of_individuals_in_node_2},
]
>>> def get_mean_number_of_individuals_in_node_2(pi, states):
...     """Mean number of class 2 individuals blocked
...     Parameters
...     ----------
...     pi : numpy.ndarray
...         steady state vector
...     states : list
...         list of tuples that contains all states
...     Returns
...     -------
...     float
...         Mean number of blocked class 2 individuals
...     """
...     states = np.array(states)
...     mean_blocked = np.sum(states[:, 0] * pi)
...     return mean_blocked
>>>
>>> round(get_mean_number_of_individuals_in_node_2(pi, all_states), 10)
0.2685797749

\end{lstlisting}


Consequently, there are some additional performance measures of interest that
are more complex to calculate.
Such performance measures are the mean waiting time in the system (for both
type 1 and type 2 individuals), the mean time blocked in node 2 (only
valid for type 2 individuals) and the proportion of individuals that wait in
node 1 within a predefined time target (for both types).
Under the scope of this study three approaches have been considered to calculate
these performance measures; a recursive algorithm, a direct approach and
a closed-form equation. 
Furthermore, different formulas arise for type 1 individuals and different
ones for type 2 individuals.


\input{chapters/03_queueing_model/03_performance_measures/01_waiting_time/main.tex}

\input{chapters/03_queueing_model/03_performance_measures/02_blocking_time/main.tex}

\input{chapters/03_queueing_model/03_performance_measures/03_proportion_of_individuals_within_time/main.tex}

\section{Performance measures}\label{sec:queueing_performance_measures}
Using vector \(\pi\) there are numerous performance measures of the model that
can be calculated.
The following equations utilise \(\pi\) to get performance measures on the
average number of individuals in node 1 and in node 2:

\begin{itemize}
    \item Mean number of individuals in the entire system:
        \begin{equation}
            L_S = \sum_{i=1}^{|\pi|} \pi_i (u_i + v_i)
        \end{equation}
    \item Mean number of individuals in node 1:
        \begin{equation}
            L_1 = \sum_{i=1}^{|\pi|} \pi_i v_i
        \end{equation}
    \item Mean number of individuals in node 2:
        \begin{equation}
            L_2 = \sum_{i=1}^{|\pi|} \pi_i u_i
        \end{equation}
\end{itemize}

The python code for these functions may be obtained quite quickly using the set
of all states and the steady state probability.

\begin{lstlisting}[
    style=pystyle,
    caption={Code snippet for getting the set of all states and the steady
    state probabilities.},
    label={lst:set_of_all_states_and_steady_state_probabilities},
]
>>> import ambulance_game as abg
>>> import numpy as np
>>> all_states = abg.markov.build_states(
...     threshold=3 ,
...     system_capacity=4,
...     buffer_capacity=2
... )
>>> Q = abg.markov.get_transition_matrix(
...     lambda_1=1,
...     lambda_2=2,
...     mu=2,
...     num_of_servers=2,
...     threshold=3,
...     system_capacity=4,
...     buffer_capacity=2
... )
>>> pi = abg.markov.get_steady_state_algebraically(
...     Q, algebraic_function=np.linalg.lstsq
... )

\end{lstlisting}

Having the set of all states and the steady state probabilities, the mean
number of individuals in the system, in node 1 and in node 2 can be calculated
as shown in code snippets~\ref{lst:mean_number_of_individuals_in_system},
\ref{lst:mean_number_of_individuals_in_node_1}
and~\ref{lst:mean_number_of_individuals_in_node_2}.

\begin{lstlisting}[
    style=pystyle,
    caption={Code snippet for calculating the mean number of individuals in
    the system.},
    label={lst:mean_number_of_individuals_in_system},
]
>>> def get_mean_number_of_individuals_in_system(pi, states):
...     """Gets the mean number of individuals in the system
...     Parameters
...     ----------
...     pi : numpy.ndarray
...         steady state vector
...     states : list
...         list of tuples that contains all states
...     Returns
...     -------
...     float
...         Mean number of individuals in the whole model
...     """
...     states = np.array(states)
...     mean_inds_in_system = np.sum((states[:, 0] + states[:, 1]) * pi)
...     return mean_inds_in_system
>>>
>>> round(get_mean_number_of_individuals_in_system(pi, all_states), 10)
2.0872927227

\end{lstlisting}

\begin{lstlisting}[
    style=pystyle,
    caption={Code snippet for calculating the mean number of individuals in
    Node 1.},
    label={lst:mean_number_of_individuals_in_node_1},
]
>>> def get_mean_number_of_individuals_in_node_1(pi, states):
...     """Mean number of individuals in node 1
...     Parameters
...     ----------
...     pi : numpy.ndarray
...         steady state vector
...     states : list
...         list of tuples that contains all states
...     Returns
...     -------
...     float
...         Mean number of individuals
...     """
...     states = np.array(states)
...     mean_inds_in_node_1 = np.sum(states[:, 1] * pi)
...     return mean_inds_in_node_1
>>>
>>> round(get_mean_number_of_individuals_in_node_1(pi, all_states), 10)
1.8187129478

\end{lstlisting}

\begin{lstlisting}[
    style=pystyle,
    caption={Code snippet for calculating the mean number of individuals in
    Node 2.},
    label={lst:mean_number_of_individuals_in_node_2},
]
>>> def get_mean_number_of_individuals_in_node_2(pi, states):
...     """Mean number of class 2 individuals blocked
...     Parameters
...     ----------
...     pi : numpy.ndarray
...         steady state vector
...     states : list
...         list of tuples that contains all states
...     Returns
...     -------
...     float
...         Mean number of blocked class 2 individuals
...     """
...     states = np.array(states)
...     mean_blocked = np.sum(states[:, 0] * pi)
...     return mean_blocked
>>>
>>> round(get_mean_number_of_individuals_in_node_2(pi, all_states), 10)
0.2685797749

\end{lstlisting}


Consequently, there are some additional performance measures of interest that
are more complex to calculate.
Such performance measures are the mean waiting time in the system (for both
type 1 and type 2 individuals), the mean time blocked in node 2 (only
valid for type 2 individuals) and the proportion of individuals that wait in
node 1 within a predefined time target (for both types).
Under the scope of this study three approaches have been considered to calculate
these performance measures; a recursive algorithm, a direct approach and
a closed-form equation. 
Furthermore, different formulas arise for type 1 individuals and different
ones for type 2 individuals.


\input{chapters/03_queueing_model/03_performance_measures/01_waiting_time/main.tex}

\input{chapters/03_queueing_model/03_performance_measures/02_blocking_time/main.tex}

\input{chapters/03_queueing_model/03_performance_measures/03_proportion_of_individuals_within_time/main.tex}