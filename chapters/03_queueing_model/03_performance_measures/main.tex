\section{Performance measures}\label{sec:queueing_performance_measures}
Using vector \(\pi\) there are numerous performance measures of the model that
can be calculated.
The following equations utilise \(\pi\) to get performance measures on the
average number of individuals in node 1 and in node 2:

\begin{itemize}
    \item Mean number of individuals in the entire system:
        \begin{equation}
            L_S = \sum_{i=1}^{|\pi|} \pi_i (u_i + v_i)
        \end{equation}
    \item Mean number of individuals in node 1:
        \begin{equation}
            L_1 = \sum_{i=1}^{|\pi|} \pi_i v_i
        \end{equation}
    \item Mean number of individuals in node 2:
        \begin{equation}
            L_2 = \sum_{i=1}^{|\pi|} \pi_i u_i
        \end{equation}
\end{itemize}

The python code for these functions may be obtained quite quickly using the set
of all states and the steady state probability.

\begin{lstlisting}[
    style=pystyle,
    caption={Code snippet for getting the set of all states and the steady
    state probabilities.},
    label={lst:set_of_all_states_and_steady_state_probabilities},
]
>>> import ambulance_game as abg
>>> import numpy as np
>>> all_states = abg.markov.build_states(
...     threshold=3 ,
...     system_capacity=4,
...     buffer_capacity=2
... )
>>> Q = abg.markov.get_transition_matrix(
...     lambda_1=1,
...     lambda_2=2,
...     mu=2,
...     num_of_servers=2,
...     threshold=3,
...     system_capacity=4,
...     buffer_capacity=2
... )
>>> pi = abg.markov.get_steady_state_algebraically(
...     Q, algebraic_function=np.linalg.lstsq
... )

\end{lstlisting}

Having the set of all states and the steady state probabilities, the mean
number of individuals in the system, in node 1 and in node 2 can be calculated
as shown in code snippets~\ref{lst:mean_number_of_individuals_in_system},
\ref{lst:mean_number_of_individuals_in_node_1}
and~\ref{lst:mean_number_of_individuals_in_node_2}.

\begin{lstlisting}[
    style=pystyle,
    caption={Code snippet for calculating the mean number of individuals in
    the system.},
    label={lst:mean_number_of_individuals_in_system},
]
>>> def get_mean_number_of_individuals_in_system(pi, states):
...     """Gets the mean number of individuals in the system
...     Parameters
...     ----------
...     pi : numpy.ndarray
...         steady state vector
...     states : list
...         list of tuples that contains all states
...     Returns
...     -------
...     float
...         Mean number of individuals in the whole model
...     """
...     states = np.array(states)
...     mean_inds_in_system = np.sum((states[:, 0] + states[:, 1]) * pi)
...     return mean_inds_in_system
>>>
>>> round(get_mean_number_of_individuals_in_system(pi, all_states), 10)
2.0872927227

\end{lstlisting}

\begin{lstlisting}[
    style=pystyle,
    caption={Code snippet for calculating the mean number of individuals in
    Node 1.},
    label={lst:mean_number_of_individuals_in_node_1},
]
>>> def get_mean_number_of_individuals_in_node_1(pi, states):
...     """Mean number of individuals in node 1
...     Parameters
...     ----------
...     pi : numpy.ndarray
...         steady state vector
...     states : list
...         list of tuples that contains all states
...     Returns
...     -------
...     float
...         Mean number of individuals
...     """
...     states = np.array(states)
...     mean_inds_in_node_1 = np.sum(states[:, 1] * pi)
...     return mean_inds_in_node_1
>>>
>>> round(get_mean_number_of_individuals_in_node_1(pi, all_states), 10)
1.8187129478

\end{lstlisting}

\begin{lstlisting}[
    style=pystyle,
    caption={Code snippet for calculating the mean number of individuals in
    Node 2.},
    label={lst:mean_number_of_individuals_in_node_2},
]
>>> def get_mean_number_of_individuals_in_node_2(pi, states):
...     """Mean number of class 2 individuals blocked
...     Parameters
...     ----------
...     pi : numpy.ndarray
...         steady state vector
...     states : list
...         list of tuples that contains all states
...     Returns
...     -------
...     float
...         Mean number of blocked class 2 individuals
...     """
...     states = np.array(states)
...     mean_blocked = np.sum(states[:, 0] * pi)
...     return mean_blocked
>>>
>>> round(get_mean_number_of_individuals_in_node_2(pi, all_states), 10)
0.2685797749

\end{lstlisting}


Consequently, there are some additional performance measures of interest that
are more complex to calculate.
Such performance measures are the mean waiting time in the system (for both
type 1 and type 2 individuals), the mean time blocked in node 2 (only
valid for type 2 individuals) and the proportion of individuals that wait in
node 1 within a predefined time target (for both types).
Under the scope of this study three approaches have been considered to calculate
these performance measures; a recursive algorithm, a direct approach and
a closed-form equation. 
Furthermore, different formulas arise for type 1 individuals and different
ones for type 2 individuals.


\subsection{Waiting time}\label{sec:waiting_time}
Waiting time is the amount of time that individuals wait in node 1 before
they start their service.
For a given set of parameters there are three different performance measures
around the mean waiting time that need to be considered.
The mean waiting time of type 1 individuals, the mean waiting time of type 2
individuals, and the overall mean waiting time.

\subsubsection{Recursive approach}\label{sec:recursive_waiting_time}

The first approach to be considered is a recursive approach to getting the
mean waiting time~\cite{banjevic1996recursive}.
To calculate the mean waiting time of type 1 individuals one must first
identify the set of states \((u, v)\) where a wait can occur.
For this particular Markov chain, these are all states that satisfy \(v > C\)
i.e. all states where the number of individuals in the node 1 exceed
the number of servers.
The set of such states is defined as the \textit{waiting states} and can be
denoted as a subset of all the states, where:

\begin{equation} \label{eq:waiting_states}
    S_w = \{(u, v) \in S \; | \; v > C \}
\end{equation}

Moreover, another element that needs to be considered is the expected waiting
time spent in each state for type \(i\) individuals.
In order to do so a variation of the Markov model has to be considered where
arrivals are removed.
Figure~\ref{fig:markov_variation_no_arrivals} shows this new Markov chain model.

\begin{figure}[H]
    \centering
    
\begin{tikzpicture}[-, node distance = 1.4cm, auto, every node/.style={scale=0.85}]

    \tikzmath{
        let \minsize = 1.8cm;
        let \nosize = 0.1cm;
    }

    \node[draw=none, minimum size=\nosize] (one) {};
    \node[state, minimum size=\minsize, right=of one] (two) {(0,T-1)};
    \node[state, minimum size=\minsize, right=of two] (three) {(0,T)};
    \node[state, minimum size=\minsize, right=of three] (four) {(0,T+1)};
    \node[draw=none, minimum size=\minsize, right=of four] (five) {};

    \node[state, minimum size=\minsize, below=of three] (three_one) {(1,T)};
    \node[state, minimum size=\minsize, below=of three_one] (three_two) {(2,T)};
    \node[state, minimum size=\minsize, below=of four] (four_one) {(1,T+1)};
    \node[state, minimum size=\minsize, below=of four_one] (four_two) {(2,T+1)};
    \node[draw=none, minimum size=\nosize, right=of four_one] (five_one) {};
    \node[draw=none, minimum size=\nosize, right=of four_two] (five_two) {};
    \node[draw=none, minimum size=\nosize, below=of three_two] (three_three) {};

    \draw[every loop]
        (two) edge node {\((T-1) \mu\)} (one)
        (three) edge node {\(T \mu\)} (two)
        (four) edge node {\((T+1) \mu\)} (three)
        (five) edge node {\((T+2) \mu\)} (four)
        (three_one) edge node {\(T \mu\)} (three)

        (four_one) edge node {\((T+1) \mu\)} (three_one)
        (five_one) edge node {\((T+2) \mu\)} (four_one)
        (three_two) edge node {\(T \mu\)} (three_one)
        (three_three) edge node {\(T \mu\)} (three_two)
        (four_two) edge node {\((T+1) \mu\)} (three_two)
        (five_two) edge node {\((T+2) \mu\)} (four_two)
        ;
\end{tikzpicture}

    \caption{Variation of Markov chain model where all arrivals are removed.
    This diagram is used as a visualisation aid to illustrate how the recursive
    algorithm works.}
    \label{fig:markov_variation_no_arrivals}
\end{figure}

For this particular Markov chain variation the expected waiting time spent at
each state \((u,v)\) for type \(i\) individuals is denoted by \(c_w^i(u,v)\).
From a type 1 individual's perspective, when they arrive at the system, no
matter how many other individuals of either type arrive after them it will not
affect their own waiting time.
The desired waiting time by acquired by calculating the inverse of the sum of
the out-flow rate of that state.
Therefore by eliminating the arrival rates of both individuals, the waiting time
at each state for type 1 individuals can be expressed as:

\begin{equation}\label{eq:waiting_time_state_type_1}
    c^{(1)}_w(u,v) =
    \begin{cases}
        0, & \textbf{if } u > 0 \textbf{ and } v = T \\
        \frac{1}{\text{min}(v,C)\mu}, & \textbf{otherwise}
    \end{cases}
\end{equation}

Note here that whenever any type 1 individual is at a state \((u,v)\) where
\(u > 0\) and \(v = T\) (i.e. all states \((1,T), (2,T) \dots, (M,T)\)) the
expected waiting time is set to \(0\).
This is done to capture the trip thorough the Markov chain from the perspective
of type 1 individuals,
meaning that individuals visit all states of the threshold column but only the
one in the first row will return a non-zero waiting time.
Additionally, in equation~\eqref{eq:waiting_time_state_type_1} the service rate
\(\mu\) is multiplied by the minimum of \(v\) and \(C\) since, when the system
is at a state \((u,v)\) where \(v \geq C\), the maximum out-flow service rate
of \( C \mu \) is reached.


Similarly from a type 2 individual's perspective the same logic holds.
The only difference is that type 2 individuals cannot have a waiting time when
they are blocked in node 2.
From the Markov chain model's perspective, type 2 individuals cannot have a wait
whenever they are at state \(u, v\) where \(u > 0\).
Thus, the waiting time at each state for type 2 individuals can be expressed as:

\begin{equation}\label{eq:waiting_time_state_type_2}
    c^{(2)}_w(u,v) =
    \begin{cases}
        0, & \textbf{if } u > 0 \\
        \frac{1}{\text{min}(v, C)\mu}, & \textbf{otherwise}
    \end{cases}
\end{equation}


Using the set of waiting states defined in~\eqref{eq:waiting_states} and
equations~\eqref{eq:waiting_time_state_type_1}
and~\eqref{eq:waiting_time_state_type_2} the following recursive formula can be
used to get the mean waiting time spent in each state.
The formula goes through all states from right to left recursively and adds the
total expected waiting time of all these states together until it reaches a
state that is not in the set of waiting states.
Thus, the expected waiting time of a type \(i\) individual when they arrive at
state \( (u,v) \) is given by:

\begin{equation} \label{eq:recursive_waiting_time_for_state}
    w^{(i)}(u,v) =
    \begin{cases}
        0, \hspace{4.85cm} & \textbf{if } (u,v) \notin S_w \\
        c^{(i)}_w(u,v) + w^{(i)}(u-1, v), & \textbf{if } u > 0 \textbf{ and } v = T \\
        c^{(i)}_w(u,v) + w^{(i)}(u, v-1), & \textbf{otherwise}
    \end{cases}
\end{equation}

Whenever the system is at state \((u,v)\) and an individual arrives, depending
on the type of the individual, the system will move to a different state.
The state that the Markov chain will transition to when a type \(i\) individual
arrives is defined as the \textit{arriving state} \(\mathcal{A}_i(u,v)\).
Using Figure~\ref{fig:adjusted_markov_model} as reference, an arrival of a type
1 individual makes the system transition to the state on the right.
Similarly an arrival of a type 2 individual makes the system transition to the
right if the threshold hasn't been reached and transition down if the threshold
has been reached.
This can be expressed mathematically as:

\begin{equation}\label{eq:arriving_state_type_1}
    \mathcal{A}_1(u,v) = (u, v + 1)
\end{equation}
\begin{equation}\label{eq:arriving_state_type_2}
    \mathcal{A}_2(u,v) =
    \begin{cases}
        (u, v + 1), & \text{if } v < T \\
        (u + 1, v), & \text{if } v \geq T \\
    \end{cases}
\end{equation}

Additionally, there are certain states in the model where arrivals cannot occur.
A type 1 individual cannot arrive whenever the model is at any state \((u, N)\)
for all \(u\), where \(N\) is the capacity of node 1.
Therefore the set of all such states that an arrival may occur are defined as
\textit{accepting states}.
The set of accepting states for type 1 individuals is denoted as:

\begin{equation}\label{eq:accepting_states_type_1}
    S_A^{(1)} = \{(u, v) \in S \; | \; v < N \}
\end{equation}

Similarly, an arrival of a type 2 individual cannot occur whenever the model is
at state \((M, v)\) for all \(v\), where \(M\) is the capacity of node 2.
The set of accepting states for type 2 individuals is denoted as:

\begin{equation}\label{eq:accepting_states_type_2}
    S_A^{(2)} = \{(u, v) \in S \; | \; u < M \}
\end{equation}



Finally, the total mean waiting time can be calculated by summing over all
expected waiting times of accepting states multiplied by the probability of
being at that state.
The different approaches that are used to get the state probabilities are
described in Section~\ref{sec:steady_state_probabilities}.
The mean waiting time in the system for type \(i\) individuals is given by:

\begin{equation}\label{eq:recursive_waiting_time_for_type_i}
    W^{(i)} = \frac{\sum_{(u,v) \in S_A^{(i)}} \pi_{(u,v)} w^{(i)}
    (\mathcal{A}_i(u,v))}{\sum_{(u,v) \in S_A^{(i)}} \pi_{(u,v)}}
\end{equation}

Consequently, using both the mean waiting time for type 1 individuals
\(W^{(1)}\) and the mean waiting time for type 2 individuals \(W^{(2)}\), the
overall mean waiting time in the system is a linear combination of the 2.
The overall waiting time can be then given by the following equation where
\(\theta_1\) and \(\theta_2\) are the coefficients of the waiting time for each
type of individual:

\begin{equation}\label{eq:overall_waiting_time_coeff}
    W = \theta_1 W^{(1)} + \theta_2 W^{(2)}
\end{equation}

The two coefficients represent the proportion of individuals of each type that
traversed through the model.
Theoretically, determining these percentages should be as quick as
looking at the arrival rates of each type \(\lambda_1\) and \(\lambda_2\), but
in practise if either node 1 or node 2 are full, some
individuals may become lost to the system.
Thus, one should account for the probability that an individual is lost to the
system.
This probability can be calculated by using the two sets of accepting
states \(S_A^{(2)}\) and \(S_A^{(1)}\) defined earlier
in~\eqref{eq:accepting_states_type_1} and~\eqref{eq:accepting_states_type_2}.
Let us define the probability that an individual of type \(i\) is not lost
to the system as \(P(L'_i)\):

\begin{equation*}
    P(L'_1) = \sum_{(u,v) \, \in S_A^{(1)}} \pi(u,v) \hspace{2cm}
    P(L'_2) = \sum_{(u,v) \, \in S_A^{(2)}} \pi(u,v)
\end{equation*}


Having defined these probabilities one may combine them with the arrival rates
of each individual type in such a way to get the expected proportions of type 1
and type 2 individuals in the model.

\begin{equation}
    \theta_1 = \frac{\lambda_1 P(L'_1)}{\lambda_2 P(L'_2) + \lambda_1 P(L'_1)},
    \hspace{1.5cm}
    \theta_2 = \frac{\lambda_2 P(L'_2)}{\lambda_2 P(L'_2) + \lambda_1 P(L'_1)}
\end{equation}

Thus, by using these values as the coefficient of
equation~\eqref{eq:overall_waiting_time_coeff}
the resultant equation can be used to get the overall waiting time.

\begin{equation}\label{eq:overall_waiting_time}
    W = \frac{\lambda_1 P(L'_1)}{\lambda_2 P(L'_2) + \lambda_1 P(L'_1)} W^{(1)}
    + \frac{\lambda_2 P(L'_2)}{\lambda_2 P(L'_2) + \lambda_1 P(L'_1)} W^{(2)}
\end{equation}


\paragraph{Implementation}\label{sec:waiting_recursive_implementation}

Implementing the recursive approach for the waiting time in python uses the
same logical structure as described in Section~\ref{sec:recursive_waiting_time}.
The first function needed is one that checks if a state belongs in the
set of waiting states that corresponds to the set defined
in~\eqref{eq:waiting_states}.

\begin{lstlisting}[
    style=pystyle,
    caption={Function that checks if a state is a waiting state.},
    label={lst:is_waiting_state},
]
>>> def is_waiting_state(state, num_of_servers):
...     """Checks if waiting occurs in the given state. In essence, all
...     states (u,v) where v > C are considered waiting states.
...     Parameters
...     ----------
...     state : tuple
...         a tuples of the form (u,v)
...     num_of_servers : int
...         the number of servers = C
...     Returns
...     -------
...     Boolean
...         An indication of whether or not any wait occurs on the given
...         state
...     """
...     return state[1] > num_of_servers
>>>
>>> is_waiting_state(state=(1, 4), num_of_servers=2)
True

\end{lstlisting}

Similarly a function that calculates the expected wait in each state is needed
that corresponds to equations~\eqref{eq:waiting_time_state_type_1}
and~\eqref{eq:waiting_time_state_type_2}.
Note here that the following function takes the individuals type as an argument
and thus only one function is needed for both expressions.

\begin{lstlisting}[
    style=pystyle,
    caption={Function for the expected waiting time in a state.},
    label={lst:expected_waiting_time_in_state},
]
>>> def expected_time_in_markov_state_ignoring_arrivals(
...     state,
...     class_type,
...     num_of_servers,
...     mu,
...     threshold,
... ):
...     """Get the expected waiting time in a Markov state when ignoring any
...     subsequent arrivals. When considering the waiting time of class 2
...     individuals, and when these individuals are in a blocked state
...     (v > 0) then by the definition of the problem the waiting time in
...     that state is set to 0. Additionally, all states where u > 0 and
...     v = T automatically get a waiting time of 0 because class 1
...     individuals only pass one of the states of that column (only state
...     (0,T) is not zero).
... 
...     Parameters
...     ----------
...     state : tuple
...         a tuples of the form (u,v)
...     class_type : int
...         A string to distinguish between class 1(=0) and class 2(=1)
...         individuals
...     num_of_servers : int
...         The number of servers = C
...     mu : float
...         The service rate = mu
... 
...     Returns
...     -------
...     float
...         The expected waiting time in the given state
...     """
...     if state[0] > 0 and (state[1] == threshold or class_type == 1):
...         return 0
...     return 1 / (min(state[1], num_of_servers) * mu)
>>>
>>> expected_time_in_markov_state_ignoring_arrivals(
...     state=(3, 4),
...     class_type=0,
...     num_of_servers=1,
...     mu=4,
...     threshold=2,
... )
0.25

\end{lstlisting}

The following block of code is the implementation of
equation~\eqref{eq:recursive_waiting_time_for_state}, where it returns the
waiting time of an individual when they arrive at a given state until they
leave that particular state.
Note that this function uses both of the functions defined earlier.

\begin{lstlisting}[
    style=pystyle,
    caption={Function for the overall expected waiting time in a state using
    recursion.},
    label={lst:expected_waiting_time_in_state_recursively},
]
>>> def get_waiting_time_for_each_state_recursively(
...     state,
...     class_type,
...     lambda_2,
...     lambda_1,
...     mu,
...     num_of_servers,
...     threshold,
...     system_capacity,
...     buffer_capacity,
... ):
...     """Performs a recursive algorithm to get the expected waiting time of
...     individuals when they enter the model at a given state. Given an
...     arriving state the algorithm moves down to all subsequent states
...     until it reaches one that is not a waiting state.
... 
...     Class 1:
...         - If (u,v) not a waiting state: return 0
...         - Next state s_d = (0, v - 1)
...         - w(u,v) = c(u,v) + w(s_d)
... 
...     Class 2:
...         - If (u,v) not a waiting state: return 0
...         - Next state:   s_n = (u-1, v),    if u >= 1 and v=T
...                         s_n = (u, v - 1),  otherwise
...         - w(u,v) = c(u,v) + w(s_n)
... 
...     Note: For all class 1 individuals the recursive formula acts in a
...     linear manner meaning that an individual will have the same waiting
...     time when arriving at any state of the same column e.g (2, 3) or
...     (5, 3).
... 
...     Parameters
...     ----------
...     state : tuple
...     class_type : int
...     lambda_2 : float
...     lambda_1 : float
...     mu : float
...     num_of_servers : int
...     threshold : int
...     system_capacity : int
...     buffer_capacity : int
... 
...     Returns
...     -------
...     float
...         The expected waiting time from the arriving state of an
...         individual until service
...     """
...     if not is_waiting_state(state, num_of_servers):
...         return 0
...     if state[0] >= 1 and state[1] == threshold:
...         next_state = (state[0] - 1, state[1])
...     else:
...         next_state = (state[0], state[1] - 1)
... 
...     wait = expected_time_in_markov_state_ignoring_arrivals(
...         state=state,
...         class_type=class_type,
...         num_of_servers=num_of_servers,
...         mu=mu,
...         threshold=threshold,
...     )
...     wait += get_waiting_time_for_each_state_recursively(
...         state=next_state,
...         class_type=class_type,
...         lambda_2=lambda_2,
...         lambda_1=lambda_1,
...         mu=mu,
...         num_of_servers=num_of_servers,
...         threshold=threshold,
...         system_capacity=system_capacity,
...         buffer_capacity=buffer_capacity,
...     )
...     return wait
>>>
>>> get_waiting_time_for_each_state_recursively(
...     state=(3, 4),
...     class_type=0,
...     lambda_2=1,
...     lambda_1=1,
...     mu=4,
...     num_of_servers=1,
...     threshold=2,
...     system_capacity=4,
...     buffer_capacity=3,
... )
0.75

\end{lstlisting}

Additionally, before getting the mean waiting time for each type of individuals,
a function for the set of accepting states described
in~\eqref{eq:accepting_states_type_1} and~\eqref{eq:accepting_states_type_1}
needs to be constructed.

\begin{lstlisting}[
    style=pystyle,
    caption={Function to check if a state is an accepting state.},
    label={lst:is_accepting_state},
]
>>> def is_accepting_state(
...     state, class_type, threshold, system_capacity, buffer_capacity
... ):
...     """
...     Checks if a state given is an accepting state. Accepting states are
...     defined as the states of the system where arrivals may occur. In
...     essence these states are all states apart from the one when the system
...     cannot accept additional arrivals. Because there are two types of
...     arrivals though, the set of accepting states is different for class 1
...     and class 2 individuals:
... 
...     Parameters
...     ----------
...     state : tuple
...         a tuples of the form (u,v)
...     class_type : int
...         A string to distinguish between class 1 (=0) and class 2
...         individuals (=1)
...     system_capacity : int
...         The capacity of the system (hospital) = N
...     buffer_capacity : int
...         The capacity of the buffer space = M
... 
...     Returns
...     -------
...     Boolean
...         An indication of whether or not an arrival of the given type
...         (class_type) can occur
...     """
...     if class_type == 1:
...         condition = (
...             (state[0] < buffer_capacity)
...             if (threshold <= system_capacity)
...             else (state[1] < system_capacity)
...         )
...     if class_type == 0:
...         condition = state[1] < system_capacity
...     return condition

\end{lstlisting}

The only thing left to do is to find the weighted average of the waiting times
for all states using the steady state probabilities.
The function defined in~\ref{lst:mean_waiting_time_recursive}
corresponds to the expression for \(W^{(i)}\)
defined in equation~\eqref{eq:recursive_waiting_time_for_type_i}.

\begin{lstlisting}[
    style=pystyle,
    caption={Function to get the mean waiting time recursively for a specific
    individual type.},
    label={lst:mean_waiting_time_recursive},
]
>>> import ambulance_game as abg
>>> import numpy as np
>>> def mean_waiting_time_formula_using_recursive_approach(
...     all_states,
...     pi,
...     class_type,
...     lambda_2,
...     lambda_1,
...     mu,
...     num_of_servers,
...     threshold,
...     system_capacity,
...     buffer_capacity,
...     **kwargs,
... ):
...     """
...     Get the mean waiting time by using a recursive formula. 
...     All w(u,v) terms are calculated recursively by going through
...     the waiting times of all previous states.
... 
...     Parameters
...     ----------
...     all_states : list
...     pi : array
...     class_type : int
...     lambda_2 : float
...     lambda_1 : float
...     mu : float
...     num_of_servers : int
...     threshold : int
...     system_capacity : int
...     buffer_capacity : int
... 
...     Returns
...     -------
...     float
...     """
...     mean_waiting_time = 0
...     probability_of_accepting = 0
...     for u, v in all_states:
...         if is_accepting_state(
...             state=(u, v),
...             class_type=class_type,
...             threshold=threshold,
...             system_capacity=system_capacity,
...             buffer_capacity=buffer_capacity,
...         ):
...             arriving_state = (u, v + 1)
...             if class_type == 1 and v >= threshold:
...                 arriving_state = (u + 1, v)
... 
...             current_state_wait = get_waiting_time_for_each_state_recursively(
...                 state=arriving_state,
...                 class_type=class_type,
...                 lambda_2=lambda_2,
...                 lambda_1=lambda_1,
...                 mu=mu,
...                 num_of_servers=num_of_servers,
...                 threshold=threshold,
...                 system_capacity=system_capacity,
...                 buffer_capacity=buffer_capacity,
...             )
...             mean_waiting_time += current_state_wait * pi[u, v]
...             probability_of_accepting += pi[u, v]
...     return mean_waiting_time / probability_of_accepting
>>>
>>> all_states = abg.markov.build_states(
...     threshold=2,
...     system_capacity=4,
...     buffer_capacity=3,
... )
>>> Q = abg.markov.get_transition_matrix(
...     lambda_1=1,
...     lambda_2=1,
...     mu=4,
...     num_of_servers=1,
...     threshold=2,
...     system_capacity=4,
...     buffer_capacity=3
... )
>>> pi = abg.markov.get_markov_state_probabilities(
...     abg.markov.get_steady_state_algebraically(
...         Q, algebraic_function=np.linalg.solve
...     ),
...     all_states,
... )
>>> round(
...     mean_waiting_time_formula_using_recursive_approach(
...         all_states=all_states,
...         pi=pi,
...         class_type=0,
...         lambda_2=1,
...         lambda_1=1,
...         mu=4,
...         num_of_servers=1,
...         threshold=2,
...         system_capacity=4,
...         buffer_capacity=3,
...     ), 10
... )
0.192140129

\end{lstlisting}

Finally the overall waiting time for both individuals can be calculated by
taking the weighted average of the waiting times for each type of individual
as described in equation~\eqref{eq:overall_waiting_time}.

\begin{lstlisting}[
    style=pystyle,
    caption={Function for the overall waiting time formula.},
    label={lst:overall_waiting_time_formula},
]
>>> def overall_waiting_time_formula(
...     all_states,
...     pi,
...     lambda_2,
...     lambda_1,
...     mu,
...     num_of_servers,
...     threshold,
...     system_capacity,
...     buffer_capacity,
...     waiting_formula,
...     **kwargs,
... ):
...     """
...     Gets the overall waiting time for all individuals by calculating both
...     class 1 and class 2 waiting times. Thus, considering the probability
...     that an individual is lost to the system (for both classes)
...     calculates the overall waiting time.
... 
...     Parameters
...     ----------
...     all_states : list
...     pi : array
...     lambda_1 : float
...     lambda_2 : float
...     mu : float
...     num_of_servers : int
...     threshold : int
...     system_capacity : int
...     buffer_capacity : int
...     waiting_formula : function
... 
...     Returns
...     -------
...     float
...         The overall mean waiting time by combining class 1 and class 2
...         individuals
...     """
...     mean_waiting_times_for_each_class = [
...         waiting_formula(
...             all_states=all_states,
...             pi=pi,
...             class_type=class_type,
...             lambda_2=lambda_2,
...             lambda_1=lambda_1,
...             mu=mu,
...             num_of_servers=num_of_servers,
...             threshold=threshold,
...             system_capacity=system_capacity,
...             buffer_capacity=buffer_capacity,
...         )
...         for class_type in range(2)
...     ]
...     prob_accept = [
...         np.sum(
...             [
...                 pi[state]
...                 for state in all_states
...                 if is_accepting_state(
...                     state=state,
...                     class_type=class_type,
...                     threshold=threshold,
...                     system_capacity=system_capacity,
...                     buffer_capacity=buffer_capacity,
...                 )
...             ]
...         )
...         for class_type in range(2)
...     ]
...     class_rates = [
...         prob_accept[class_type]
...         / ((lambda_2 * prob_accept[1]) + (lambda_1 * prob_accept[0]))
...         for class_type in range(2)
...     ]
...     class_rates[0] *= lambda_1
...     class_rates[1] *= lambda_2
...     mean_waiting_time = np.sum(
...         [
...             mean_waiting_times_for_each_class[class_type]
...             * class_rates[class_type]
...             for class_type in range(2)
...         ]
...     )
...     return mean_waiting_time
>>>
>>> round(overall_waiting_time_formula(
...     all_states=all_states,
...     pi=pi,
...     lambda_2=1,
...     lambda_1=1,
...     mu=4,
...     num_of_servers=1,
...     threshold=2,
...     system_capacity=4,
...     buffer_capacity=3,
...     waiting_formula=mean_waiting_time_formula_using_recursive_approach,
... ), 10)
0.1572461886

\end{lstlisting}


\subsubsection{Direct approach}\label{sec:direct_waiting_time}

The direct approach uses similar concepts to the recursive approach of
Section~\ref{sec:recursive_waiting_time}.
Instead of using recursion, a linear system of the set of equations generated by
equation~\eqref{eq:recursive_waiting_time_for_state} for every state \((u,v)\)
is solved.
The set of equations that need to be solved for individuals of type \(i\) are
all \( w^{(i)}(u, v) \) for all possible states \((u,v) \in S\).


\begin{equation*}
    w^{(i)}(u,v) =
    \begin{cases}
        0, \hspace{4.85cm} & \textbf{if } (u,v) \notin S_w \\
        c^{(i)}_w(u,v) + w^{(i)}(u-1, v), & \textbf{if } u > 0
        \textbf{ and } v = T \\
        c^{(i)}_w(u,v) + w^{(i)}(u, v-1), & \textbf{otherwise}
    \end{cases}
\end{equation*}

Consider a relatively small model where \(C=1, T=2, N=3, M=1\).
All possible equations \(w^{(i)}(u,v)\) are given by
equations~\eqref{eq:first_eq_of_waiting_example}
-~\eqref{eq:last_eq_of_waiting_example}.

\begin{minipage}{0.45\textwidth}
    \begin{figure}[H]
        \centering
        \scalebox{0.65}{\begin{tikzpicture}[-, node distance = 1cm, auto]
\node[state] (u0v0) {(0,0)};
\node[state, right=of u0v0] (u0v1) {(0,1)};
\draw[->](u0v0) edge[bend left] node {\( \Lambda \)} (u0v1);
\draw[->](u0v1) edge[bend left] node {\(\mu \)} (u0v0);
\node[state, right=of u0v1] (u0v2) {(0,2)};
\draw[->](u0v1) edge[bend left] node {\( \Lambda \)} (u0v2);
\draw[->](u0v2) edge[bend left] node {\(\mu \)} (u0v1);
\node[state, below=of u0v2] (u1v2) {(1,2)};
\draw[->](u0v2) edge[bend left] node {\( \lambda_2 \)} (u1v2);
\draw[->](u1v2) edge[bend left] node {\(\mu \)} (u0v2);
\node[state, right=of u0v2] (u0v3) {(0,3)};
\draw[->](u0v2) edge[bend left] node {\( \lambda_1 \)} (u0v3);
\draw[->](u0v3) edge[bend left] node {\(\mu \)} (u0v2);
\node[state, right=of u1v2] (u1v3) {(1,3)};
\draw[->](u1v2) edge[bend left] node {\( \lambda_1 \)} (u1v3);
\draw[->](u1v3) edge[bend left] node {\(\mu \)} (u1v2);
\draw[->](u0v3) edge node {\( \lambda_2 \)} (u1v3);
\end{tikzpicture}}
        \caption{Markov chain example with \(C=1, T=2, N=3, M=1\)}
        \label{fig:example_algeb_waiting}
    \end{figure}
\end{minipage}
\begin{minipage}{0.5\textwidth}
    \begin{align}
        w^{(i)}(0,0) &= 0 \\
        w^{(i)}(0,1) &= 0 \\
        w^{(i)}(0,2) &= c^{(i)}_w(0,2) + w^{(i)}(0,1) \label{eq:first_eq_of_waiting_example} \\
        w^{(i)}(0,3) &= c^{(i)}_w(0,3) + w^{(i)}(0,2) \\
        w^{(i)}(1,2) &= c^{(i)}_w(1,2) + w^{(i)}(0,2) \\
        w^{(i)}(1,3) &= c^{(i)}_w(1,3) + w^{(i)}(1,2) \label{eq:last_eq_of_waiting_example}
    \end{align}
\end{minipage}

\vspace{0.5cm}

Additionally, the above equations can be transformed into a linear system of
the form \(Zx=y\) where:

\begin{equation}\label{eq:example_direct_approach_waiting_time}
    Z=
    \begin{pmatrix}
       -1 &  0 &  0 &  0 &  0 &  0 \\  %(0,0)
        0 & -1 &  0 &  0 &  0 &  0 \\  %(0,1)
        0 &  1 & -1 &  0 &  0 &  0 \\  %(0,2)
        0 &  0 &  1 & -1 &  0 &  0 \\  %(0,3)
        0 &  0 &  1 &  0 & -1 &  0 \\  %(1,2)
        0 &  0 &  0 &  0 &  1 & -1 \\  %(1,3)
    \end{pmatrix},
    x=
    \begin{pmatrix}
        w^{(i)}(0,0) \\
        w^{(i)}(0,1) \\
        w^{(i)}(0,2) \\
        w^{(i)}(0,3) \\
        w^{(i)}(1,2) \\
        w^{(i)}(1,3) \\
    \end{pmatrix},
    y=
    \begin{pmatrix}
        0 \\
        0 \\
        -c^{(i)}_w(0,2) \\
        -c^{(i)}_w(0,3) \\
        -c^{(i)}_w(1,2) \\
        -c^{(i)}_w(1,3) \\
    \end{pmatrix}
\end{equation}

A more generalised form of the equations
in~\eqref{eq:example_direct_approach_waiting_time} can be given for any value of
\(C,T,N,M\) by:

\begin{align}
    w^{(i)}(0, 0) &= 0 \label{eq:first_eq_of_waiting_general} \\
    w^{(i)}(0, 1) &= c^{(i)}_w(0,1) + w^{(i)}(0,0) \\
    w^{(i)}(0, 2) &= c^{(i)}_w(0,2) + w^{(i)}(0,1) \\
    & \vdots \nonumber \\
    w^{(i)}(0, T-1) &= c^{(i)}_w(0,T-1) + w^{(i)}(0,T-2) \\
    w^{(i)}(0, T) &= c^{(i)}_w(0, T) + w^{(i)}(0, T - 1) \\
    w^{(i)}(0, T + 1) =& c^{(i)}_w(0, T + 1) + w^{(i)}(0, T) \\
    w^{(i)}(0, T + 2) =& c^{(i)}_w(0, T + 2) + w^{(i)}(0, T + 1) \\
    & \vdots \nonumber \\
    w^{(i)}(0, N) =& c^{(i)}_w(0, N) + w^{(i)}(0, N - 1) \\
    w^{(i)}(1, T) =& c^{(i)}_w(1, T) + w^{(i)}(0, T) \\
    w^{(i)}(1, T + 1) =& c^{(i)}_w(1, T + 1) + w^{(i)}(1, T) \\
    & \vdots \nonumber \\
    w^{(i)}(M, N) =& c^{(i)}_w(M, N) + w^{(i)}(M, N-1)
    \label{eq:last_eq_of_waiting_general}
\end{align}

The equivalent matrix form of the linear system of
equations~\eqref{eq:first_eq_of_waiting_general}
-~\eqref{eq:last_eq_of_waiting_general}
is given by \(Zx=y\), where:

\newcommand{\allthedots}{\vdots & \vdots & \vdots & \ddots & \vdots & \vdots &
\vdots & \vdots & \ddots & \vdots & \vdots & \vdots & \ddots & \vdots}

\footnotesize
\begin{equation*}\label{eq:general_direct_approach_waiting_time}
    Z =
    \begin{pmatrix}
        -1 &  0 &  0 & \dots &  0 &  0 &  0 &  0 & \dots &  0 &  0 &  0 & \dots &  0 \\    %(0, 0)
         1 & -1 &  0 & \dots &  0 &  0 &  0 &  0 & \dots &  0 &  0 &  0 & \dots &  0 \\    %(0, 1)
         0 &  1 & -1 & \dots &  0 &  0 &  0 &  0 & \dots &  0 &  0 &  0 & \dots &  0 \\    %(0, 2)
         \allthedots \\
         0 &  0 &  0 & \dots & -1 &  0 &  0 &  0 & \dots &  0 &  0 &  0 & \dots &  0 \\    %(0, T-1)
         0 &  0 &  0 & \dots &  1 & -1 &  0 &  0 & \dots &  0 &  0 &  0 & \dots &  0 \\    %(0,T)
         0 &  0 &  0 & \dots &  0 &  1 & -1 &  0 & \dots &  0 &  0 &  0 & \dots &  0 \\    %(0,T+1)
         0 &  0 &  0 & \dots &  0 &  0 &  1 & -1 & \dots &  0 &  0 &  0 & \dots &  0 \\    %(0,T+2)
         \allthedots \\
         0 &  0 &  0 & \dots &  0 &  0 &  0 &  0 & \dots & -1 &  0 &  0 & \dots &  0 \\   %(0,N)
         0 &  0 &  0 & \dots &  0 &  1 &  0 &  0 & \dots &  0 & -1 &  0 & \dots &  0 \\   %(1,T)
         0 &  0 &  0 & \dots &  0 &  0 &  1 &  0 & \dots &  0 &  0 & -1 & \dots &  0 \\   %(1,T+1)
         \allthedots \\
         0 &  0 &  0 & \dots &  0 &  0 &  0 &  0 & \dots &  0 &  0 &  0 & \dots & -1 \\   %(M,N)
    \end{pmatrix}
\end{equation*}
\normalsize

\begin{equation}
    x =
    \begin{pmatrix}
        w^{(i)}(0, 0) \\
        w^{(i)}(0, 1) \\
        w^{(i)}(0, 2) \\
        \vdots \\
        w^{(i)}(0, T-1) \\
        w^{(i)}(0, T) \\
        w^{(i)}(0, T + 1) \\
        w^{(i)}(0, T + 2) \\
        \vdots \\
        w^{(i)}(0, N) \\
        w^{(i)}(1, T) \\
        w^{(i)}(1, T + 1) \\
        \vdots \\
        w^{(i)}(M, N) \\
    \end{pmatrix},
    y=
    \begin{pmatrix}
        0 \\ %new
        -c^{(i)}_w(0,1) \\ %new
        -c^{(i)}_w(0,2) \\ %new
        \vdots \\
        -c^{(i)}_w(0,T - 1) \\ %new
        -c^{(i)}_w(0,T) \\ %new
        -c^{(i)}_w(0,T+1) \\
        -c^{(i)}_w(0,T+2) \\
        \vdots \\
        -c^{(i)}_w(0,N) \\
        -c^{(i)}_w(1,T) \\
        -c^{(i)}_w(1,T+1) \\
        \vdots \\
        -c^{(i)}_w(M,N) \\
    \end{pmatrix}
\end{equation}

Thus, solving for \(x\) gets the values of all \(w^{(i)}(u,v)\) for all states
\((u,v) \in S\).
These values can then be used with
equation~\eqref{eq:recursive_waiting_time_for_type_i} to compute the mean
waiting time for type \(i\) individuals \(W^{(i)}\).
Now, having \(W^{(1)}\) and \(W^{(2)}\),
equation~\eqref{eq:overall_waiting_time}
can be utilised once more to compute the overall mean waiting time for both
individual types.


\paragraph{Implementation}\label{sec:waiting_direct_implementation}

Similar to the implementation of the recursive approach from
Section~\ref{sec:waiting_recursive_implementation} the functions that correspond
to equations~\eqref{eq:waiting_states}, \eqref{eq:waiting_time_state_type_1},
\eqref{eq:waiting_time_state_type_2}, \eqref{eq:accepting_states_type_1},
\eqref{eq:accepting_states_type_2} will be used again.
For the implementation of the direct approach the aim is to construct matrix
\(Z\) and vector \(y\) as described in Section
\ref{sec:direct_waiting_time} in order to solve the system of linear
equations described in~\eqref{eq:general_direct_approach_waiting_time}.

The block of code in~\ref{lst:get_coefficients_row_for_state} returns the
values of one row of matrix \(Z\) that corresponds to the state \((u,v)\) and
the value of vector \(y\) that correspond to the state \((u,v)\).

\begin{lstlisting}[
    style=pystyle,
    caption={Function to get a row of the coefficients matrix and the
    corresponding value of the right hand side vector.},
    label={lst:get_coefficients_row_for_state},
]
>>> import itertools
>>> def get_coefficients_row_of_array_for_state(
...     state,
...     class_type,
...     mu,
...     num_of_servers,
...     threshold,
...     system_capacity,
...     buffer_capacity
... ):
...     """
...     For direct approach: Constructs a row of the coefficients matrix.
...     The row to be constructed corresponds to the waiting time equation
...     for a given state (u,v)
...     """
...     lhs_coefficient_row = np.zeros(
...         [buffer_capacity + 1, system_capacity + 1]
...     )
...     lhs_coefficient_row[state[0], state[1]] = -1
...     for (u, v) in itertools.product(
...         range(1, buffer_capacity + 1), range(threshold)
...     ):
...         lhs_coefficient_row[u, v] = np.NaN
... 
...     rhs_value = 0
...     if is_waiting_state(state, num_of_servers):
...         if state[0] >= 1 and state[1] == threshold:
...             next_state = (state[0] - 1, state[1])
...         else:
...             next_state = (state[0], state[1] - 1)
... 
...         lhs_coefficient_row[next_state[0], next_state[1]] = 1
...         rhs_value = -expected_time_in_markov_state_ignoring_arrivals(
...             state=state,
...             class_type=class_type,
...             mu=mu,
...             num_of_servers=num_of_servers,
...             threshold=threshold,
...         )
...     vectorised_array = np.hstack(
...         (
...             lhs_coefficient_row[0, :threshold],
...             lhs_coefficient_row[:, threshold:].flatten("F"),
...         )
...     )
...     return vectorised_array, rhs_value
>>>
>>> get_coefficients_row_of_array_for_state(
...     state=(2,3),
...     class_type=0,
...     mu=4,
...     num_of_servers=1,
...     threshold=2,
...     system_capacity=3,
...     buffer_capacity=2
... )
(array([ 0.,  0.,  0.,  0.,  1.,  0.,  0., -1.]), -0.25)

\end{lstlisting}

In code snippet~\ref{lst:get_coefficients_row_for_state}, the function returns
a tuple with two elements; the row of
matrix \(Z\) and the value of vector \(y\) that corresponds to state \((2,3)\).
Using the function defined in~\ref{lst:get_coefficients_row_for_state}
the matrix \(Z\) and vector \(y\) can be constructed
by considering all states of the Markov chain.

\begin{lstlisting}[
    style=pystyle,
    caption={Function that formulates (but does not solve) the linear system
    needed to get the waiting time.},
    label={lst:get_waiting_time_linear_system},
]
>>> def get_waiting_time_linear_system(
...     class_type,
...     mu,
...     num_of_servers,
...     threshold,
...     system_capacity,
...     buffer_capacity
... ):
...     """
...     For direct approach: Obtain the linear system Z X = y by finding
...     the array Z and the column vector y that are required. Here Z is
...     denoted as "all_coefficients_array" and y as "constant_column".
...     The function stacks the outputs of
...     get_coefficients_row_of_array_for_state() for all states. In
...     essence all outputs are stacked together to form a square matrix
...     (|) and equivalently a column vector (y) that will be used to find
...     X s.t. Z*X=y
...     """
...     all_coefficients_array = np.array([])
...     all_states = abg.markov.build_states(
...         threshold=threshold,
...         system_capacity=system_capacity,
...         buffer_capacity=buffer_capacity,
...     )
...     for state in all_states:
...         lhs_vector, rhs_value = get_coefficients_row_of_array_for_state(
...             state=state,
...             class_type=class_type,
...             mu=mu,
...             num_of_servers=num_of_servers,
...             threshold=threshold,
...             system_capacity=system_capacity,
...             buffer_capacity=buffer_capacity,
...         )
...         if len(all_coefficients_array) == 0:
...             all_coefficients_array = [lhs_vector]
...             constant_column = [rhs_value]
...         else:
...             all_coefficients_array = np.vstack(
...                 [all_coefficients_array, lhs_vector]
...             )
...             constant_column.append(rhs_value)
...     return all_coefficients_array, constant_column
>>>
>>> Z, y = get_waiting_time_linear_system(
...     class_type=0,
...     mu=4,
...     num_of_servers=1,
...     threshold=2,
...     system_capacity=3,
...     buffer_capacity=2
... )
>>> Z
array([[-1.,  0.,  0.,  0.,  0.,  0.,  0.,  0.],
       [ 0., -1.,  0.,  0.,  0.,  0.,  0.,  0.],
       [ 0.,  1., -1.,  0.,  0.,  0.,  0.,  0.],
       [ 0.,  0.,  1., -1.,  0.,  0.,  0.,  0.],
       [ 0.,  0.,  0.,  1., -1.,  0.,  0.,  0.],
       [ 0.,  0.,  1.,  0.,  0., -1.,  0.,  0.],
       [ 0.,  0.,  0.,  1.,  0.,  0., -1.,  0.],
       [ 0.,  0.,  0.,  0.,  1.,  0.,  0., -1.]])
>>> y
[0, 0, -0.25, 0, 0, -0.25, -0.25, -0.25]

\end{lstlisting}

The piece of code in~\ref{lst:waiting_time_for_each_state_direct_approach}
solves the linear system \(Z X = y\) to obtain the
vector \(X\) containing the waiting times for all states of the Markov chain.
After solving the linear system for vector \(X\) it also converts the
1-dimensional array into a 2-dimensional array where the entry at row \(u\) and
column \(v\) corresponds to the expected waiting time that an individual will
have to wait when arriving and the Markov chain is at state \((u,v)\).

\begin{lstlisting}[
    style=pystyle,
    caption={Functions to solve the linear system and get waiting time for
    each state},
    label={lst:waiting_time_for_each_state_direct_approach},
]
>>> def convert_solution_to_correct_array_format(
...     array, all_states, system_capacity, buffer_capacity
... ):
...     """
...     For direct approach: Convert the solution into a format that matches
...     the state probabilities array. The given array is a one-dimensional
...     array with the waiting times of each state
...     """
...     array_with_correct_shape = np.zeros(
...         [buffer_capacity + 1, system_capacity + 1]
...     )
...     for index, (u, v) in enumerate(all_states):
...         array_with_correct_shape[u, v] = array[index]
...     return array_with_correct_shape
>>>
>>>
>>> def get_waiting_times_of_all_states_using_direct_approach(
...     class_type,
...     all_states,
...     mu,
...     num_of_servers,
...     threshold,
...     system_capacity,
...     buffer_capacity,
... ):
...     """
...     For direct approach: Solve M*X = b using numpy.linalg.solve() where:
...         M = The array containing the coefficients of all w(u,v) equations
...         b = Vector of constants of equations
...         X = All w(u,v) variables of the equations
...     """
...     M, b = get_waiting_time_linear_system(
...         class_type=class_type,
...         mu=mu,
...         num_of_servers=num_of_servers,
...         threshold=threshold,
...         system_capacity=system_capacity,
...         buffer_capacity=buffer_capacity,
...     )
...     state_waiting_times = np.linalg.solve(M, b)
...     state_waiting_times = convert_solution_to_correct_array_format(
...         array=state_waiting_times,
...         all_states=all_states,
...         system_capacity=system_capacity,
...         buffer_capacity=buffer_capacity,
...     )
...     return state_waiting_times
>>>
>>> all_states = abg.markov.build_states(
...     threshold=2,
...     system_capacity=3,
...     buffer_capacity=2,
... )
>>>
>>> get_waiting_times_of_all_states_using_direct_approach(
...     class_type=0,
...     all_states=all_states,
...     mu=4,
...     num_of_servers=1,
...     threshold=2,
...     system_capacity=3,
...     buffer_capacity=2
... )
array([[-0.  , -0.  ,  0.25,  0.5 ],
       [ 0.  ,  0.  ,  0.25,  0.5 ],
       [ 0.  ,  0.  ,  0.25,  0.5 ]])

\end{lstlisting}

Finally, similar to Section~\ref{sec:waiting_recursive_implementation}, using
equation~\eqref{eq:recursive_waiting_time_for_type_i} the mean waiting time
for either type of individuals can be calculated as shown
in~\ref{lst:get_mean_waiting_time_direct_approach}.

\begin{lstlisting}[
    style=pystyle,
    caption={Function to calculate the mean waiting time for a given individual
    type using the direct approach},
    label={lst:get_mean_waiting_time_direct_approach},
] 
>>> def mean_waiting_time_formula_using_direct_approach(
...     all_states,
...     pi,
...     class_type,
...     lambda_2,
...     lambda_1,
...     mu,
...     num_of_servers,
...     threshold,
...     system_capacity,
...     buffer_capacity,
...     **kwargs,
... ):
...     """
...     Get the mean waiting time by using a direct approach.
...     """
...     wait_times = get_waiting_times_of_all_states_using_direct_approach(
...         class_type=class_type,
...         all_states=all_states,
...         mu=mu,
...         num_of_servers=num_of_servers,
...         threshold=threshold,
...         system_capacity=system_capacity,
...         buffer_capacity=buffer_capacity,
...     )
...     mean_waiting_time, prob_accept_class_2_ind = 0, 0
...     for (u, v) in all_states:
...         if is_accepting_state(
...             state=(u, v),
...             class_type=class_type,
...             threshold=threshold,
...             system_capacity=system_capacity,
...             buffer_capacity=buffer_capacity,
...         ):
...             arriving_state = (u, v + 1)
...             if class_type == 1 and v >= threshold:
...                 arriving_state = (u + 1, v)
...             mean_waiting_time += wait_times[arriving_state] * pi[u, v]
...             prob_accept_class_2_ind += pi[u, v]
... 
...     return mean_waiting_time / prob_accept_class_2_ind
>>>
>>> all_states = abg.markov.build_states(
...     threshold=2,
...     system_capacity=4,
...     buffer_capacity=3,
... )
>>> Q = abg.markov.get_transition_matrix(
...     lambda_1=1,
...     lambda_2=1,
...     mu=4,
...     num_of_servers=1,
...     threshold=2,
...     system_capacity=4,
...     buffer_capacity=3
... )
>>> pi = abg.markov.get_markov_state_probabilities(
...     abg.markov.get_steady_state_algebraically(
...         Q, algebraic_function=np.linalg.solve
...     ),
...     all_states,
... )
>>> round(
...     mean_waiting_time_formula_using_direct_approach(
...         all_states=all_states,
...         pi=pi,
...         class_type=0,
...         lambda_2=1,
...         lambda_1=1,
...         mu=4,
...         num_of_servers=1,
...         threshold=2,
...         system_capacity=4,
...         buffer_capacity=3,
...     ), 10
... )
0.192140129

\end{lstlisting}


\subsubsection{Closed-form approach}\label{sec:closed_form_waiting_time}

The final approach for getting the mean waiting time is to use a closed-form
approach.
This approach is an immediate simplification of the recursive approach
described in Section~\ref{sec:recursive_waiting_time}.

\begin{equation} \label{eq:closed_form_waiting_type_1}
    W^{(1)} = \frac{\sum_{\substack{(u,v) \, \in S_A^{(1)} \\ v \geq C}}
    \frac{1}{C \mu} \times (v-C+1) \times \pi(u,v)}{\sum_{(u,v) \,
    \in S_A^{(1)}} \pi(u,v)}
\end{equation}

The mean waiting time of type 2 individuals:

\begin{equation}\label{eq:closed_form_waiting_type_2}
    W^{(2)} = \frac{\sum_{\substack{(u,v) \, \in S_A^{(2)} \\ min(v,T) \geq C}}
    \frac{1}{C \mu} \times (\min(v+1,T)-C) \times \pi(u,v)}{\sum_{(u,v) \,
    \in S_A^{(2)}} \pi(u,v)}
\end{equation}

Having \(W^{(1)}\) and \(W^{(2)}\), equation~\eqref{eq:overall_waiting_time} can
then be used to compute \(W\), the overall mean waiting time for both types.


\paragraph{Implementation}\label{sec:waiting_closed_form_implementation}

The closed-form method can be implemented in one function shown
in code snippet~\ref{lst:mean_waiting_time_closed_form_approach}.
The function is broken down in two parts for the case of each individual type.

\begin{lstlisting}[
    style=pystyle,
    caption={Function to calculate the mean waiting time for a given individual
    type using the closed-form approach},
    label={lst:mean_waiting_time_closed_form_approach},
]
>>> def mean_waiting_time_formula_using_closed_form_approach(
...     all_states,
...     pi,
...     class_type,
...     mu,
...     num_of_servers,
...     threshold,
...     system_capacity,
...     buffer_capacity,
...     **kwargs,
... ):
...     """
...     Get the mean waiting time by using a closed-form method.
...     """
...     sojourn_time = 1 / (num_of_servers * mu)
...     if class_type == 0:
...         mean_waiting_time = np.sum(
...             [
...                 (state[1] - num_of_servers + 1) * pi[state]
...                 * sojourn_time
...                 for state in all_states
...                 if is_accepting_state(
...                     state=state,
...                     class_type=class_type,
...                     threshold=threshold,
...                     system_capacity=system_capacity,
...                     buffer_capacity=buffer_capacity,
...                 )
...                 and state[1] >= num_of_servers
...             ]
...         ) / np.sum(
...             [
...                 pi[state]
...                 for state in all_states
...                 if is_accepting_state(
...                     state=state,
...                     class_type=class_type,
...                     threshold=threshold,
...                     system_capacity=system_capacity,
...                     buffer_capacity=buffer_capacity,
...                 )
...             ]
...         )
...
...     if class_type == 1:
...         mean_waiting_time = np.sum(
...             [
...                 (min(state[1] + 1, threshold) - num_of_servers)
...                 * pi[state]
...                 * sojourn_time
...                 for state in all_states
...                 if is_accepting_state(
...                     state=state,
...                     class_type=class_type,
...                     threshold=threshold,
...                     system_capacity=system_capacity,
...                     buffer_capacity=buffer_capacity,
...                 )
...                 and min(state[1], threshold) >= num_of_servers
...             ]
...         ) / np.sum(
...             [
...                 pi[state]
...                 for state in all_states
...                 if is_accepting_state(
...                     state=state,
...                     class_type=class_type,
...                     threshold=threshold,
...                     system_capacity=system_capacity,
...                     buffer_capacity=buffer_capacity,
...                 )
...             ]
...         )
...     return mean_waiting_time
>>>
>>> all_states = abg.markov.build_states(
...     threshold=2,
...     system_capacity=4,
...     buffer_capacity=3,
... )
>>> Q = abg.markov.get_transition_matrix(
...     lambda_1=1,
...     lambda_2=1,
...     mu=4,
...     num_of_servers=1,
...     threshold=2,
...     system_capacity=4,
...     buffer_capacity=3
... )
>>> pi = abg.markov.get_markov_state_probabilities(
...     abg.markov.get_steady_state_algebraically(
...         Q, algebraic_function=np.linalg.solve
...     ),
...     all_states,
... )
>>> round(
...     mean_waiting_time_formula_using_closed_form_approach(
...         all_states=all_states,
...         pi=pi,
...         class_type=0,
...         mu=4,
...         num_of_servers=1,
...         threshold=2,
...         system_capacity=4,
...         buffer_capacity=3,
...     ), 10
... )
0.192140129

\end{lstlisting}

A numeric comparison of the 3 approaches used to compute the mean waiting time
can be found in Section~\ref{sec:waiting_time_approach_comparison}.


\subsection{Blocking time}\label{sec:blocking_time}

Unlike the waiting time in Section~\ref{sec:waiting_time}, 
the blocking time is only calculated for type 2 individuals.
That is because type 1 individuals cannot be blocked.
Thus, one only needs to consider the pathway of type 2 individuals to get the
mean blocking time of the system.

For the waiting time formula described in
equation~\eqref{eq:recursive_waiting_time_for_type_i} in
Section~\ref{sec:recursive_waiting_time}
the expected waiting time for each state was considered by ignoring all
arrivals.
Here, the same approach is used but ignoring only arrivals of type 2
individuals.
That is because for the waiting time formula, once an individual enters
node 1 (i.e. starts waiting) any individual arriving after them will
not affect their pathway.
That is not the case for the blocking time.
When a type 2 individual is blocked, any type 1 individual that arrives will
cause the blocked individual to stay blocked for more time.
Therefore, unlike Figure~\ref{fig:markov_variation_no_arrivals}, type 1 arrivals
are considered here.
Once again a variation of the already existing Markov chain model described in
Figure~\ref{fig:adjusted_markov_model} can be seen in
Figure~\ref{fig:markov_variation_no_type_2_arrivals} where type 2 arrivals are
ignored.

\begin{figure}[ht]
    \centering
    
\begin{tikzpicture}[-, node distance = 1.4cm, auto, every node/.style={scale=0.85}]

    \tikzmath{
        let \minsize = 1.8cm;
        let \nosize = 1cm;
    }

    \node[draw=none, minimum size=\nosize] (one) {};
    \node[state, minimum size=\minsize, right=of one] (two) {(0,T-1)};
    \node[state, minimum size=\minsize, right=of two] (three) {(0,T)};
    \node[state, minimum size=\minsize, right=of three] (four) {(0,T+1)};
    \node[draw=none, minimum size=\nosize, right=of four] (five) {};

    \node[state, minimum size=\minsize, below=of three] (three_one) {(1,T)};
    \node[state, minimum size=\minsize, below=of three_one] (three_two) {(2,T)};
    \node[state, minimum size=\minsize, below=of four] (four_one) {(1,T+1)};
    \node[state, minimum size=\minsize, below=of four_one] (four_two) {(2,T+1)};
    \node[draw=none, minimum size=\nosize, right=of four_one] (five_one) {};
    \node[draw=none, minimum size=\nosize, right=of four_two] (five_two) {};
    \node[draw=none, minimum size=\nosize, below=of three_two] (three_three) {};

    \draw[every loop]
        (two) edge[bend left] node {\((T-1) \mu\)} (one)
        (one) edge[bend left] node {\(\lambda_1\)} (two)
        (three) edge[bend left] node {\(T \mu\)} (two)
        (two) edge[bend left] node {\(\lambda_1\)} (three)
        (four) edge[bend left] node {\((T+1) \mu\)} (three)
        (three) edge[bend left] node {\(\lambda_1\)} (four)
        (five) edge[bend left] node {\((T+2) \mu\)} (four)
        (four) edge[bend left] node {\(\lambda_1\)} (five)
        (three_one) edge node {\(T \mu\)} (three)
        (four_one) edge[bend left] node {\((T+1) \mu\)} (three_one)
        (three_one) edge[bend left] node {\(\lambda_1\)} (four_one)
        (five_one) edge[bend left] node {\((T+2) \mu\)} (four_one)
        (four_one) edge[bend left] node {\(\lambda_1\)} (five_one)
        (three_two) edge node {\(T \mu\)} (three_one)
        (three_three) edge node {\(T \mu\)} (three_two)
        (four_two) edge[bend left] node {\((T+1) \mu\)} (three_two)
        (three_two) edge[bend left] node {\(\lambda_1\)} (four_two)
        (five_two) edge[bend left] node {\((T+2) \mu\)} (four_two)
        (four_two) edge[bend left] node {\(\lambda_1\)} (five_two)
        ;
\end{tikzpicture}
    \caption{Variation of Markov chain model where type 2 arrivals are removed
    (i.e. all arrows pointing down with a rate of \(\lambda_1\) are removed).
    This diagram is used as a visualisation aid for the blocking time formula.}
    \label{fig:markov_variation_no_type_2_arrivals}
\end{figure}

By the nature of this new Markov chain variation a similar recursive approach
to the waiting time cannot be used here.
Since both service completions and new arrivals can occur, the path of an
individual from arrival to departure is not fixed.
For example, for a particular Markov chain model with a threshold of \(T=2\),
an individual arriving at state \((2, 3)\) may have multiple different pathways.
Both of these are valid paths:
\begin{itemize}
    \item \((2, 3) \rightarrow (2, 2) \rightarrow (1, 2) \rightarrow (0, 2)\)
    \item \((2, 3) \rightarrow (2, 4) \rightarrow (2, 3) \rightarrow (2, 2)
    \rightarrow (1, 2) \rightarrow (0, 2)\)
\end{itemize}

Similar to equations~\eqref{eq:waiting_time_state_type_1}
and~\eqref{eq:waiting_time_state_type_2} the expected time spent in each state
here is denoted as:

\begin{equation}\label{eq:blocking_time_state}
    c_b(u,v) =
    \begin{cases}
        \frac{1}{\min(v,C) \mu}, & \text{if } v \leq C\\
        \frac{1}{\min(v,C) \mu + \lambda_1}, & \text{otherwise}
    \end{cases}
\end{equation}

In equation~\eqref{eq:blocking_time_state}, both service completions and
type 1 arrivals are considered.
Thus, from a blocked individual's perspective whenever the system moves from one
state \((u,v)\) to another state it can either be:

\begin{itemize}
    \item because of a service being completed: we will denote the
    probability of this happening by \(p_s(u,v)\).
    \item because of an arrival of an individual of type 1: denoting such
    probability by \(p_o(u,v)\).
\end{itemize}

These probabilities are given by:

\begin{equation}\label{eq:blocking_time_probs}
    p_s(u,v) = \frac{\min(v,C)\mu}{\lambda_1 + \min(v,C)\mu}, \qquad
    p_o(u,v) = \frac{\lambda_1}{\lambda_1 + \min(v,C)\mu}
\end{equation}




The set of states where blocking can occur is defined as the \textit{blocking
states} and consists of all states \((u,v)\) where \(u\) is non-zero.
In essence, the set of blocking state \(S_b\) is defined as:

\begin{equation}\label{eq:blocking_states}
    S_b = \{(u,v) \in S \; | \; u > 0\}
\end{equation}

From Figure~\ref{fig:markov_variation_no_type_2_arrivals} the set \(S_b\)
consists of all states below the first line of Markov chain.
In addition, in order to not consider individuals that will be lost to the
system, the set of accepting states needs to be taken into account.
As defined in Section~\ref{sec:recursive_waiting_time}, the set of accepting
states \(S_A^{(2)}\) is given by equation~\eqref{eq:accepting_states_type_2}.

Having defined \(c_b(u,v)\) and \(S_b\) a formula for the expected blocking time
at each state can be given by:

\small
\begin{equation}\label{eq:expected_blocking_time_at_each_state}
    b(u,v) =
    \begin{cases}
        0, & \textbf{if } (u,v) \notin S_b \\
        c_b(u,v) + b(u - 1, v), & \textbf{if } v = N = T\\
        c_b(u,v) + b(u, v-1), & \textbf{if } v = N \neq T \\
        c_b(u,v) + p_s(u,v) b(u-1, v) + p_o(u,v) b(u, v+1), & \textbf{if } u > 0
        \textbf{ and } v = T \\
        c_b(u,v) + p_s(u,v) b(u, v-1) + p_o(u,v) b(u, v+1), & \textbf{otherwise} \\
    \end{cases}
\end{equation}
\normalsize

Unlike equation~\eqref{eq:recursive_waiting_time_for_type_i},
equation~\eqref{eq:expected_blocking_time_at_each_state} cannot be solved
recursively.
Only a direct approach will be used to solve this equation.
By enumerating all possible equations generated
by~\eqref{eq:expected_blocking_time_at_each_state} for all states \((u,v)\) that
belong in \(S_b\) a system of linear equations arises where the unknown
variables are all the \(b(u,v)\) terms.
For instance, let us consider a Markov model where \(C=2, T=3, N=6, M=2\).
The Markov model is shown in Figure~\ref{fig:example_algeb_blocking}
and the equivalent equations are~\eqref{eq:first_eq_of_blocking_example}
-~\eqref{eq:last_eq_of_blocking_example}.
The equations considered here are only the ones that correspond to the blocking
states.

\begin{minipage}{0.49\textwidth}
    \begin{figure}[H]
        \scalebox{0.5}{\begin{tikzpicture}[-, node distance = 1cm, auto]
\node[state] (u0v0) {(0,0)};
\node[state, right=of u0v0] (u0v1) {(0,1)};
\draw[->](u0v0) edge[bend left] node {\( \Lambda \)} (u0v1);
\draw[->](u0v1) edge[bend left] node {\(\mu \)} (u0v0);
\node[state, right=of u0v1] (u0v2) {(0,2)};
\draw[->](u0v1) edge[bend left] node {\( \Lambda \)} (u0v2);
\draw[->](u0v2) edge[bend left] node {\(2\mu \)} (u0v1);
\node[state, below=of u0v2] (u1v2) {(1,2)};
\draw[->](u0v2) edge[bend left] node {\( \lambda_2 \)} (u1v2);
\draw[->](u1v2) edge[bend left] node {\(2\mu \)} (u0v2);
\node[state, below=of u1v2] (u2v2) {(2,2)};
\draw[->](u1v2) edge[bend left] node {\( \lambda_2 \)} (u2v2);
\draw[->](u2v2) edge[bend left] node {\(2\mu \)} (u1v2);
\node[state, right=of u0v2] (u0v3) {(0,3)};
\draw[->](u0v2) edge[bend left] node {\( \lambda_1 \)} (u0v3);
\draw[->](u0v3) edge[bend left] node {\(2\mu \)} (u0v2);
\node[state, right=of u1v2] (u1v3) {(1,3)};
\draw[->](u1v2) edge[bend left] node {\( \lambda_1 \)} (u1v3);
\draw[->](u1v3) edge[bend left] node {\(2\mu \)} (u1v2);
\draw[->](u0v3) edge node {\( \lambda_2 \)} (u1v3);
\node[state, right=of u2v2] (u2v3) {(2,3)};
\draw[->](u2v2) edge[bend left] node {\( \lambda_1 \)} (u2v3);
\draw[->](u2v3) edge[bend left] node {\(2\mu \)} (u2v2);
\draw[->](u1v3) edge node {\( \lambda_2 \)} (u2v3);
\node[state, right=of u0v3] (u0v4) {(0,4)};
\draw[->](u0v3) edge[bend left] node {\( \lambda_1 \)} (u0v4);
\draw[->](u0v4) edge[bend left] node {\(2\mu \)} (u0v3);
\node[state, right=of u1v3] (u1v4) {(1,4)};
\draw[->](u1v3) edge[bend left] node {\( \lambda_1 \)} (u1v4);
\draw[->](u1v4) edge[bend left] node {\(2\mu \)} (u1v3);
\draw[->](u0v4) edge node {\( \lambda_2 \)} (u1v4);
\node[state, right=of u2v3] (u2v4) {(2,4)};
\draw[->](u2v3) edge[bend left] node {\( \lambda_1 \)} (u2v4);
\draw[->](u2v4) edge[bend left] node {\(2\mu \)} (u2v3);
\draw[->](u1v4) edge node {\( \lambda_2 \)} (u2v4);
\end{tikzpicture}}
        \caption{Example of Markov model with \(C=2, T=3, N=6, M=2\).}
        \label{fig:example_algeb_blocking}
    \end{figure}
\end{minipage}
\begin{minipage}{0.5\textwidth}
    \scriptsize
    \begin{align}
        b(1,2) &= c_b(1,2) + p_o b(1,3) \label{eq:first_eq_of_blocking_example} \\
        b(1,3) &= c_b(1,3) + p_s b(1,2) + p_o b(1,4) \\
        b(1,4) &= c_b(1,4) + b(1,3) \\
        b(2,2) &= c_b(2,2) + p_s b(1,2) + p_o b(2,3) \\
        b(2,3) &= c_b(2,3) + p_s b(2,2) + p_o b(1,4) \\
        b(2,4) &= c_b(2,4) + b(2,3)\label{eq:last_eq_of_blocking_example}
    \end{align}
    \normalsize
\end{minipage}
\vspace{0.5cm}

Additionally, the above equations can be transformed into a linear system of the
form \(Zx=y\) where:

\begin{equation}\label{eq:example_direct_approach_blocking_time}
    Z=
    \begin{pmatrix}
         -1 & p_o &   0 &   0 &   0 &   0 \\ %(1,2)
        p_s &  -1 & p_o &   0 &   0 &   0 \\ %(1,3)
          0 &   1 & - 1 &   0 &   0 &   0 \\ %(1,4)
        p_s &   0 &   0 &  -1 & p_o &   0 \\ %(2,2)
          0 &   0 &   0 & p_s &  -1 & p_o \\ %(2,3)
          0 &   0 &   0 &   0 &   1 &  -1 \\ %(2,4)
    \end{pmatrix},
    x=
    \begin{pmatrix}
        b(1,2) \\
        b(1,3) \\
        b(1,4) \\
        b(2,2) \\
        b(2,3) \\
        b(2,4) \\
    \end{pmatrix},
    y=
    \begin{pmatrix}
        -c_b(1,2) \\
        -c_b(1,3) \\
        -c_b(1,4) \\
        -c_b(2,2) \\
        -c_b(2,3) \\
        -c_b(2,4) \\
    \end{pmatrix}
\end{equation}

A more generalised form of the linear system
of~\eqref{eq:example_direct_approach_blocking_time} can thus be given for any
value of \(C,T,N,M\) by:

\begin{align}
    b(1,T) =& c_b(1, T) + p_o b(1, T + 1)
    \label{eq:first_eq_of_blocking_general}\\
    b(1,T + 1) =& c_b(1, T + 1) + p_s(1, T) + p_o b(1, T + 1) \\
    b(1,T + 2) =& c_b(1, T + 2) + p_s(1, T + 1) + p_o b(1, T + 3) \\
    & \vdots \\
    b(1, N) =& c_b(1, N) + b(1, N - 1) \\
    b(2, T) =& c_b(2, T) + p_s b(1, T) + p_o b(2, T + 1) \\
    b(2, T + 1) =& c_b(2, T + 1) + p_s b(2, T) + p_o b(2, T + 2) \\
    & \vdots \\
    b(M, T) =& c_b(M, T) + b(M, T-1) \label{eq:last_eq_of_blocking_general}
\end{align}

The equivalent matrix form of the linear system of
equations~\eqref{eq:first_eq_of_blocking_general}
-~\eqref{eq:last_eq_of_blocking_general}
is given by \(Zx=y\), where:

\newcommand{\secondallthedots}{\vdots & \vdots & \vdots & \ddots & \vdots & \vdots &
\vdots & \vdots & \vdots & \ddots & \vdots & \vdots}

\begin{equation}
    Z =
    \begin{pmatrix}
        -1 & p_o & 0 & \dots & 0 & 0 & 0 & 0 & 0 & \dots & 0 & 0 \\ %(1,T)
        p_s & -1 & p_o & \dots & 0 & 0 & 0 & 0 & 0 & \dots & 0 & 0 \\ %(1,T+1)
        0 & p_s & -1 & \dots & 0 & 0 & 0 & 0 & 0 & \dots & 0 & 0 \\ %(1,T+2)
        \secondallthedots \\
        0 & 0 & 0 & \dots & 1 & -1 & 0 & 0 & 0 & \dots & 0 & 0 \\ %(1,N)
        p_s & 0 & 0 & \dots & 0 & 0 & -1 & p_o & 0 & \dots & 0 & 0 \\ %(2,T)
        0 & 0 & 0 & \dots & 0 & 0 & p_s & -1 & p_o & \dots & 0 & 0 \\ %(2,T+1)
        \secondallthedots \\
        0 & 0 & 0 & \dots & 0 & 0 & 0 & 0 & 0 & \dots & 1 & -1 \\ %(M,T)
    \end{pmatrix},
\end{equation}
\begin{equation}\label{eq:general_direct_approach_blocking_time}
    x =
    \begin{pmatrix}
        b(1,T) \\
        b(1,T+1) \\
        b(1,T+2) \\
        \vdots \\
        b(1,N) \\
        b(2,T) \\
        b(2,T+1) \\
        \vdots \\
        b(M,T) \\
    \end{pmatrix},
    y=
    \begin{pmatrix}
        -c_b(1,T) \\
        -c_b(1,T+1) \\
        -c_b(1,T+2) \\
        \vdots \\
        -c_b(1,N) \\
        -c_b(2,T) \\
        -c_b(2,T+1) \\
        \vdots \\
        -c_b(M,T) \\
    \end{pmatrix}
\end{equation}

Thus, having calculated the mean blocking time \(b(u,v)\) for every blocking
state individually, a similar formula to
equation~\eqref{eq:recursive_waiting_time_for_type_i} can be derived.
The resultant blocking time formula is given by:

\begin{equation}\label{eq:blocking_time_formula}
    B = \frac{\sum_{(u,v) \in S_A} \pi_{(u,v)} \; b(\mathcal{A}_2(u,v))}{
        \sum_{(u,v) \in S_A}\pi_{(u,v)}}
\end{equation}

Note here that \(\pi_(u,v)\) is the steady state probability that the Markov
chain model is at state \((u,v)\) described in
Section~\ref{sec:steady_state_probabilities}.

\subsubsection{Implementation}\label{sec:implementation_blocking_time}

The mean blocking time is only calculated using a direct approach
similar to the one described in Section~\ref{sec:waiting_direct_implementation}.
Since this implementation is the same as the waiting time one, the code
snippet shown in~\ref{lst:mean_blocking_time_usage} shows only the usage of the
function rather than the function itself.

\begin{lstlisting}[
    style=pystyle,
    caption={Usage of the function to calculate the mean blocking time.},
    label={lst:mean_blocking_time_usage},
]
>>> import ambulance_game as abg
>>> import numpy as np
>>>
>>> all_states = abg.markov.build_states(
...     threshold=2,
...     system_capacity=4,
...     buffer_capacity=3,
... )
>>> Q = abg.markov.get_transition_matrix(
...     lambda_1=1,
...     lambda_2=1,
...     mu=4,
...     num_of_servers=1,
...     threshold=2,
...     system_capacity=4,
...     buffer_capacity=3
... )
>>> pi = abg.markov.get_markov_state_probabilities(
...     abg.markov.get_steady_state_algebraically(
...         Q, algebraic_function=np.linalg.solve
...     ),
...     all_states,
... )
>>> round(
...     abg.markov.mean_blocking_time_formula_using_direct_approach(
...         all_states=all_states,
...         pi=pi,
...         lambda_1=1,
...         mu=4,
...         num_of_servers=1,
...         threshold=2,
...         system_capacity=4,
...         buffer_capacity=3,
...     ), 10
... )
0.1287843179

\end{lstlisting}


\subsection{Proportion of individuals within target}
\label{sec:proportion_of_individuals_within_time}

Another performance measure that needs to be taken into consideration is the
proportion of individuals whose waiting and service times are within a
specified time target.
In order to consider such a measure though one would need to obtain the
distribution of time in the system for all individuals.
The complexity of such a task comes from the fact that different individuals
arrive at different states of the Markov model.
Consider the case when an arrival occurs when the model is at a specific state.

\subsubsection{Distribution of time at a specific state (with 1 server)}

\begin{figure}[H]
    \centering
    \scalebox{0.75}{\begin{tikzpicture}[-, node distance = 1cm, auto]
\node[state] (u0v0) {(0,0)};
\node[state, right=of u0v0] (u0v1) {(0,1)};
\draw[->](u0v0) edge[bend left] node {\( \Lambda \)} (u0v1);
\draw[->](u0v1) edge[bend left] node {\(\mu \)} (u0v0);
\node[state, right=of u0v1] (u0v2) {(0,2)};
\draw[->](u0v1) edge[bend left] node {\( \Lambda \)} (u0v2);
\draw[->](u0v2) edge[bend left] node {\(\mu \)} (u0v1);
\node[state, below=of u0v2] (u1v2) {(1,2)};
\draw[->](u0v2) edge[bend left] node {\( \lambda_2 \)} (u1v2);
\draw[->](u1v2) edge[bend left] node {\(\mu \)} (u0v2);
\node[state, below=of u1v2] (u2v2) {(2,2)};
\draw[->](u1v2) edge[bend left] node {\( \lambda_2 \)} (u2v2);
\draw[->](u2v2) edge[bend left] node {\(\mu \)} (u1v2);
\node[state, right=of u0v2] (u0v3) {(0,3)};
\draw[->](u0v2) edge[bend left] node {\( \lambda_1 \)} (u0v3);
\draw[->](u0v3) edge[bend left] node {\(\mu \)} (u0v2);
\node[state, right=of u1v2] (u1v3) {(1,3)};
\draw[->](u1v2) edge[bend left] node {\( \lambda_1 \)} (u1v3);
\draw[->](u1v3) edge[bend left] node {\(\mu \)} (u1v2);
\draw[->](u0v3) edge node {\( \lambda_2 \)} (u1v3);
\node[state, right=of u2v2] (u2v3) {(2,3)};
\draw[->](u2v2) edge[bend left] node {\( \lambda_1 \)} (u2v3);
\draw[->](u2v3) edge[bend left] node {\(\mu \)} (u2v2);
\draw[->](u1v3) edge node {\( \lambda_2 \)} (u2v3);
\node[state, right=of u0v3] (u0v4) {(0,4)};
\draw[->](u0v3) edge[bend left] node {\( \lambda_1 \)} (u0v4);
\draw[->](u0v4) edge[bend left] node {\(\mu \)} (u0v3);
\node[state, right=of u1v3] (u1v4) {(1,4)};
\draw[->](u1v3) edge[bend left] node {\( \lambda_1 \)} (u1v4);
\draw[->](u1v4) edge[bend left] node {\(\mu \)} (u1v3);
\draw[->](u0v4) edge node {\( \lambda_2 \)} (u1v4);
\node[state, right=of u2v3] (u2v4) {(2,4)};
\draw[->](u2v3) edge[bend left] node {\( \lambda_1 \)} (u2v4);
\draw[->](u2v4) edge[bend left] node {\(\mu \)} (u2v3);
\draw[->](u1v4) edge node {\( \lambda_2 \)} (u2v4);
\end{tikzpicture}}
    \caption{Example of Markov model with \(C=1, T=2, N=4, M=2\)}
    \label{fig:distribution_of_time_at_specific_state_1_server}
\end{figure}

Consider the Markov model of
Figure~\ref{fig:distribution_of_time_at_specific_state_1_server}
with one server (i.e.
the rate of service completion is \(\mu\) throughout the Markov model)
and a threshold of two individuals.
Assume that a type 1 individual arrives when the model is at state
\((0,3)\), thus forcing the model to move to state \((0,4)\).
The distribution of the time needed for the specified individual to exit the
system from state \((0,4)\) is given by the sum of exponentially distributed
random variables with the same parameter \(\mu\).
The sum of such random variables form the Erlang distribution which is defined
by the number of random variables \(k\) that are added together and their
exponential parameter \(\mu\).

\begin{align}
    & X_i \sim \text{Exp}(\mu) \nonumber \\
    X_1 + X_2 + & \dots + X_k \sim \text{Erlang}(k,\mu)
    \label{eq:erlang_distribution_definition}
\end{align}

Note here that these random variables represent the individual's pathway from
the perspective of the individual.
Thus, \(X_i\) represents the random variable of the time that it takes for an
individual to move from the \(i^{\text{th}}\) position of the queue to the
\((i-1)^{\text{th}}\) position (i.e. for someone in front of them to finish
their service) and \(X_0\) is the time it takes that individual from
starting their service to exiting the system.


\begin{align}
    (0,4) \Rightarrow \quad & X_3 \sim Exp(\mu) \nonumber \\
    (0,3) \Rightarrow \quad & X_2 \sim Exp(\mu) \nonumber \\
    (0,2) \Rightarrow \quad & X_1 \sim Exp(\mu) \nonumber \\
    (0,1) \Rightarrow \quad & X_0 \sim Exp(\mu) \nonumber \\
    S = X_3 + X_2 + & X_1 + X_0 = Erlang(4, \mu)
\end{align}

Thus, the waiting and service time of an individual in the model of
Figure~\ref{fig:distribution_of_time_at_specific_state_1_server}
can be captured by an
erlang distributed random variable.
The general CDF of the erlang distribution \(Erlang(k, \mu)\) is given by:

\begin{equation} \label{eq:cdf_erlang}
    P(S < t) = 1 - \sum_{i=0}^{k-1} \frac{1}{i!} e^{-\mu t} (\mu t)^i
\end{equation}

Unfortunately, the erlang distribution can only be used for the sum of
identically distributed random variables from the exponential distribution.
Therefore, this approach cannot be used when one of the random variables has a
different parameter than the others.
In fact the only case where this can be use is only when the number of servers 
are \(C=1\), similar to the explored example, or when an individual arrives
and goes straight to service (i.e. when there is no other individual waiting
and there is an empty server).


\subsubsection{Distribution of time at a specific state (with multiple servers)}

\begin{figure}[H]
    \centering
    \scalebox{0.75}{\begin{tikzpicture}[-, node distance = 1cm, auto]
\node[state] (u0v0) {(0,0)};
\node[state, right=of u0v0] (u0v1) {(0,1)};
\draw[->](u0v0) edge[bend left] node {\( \Lambda \)} (u0v1);
\draw[->](u0v1) edge[bend left] node {\(\mu \)} (u0v0);
\node[state, right=of u0v1] (u0v2) {(0,2)};
\draw[->](u0v1) edge[bend left] node {\( \Lambda \)} (u0v2);
\draw[->](u0v2) edge[bend left] node {\(2\mu \)} (u0v1);
\node[state, below=of u0v2] (u1v2) {(1,2)};
\draw[->](u0v2) edge[bend left] node {\( \lambda_2 \)} (u1v2);
\draw[->](u1v2) edge[bend left] node {\(2\mu \)} (u0v2);
\node[state, below=of u1v2] (u2v2) {(2,2)};
\draw[->](u1v2) edge[bend left] node {\( \lambda_2 \)} (u2v2);
\draw[->](u2v2) edge[bend left] node {\(2\mu \)} (u1v2);
\node[state, right=of u0v2] (u0v3) {(0,3)};
\draw[->](u0v2) edge[bend left] node {\( \lambda_1 \)} (u0v3);
\draw[->](u0v3) edge[bend left] node {\(2\mu \)} (u0v2);
\node[state, right=of u1v2] (u1v3) {(1,3)};
\draw[->](u1v2) edge[bend left] node {\( \lambda_1 \)} (u1v3);
\draw[->](u1v3) edge[bend left] node {\(2\mu \)} (u1v2);
\draw[->](u0v3) edge node {\( \lambda_2 \)} (u1v3);
\node[state, right=of u2v2] (u2v3) {(2,3)};
\draw[->](u2v2) edge[bend left] node {\( \lambda_1 \)} (u2v3);
\draw[->](u2v3) edge[bend left] node {\(2\mu \)} (u2v2);
\draw[->](u1v3) edge node {\( \lambda_2 \)} (u2v3);
\node[state, right=of u0v3] (u0v4) {(0,4)};
\draw[->](u0v3) edge[bend left] node {\( \lambda_1 \)} (u0v4);
\draw[->](u0v4) edge[bend left] node {\(2\mu \)} (u0v3);
\node[state, right=of u1v3] (u1v4) {(1,4)};
\draw[->](u1v3) edge[bend left] node {\( \lambda_1 \)} (u1v4);
\draw[->](u1v4) edge[bend left] node {\(2\mu \)} (u1v3);
\draw[->](u0v4) edge node {\( \lambda_2 \)} (u1v4);
\node[state, right=of u2v3] (u2v4) {(2,4)};
\draw[->](u2v3) edge[bend left] node {\( \lambda_1 \)} (u2v4);
\draw[->](u2v4) edge[bend left] node {\(2\mu \)} (u2v3);
\draw[->](u1v4) edge node {\( \lambda_2 \)} (u2v4);
\end{tikzpicture}}
    \caption{Example of Markov model with \(C=2, T=2, N=4, M=2\)}
    \label{fig:distribution_of_time_at_specific_state_2_servers}
\end{figure}

Figure~\ref{fig:distribution_of_time_at_specific_state_2_servers} represents the
same Markov model as
Figure~\ref{fig:distribution_of_time_at_specific_state_1_server} with the only
exception that there are 2 servers here.
By applying the same logic, assuming that an individual arrives at state
\((0,4)\), the sum of the following random variables arises.

\begin{align}
    (0,4) \Rightarrow \quad & X_2 \sim Exp(2\mu) \nonumber \\
    (0,3) \Rightarrow \quad & X_1 \sim Exp(2\mu) \\
    (0,2) \Rightarrow \quad & X_0 \sim Exp(\mu) \nonumber
\end{align}

Since these exponentially distributed random variables do not share the same
parameter, an erlang distribution cannot be used.
In fact, the problem can now be viewed as the sum of exponentially
distributed random variables with different parameters, which is in turn the
sum of erlang distributed random variables.
The sum of erlang distributed random variables is said to follow the
hypoexponential distribution.
The hypoexponential distribution is defined with two vectors of size equal
to the number of Erlang random variables that are added
together~\cite{Akkouchi2008, Smaili2013}.
For this particular example:

\footnotesize
\begin{align} \label{eq:multiple_servers_distribution_example}
    \begin{rcases}
        \begin{rcases}
            X_2 \sim Exp(2\mu) \\
            X_1 \sim Exp(2\mu)
        \end{rcases}
        X_1 + X_2 = S_1 \sim Erlang(2, 2\mu) \\
        X_0 \sim Exp(\mu) \Rightarrow \hspace{1cm} X_0 = S_2 \sim Erlang(1, \mu)
    \end{rcases}
    S_1 + S_2 = H \sim Hypo((2,1), (2\mu, \mu))
\end{align}
\normalsize

The random variable \(H\) from
equation~\eqref{eq:multiple_servers_distribution_example} follows a
hypoexponential
distribution with two vector parameters (\((2,1)\) and \((2\mu, \mu)\)).
The CDF of this distribution can be therefore used to get the probability of the
time spent in the system being less than a given target.
The CDF of the general hypoexponential distribution \(Hypo(\vec{r},
\vec{\lambda})\), is given by the~\eqref{eq:general_cdf_hypoexponential}, where
vector \(\vec{r}\) contains all \(k\)-values of the erlang distributions
defined in~\eqref{eq:erlang_distribution_definition} and \(\vec{\lambda}\)
is a vector of the distinct parameters~\cite{Favaro2010}.

\begin{align} \label{eq:general_cdf_hypoexponential}
    & P(H < t) = 1 - \left( \prod_{j=1}^{\mid \vec{r} \mid} \lambda_j^{r_j}
    \right) \sum_{k=1}^{\mid \vec{r} \mid} \sum_{l=1}^{r_k}
    \frac{\Psi_{k,l}(-\lambda_k)t^{r_k - l} e^{-\lambda_k t}}
    {(r_k - l)! (l - 1)!} \nonumber \\
    & \textbf{where} \qquad \Psi_{k,l}(t) = - \frac{\partial^{l - 1}}
    {\partial t ^{l - 1}} \left( \prod_{j = 0, j \neq k}^{\mid \vec{r} \mid}
    (\lambda_j + t)^{-r_j} \right) \nonumber \\
    & \textbf{and} \quad \qquad \lambda_0 = 0, r_0 = 1
\end{align}


The computation of the derivative makes
equation~\eqref{eq:general_cdf_hypoexponential} computationally expensive.
In~\cite{Legros2015} an alternative linear version of that CDF is explored via
matrix analysis, and is given by the following formula:

\small
\begin{equation} \label{eq:linear_general_cdf_hypoexponential}
    \begin{split}
        F(x) = &1 - \sum_{k=1}^{n} \sum_{l=0}^{k-1} (-1)^{k-1} \binom{n}{k}
            \binom{k-1}{l} \sum_{j=1}^{n} \sum_{s=1}^{j-1} e^{-x \lambda_s}
            \prod_{l=1}^{s-1} \left( \frac{\lambda_l}{\lambda_l - \lambda_s} \right)
            ^ {k_s} \\
        & \times \sum_{s < a_1 < \dots < a_{l-1} < j}
            \left( \frac{\lambda_s}{\lambda_s - \lambda_{a_1}} \right) ^ {k_s}
            \prod_{m=s+1}^{a_1-1} \left( \frac{\lambda_m}{\lambda_m -
            \lambda_{a_1}}\right) ^ {k_m}
            \prod_{n=a_1}^{a_2-1} \left( \frac{\lambda_n}{\lambda_n -
            \lambda_{a_2}}\right) ^ {k_n} \\
        & \dots \prod_{r=a_l-1}^{j-1} \left( \frac{\lambda_r}{\lambda_r -
            \lambda_{a_j}}\right) ^ {k_r}
            \sum_{q=0}^{k_s - 1} \frac{((\lambda_s - \lambda_{a_1})x)^q}{q!}, \\
        & \text{for } \geq 0
    \end{split}
\end{equation}
\normalsize

Although equation~\eqref{eq:linear_general_cdf_hypoexponential} is a simplified
version of equation~\eqref{eq:general_cdf_hypoexponential} it still has some
unnecessary complexity.
The described expressions are general expressions used to get the CDF of
the hypoexponential distribution, which is in turn the sum of multiple
Erlang distributed random variables.
However, the random variable \(H\), described
in~\eqref{eq:multiple_servers_distribution_example}, is the sum of only two
different erlang distributed random variables.
Thus, perhaps a more simplified version of the above expressions can be derived
that is specific to the case of two erlang distributed random variables.


\subsubsection{Specific CDF of hypoexponential distribution}
Equations~\eqref{eq:general_cdf_hypoexponential}
and~\eqref{eq:linear_general_cdf_hypoexponential} refers to the general CDF
of the
hypoexponential distribution where the size of the vector parameters can be of
any size~\cite{Favaro2010}.
In the Markov chain models described in
Figures~\ref{fig:distribution_of_time_at_specific_state_1_server} and
\ref{fig:distribution_of_time_at_specific_state_2_servers} the parameter vectors
of the hypoexponential distribution are of size two, and in fact, for any
possible version of the investigated Markov chain model the vectors can only be
of size two.
This is true since for any dimensions of this Markov chain model there will
always be at most two distinct exponential parameters; the parameter for
finishing a service (\(\mu\)) and the parameter for moving forward in the queue
(\(C \mu\)).
For the unique case of \(C=1\) the hypoexponential distribution will not be
used as this is equivalent to an erlang distribution.
Therefore, by fixing the sizes of \(\vec{r}\) and \(\vec{\lambda}\) to 2, the
following specific expression for the CDF of the hypoexponential distribution
arises, where the derivative is removed:


\begin{align} \label{eq:specific_cdf_hypoexponential}
    & P(H < t) = 1 - \left( \prod_{j=1}^{\mid \vec{r} \mid} \lambda_j^{r_j}
    \right) \sum_{k=1}^{\mid \vec{r} \mid} \sum_{l=1}^{r_k}
    \frac{\Psi_{k,l}(-\lambda_k)t^{r_k - l} e^{-\lambda_k t}}{(r_k - l)!
    (l - 1)!} \nonumber \\
    & \textbf{where} \qquad \Psi_{k,l}(t) =
    \begin{cases}
        \frac{(-1)^{l} (l-1)!}{\lambda_2} \left[\frac{1}{t^l} - \frac{1}
        {(t + \lambda_2)^l}\right] , & k=1 \\
        - \frac{1}{t (t + \lambda_1)^{r_1}}, & k=2
    \end{cases} \nonumber \\
    & \textbf{and} \quad \qquad \lambda_0 = 0, r_0 = 1
\end{align}

Note here that the only difference between
equation~\eqref{eq:general_cdf_hypoexponential}
and~\eqref{eq:specific_cdf_hypoexponential} is the \(\Psi\), where it is now
only computed for \(k=1,2\).
The following subsection proves the following expression:

\begin{equation} \label{eq:hypoexponential_expression_to_proof}
    - \frac{\partial^{l - 1}}{\partial t ^{l - 1}}
    \left(
        \prod_{j = 0, j \neq k}^{\mid \vec{r} \mid} (\lambda_j + t)^{-r_j}
    \right) =
    \begin{cases}
        \frac{(-1)^{l} (l-1)!}{\lambda_2} \left[\frac{1}{t^l} - \frac{1}
        {(t + \lambda_2)^l}\right] , & k=1 \\
        - \frac{1}{t (t + \lambda_1)^{r_1}}, & k=2
    \end{cases}
\end{equation}



\subsubsection{Proof of specific hypoexponential distribution
(eq.~\eqref{eq:hypoexponential_expression_to_proof})}

This section aims to show that there exists a simplified expression of
equation~\eqref{eq:general_cdf_hypoexponential} that is specific to the proposed
Markov model.
Function \(\Psi\) is defined using the parameter \(t\) and the variables \(k\)
and \(l\).
Given the Markov model, the range of values that \(k\) and \(l\) can take can be
bounded.
First, from the range of the double summation in
equation~\eqref{eq:general_cdf_hypoexponential}, it can be seen that
\(k = 1, 2, \dots, \mid \vec{r} \mid\).
Now, \(\mid \vec{r} \mid\) represents the size of the parameter vectors that,
for the Markov model, will always be 2.
That is because, for all the exponentially distributed random variables that are
added together to form the new distribution, there only two distinct parameters,
thus forming two erlang distributions. Therefore:

\begin{equation*}
    k = 1, 2
\end{equation*}

By observing equation~\eqref{eq:general_cdf_hypoexponential} once more, the
range of values that \(l\) takes are \(l = 1, 2, \dots, r_k\), where \(r_1\) is
subject to the individual's position in the queue and \(r_2 = 1\).
In essence, the hypoexponential distribution will be used with these bounds:

\begin{align}
    k = 1 & \qquad \Rightarrow \qquad l = 1, 2, \dots, r_1 \nonumber \\
    k = 2 & \qquad \Rightarrow \qquad l = 1
\end{align}

Thus the left hand side of equation~\eqref{eq:hypoexponential_expression_to_proof}
needs only to be defined for these bounds.
The specific hypoexponential distribution investigated here is of the form:
\[
    Hypo((r_1, 1)(\lambda_1, \lambda_2))
\]
Note the initial conditions \(\lambda_0=0, r_0=1\) defined in
equation~\eqref{eq:general_cdf_hypoexponential} also hold here.
Thus the proof is split into two parts, for \(k=1\) and \(k=2\).



\begin{itemize}
    \item \(k = 2, l = 1\)
    \begin{equation*}
        \begin{split}
            LHS &= - \frac{\partial^{1-1}}{\partial t^{1-1}}
            \left( \prod_{j=0, j \neq 2}^{2} (\lambda_j + t)^{-r_j} \right) \\
            &=-\left( (\lambda_0 + t)^{-r_0} \times (\lambda_1 + t)^{-r_1}
            \right) \\
            &=-\left( t^{-1} \times (\lambda_1 + t)^{-r_1} \right) \\
            &= - \frac{1}{t(t + \lambda_1)^{r_1}} \\
            & \hspace{7cm} \square
        \end{split}
    \end{equation*}
    \item \(k = 1, l = 1, \dots, r_1\)
    \begin{equation*}
        \begin{split}
            LHS &= -\frac{\partial^{l-1}}{\partial t^{l-1}}
            \left( \prod_{j=0, j \neq 1}^{2} (\lambda_j + t)^{-r_j} \right) \\
            &= -\frac{\partial^{l-1}}{\partial t^{l-1}}
            \left( (\lambda_o + t)^{-r_0} \times (\lambda_2 + t)^{-r_2}
            \right) \\
            &= -\frac{\partial^{l-1}}{\partial t^{l-1}}
            \left( \frac{1}{t(t + \lambda_2)}\right)
        \end{split}
    \end{equation*}

    In essence, the final part of the proof is to show that:

    \[
        -\frac{\partial^{l-1}}{\partial t^{l-1}}
        \left( \frac{1}{t(t + \lambda_2)}\right) =
        \frac{(-1)^{l} (l-1)!}{\lambda_2}\left[\frac{1}{t^l} - \frac{1}{(t +
        \lambda_2)^l}\right]
    \]

    \textbf{Proof by Induction:}
    \begin{enumerate}
        \item Base case (\(l=1\)):
        \begin{equation*}
            \begin{split}
                LHS &= -\frac{\partial^{1-1}}{\partial t^{1-1}}
                \left( \frac{1}{t(t + \lambda_2)}\right) =
                - \frac{1}{t(t + \lambda_2)} \\
                RHS &= \frac{(-1)^{1} (1-1)!}{\lambda_2}
                \left[\frac{1}{t^1} - \frac{1}{(t + \lambda_2)^1}\right] \\
                &= - \frac{t + \lambda_2 - t}{\lambda_2 t (t + \lambda_2)} \\
                &= - \frac{1}{t (t + \lambda_2)} \\
                LHS &= RHS
            \end{split}
        \end{equation*}
        \item Assume true for \(l = x\):
        \begin{equation*}
            -\frac{\partial^{x-1}}{\partial t^{x-1}}
            \left( \frac{1}{t(t + \lambda_2)}\right) =
            \frac{(-1)^{x} (x-1)!}{\lambda_2}
            \left[\frac{1}{t^x} - \frac{1}{(t + \lambda_2)^x}\right]
        \end{equation*}
        \item Prove true for \(l = x + 1\):
        
        \small
        \[
            \Bigg( \text{Show that:} \hspace{0.5cm}
            \frac{\partial^x}{\partial t ^ x}
            \left( \frac{-1}{t (t + \lambda_2)} \right) =
            \frac{(-1)^{x + 1} (x)!}{\lambda_2}
            \left[\frac{1}{t^{x+1}}-\frac{1}{(t + \lambda_2)^{x+1}}\right]\Bigg)
        \]
        \normalsize

        \begin{equation*}
            \begin{split}
                LHS &= \frac{\partial}{\partial t}
                \left[ \frac{\partial^{x-1}}{\partial t ^ {x-1}}
                \left( \frac{-1}{t (t + \lambda_2)} \right) \right] \\
                &= \frac{\partial}{\partial t} \left[
                    \frac{(-1)^x (x-1)!}{\lambda_2} \left(
                        \frac{1}{t^x} - \frac{1}{(t + \lambda_2)^x}
                    \right)
                \right] \\
                &= \frac{(-1)^x (x-1)!}{\lambda_2} \left(
                    \frac{(-x)}{t^{x+1}} - \frac{(-x)}{(t + \lambda_2)^x}
                \right) \\
                &= \frac{(-1)^x (x-1)! (-x)}{\lambda_2} \left(
                    \frac{1}{t^{x+1}} - \frac{1}{(t + \lambda_2)^x}
                \right) \\
                &= \frac{(-1)^{x+1} (x)!}{\lambda_2} \left(
                    \frac{1}{t^{x+1}} - \frac{1}{(t + \lambda_2)^x}
                \right) \\
                & = RHS \\
                & \hspace{7cm} \square
            \end{split}
        \end{equation*}
    \end{enumerate}
\end{itemize}

\subsubsection{Proportion within target for type 1 and type 2 individuals}

Given the two CDFs of the Erlang and Hypoexponential distributions
(equations~\eqref{eq:erlang_distribution_definition}
and~\eqref{eq:specific_cdf_hypoexponential}) a new function has to be defined to
decide which one to use between the two.
Based on the state of the model, there can be three scenarios when an individual
arrives.
\begin{enumerate}
    \item There is a free server and the individual does not have to wait
    \begin{equation*}
        X_{(u,v)} \sim Erlang(1, \mu)
    \end{equation*}
    \item The individual arrives at the queue at the \(n^{th}\) position and the
    model has \(C > 1\) servers
    \begin{equation*}
        X_{(u,v)} \sim Hypo((n, 1), (C \mu, \mu))
    \end{equation*}
    \item The individual arrives at a queue at the \(n^{th}\) position and the
    model has \(C = 1\) servers
    \begin{equation*}
        X_{(u,v)} \sim Erlang(n + 1, \mu)
    \end{equation*}
\end{enumerate}

Note here that for the first case \(Erlang(1, \mu)\) is equivalent to
\(Exp(\mu)\).
Define \(X_{(u,v)}^{(1)}\) as the distribution of type 1 individuals and
\(X_{(u,v)}^{(2)}\) as the distribution of type 2 individuals, when arriving at
state \((u,v)\) of the model.

\small
\begin{equation}
    X_{(u,v)}^{(1)} \sim
    \begin{cases}
        \textbf{Erlang}(v, \mu), & \textbf{if } C = 1 \textbf{ and } v>1 \\
        \textbf{Hypo}\left(\vec{r}=(v - C, 1), \vec{\lambda}=(C \mu, \mu)\right),
            & \textbf{if } C > 1 \textbf{ and } v>C \\
        \textbf{Erlang}(1, \mu), & \textbf{if } v \leq C
    \end{cases}
\end{equation}

\begin{equation}
    X_{(u,v)}^{(2)} \sim
    \begin{cases}
        \textbf{Erlang}(\min(v, T), \mu), & \textbf{if } C = 1
            \textbf{ and } v, T > 1 \\
        \textbf{Hypo}\left(\vec{r}=(\min(v, T) - C, 1), \vec{\lambda}=(C \mu, \mu)\right), &
            \textbf{if } C > 1 \textbf{ and } v, T  > C \\
        \textbf{Erlang}(1, \mu), & \textbf{if } v \leq C \textbf{ or } T \leq C
    \end{cases}
\end{equation}
\normalsize


Equations~\eqref{eq:cdf_erlang} and~\eqref{eq:specific_cdf_hypoexponential} can
now be used.
Therefore, the probability that an individual arriving at a specific state is
within a given time target \(t\) is given by the following formulas:


\footnotesize
\begin{equation}
    P(X_{(u,v)}^{(1)} < t) =
    \begin{cases}
        1 - \sum_{i=0}^{v-1} \frac{1}{i!} e^{-\mu t} (\mu t)^i,
            & \textbf{if } C = 1 \\
        & \textbf{and } v > 1 \\
        & \\
        1 - (\mu C)^{v-C} \mu
            \sum_{k=1}^{\mid \vec{r} \mid} \sum_{l=1}^{r_k}
            \frac{\Psi_{k,l}(-\lambda_k)t^{r_k - l}
            e^{-\lambda_k t}}{(r_k - l)! (l - 1)!},
            & \textbf{if } C > 1 \\
        \textbf{where } \vec{r}=(v - C, 1) \textbf{ and }
            \vec{\lambda}=(C \mu, \mu) & \textbf{and } v > C \\
        & \\
        1 - e^{-\mu t},  & \textbf{if } v \leq C
    \end{cases}
\end{equation}

\begin{equation}
    P(X_{(u,v)}^{(2)} < t) =
    \begin{cases}
        1 - \sum_{i=0}^{\min(v,T)-1} \frac{1}{i!} e^{-\mu t} (\mu t)^i,
            & \textbf{if } C = 1 \\
        & \textbf{and } v, T > 1 \\
        & \\
        1 - (C \mu ) ^ {\min(v,T) - C} \mu
            \sum_{k=1}^{\mid \vec{r} \mid} \sum_{l=1}^{r_k}
            \frac{\Psi_{k,l}(-\lambda_k)t^{r_k - l}
            e^{-\lambda_k t}}{(r_k - l)! (l - 1)!},
            & \textbf{if } C > 1 \\
        \textbf{where } \vec{r}=(\min(v, T) - C, 1) \textbf{ and }
            \vec{\lambda}=(C \mu, \mu) & \textbf{and } v, T  > C \\
        & \\
        1 - e^{-\mu t}, & \textbf{if } v \leq C \\
        & \textbf{or } T \leq C \\
    \end{cases}
\end{equation}
\normalsize

In addition the set of accepting states for type 1 (\(S_A^{(1)}\)) and type 2
(\(S_A^{(2)}\)) individuals defined in
equations~\eqref{eq:accepting_states_type_1}
and~\eqref{eq:accepting_states_type_2} are also needed here.
Note here that, \(S\) denotes the set of all states of the Markov chain model.


Having defined everything, a formula similar to the ones
of equations~\eqref{eq:recursive_waiting_time_for_type_i}
and~\eqref{eq:example_direct_approach_blocking_time} can be generated.
The following formula uses the state probability vector \(\pi\) to get the
weighted average of the probability below target of all states in the Markov
model.

\begin{equation}\label{eq:proportion_of_inds_type_1}
    P(X^{(1)} < t) = \frac{\sum_{(u,v) \in S_A^{(1)}} P(X_{u,v}^{(1)} < t)
    \pi_{u,v} }{\sum_{(u,v) \in S_A^{(1)}} \pi_{u,v}}
\end{equation}

\begin{equation}\label{eq:proportion_of_inds_type_2}
    P(X^{(2)} < t) = \frac{\sum_{(u,v) \in S_A^{(2)}} P(X_{u,v}^{(2)} < t)
    \pi_{u,v} }{\sum_{(u,v) \in S_A^{(2)}} \pi_{u,v}}
\end{equation}


\subsubsection{Overall proportion within target}

The overall proportion of individuals for both type 1 and type 2 individuals
is given by the equivalent formula of
equations~\eqref{eq:overall_waiting_time_coeff}
and~\eqref{eq:overall_waiting_time}.
The following formula uses the probability of lost individuals from both types
to get the weighted sum of the two already existing probabilities.

\begin{equation*}
    P(L'_1) = \sum_{(u,v) \, \in S_A^{(1)}} \pi(u,v), \hspace{1.5cm}
    P(L'_2) = \sum_{(u,v) \, \in S_A^{(2)}} \pi(u,v)
\end{equation*}

\small
\begin{equation}\label{eq:overall_proportion_within_target}
    P(X < t)= \frac{\lambda_1 P(L'_1)}{\lambda_2 P(L'_2)+\lambda_1 P(L'_1)}
    P(X^{(1)} < t) + \frac{\lambda_2 P(L'_2)}{\lambda_2 P(L'_2) + \lambda_1
    P(L'_1)} P(X^{(2)} < t)
\end{equation}
\normalsize

\subsubsection{Implementation}\label{sec:implementation_proportion_individuals}

This section focuses on the implementation of all necessary equations to 
calculate the proportion of individuals within target as described
in Section~\ref{sec:proportion_of_individuals_within_time}.
The first equation to be considered is the simplified version of
\(\Psi_{k, \lambda}(t)\) described in
equation~\eqref{eq:specific_cdf_hypoexponential}.
Code snippet~\ref{lst:specific_psi_function} shows the implementation of this
equation in python.

\begin{lstlisting}[
    style=pystyle,
    caption={Function for the simplified version of \(\Psi_{k, \lambda}(t)\)},
    label={lst:specific_psi_function},
]
>>> def specific_psi_function(
...     arg, k, l, exp_rates, freq, a
... ):
...     """
...     The specific version of the Psi function that is used for the
...     purpose of this study. Due to the way the hypoexponential cdf
...     works the function is called only for values of k=1 and k=2. 
...     For these values the following hold:
...         - k = 1 -> l = 1, ..., n
...         - k = 2 -> l = 1
...     """
...     if k == 1:
...         psi_val = (1 / (arg**l)) - (1 / (arg + exp_rates[2]) ** l)
...         psi_val *= (-1) ** l * math.factorial(l - 1) / exp_rates[2]
...         return psi_val
...     if k == 2:
...         psi_val = -1 / (arg * (arg + exp_rates[1]) ** freq[1])
...         return psi_val
...     return 0

\end{lstlisting}

The piece of code shown in~\ref{lst:hypoexponential_cdf} returns the cumulative
distribution function of the hypoexponential distribution.
In essence this is the value of \(P(H<t)\) outlined in
equation~\eqref{eq:specific_cdf_hypoexponential}.

\begin{lstlisting}[
    style=pystyle,
    caption={Function for the cumulative distribution of the Hypoexponential
    distribution},
    label={lst:hypoexponential_cdf},
]
>>> def hypoexponential_cdf(
...     x, exp_rates, freq, psi_func=specific_psi_function
... ):
...     """
...     The function represents the cumulative distribution function of the
...     hypoexponential distribution. It calculates the probability that a
...     hypoexponentially distributed random variable has a value less than
...     x. In other words calculate P(S < x) where S ~ Hypo(lambda, r)
...     where: lambda is a vector with distinct exponential parameters and 
...     r is a vector with the frequency of each distinct parameter
...     Note that: a Hypoexponentially distributed random variable can be
...     described as the sum of Erlang distributed random variables
...     Parameters
...     ----------
...     x : float
...         The target we want to calculate the probability for
...     exp_rates : tuple
...         The distinct exponential parameters
...     freq : tuple
...         The frequency of the exponential parameters
...     psi_func : function, optional
...         The function to be used to get Psi, by default
...         specific_psi_function
...     Returns
...     -------
...     float
...         P(S < x) where S ~ Hypo(lambda, r)
...     """
...     a = len(exp_rates)
...     exp_rates = (0,) + exp_rates
...     freq = (1,) + freq
...     summation = 0
...     for k in range(1, a + 1):
...         for l in range(1, (freq[k] + 1)):
...             psi = psi_func(
...                 arg=-exp_rates[k],
...                 k=k,
...                 l=l,
...                 exp_rates=exp_rates,
...                 freq=freq,
...                 a=a,
...             )
...             iteration = (
...                 psi * (x ** (freq[k] - l)) * np.exp(-exp_rates[k] * x)
...             )
...             iteration /= (
...                 np.math.factorial(freq[k] - l)*np.math.factorial(l - 1))
...             summation += float(iteration)
...     output = 1 - (
...         product_of_all_elements(
...             [exp_rates[j] ** freq[j] for j in range(1, a + 1)]
...         ) * summation
...     )
...     return output

\end{lstlisting}

Similarly the cumulative distribution function of the erlang distribution is
also needed here.
The code snippet in~\ref{lst:erlang_cdf} shows the implementation of this
function as described in equation~\eqref{eq:cdf_erlang}.

\begin{lstlisting}[
    style=pystyle,
    caption={Function for the cumulative distribution of the Erlang
    distribution},
    label={lst:erlang_cdf},
] 
>>> def erlang_cdf(mu, n, x):
...     """
...     Cumulative distribution function of the erlang distribution.
...     P(X < x) where X ~ Erlang(mu, n)
...     Parameters
...     ----------
...     mu : float
...         The parameter of the Erlang distribution
...     n : int
...         The number of Exponential distributions that are added together
...     x : float
...         The argument of the function
...     Returns
...     -------
...     float
...         The probability that the erlang distributed r.v. is less than x
...     """
...     return 1 - np.sum([
...         np.math.exp(-mu * x) * (mu * x) ** i
...         * (1 / np.math.factorial(i))
...         for i in range(n)
...     ])

\end{lstlisting}

Having defined all functions necessary the code snippet
in~\ref{lst:distribution_decision_function} chooses which of the two 
distributions to use to calculate the probability of an individual being within
a given time target.

\begin{lstlisting}[
    style=pystyle,
    caption={Function for deciding which distribution to use to calculate
    the probability of an individual being within a given time target.},
    label={lst:distribution_decision_function},
]
>>> def get_probability_of_waiting_time_in_system_less_than_target_for_state(
...     state,
...     class_type,
...     mu,
...     num_of_servers,
...     threshold,
...     target,
...     psi_func=specific_psi_function,
... ):
...     """
...     The function decides what probability distribution to use based on
...     the state we are currently on and the class type given. The two
...     distributions that are used are the Erlang and the Hypoexponential
...     distribution. The time it takes the system to exit a state and enter
...     the next one is known to be exponentially distributed. The sum of
...     exponentially distributed random variables is known to result in
...     either an Erlang distribution or a Hypoexponential distribution
...     (where the former is used when the exponentially distributed r.v.
...     that we are summing have the same parameters and the latter when
...     they have at least two distinct parameters). The function works as
...     follows:
...     - Checks whether the arriving individual will have to wait
...     - Finds the total number of states an individual will have to visit
...     - Depending on whether the parameters of the distributions to sum are
...     the same or not, call the appropriate cdf function.
...     Parameters
...     ----------
...     state : tuple
...     class_type : int
...     mu : float
...     num_of_servers : int
...     threshold : int
...     target : int
...     psi_func : function, optional
...     Returns
...     -------
...     float
...         The probability of spending less time than the target in the
...         system when the individual has arrived at a given state
...     """
...     if class_type == 0:
...         arrive_on_waiting_space = state[1] > num_of_servers
...         rep = state[1] - num_of_servers
...     elif class_type == 1:
...         arrive_on_waiting_space = (
...             state[1] > num_of_servers and threshold > num_of_servers
...         )
...         rep = min(state[1], threshold) - num_of_servers
...     else:
...         raise ValueError("Class type bust be 0 or 1")
... 
...     if arrive_on_waiting_space:
...         if num_of_servers == 1:
...             prob = erlang_cdf(mu=mu, n=rep + 1, x=target)
...         else:
...             param = num_of_servers * mu
...             prob = hypoexponential_cdf(
...                 x=target,
...                 exp_rates=(param, mu),
...                 freq=(rep, 1),
...                 psi_func=psi_func,
...             )
...     else:
...         prob = erlang_cdf(mu=mu, n=1, x=target)
...     return prob

\end{lstlisting}

Finally, putting everything together, going through all the states of a given
Markov chain model, the function defined in~\ref{lst:proportion_within_target}
calculates the probability of
spending less time than the target in the system for a given individual type.
This corresponds to both equations~\eqref{eq:proportion_of_inds_type_1}
and~\eqref{eq:proportion_of_inds_type_2}.

\begin{lstlisting}[
    style=pystyle,
    caption={Function for calculating the probability of spending less time
    than the target in the system for a given individual type.},
    label={lst:proportion_within_target},
]
>>> def get_proportion_of_individuals_within_time_target(
...     all_states,
...     pi,
...     class_type,
...     mu,
...     num_of_servers,
...     threshold,
...     system_capacity,
...     buffer_capacity,
...     target,
...     psi_func=specific_psi_function,
...     **kwargs,
... ):
...     """
...     Gets the probability that a certain class of individuals is within a
...     given time target. This functions runs for every state the function
...     get_probability_of_waiting_time_in_system_less_than_target_for_state
...     and by using the state probabilities to get the average proportion of
...     individuals within target.
...     Parameters
...     ----------
...     all_states : list
...     pi : numpy.array
...     class_type : int
...     mu : float
...     num_of_servers : int
...     threshold : int
...     system_capacity : int
...     buffer_capacity : int
...     target : float
...     psi_func : function, optional
...     Returns
...     -------
...     float
...         The probability of spending less time than the target in the
...         system
...     """
...     proportion_within_limit = 0
...     probability_of_accepting = 0
...     for (u, v) in all_states:
...         if abg.markov.utils.is_accepting_state(
...             state=(u, v),
...             class_type=class_type,
...             threshold=threshold,
...             system_capacity=system_capacity,
...             buffer_capacity=buffer_capacity,
...         ):
...             arriving_state = (u, v + 1)
...             if class_type == 1 and v >= threshold:
...                 arriving_state = (u + 1, v)
... 
...             proportion_within_limit_at_state = (
...                 get_probability_of_waiting_time_in_system_less_than_target_for_state(
...                     state=arriving_state,
...                     class_type=class_type,
...                     mu=mu,
...                     num_of_servers=num_of_servers,
...                     threshold=threshold,
...                     target=target,
...                     psi_func=psi_func,
...                 )
...             )
...             proportion_within_limit += (
...                 pi[u, v] * proportion_within_limit_at_state
...             )
...             probability_of_accepting += pi[u, v]
...     return proportion_within_limit / probability_of_accepting

\end{lstlisting}

Using all functions created so far, the proportion of individuals within target
can be calculated for a given Markov chain model and a given individual type.

\begin{lstlisting}[
    style=pystyle,
    caption={Combining all functions to calculate the proportion of type 1 
    individuals within a time target of \(1\) time unit.},
    label={lst:proportion_within_target_example},
]
>>> import ambulance_game as abg
>>> import numpy as np
>>> all_states = abg.markov.build_states(
...     threshold=2,
...     system_capacity=4,
...     buffer_capacity=3,
... )
>>> Q = abg.markov.get_transition_matrix(
...     lambda_1=1,
...     lambda_2=1,
...     mu=4,
...     num_of_servers=1,
...     threshold=2,
...     system_capacity=4,
...     buffer_capacity=3
... )
>>> pi = abg.markov.get_markov_state_probabilities(
...     abg.markov.get_steady_state_algebraically(
...         Q, algebraic_function=np.linalg.solve
...     ),
...     all_states,
... )
>>> round(
...     get_proportion_of_individuals_within_time_target(
...         all_states=all_states,
...         pi=pi,
...         class_type=0,
...         mu=4,
...         num_of_servers=1,
...         threshold=2,
...         system_capacity=3,
...         buffer_capacity=4,
...         target=1
...     ), 10
... )
0.9190401179

\end{lstlisting}

This shows that for the given set of parameters and for type 1 individuals
the probability of spending less than 1 unit of time in the system is 91.9\%.
