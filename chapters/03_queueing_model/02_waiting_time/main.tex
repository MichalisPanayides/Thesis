\subsubsection{Waiting time}\label{sec:waiting_time}
Waiting time is the amount of time that individuals wait in node 1 before
they start their service.
For a given set of parameters there are three different performance measures
around the mean waiting time that need to be considered.
The mean waiting time of type 1 individuals, the mean waiting time of type 2
individuals, and the overall mean waiting time.

\paragraph{Recursive approach}\label{sec:recursive_waiting_time}

The first approach to be considered is a recursive approach to getting the
mean waiting time.
To calculate the mean waiting time of type 1 individuals one must first
identify the set of states \((u, v)\) where a wait can occur.
For this particular Markov chain, these are all states that satisfy \(v > C\)
i.e. all states where the number of individuals in the node 1 exceed
the number of servers.
The set of such states is defined as the \textit{waiting states} and can be
denoted as a subset of all the states, where:

\begin{equation} \label{eq:waiting_states}
    S_w = \{(u, v) \in S \; | \; v > C \}
\end{equation}

Moreover, another element that needs to be considered is the expected waiting
time spent in each state for type \(i\) individuals.
In order to do so a variation of the Markov model has to be considered where
arrivals are removed.
Figure~\ref{fig:markov_variation_no_arrivals} shows this new Markov chain model.

\begin{figure}[h]
    \centering
    
\begin{tikzpicture}[-, node distance = 1.4cm, auto, every node/.style={scale=0.85}]

    \tikzmath{
        let \minsize = 1.8cm;
        let \nosize = 0.1cm;
    }

    \node[draw=none, minimum size=\nosize] (one) {};
    \node[state, minimum size=\minsize, right=of one] (two) {(0,T-1)};
    \node[state, minimum size=\minsize, right=of two] (three) {(0,T)};
    \node[state, minimum size=\minsize, right=of three] (four) {(0,T+1)};
    \node[draw=none, minimum size=\minsize, right=of four] (five) {};

    \node[state, minimum size=\minsize, below=of three] (three_one) {(1,T)};
    \node[state, minimum size=\minsize, below=of three_one] (three_two) {(2,T)};
    \node[state, minimum size=\minsize, below=of four] (four_one) {(1,T+1)};
    \node[state, minimum size=\minsize, below=of four_one] (four_two) {(2,T+1)};
    \node[draw=none, minimum size=\nosize, right=of four_one] (five_one) {};
    \node[draw=none, minimum size=\nosize, right=of four_two] (five_two) {};
    \node[draw=none, minimum size=\nosize, below=of three_two] (three_three) {};

    \draw[every loop]
        (two) edge node {\((T-1) \mu\)} (one)
        (three) edge node {\(T \mu\)} (two)
        (four) edge node {\((T+1) \mu\)} (three)
        (five) edge node {\((T+2) \mu\)} (four)
        (three_one) edge node {\(T \mu\)} (three)

        (four_one) edge node {\((T+1) \mu\)} (three_one)
        (five_one) edge node {\((T+2) \mu\)} (four_one)
        (three_two) edge node {\(T \mu\)} (three_one)
        (three_three) edge node {\(T \mu\)} (three_two)
        (four_two) edge node {\((T+1) \mu\)} (three_two)
        (five_two) edge node {\((T+2) \mu\)} (four_two)
        ;
\end{tikzpicture}

    \caption{Variation of Markov chain model where all arrivals are removed}
    \label{fig:markov_variation_no_arrivals}
\end{figure}

For this particular Markov chain variation the expected waiting time spent at
each state \((u,v)\) for type \(i\) individuals is denoted by \(c_w^i(u,v)\).
From a type 1 individual's perspective, when they arrive at the system, no
matter how many other individuals of either type arrive after them it will not
affect their own waiting time.
The desired waiting time by acquired by calculating the inverse of the sum of
the out-flow rate of that state.
Therefore by eliminating the arrival rates of both individuals, the waiting time
at each state for type 1 individuals can be expressed as:

\begin{equation}\label{eq:waiting_time_state_type_1}
    c^{(1)}_w(u,v) =
    \begin{cases}
        0, & \textbf{if } u > 0 \textbf{ and } v = T \\
        \frac{1}{\text{min}(v,C)\mu}, & \textbf{otherwise}
    \end{cases}
\end{equation}

Note here that whenever any type 1 individual is at a state \((u,v)\) where
\(u > 0\) and \(v = T\) (i.e. all states \((1,T), (2,T) \dots, (M,T)\)) the
expected waiting time is set to \(0\).
This is done to capture the trip thorough the Markov chain from the perspective
of type 1 individuals.
Meaning that individuals visit all states of the threshold column but only the
one in the first row will return a non-zero waiting time.
Additionally, in equation (\ref{eq:waiting_time_state_type_1}) the service rate
\(\mu\) is multiplied by the minimum of \(v\) and \(C\) since, when the system
is at a state \((u,v)\) where \(v \geq C\), the maximum out-flow service rate
of \( C \mu \) is reached.


Similarly from a type 2 individual's perspective the same logic holds.
The only difference is that type 2 individuals cannot have a waiting time when
they are blocked in node 2.
From the Markov chain model's perspective, type 2 individuals cannot have a wait
whenever they are at state \(u, v\) where \(u > 0\).
Thus, the waiting time at each state for type 2 individuals can be expressed as:

\begin{equation}\label{eq:waiting_time_state_type_2}
    c^{(2)}_w(u,v) =
    \begin{cases}
        0, & \textbf{if } u > 0 \\
        \frac{1}{\text{min}(v, C)\mu}, & \textbf{otherwise}
    \end{cases}
\end{equation}


Thus using the set of waiting states defined in (\ref{eq:waiting_states}) and
equations (\ref{eq:waiting_time_state_type_1}) and
(\ref{eq:waiting_time_state_type_2}) the following recursive formula can be used
to get the mean waiting time spent in each state.
The formula goes through all states from right to left recursively and adds the
total expected waiting time of all these states together until it reaches a
state that is not in the set of waiting states.
Thus, the expected waiting time of a type \(i\) individual when they arrive at
state \( (u,v) \) is given by:

\begin{equation} \label{eq:recursive_waiting_time_for_state}
    w^{(i)}(u,v) =
    \begin{cases}
        0, \hspace{4.85cm} & \textbf{if } (u,v) \notin S_w \\
        c^{(i)}_w(u,v) + w^{(i)}(u-1, v), & \textbf{if } u > 0 \textbf{ and } v = T \\
        c^{(i)}_w(u,v) + w^{(i)}(u, v-1), & \textbf{otherwise}
    \end{cases}
\end{equation}

Whenever the system is at state \((u,v)\) and an individual arrives, depending
on the type of the individual, the system will move to a different state.
The state that the Markov chain will transition to when a type \(i\) individual
arrives is defined as the \textit{arriving state} \(\mathcal{A}_i(u,v)\).
Using Figure~\ref{fig:adjusted_markov_model} as reference, an arrival of a type
1 individual makes the system transition to the state on the right.
Similarly an arrival of a type 2 individual makes the system transition to the
right if the threshold hasn't been reached and transition down if the threshold
has been reached.
This can be expressed mathematically as:

\begin{equation}\label{eq:arriving_state_type_1}
    \mathcal{A}_1(u,v) = (u, v + 1)
\end{equation}
\begin{equation}\label{eq:arriving_state_type_2}
    \mathcal{A}_2(u,v) =
    \begin{cases}
        (u, v + 1), & \text{if } v < T \\
        (u + 1, v), & \text{if } v \geq T \\
    \end{cases}
\end{equation}

Additionally, there are certain states in the model where arrivals cannot occur.
A type 1 individual cannot arrive whenever the model is at any state \((u, N)\)
for all \(u\), where \(N\) is the capacity of node 1.
Therefore the set of all such states that an arrival may occur are defined as
\textit{accepting states}.
The set of accepting states for type 1 individuals is denoted as:

\begin{equation}\label{eq:accepting_states_type_1}
    S_A^{(1)} = \{(u, v) \in S \; | \; v < N \}
\end{equation}

Similarly, an arrival of a type 2 individual cannot occur whenever the model is
at state \((M, v)\) for all \(v\), where \(M\) is the capacity of node 2.
The set of accepting states for type 2 individuals is denoted as:

\begin{equation}\label{eq:accepting_states_type_2}
    S_A^{(2)} = \{(u, v) \in S \; | \; u < M \}
\end{equation}



Finally, the total mean waiting time can be calculated by summing over all
expected waiting times of accepting states multiplied by the probability of
being at that state.
The different approaches that are used to get the state probabilities are
described in Section~\ref{sec:steady_state_probabilities}.
The mean waiting time in the system for type \(i\) individuals is given by:

\begin{equation}\label{eq:recursive_waiting_time_for_type_i}
    W^{(i)} = \frac{\sum_{(u,v) \in S_A^{(i)}} \pi_{(u,v)} w^{(i)}
    (\mathcal{A}_i(u,v))}{\sum_{(u,v) \in S_A^{(i)}} \pi_{(u,v)}}
\end{equation}

Consequently, using both the mean waiting time for type 1 individuals
\(W^{(1)}\) and the mean waiting time for type 2 individuals \(W^{(2)}\), the
overall mean waiting time in the system is a linear combination of the 2.
The overall waiting time can be then given by the following equation where
\(\theta_1\) and \(\theta_2\) are the coefficients of the waiting time for each
type of individual:

\begin{equation}\label{eq:overall_waiting_time_coeff}
    W = \theta_1 W^{(1)} + \theta_2 W^{(2)}
\end{equation}

The two coefficients represent the proportion of individuals of each type that
traversed through the model.
Theoretically, getting these percentages should be as straightforward as
looking at the arrival rates of each type \(\lambda_1\) and \(\lambda_2\), but
in practise if either node 1 or node 2 are full, some
individuals may become lost to the system.
Thus, one should account for the probability that an individual is lost to the
system.
This probability can be calculated by using the two sets of accepting
states \(S_A^{(2)}\) and \(S_A^{(1)}\) defined earlier in
(\ref{eq:accepting_states_type_1}) and (\ref{eq:accepting_states_type_2}).
Let us define the probability that an individual of type \(i\) is not lost
to the system as \(P(L'_i)\):

\begin{equation*}
    P(L'_1) = \sum_{(u,v) \, \in S_A^{(1)}} \pi(u,v) \hspace{2cm}
    P(L'_2) = \sum_{(u,v) \, \in S_A^{(2)}} \pi(u,v)
\end{equation*}


Having defined these probabilities one may combine them with the arrival rates
of each individual type in such a way to get the expected proportions of type 1
and type 2 individuals in the model.

\begin{equation}
    \theta_1 = \frac{\lambda_1 P(L'_1)}{\lambda_2 P(L'_2) + \lambda_1 P(L'_1)},
    \hspace{1.5cm}
    \theta_2 = \frac{\lambda_2 P(L'_2)}{\lambda_2 P(L'_2) + \lambda_1 P(L'_1)}
\end{equation}

Thus, by using these values as the coefficient of equation
(\ref{eq:overall_waiting_time_coeff})
the resultant equation can be used to get the overall waiting time.

\begin{equation}\label{eq:overall_waiting_time}
    W = \frac{\lambda_1 P(L'_1)}{\lambda_2 P(L'_2) + \lambda_1 P(L'_1)} W^{(1)}
    + \frac{\lambda_2 P(L'_2)}{\lambda_2 P(L'_2) + \lambda_1 P(L'_1)} W^{(2)}
\end{equation}


\subparagraph{Implementation}\label{sec:waiting_recursive_implementation}

Implementing the recursive approach for the waiting time in python uses the
same logical structure as described in Section~\ref{sec:recursive_waiting_time}.
The first function needed is one that checks if a state belongs in the
set of waiting states that corresponds to the set defined in
(\ref{eq:waiting_states}).

\begin{lstlisting}[style=pystyle]
>>> def is_waiting_state(state, num_of_servers):
...     """Checks if waiting occurs in the given state. In essence, all
...     states (u,v) where v > C are considered waiting states.
...     Parameters
...     ----------
...     state : tuple
...         a tuples of the form (u,v)
...     num_of_servers : int
...         the number of servers = C
...     Returns
...     -------
...     Boolean
...         An indication of whether or not any wait occurs on the given
...         state
...     """
...     return state[1] > num_of_servers
>>>
>>> is_waiting_state(state=(1, 4), num_of_servers=2)
True

\end{lstlisting}

Similarly a function that calculates the expected wait in each state is needed
that corresponds to equations (\ref{eq:waiting_time_state_type_1}) and
(\ref{eq:waiting_time_state_type_2}).
Note here that the following function takes the individuals type as an argument
and thus only one function is needed for both expressions.

\begin{lstlisting}[style=pystyle]
>>> def expected_time_in_markov_state_ignoring_arrivals(
...     state,
...     class_type,
...     num_of_servers,
...     mu,
...     threshold,
... ):
...     """Get the expected waiting time in a Markov state when ignoring any
...     subsequent arrivals. When considering the waiting time of class 2
...     individuals, and when these individuals are in a blocked state
...     (v > 0) then by the definition of the problem the waiting time in
...     that state is set to 0. Additionally, all states where u > 0 and
...     v = T automatically get a waiting time of 0 because class 1
...     individuals only pass one of the states of that column (only state
...     (0,T) is not zero).
... 
...     Parameters
...     ----------
...     state : tuple
...         a tuples of the form (u,v)
...     class_type : int
...         A string to distinguish between class 1(=0) and class 2(=1)
...         individuals
...     num_of_servers : int
...         The number of servers = C
...     mu : float
...         The service rate = mu
... 
...     Returns
...     -------
...     float
...         The expected waiting time in the given state
...     """
...     if state[0] > 0 and (state[1] == threshold or class_type == 1):
...         return 0
...     return 1 / (min(state[1], num_of_servers) * mu)
>>>
>>> expected_time_in_markov_state_ignoring_arrivals(
...     state=(3, 4),
...     class_type=0,
...     num_of_servers=1,
...     mu=4,
...     threshold=2,
... )
0.25

\end{lstlisting}

The following block of code is the implementation of equation
(\ref{eq:recursive_waiting_time_for_state}), where it returns the waiting time
of an individual when they arrive at a given state until they leave that
particular state.
Note that this function uses both of the functions defined earlier.

\begin{lstlisting}[style=pystyle]
>>> def get_waiting_time_for_each_state_recursively(
...     state,
...     class_type,
...     lambda_2,
...     lambda_1,
...     mu,
...     num_of_servers,
...     threshold,
...     system_capacity,
...     buffer_capacity,
... ):
...     """Performs a recursive algorithm to get the expected waiting time of
...     individuals when they enter the model at a given state. Given an
...     arriving state the algorithm moves down to all subsequent states
...     until it reaches one that is not a waiting state.
... 
...     Class 1:
...         - If (u,v) not a waiting state: return 0
...         - Next state s_d = (0, v - 1)
...         - w(u,v) = c(u,v) + w(s_d)
... 
...     Class 2:
...         - If (u,v) not a waiting state: return 0
...         - Next state:   s_n = (u-1, v),    if u >= 1 and v=T
...                         s_n = (u, v - 1),  otherwise
...         - w(u,v) = c(u,v) + w(s_n)
... 
...     Note: For all class 1 individuals the recursive formula acts in a
...     linear manner meaning that an individual will have the same waiting
...     time when arriving at any state of the same column e.g (2, 3) or
...     (5, 3).
... 
...     Parameters
...     ----------
...     state : tuple
...     class_type : int
...     lambda_2 : float
...     lambda_1 : float
...     mu : float
...     num_of_servers : int
...     threshold : int
...     system_capacity : int
...     buffer_capacity : int
... 
...     Returns
...     -------
...     float
...         The expected waiting time from the arriving state of an
...         individual until service
...     """
...     if not is_waiting_state(state, num_of_servers):
...         return 0
...     if state[0] >= 1 and state[1] == threshold:
...         next_state = (state[0] - 1, state[1])
...     else:
...         next_state = (state[0], state[1] - 1)
... 
...     wait = expected_time_in_markov_state_ignoring_arrivals(
...         state=state,
...         class_type=class_type,
...         num_of_servers=num_of_servers,
...         mu=mu,
...         threshold=threshold,
...     )
...     wait += get_waiting_time_for_each_state_recursively(
...         state=next_state,
...         class_type=class_type,
...         lambda_2=lambda_2,
...         lambda_1=lambda_1,
...         mu=mu,
...         num_of_servers=num_of_servers,
...         threshold=threshold,
...         system_capacity=system_capacity,
...         buffer_capacity=buffer_capacity,
...     )
...     return wait
>>>
>>> get_waiting_time_for_each_state_recursively(
...     state=(3, 4),
...     class_type=0,
...     lambda_2=1,
...     lambda_1=1,
...     mu=4,
...     num_of_servers=1,
...     threshold=2,
...     system_capacity=4,
...     buffer_capacity=3,
... )
0.75

\end{lstlisting}

Additionally, before getting the mean waiting time for each type of individuals,
a function for the set of accepting states described in
(\ref{eq:accepting_states_type_1}) and (\ref{eq:accepting_states_type_1}) needs
to be constructed.

\begin{lstlisting}[style=pystyle]
>>> def is_accepting_state(
...     state, class_type, threshold, system_capacity, buffer_capacity
... ):
...     """
...     Checks if a state given is an accepting state. Accepting states are
...     defined as the states of the system where arrivals may occur. In
...     essence these states are all states apart from the one when the system
...     cannot accept additional arrivals. Because there are two types of
...     arrivals though, the set of accepting states is different for class 1
...     and class 2 individuals:
... 
...     Parameters
...     ----------
...     state : tuple
...         a tuples of the form (u,v)
...     class_type : int
...         A string to distinguish between class 1 (=0) and class 2
...         individuals (=1)
...     system_capacity : int
...         The capacity of the system (hospital) = N
...     buffer_capacity : int
...         The capacity of the buffer space = M
... 
...     Returns
...     -------
...     Boolean
...         An indication of whether or not an arrival of the given type
...         (class_type) can occur
...     """
...     if class_type == 1:
...         condition = (
...             (state[0] < buffer_capacity)
...             if (threshold <= system_capacity)
...             else (state[1] < system_capacity)
...         )
...     if class_type == 0:
...         condition = state[1] < system_capacity
...     return condition

\end{lstlisting}

The only thing left to do is to find the weighted average of the waiting times
for all states using the steady state probabilities.
The function following function corresponds to the expression for \(W^{(i)}\)
defined in equation (\ref{eq:recursive_waiting_time_for_type_i}).

\begin{lstlisting}[style=pystyle]
>>> import ambulance_game as abg
>>> import numpy as np
>>> def mean_waiting_time_formula_using_recursive_approach(
...     all_states,
...     pi,
...     class_type,
...     lambda_2,
...     lambda_1,
...     mu,
...     num_of_servers,
...     threshold,
...     system_capacity,
...     buffer_capacity,
...     **kwargs,
... ):
...     """
...     Get the mean waiting time by using a recursive formula. 
...     All w(u,v) terms are calculated recursively by going through
...     the waiting times of all previous states.
... 
...     Parameters
...     ----------
...     all_states : list
...     pi : array
...     class_type : int
...     lambda_2 : float
...     lambda_1 : float
...     mu : float
...     num_of_servers : int
...     threshold : int
...     system_capacity : int
...     buffer_capacity : int
... 
...     Returns
...     -------
...     float
...     """
...     mean_waiting_time = 0
...     probability_of_accepting = 0
...     for u, v in all_states:
...         if is_accepting_state(
...             state=(u, v),
...             class_type=class_type,
...             threshold=threshold,
...             system_capacity=system_capacity,
...             buffer_capacity=buffer_capacity,
...         ):
...             arriving_state = (u, v + 1)
...             if class_type == 1 and v >= threshold:
...                 arriving_state = (u + 1, v)
... 
...             current_state_wait = get_waiting_time_for_each_state_recursively(
...                 state=arriving_state,
...                 class_type=class_type,
...                 lambda_2=lambda_2,
...                 lambda_1=lambda_1,
...                 mu=mu,
...                 num_of_servers=num_of_servers,
...                 threshold=threshold,
...                 system_capacity=system_capacity,
...                 buffer_capacity=buffer_capacity,
...             )
...             mean_waiting_time += current_state_wait * pi[u, v]
...             probability_of_accepting += pi[u, v]
...     return mean_waiting_time / probability_of_accepting
>>>
>>> all_states = abg.markov.build_states(
...     threshold=2,
...     system_capacity=4,
...     buffer_capacity=3,
... )
>>> Q = abg.markov.get_transition_matrix(
...     lambda_1=1,
...     lambda_2=1,
...     mu=4,
...     num_of_servers=1,
...     threshold=2,
...     system_capacity=4,
...     buffer_capacity=3
... )
>>> pi = abg.markov.get_markov_state_probabilities(
...     abg.markov.get_steady_state_algebraically(
...         Q, algebraic_function=np.linalg.solve
...     ),
...     all_states,
... )
>>> mean_waiting_time_formula_using_recursive_approach(
...     all_states=all_states,
...     pi=pi,
...     class_type=0,
...     lambda_2=1,
...     lambda_1=1,
...     mu=4,
...     num_of_servers=1,
...     threshold=2,
...     system_capacity=4,
...     buffer_capacity=3,
... )
0.1921401289632943

\end{lstlisting}

Finally the overall waiting time for both individuals can be calculated by
taking the weighted average of the waiting times for each type of individual
as described in equation (\ref{eq:overall_waiting_time}).

\begin{lstlisting}[style=pystyle]
>>> def overall_waiting_time_formula(
...     all_states,
...     pi,
...     lambda_2,
...     lambda_1,
...     mu,
...     num_of_servers,
...     threshold,
...     system_capacity,
...     buffer_capacity,
...     waiting_formula,
...     **kwargs,
... ):
...     """
...     Gets the overall waiting time for all individuals by calculating both
...     class 1 and class 2 waiting times. Thus, considering the probability
...     that an individual is lost to the system (for both classes)
...     calculates the overall waiting time.
... 
...     Parameters
...     ----------
...     all_states : list
...     pi : array
...     lambda_1 : float
...     lambda_2 : float
...     mu : float
...     num_of_servers : int
...     threshold : int
...     system_capacity : int
...     buffer_capacity : int
...     waiting_formula : function
... 
...     Returns
...     -------
...     float
...         The overall mean waiting time by combining class 1 and class 2
...         individuals
...     """
...     mean_waiting_times_for_each_class = [
...         waiting_formula(
...             all_states=all_states,
...             pi=pi,
...             class_type=class_type,
...             lambda_2=lambda_2,
...             lambda_1=lambda_1,
...             mu=mu,
...             num_of_servers=num_of_servers,
...             threshold=threshold,
...             system_capacity=system_capacity,
...             buffer_capacity=buffer_capacity,
...         )
...         for class_type in range(2)
...     ]
...     prob_accept = [
...         np.sum(
...             [
...                 pi[state]
...                 for state in all_states
...                 if is_accepting_state(
...                     state=state,
...                     class_type=class_type,
...                     threshold=threshold,
...                     system_capacity=system_capacity,
...                     buffer_capacity=buffer_capacity,
...                 )
...             ]
...         )
...         for class_type in range(2)
...     ]
...     class_rates = [
...         prob_accept[class_type]
...         / ((lambda_2 * prob_accept[1]) + (lambda_1 * prob_accept[0]))
...         for class_type in range(2)
...     ]
...     class_rates[0] *= lambda_1
...     class_rates[1] *= lambda_2
...     mean_waiting_time = np.sum(
...         [
...             mean_waiting_times_for_each_class[class_type]
...             * class_rates[class_type]
...             for class_type in range(2)
...         ]
...     )
...     return mean_waiting_time
>>>
>>> overall_waiting_time_formula(
...     all_states=all_states,
...     pi=pi,
...     lambda_2=1,
...     lambda_1=1,
...     mu=4,
...     num_of_servers=1,
...     threshold=2,
...     system_capacity=4,
...     buffer_capacity=3,
...     waiting_formula=mean_waiting_time_formula_using_recursive_approach,
... )
0.15724618859452746

\end{lstlisting}


\paragraph{Direct approach}\label{sec:direct_waiting_time}

The direct approach uses similar concepts to the recursive approach of
Section~\ref{sec:recursive_waiting_time}.
Instead of using recursion, a linear system of the set of equations generated by
equation (\ref{eq:recursive_waiting_time_for_state}) for every state \((u,v)\)
is solved.
The set of equations that need to be solved for individuals of type \(i\) are
all \( w^{(i)}(u, v) \) for all possible states \((u,v) \in S\).


\begin{equation*}
    w^{(i)}(u,v) =
    \begin{cases}
        0, \hspace{4.85cm} & \textbf{if } (u,v) \notin S_w \\
        c^{(i)}_w(u,v) + w^{(i)}(u-1, v), & \textbf{if } u > 0
        \textbf{ and } v = T \\
        c^{(i)}_w(u,v) + w^{(i)}(u, v-1), & \textbf{otherwise}
    \end{cases}
\end{equation*}

Consider a relatively small model where \(C=1, T=2, N=3, M=1\).
All possible equations \(w^{(i)}(u,v)\) are given by equations
(\ref{eq:first_eq_of_waiting_example} - \ref{eq:last_eq_of_waiting_example}).

\begin{minipage}{0.45\textwidth}
    \begin{figure}[H]
        \centering
        \scalebox{0.65}{\begin{tikzpicture}[-, node distance = 1cm, auto]
\node[state] (u0v0) {(0,0)};
\node[state, right=of u0v0] (u0v1) {(0,1)};
\draw[->](u0v0) edge[bend left] node {\( \Lambda \)} (u0v1);
\draw[->](u0v1) edge[bend left] node {\(\mu \)} (u0v0);
\node[state, right=of u0v1] (u0v2) {(0,2)};
\draw[->](u0v1) edge[bend left] node {\( \Lambda \)} (u0v2);
\draw[->](u0v2) edge[bend left] node {\(\mu \)} (u0v1);
\node[state, below=of u0v2] (u1v2) {(1,2)};
\draw[->](u0v2) edge[bend left] node {\( \lambda_2 \)} (u1v2);
\draw[->](u1v2) edge[bend left] node {\(\mu \)} (u0v2);
\node[state, right=of u0v2] (u0v3) {(0,3)};
\draw[->](u0v2) edge[bend left] node {\( \lambda_1 \)} (u0v3);
\draw[->](u0v3) edge[bend left] node {\(\mu \)} (u0v2);
\node[state, right=of u1v2] (u1v3) {(1,3)};
\draw[->](u1v2) edge[bend left] node {\( \lambda_1 \)} (u1v3);
\draw[->](u1v3) edge[bend left] node {\(\mu \)} (u1v2);
\draw[->](u0v3) edge node {\( \lambda_2 \)} (u1v3);
\end{tikzpicture}}
        \caption{Markov chain example}
        \label{fig:example-algeb-waiting}
    \end{figure}
\end{minipage}
\begin{minipage}{0.5\textwidth}
    \begin{align}
        w^{(i)}(0,0) &= 0 \\
        w^{(i)}(0,1) &= 0 \\
        w^{(i)}(0,2) &= c^{(i)}_w(0,2) + w^{(i)}(0,1) \label{eq:first_eq_of_waiting_example} \\
        w^{(i)}(0,3) &= c^{(i)}_w(0,3) + w^{(i)}(0,2) \\
        w^{(i)}(1,2) &= c^{(i)}_w(1,2) + w^{(i)}(0,2) \\
        w^{(i)}(1,3) &= c^{(i)}_w(1,3) + w^{(i)}(1,2) \label{eq:last_eq_of_waiting_example}
    \end{align}
\end{minipage}

\vspace{0.5cm}

Additionally, the above equations can be transformed into a linear system of
the form \(Zx=y\) where:

\begin{equation}\label{eq:example_direct_approach_waiting_time}
    Z=
    \begin{pmatrix}
       -1 &  0 &  0 &  0 &  0 &  0 \\  %(0,0)
        0 & -1 &  0 &  0 &  0 &  0 \\  %(0,1)
        0 &  1 & -1 &  0 &  0 &  0 \\  %(0,2)
        0 &  0 &  1 & -1 &  0 &  0 \\  %(0,3)
        0 &  0 &  1 &  0 & -1 &  0 \\  %(1,2)
        0 &  0 &  0 &  0 &  1 & -1 \\  %(1,3)
    \end{pmatrix},
    x=
    \begin{pmatrix}
        w^{(i)}(0,0) \\
        w^{(i)}(0,1) \\
        w^{(i)}(0,2) \\
        w^{(i)}(0,3) \\
        w^{(i)}(1,2) \\
        w^{(i)}(1,3) \\
    \end{pmatrix},
    y=
    \begin{pmatrix}
        0 \\
        0 \\
        -c^{(i)}_w(0,2) \\
        -c^{(i)}_w(0,3) \\
        -c^{(i)}_w(1,2) \\
        -c^{(i)}_w(1,3) \\
    \end{pmatrix}
\end{equation}

A more generalised form of the equations in
(\ref{eq:example_direct_approach_waiting_time}) can be given for any value of
\(C,T,N,M\) by:

\begin{align}
    w^{(i)}(0, 0) &= 0 \label{eq:first_eq_of_waiting_general} \\
    w^{(i)}(0, 1) &= c^{(i)}_w(0,1) + w^{(i)}(0,0) \\
    w^{(i)}(0, 2) &= c^{(i)}_w(0,2) + w^{(i)}(0,1) \\
    & \vdots \nonumber \\
    w^{(i)}(0, T-1) &= c^{(i)}_w(0,T-1) + w^{(i)}(0,T-2) \\
    w^{(i)}(0, T) &= c^{(i)}_w(0, T) + w^{(i)}(0, T - 1) \\
    w^{(i)}(0, T + 1) =& c^{(i)}_w(0, T + 1) + w^{(i)}(0, T) \\
    w^{(i)}(0, T + 2) =& c^{(i)}_w(0, T + 2) + w^{(i)}(0, T + 1) \\
    & \vdots \nonumber \\
    w^{(i)}(0, N) =& c^{(i)}_w(0, N) + w^{(i)}(0, N - 1) \\
    w^{(i)}(1, T) =& c^{(i)}_w(1, T) + w^{(i)}(0, T) \\
    w^{(i)}(1, T + 1) =& c^{(i)}_w(1, T + 1) + w^{(i)}(1, T) \\
    & \vdots \nonumber \\
    w^{(i)}(M, N) =& c^{(i)}_w(M, N) + w^{(i)}(M, N-1)
    \label{eq:last_eq_of_waiting_general}
\end{align}

The equivalent matrix form of the linear system of equations
(\ref{eq:first_eq_of_waiting_general} - \ref{eq:last_eq_of_waiting_general})
is given by \(Zx=y\), where:

\newcommand{\allthedots}{\vdots & \vdots & \vdots & \ddots & \vdots & \vdots &
\vdots & \vdots & \ddots & \vdots & \vdots & \vdots & \ddots & \vdots}

\footnotesize
\begin{equation*}\label{eq:general_direct_approach_waiting_time}
    Z =
    \begin{pmatrix}
        -1 &  0 &  0 & \dots &  0 &  0 &  0 &  0 & \dots &  0 &  0 &  0 & \dots &  0 \\    %(0, 0)
         1 & -1 &  0 & \dots &  0 &  0 &  0 &  0 & \dots &  0 &  0 &  0 & \dots &  0 \\    %(0, 1)
         0 &  1 & -1 & \dots &  0 &  0 &  0 &  0 & \dots &  0 &  0 &  0 & \dots &  0 \\    %(0, 2)
         \allthedots \\
         0 &  0 &  0 & \dots & -1 &  0 &  0 &  0 & \dots &  0 &  0 &  0 & \dots &  0 \\    %(0, T-1)
         0 &  0 &  0 & \dots &  1 & -1 &  0 &  0 & \dots &  0 &  0 &  0 & \dots &  0 \\    %(0,T)
         0 &  0 &  0 & \dots &  0 &  1 & -1 &  0 & \dots &  0 &  0 &  0 & \dots &  0 \\    %(0,T+1)
         0 &  0 &  0 & \dots &  0 &  0 &  1 & -1 & \dots &  0 &  0 &  0 & \dots &  0 \\    %(0,T+2)
         \allthedots \\
         0 &  0 &  0 & \dots &  0 &  0 &  0 &  0 & \dots & -1 &  0 &  0 & \dots &  0 \\   %(0,N)
         0 &  0 &  0 & \dots &  0 &  1 &  0 &  0 & \dots &  0 & -1 &  0 & \dots &  0 \\   %(1,T)
         0 &  0 &  0 & \dots &  0 &  0 &  1 &  0 & \dots &  0 &  0 & -1 & \dots &  0 \\   %(1,T+1)
         \allthedots \\
         0 &  0 &  0 & \dots &  0 &  0 &  0 &  0 & \dots &  0 &  0 &  0 & \dots & -1 \\   %(M,N)
    \end{pmatrix}
\end{equation*}
\normalsize

\begin{equation}
    x =
    \begin{pmatrix}
        w^{(i)}(0, 0) \\
        w^{(i)}(0, 1) \\
        w^{(i)}(0, 2) \\
        \vdots \\
        w^{(i)}(0, T-1) \\
        w^{(i)}(0, T) \\
        w^{(i)}(0, T + 1) \\
        w^{(i)}(0, T + 2) \\
        \vdots \\
        w^{(i)}(0, N) \\
        w^{(i)}(1, T) \\
        w^{(i)}(1, T + 1) \\
        \vdots \\
        w^{(i)}(M, N) \\
    \end{pmatrix},
    y=
    \begin{pmatrix}
        0 \\ %new
        -c^{(i)}_w(0,1) \\ %new
        -c^{(i)}_w(0,2) \\ %new
        \vdots \\
        -c^{(i)}_w(0,T - 1) \\ %new
        -c^{(i)}_w(0,T) \\ %new
        -c^{(i)}_w(0,T+1) \\
        -c^{(i)}_w(0,T+2) \\
        \vdots \\
        -c^{(i)}_w(0,N) \\
        -c^{(i)}_w(1,T) \\
        -c^{(i)}_w(1,T+1) \\
        \vdots \\
        -c^{(i)}_w(M,N) \\
    \end{pmatrix}
\end{equation}

Thus, solving for \(x\) gets the values of all \(w^{(i)}(u,v)\) for all states
\((u,v) \in S\).
These values can then be used with equation
(\ref{eq:recursive_waiting_time_for_type_i}) to compute the mean waiting time
for type \(i\) individuals \(W^{(i)}\).
Now, having \(W^{(1)}\) and \(W^{(2)}\), equation (\ref{eq:overall_waiting_time})
can be utilised once more to compute the overall mean waiting time for both
individual types.


\subparagraph{Implementation}\label{sec:waiting_direct_implementation}

Similar to the implementation of the recursive approach from
Section~\ref{sec:waiting_recursive_implementation} the functions that correspond
to equations (\ref{eq:waiting_states}), (\ref{eq:waiting_time_state_type_1}),
(\ref{eq:waiting_time_state_type_2}), (\ref{eq:accepting_states_type_1}),
(\ref{eq:accepting_states_type_2}) will be used again.
For the implementation of the direct approach the aim is to construct matrix
\(Z\) and vector \(y\) as described in Section
\ref{sec:direct_waiting_time} in order to solve the system of linear
equations described in (\ref{eq:general_direct_approach_waiting_time}).

The following block of code Returns the values of one row of matrix \(Z\) that
corresponds to the state \((u,v)\) and the value of vector \(y\) that
correspond to the state \((u,v)\).

\begin{lstlisting}[style=pystyle]
>>> import itertools
>>> def get_coefficients_row_of_array_for_state(
...     state,
...     class_type,
...     mu,
...     num_of_servers,
...     threshold,
...     system_capacity,
...     buffer_capacity
... ):
...     """
...     For direct approach: Constructs a row of the coefficients matrix.
...     The row to be constructed corresponds to the waiting time equation
...     for a given state (u,v)
...     """
...     lhs_coefficient_row = np.zeros(
...         [buffer_capacity + 1, system_capacity + 1]
...     )
...     lhs_coefficient_row[state[0], state[1]] = -1
...     for (u, v) in itertools.product(
...         range(1, buffer_capacity + 1), range(threshold)
...     ):
...         lhs_coefficient_row[u, v] = np.NaN
... 
...     rhs_value = 0
...     if is_waiting_state(state, num_of_servers):
...         if state[0] >= 1 and state[1] == threshold:
...             next_state = (state[0] - 1, state[1])
...         else:
...             next_state = (state[0], state[1] - 1)
... 
...         lhs_coefficient_row[next_state[0], next_state[1]] = 1
...         rhs_value = -expected_time_in_markov_state_ignoring_arrivals(
...             state=state,
...             class_type=class_type,
...             mu=mu,
...             num_of_servers=num_of_servers,
...             threshold=threshold,
...         )
...     vectorised_array = np.hstack(
...         (
...             lhs_coefficient_row[0, :threshold],
...             lhs_coefficient_row[:, threshold:].flatten("F"),
...         )
...     )
...     return vectorised_array, rhs_value
>>>
>>> get_coefficients_row_of_array_for_state(
...     state=(2,3),
...     class_type=0,
...     mu=4,
...     num_of_servers=1,
...     threshold=2,
...     system_capacity=3,
...     buffer_capacity=2
... )
(array([ 0.,  0.,  0.,  0.,  1.,  0.,  0., -1.]), -0.25)

\end{lstlisting}

In the above code, the function returns a tuple with two elements; the row of
matrix \(Z\) and the value of vector \(y\) that corresponds to state \((2,3)\).
Using the function above, the matrix \(Z\) and vector \(y\) can be constructed
by considering all states of the Markov chain.

\begin{lstlisting}[style=pystyle]
>>> def get_waiting_time_linear_system(
...     class_type,
...     mu,
...     num_of_servers,
...     threshold,
...     system_capacity,
...     buffer_capacity
... ):
...     """
...     For direct approach: Obtain the linear system Z X = y by finding
...     the array Z and the column vector y that are required. Here Z is
...     denoted as "all_coefficients_array" and y as "constant_column".
...     The function stacks the outputs of
...     get_coefficients_row_of_array_for_state() for all states. In
...     essence all outputs are stacked together to form a square matrix
...     (|) and equivalently a column vector (y) that will be used to find
...     X s.t. Z*X=y
...     """
...     all_coefficients_array = np.array([])
...     all_states = abg.markov.build_states(
...         threshold=threshold,
...         system_capacity=system_capacity,
...         buffer_capacity=buffer_capacity,
...     )
...     for state in all_states:
...         lhs_vector, rhs_value = get_coefficients_row_of_array_for_state(
...             state=state,
...             class_type=class_type,
...             mu=mu,
...             num_of_servers=num_of_servers,
...             threshold=threshold,
...             system_capacity=system_capacity,
...             buffer_capacity=buffer_capacity,
...         )
...         if len(all_coefficients_array) == 0:
...             all_coefficients_array = [lhs_vector]
...             constant_column = [rhs_value]
...         else:
...             all_coefficients_array = np.vstack(
...                 [all_coefficients_array, lhs_vector]
...             )
...             constant_column.append(rhs_value)
...     return all_coefficients_array, constant_column
>>>
>>> Z, y = get_waiting_time_linear_system(
...     class_type=0,
...     mu=4,
...     num_of_servers=1,
...     threshold=2,
...     system_capacity=3,
...     buffer_capacity=2
... )
>>> Z
array([[-1.,  0.,  0.,  0.,  0.,  0.,  0.,  0.],
       [ 0., -1.,  0.,  0.,  0.,  0.,  0.,  0.],
       [ 0.,  1., -1.,  0.,  0.,  0.,  0.,  0.],
       [ 0.,  0.,  1., -1.,  0.,  0.,  0.,  0.],
       [ 0.,  0.,  0.,  1., -1.,  0.,  0.,  0.],
       [ 0.,  0.,  1.,  0.,  0., -1.,  0.,  0.],
       [ 0.,  0.,  0.,  1.,  0.,  0., -1.,  0.],
       [ 0.,  0.,  0.,  0.,  1.,  0.,  0., -1.]])
>>> y
[0, 0, -0.25, 0, 0, -0.25, -0.25, -0.25]

\end{lstlisting}

The following piece of code solves the linear system \(Z X = y\) to obtain the
vector \(X\) containing the waiting times for all states of the Markov chain.
After solving the linear system for vector \(X\) it also converts the
1-dimensional array into a 2-dimensional array where the entry at row \(u\) and
column \(v\) corresponds to the expected waiting time that an individual will
have to wait when arriving and the Markov chain is at state \((u,v)\).

\begin{lstlisting}[style=pystyle]
>>> def convert_solution_to_correct_array_format(
...     array, all_states, system_capacity, buffer_capacity
... ):
...     """
...     For direct approach: Convert the solution into a format that matches
...     the state probabilities array. The given array is a one-dimensional
...     array with the waiting times of each state
...     """
...     array_with_correct_shape = np.zeros(
...         [buffer_capacity + 1, system_capacity + 1]
...     )
...     for index, (u, v) in enumerate(all_states):
...         array_with_correct_shape[u, v] = array[index]
...     return array_with_correct_shape
>>>
>>>
>>> def get_waiting_times_of_all_states_using_direct_approach(
...     class_type,
...     all_states,
...     mu,
...     num_of_servers,
...     threshold,
...     system_capacity,
...     buffer_capacity,
... ):
...     """
...     For direct approach: Solve M*X = b using numpy.linalg.solve() where:
...         M = The array containing the coefficients of all w(u,v) equations
...         b = Vector of constants of equations
...         X = All w(u,v) variables of the equations
...     """
...     M, b = get_waiting_time_linear_system(
...         class_type=class_type,
...         mu=mu,
...         num_of_servers=num_of_servers,
...         threshold=threshold,
...         system_capacity=system_capacity,
...         buffer_capacity=buffer_capacity,
...     )
...     state_waiting_times = np.linalg.solve(M, b)
...     state_waiting_times = convert_solution_to_correct_array_format(
...         array=state_waiting_times,
...         all_states=all_states,
...         system_capacity=system_capacity,
...         buffer_capacity=buffer_capacity,
...     )
...     return state_waiting_times
>>>
>>> all_states = abg.markov.build_states(
...     threshold=2,
...     system_capacity=3,
...     buffer_capacity=2,
... )
>>>
>>> get_waiting_times_of_all_states_using_direct_approach(
...     class_type=0,
...     all_states=all_states,
...     mu=4,
...     num_of_servers=1,
...     threshold=2,
...     system_capacity=3,
...     buffer_capacity=2
... )
array([[-0.  , -0.  ,  0.25,  0.5 ],
       [ 0.  ,  0.  ,  0.25,  0.5 ],
       [ 0.  ,  0.  ,  0.25,  0.5 ]])

\end{lstlisting}

Finally, similar to Section~\ref{sec:waiting_recursive_implementation}, using
equation (\ref{eq:recursive_waiting_time_for_type_i}) the mean waiting time
for either type of individuals can be calculated as follows:

\begin{lstlisting}[style=pystyle] 
>>> def mean_waiting_time_formula_using_direct_approach(
...     all_states,
...     pi,
...     class_type,
...     lambda_2,
...     lambda_1,
...     mu,
...     num_of_servers,
...     threshold,
...     system_capacity,
...     buffer_capacity,
...     **kwargs,
... ):
...     """
...     Get the mean waiting time by using a direct approach.
...     """
...     wait_times = get_waiting_times_of_all_states_using_direct_approach(
...         class_type=class_type,
...         all_states=all_states,
...         mu=mu,
...         num_of_servers=num_of_servers,
...         threshold=threshold,
...         system_capacity=system_capacity,
...         buffer_capacity=buffer_capacity,
...     )
...     mean_waiting_time, prob_accept_class_2_ind = 0, 0
...     for (u, v) in all_states:
...         if is_accepting_state(
...             state=(u, v),
...             class_type=class_type,
...             threshold=threshold,
...             system_capacity=system_capacity,
...             buffer_capacity=buffer_capacity,
...         ):
...             arriving_state = (u, v + 1)
...             if class_type == 1 and v >= threshold:
...                 arriving_state = (u + 1, v)
...             mean_waiting_time += wait_times[arriving_state] * pi[u, v]
...             prob_accept_class_2_ind += pi[u, v]
... 
...     return mean_waiting_time / prob_accept_class_2_ind
>>>
>>> all_states = abg.markov.build_states(
...     threshold=2,
...     system_capacity=4,
...     buffer_capacity=3,
... )
>>> Q = abg.markov.get_transition_matrix(
...     lambda_1=1,
...     lambda_2=1,
...     mu=4,
...     num_of_servers=1,
...     threshold=2,
...     system_capacity=4,
...     buffer_capacity=3
... )
>>> pi = abg.markov.get_markov_state_probabilities(
...     abg.markov.get_steady_state_algebraically(
...         Q, algebraic_function=np.linalg.solve
...     ),
...     all_states,
... )
>>> mean_waiting_time_formula_using_direct_approach(
...     all_states=all_states,
...     pi=pi,
...     class_type=0,
...     lambda_2=1,
...     lambda_1=1,
...     mu=4,
...     num_of_servers=1,
...     threshold=2,
...     system_capacity=4,
...     buffer_capacity=3,
... )
0.1921401289632943

\end{lstlisting}


\paragraph{Closed-form approach}\label{sec:closed_form_waiting_time}

The final approach for getting the mean waiting time is to use a closed-form
approach.
This approach is an immediate simplification of the recursive approach
described in Section~\ref{sec:recursive_waiting_time}.

\begin{equation} \label{eq:closed_form_waiting_type_1}
    W^{(1)} = \frac{\sum_{\substack{(u,v) \, \in S_A^{(1)} \\ v \geq C}}
    \frac{1}{C \mu} \times (v-C+1) \times \pi(u,v)}{\sum_{(u,v) \,
    \in S_A^{(1)}} \pi(u,v)}
\end{equation}

The mean waiting time of type 2 individuals:

\begin{equation}\label{eq:closed_form_waiting_type_2}
    W^{(2)} = \frac{\sum_{\substack{(u,v) \, \in S_A^{(2)} \\ min(v,T) \geq C}}
    \frac{1}{C \mu} \times (\min(v+1,T)-C) \times \pi(u,v)}{\sum_{(u,v) \,
    \in S_A^{(2)}} \pi(u,v)}
\end{equation}

Having \(W^{(1)}\) and \(W^{(2)}\), equation (\ref{eq:overall_waiting_time}) can
then be used to compute \(W\), the overall mean waiting time for both types.


\subparagraph{Implementation}\label{sec:waiting_closed_form_implementation}

The closed-form formula can be implemented in one function.
The functions is broken down in two parts for the case of each individual type.

\begin{lstlisting}[style=pystyle]
>>> def mean_waiting_time_formula_using_closed_form_approach(
...     all_states,
...     pi,
...     class_type,
...     mu,
...     num_of_servers,
...     threshold,
...     system_capacity,
...     buffer_capacity,
...     **kwargs,
... ):
...     """
...     Get the mean waiting time by using a closed-form formula.
...     """
...     sojourn_time = 1 / (num_of_servers * mu)
...     if class_type == 0:
...         mean_waiting_time = np.sum(
...             [
...                 (state[1] - num_of_servers + 1) * pi[state]
...                 * sojourn_time
...                 for state in all_states
...                 if is_accepting_state(
...                     state=state,
...                     class_type=class_type,
...                     threshold=threshold,
...                     system_capacity=system_capacity,
...                     buffer_capacity=buffer_capacity,
...                 )
...                 and state[1] >= num_of_servers
...             ]
...         ) / np.sum(
...             [
...                 pi[state]
...                 for state in all_states
...                 if is_accepting_state(
...                     state=state,
...                     class_type=class_type,
...                     threshold=threshold,
...                     system_capacity=system_capacity,
...                     buffer_capacity=buffer_capacity,
...                 )
...             ]
...         )
...
...     if class_type == 1:
...         mean_waiting_time = np.sum(
...             [
...                 (min(state[1] + 1, threshold) - num_of_servers)
...                 * pi[state]
...                 * sojourn_time
...                 for state in all_states
...                 if is_accepting_state(
...                     state=state,
...                     class_type=class_type,
...                     threshold=threshold,
...                     system_capacity=system_capacity,
...                     buffer_capacity=buffer_capacity,
...                 )
...                 and min(state[1], threshold) >= num_of_servers
...             ]
...         ) / np.sum(
...             [
...                 pi[state]
...                 for state in all_states
...                 if is_accepting_state(
...                     state=state,
...                     class_type=class_type,
...                     threshold=threshold,
...                     system_capacity=system_capacity,
...                     buffer_capacity=buffer_capacity,
...                 )
...             ]
...         )
...     return mean_waiting_time
>>>
>>> all_states = abg.markov.build_states(
...     threshold=2,
...     system_capacity=4,
...     buffer_capacity=3,
... )
>>> Q = abg.markov.get_transition_matrix(
...     lambda_1=1,
...     lambda_2=1,
...     mu=4,
...     num_of_servers=1,
...     threshold=2,
...     system_capacity=4,
...     buffer_capacity=3
... )
>>> pi = abg.markov.get_markov_state_probabilities(
...     abg.markov.get_steady_state_algebraically(
...         Q, algebraic_function=np.linalg.solve
...     ),
...     all_states,
... )
>>> mean_waiting_time_formula_using_closed_form_approach(
...     all_states=all_states,
...     pi=pi,
...     class_type=0,
...     mu=4,
...     num_of_servers=1,
...     threshold=2,
...     system_capacity=4,
...     buffer_capacity=3,
... )
0.19214012896329433

\end{lstlisting}

A numeric comparison of the 3 approaches used to compute the mean waiting time
can be found in Section 
% TODO: reference numeric examples section