\chapter{Conclusions}\label{sec:conclusion}

This Chapter will act as a summary of the research presented in this thesis.
It will also provide a reflection on the research process and the
contributions that have been made.
Finally, it will provide some recommendations for future work.
Each chapter of this thesis included a \textit{Chapter Summary} section, and
so the summary here will be brief.


\section{Research Overview}

The motivation behind this thesis has been the fact that emergency departments
are under a lot of pressure to treat patients.
This is, in practice, often centrally controlled through a mechanism of some
sort of performance measure target.
The research presented in this thesis shows how this can negatively impact the
pathway of both the ambulance patients and the ambulance service itself.
Due to some managerial decision making that takes place at the ED, ambulances
stay blocked outside of the ED at the hospital's parking zone in an attempt to
satisfy these regulations

This thesis presents a queueing network model that is used to describe an
Emergency Department (ED) that accepts patients arriving by ambulance and
patients that arrive by other means.
The model is then used to construct a game theoretic model that is informed
by the queueing network model.
The game theoretic model is presented as a 3-player game between the
Emergency Medical Services (EMS) and two EDs.
The game theoretic model is then used to explore the impact of different
strategies on the performance of the EDs.

Chapter~\ref{sec:introduction} provides an overview the Operational Research
(OR) field and the problem of congestion in healthcare which served as the
motivation for this research.
Chapter~\ref{sec:lit_review} then provides a literature review of the
relevant research that has been done.
Namely, a review of the literature on OR models applied to healthcare
systems, a review of the conjunction of queueing theory and game theory and a
review of the literature on game theoretic models applied to healthcare has
been presented.
Moreover, a brief review on behavioural OR is also presented to provide some
context for the agent-based model that is presented in this thesis.
Chapter~\ref{sec:queueing_section} then introduces a queueing network model
with two waiting spaces that accepts two types of individuals.
The types of individuals are then used to describe an ED that accepts
patients arriving by ambulance and patients that arrive by other means.
The modelling approaches along with the calculations for the model's
performance measures are also presented.
In addition, some numerical comparisons of the different approaches are
given as a form of validation of the different approaches.
Chapter~\ref{sec:game_theoretic_model} introduces a game theoretic model
using the queueing network model as a basis.
Essentially, the queueing network model is used to construct a 3-player game
between the decision makers of two queueing networks and a provider that
distributes individuals between the two queueing networks.
This is later mapped to a 3-player game between the EMS and two EDs.
The methodology included using Brent's algorithm to find the best response of
the third player (the EMS) to all possible combinations of strategies of the
other two players (the two EDs).
The resultant game is then reduced to a 2-player game between the two EDs
and already existing Nash Equilibrium algorithms and evolutionary learning
algorithms could be applied.
Some results of the numerical experiments are then presented and discussed in
Chapter~\ref{sec:numerical_results}.
The particular scenario that was explored was one between two EDs that
were heavily congested.
Several what-if scenarios were investigated to determine ways to reduce the
ambulance congestion at the EDs.

In addition to the game theoretic model, an agent-based model was also
developed in Chapter~\ref{sec:agent_based_extension}.
Instead of the previous constant service times, the queueing model was expanded
to use state- and server-dependent service times.
As a result, an agent-based model is built with different service times for
every server and system state.
The learning that occurs when servers determine the speed at which they serve
customers in order to maximise some utility is then observed using a
reinforcement learning algorithm.
Some numerical results are then presented and discussed in
Section~\ref{sec:reiforcement_learning_numeric_results}.

The motivation for the agent-based extension came from certain ethnographic
insights where it was observed that ambulance staff and ED staff were in fact
playing a more cooperative game.
In fact, the players were not only trying to maximise their own utility but
also the utility of the system.
Thus, some of the utility functions used for the agent-based model were
structured in a way that aimed to increase both each staff member's happiness
and the overall social welfare of the system.



\section{Contributions}

The research presented in this thesis has made some novel contributions to the
literature on Operational Research (OR) models and healthcare applications.
The findings of this thesis that relate to the queueing network and the game
theoretic model have also been published and are presented
in~\cite{panayides2023game}.
The contributions are as follows:

\begin{itemize}
    \item A novel queueing network model with two waiting spaces where one
    serves as a buffer for the other.
    The model is used to describe an ED that accepts patients arriving by
    ambulance and patients that arrive by other means.
    \item Performance measure calculations for the queueing network model.
    Such performance metrics include the average number of individuals in the
    system, mean waiting time, mean blocking time and proportion of individuals
    that are served within a certain time.
    \item A 3-player game theoretic model between two queueing networks and a
    provider that distributes individuals between the two queueing networks.
    The game is then mapped to a 3-player game between the EMS and two EDs.
    \item Numerical experiments showing emergent behaviour of gaming between
    EDs and the EMS.
    A scenario where two EDs are heavily congested is explored and several
    what-if scenarios are investigated to determine ways to reduce the
    ambulance congestion at the EDs.
    \item An agent-based model with reinforcement learning that is used to
    explore the learning that occurs when servers determine the speed at which
    they serve customers in order to maximise some utility.
    The model is built using the queueing network model as a basis.
    \item Numerical experiments using the agent-based model with reinforcement
    learning to explore the learning that occurs.
\end{itemize}

Although this research is motivated by the particular EMS-ED example, the
developed modelling framework and behavioural insights has applications to
similar systems across a range of sectors and settings.
The queueing model can be applied to any setting where individuals may be
blocked on a separate queue.
An example of such setting can be any type of delivery service where customers
can purchase goods either online or in-person.
At busier times, the person delivering the product may be blocked outside the
store in an attempt to improve the waiting times for walk-in customers.


The key findings from this paper that were observed when playing the game
between two EDs and the EMS are:

\begin{itemize}
    \item Inefficiencies can be learned and emerge naturally;
    \item Targeted incentivisation of behaviours can help escape inefficiencies.
\end{itemize}

The former relates to the results of asymmetric replicator dynamics that shown
in Chapter~\ref{sec:numerical_results}.
The results showed that inefficient scenarios can arise by letting the players
play the game by prioritising their best interests, while the latter
implies that these learned inefficiencies can be escaped by carefully applying
certain incentives to the players.
In theory, this careful incentivisation of behaviours is done by applying some
form of penalty to the payoff matrices of the players to force them change
their strategy.
In practice, applying this incentive mechanism to an ED would be difficult
because the EDs are not a single entity but rather a collection of individuals
that could be incentivised in different ways.


Apart from the theoretical contributions, this thesis also made some
contributions to open-source software.
The following software contributions were made as part of this thesis:

\begin{itemize}
    \item \texttt{ambulance\_game}: A Python package that implements the
    queueing network model and the game theoretic model presented in this
    thesis.
    A detailed description of the package can be found in
    Appendix % TODO: ~\ref{app:ambulance_game}.
    \item \texttt{nashpy} contribution: Implemented the asymmetric replicator
    dynamics algorithm in the \texttt{nashpy} Python library.
    \item \texttt{ciw} contribution: Implemented custom server priorities in
    the \texttt{ciw} Python library.
    \item \texttt{ciw} contribution: Implemented server dependent distributions
    in the \texttt{ciw} Python library.
\end{itemize}




\section{Future Work}

The model that is being discussed here presupposes the presence of only two
players that can receive individual.
However, in a realistic healthcare scenario an ambulance may have to decide
among multiple EDs.
An immediate extension of this work would be to consider a multiplayer system
that could represent a group of hospitals in a concentrated area.
Moreover, the developed game theoretic model employs a discrete strategy space
for the EDs (something that is also present in various related
literature~\cite{deo2011centralized,knight_measuring_poa}).
The single threshold parameter that is used for the ED's decision may not be
a good representation of the way EDs actually operate.
In reality ED managers might adopt far more complex parameters for their
decision making process.
Moreover, the game theoretic model of this work assumes that the EMS and EDs
act in a selfish and rational way by only aiming to satisfy their own
objectives.
In some settings, cooperation may be observed and would therefore require
an adapted modelling approach.
The creation of the agent-based model that was introduced in
Chapter~\ref{sec:agent_based_model} was motivated by the potential cooperation
between the EMS and EDs.
Further research could be done to explore the potential for cooperation in
this setting and how it could be further investigated.
Finally, future work could touch upon the completion of the work presented in
Section~\ref{sec:state_probs_closed_form}, where an attempt to develop a closed
form formula for the steady state probabilities of the queueing model was
made.
The formula was not completed and therefore further research could be done to
derive the formula and investigate its properties.
