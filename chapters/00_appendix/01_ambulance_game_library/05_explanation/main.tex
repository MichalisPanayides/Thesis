\section{Explanation}\label{sec:ambulance_game_explanation}

This section provides some additional information about the
\texttt{ambulance\_game} library.
The information provided in this section is not necessary to use the library,
but it may be useful to understand how the library works.

\subsection{Additional information}

Some of the functions and general functionality of the library has not been
explained in the previous sections and are not necessary to use the library.

One of the functions that has not been explained is one that relates to the
transition matrix and is the \texttt{get\_symbolic\_transition\_matrix}
function.
This function is part of the \texttt{markov.py} module and is used to calculate
a symbolic version of the transition matrix of the Markov chain.
The function makes use of the \texttt{sympy} library~\cite{sympy} to get a
symbolic version of the transition matrix where the entries are symbols.
In essence, the function returns a matrix in terms of \(\lambda_1, \lambda_2
\text{ and } \mu\).
An additional function is provided to convert the symbolic version of the
transition matrix to a numerical version of the transition matrix.
This function is called \texttt{convert\_symbolic\_transition\_matrix}.

Another functionality that has not been explained is the way the transition
matrix itself is calculated.
Initially, the transition matrix was calculated by iterating through all
possible states and calculating the entry of the transition matrix for each
state.
This method was not computationally efficient and was replaced by a more
efficient method.
The new method creates a matrix with zeros and visits only the entries that
will have a non-zero value.
The new method makes use of the function
\texttt{get\_all\_pairs\_of\_states\_with\_non\_zero\_entries}.

The module \texttt{tikz.py} has been created for faster creation of
Markov chain tikz figures.
There are two main functionalities of this module.
The first functionality is the ability to generate a tikz figure of the
specific Markov chain that is described in this thesis with any set of
parameters.
This is done by using the function~\texttt{generate\_code\_for\_tikz\_figure}.
The second functionality is the ability to generate a tikz figure of all
possible spanning trees rooted at state \((0, 0)\) of a Markov
chain~\cite{sung2016enumeration, levine2011sandpile}.
This is done by using the
function~\texttt{generate\_code\_for\_tikz\_spanning\_trees\_rooted\_at\_00}.
More information about the investigation between spanning trees and Markov
chains that was done in this thesis can be found in
appendix~\ref{app:steady_state_probs_closed_form}.


\subsection{Other libraries}

Numerous libraries were used in the construction of this library.
Some of the key libraries that were used are:

\begin{itemize}
    \item \texttt{numpy}~\cite{2020NumPy-Array}
    \item \texttt{scipy}~\cite{2020SciPy-NMeth}
    \item \texttt{ciw}~\cite{ciwpython}
    \item \texttt{nashpy}~\cite{thenashpyproject}
\end{itemize}

In particular the \texttt{numpy} library was used to get the steady state
probabilities of the Markov chain algebraically and using the least squares
method.
Similarly, the \texttt{scipy} library was used to get the steady state
probabilities of the Markov chain numerically using the \texttt{odeint} and
\texttt{solve\_ivp} functions.
Apart from that, the \texttt{scipy} library was also used to get the find
the best response of the ambulance service using the \texttt{brentq} function.
The \texttt{brentq} function is part of the \texttt{optimize} module of the
\texttt{scipy} library and implements Brent's algorithm which is a
root-finding algorithm.

The primary tool that was used in the construction of the discrete event
simulation model was the python library \texttt{ciw}.
See Ciw's documentation for a more detailed explanation of how it works and
what are its capabilities~\cite{ciwpython}.
The way the library is structured, allowed for the creation of a custom
node class that inherits from the \texttt{Node} class of \texttt{ciw} and
was used to create a waiting area that individuals could be blocked in if
there were more individuals in the next node.

Finally, the \texttt{nashpy} library~\cite{thenashpyproject} was used for all
the calculations related to the Nash equilibrium and the learning algorithms
that were applied to the game.
The \texttt{nashpy} library is a game theoretic Python library that provides
tools for the solution of two-player normal form games.
See the documentation of \texttt{nashpy} for a more detailed list of the
functionality that is provided by the library~\cite{thenashpyproject}.
