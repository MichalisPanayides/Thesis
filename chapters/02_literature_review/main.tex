\chapter{Literature review}
\label{sec:lit_review}

\section{Introduction}\label{sec:lit_review_intro}

Chapter~\ref{sec:introduction} introduced the problem of congestion in
emergency departments (EDs) and the need for a model that can be used to
understand the impact of different policies on the system.
This chapter provides a review of the current relevant literature in the field
of queueing theory and game theory as well as the combination of the two and
their application to healthcare systems.
This is achieved by partitioning the literature in four different sections each
reviewing a different aspect of research.
The literature review is structured in the following way:

\begin{itemize}
    \item Section~\ref{sec:lit_review_OR_healthcare} provides a review of the
        literature of the techniques used in this thesis along with their
        application to healthcare systems.
        This includes Markov chain models and simulation models.
    \item Section~\ref{sec:lit_review_game_and_queueing_theory} provides a
        review of the literature on the combination of game theory and queueing
        theory.
    \item Section~\ref{sec:lit_review_game_theory_in_healthcare} gives an
        overview of some examples of game theoretic models applied in
        healthcare systems.
    \item Section~\ref{sec:lit_review_behavioural_modelling} discusses the
        general literature on behavioural modelling and some of its
        applications.
\end{itemize}

While the literature review is not exhaustive, it provides a good overview of
the field of queueing theory and game theory as well as their application to
healthcare systems.
This literature review builds on the literature review provided
in~\cite{panayides2023game}.


\section{Operational Research models and Healthcare}
\label{sec:lit_review_OR_healthcare}

This section aims to provide a review of Operational Research (OR) techniques
and some applications to healthcare systems.
OR is a discipline that consists of numerous mathematical tools and techniques
that can be used to solve problems in a variety of fields.
Some of these fields are healthcare, transportation, logistics, manufacturing,
finance and many more.

Markov chains were originally developed by Russian mathematician Andrey Markov
in 1906 who is known for their work in probability theory, analysis and number
theory.
They were originally developed to model the distribution of vowels and
consonants in Pushkin's poem \textit{Eugeny Onegin}~\cite{pushkin2003eugene}.
Markov extended the weak law of large numbers and the central limit theorem to
certain sequences of dependent random variables that were then known as
Markov chains~\cite{Markov_life_and_work}.

Markov chains are a mathematical tool that can be used to model a system and
how it evolves over time.
A markov chain is a stochastic model that consists of a set of states and a
transition probability matrix that describes the probability of moving from one
state to another.
The following papers are examples of Markov chains being used in healthcare
systems.

Even specifically in healthcare systems, Markov chains can be used to model a
range of different scenarios.
For example in~\cite{McClean2006256} a Markov reward model is developed for a
healthcare system to model the movement of patients between hospital states
where patients arrive at a constant rate.
An additional Markov model is also developed to determine patient numbers and
costs at any time where arrivals are taken from a waiting list and a fixed
growth of arrivals that is slowly declining to zero is introduced in the
waiting list.
The model is then applied to geriatric patients to determine costs over time.
In~\cite{Hamdani2015} a Markov chain model is created to analyse the elderly
people flow in the French Healthcare system (FHS) and model their pathway in
hospital.
The model is then applied to a French Hospital to understand the dynamics of
elderly patients flow.
The authors in~\cite{Gosavi2020399} present a Markov-chain model to analyse
the progression of opioid addiction in order to develop treatments.
The Markov chain model is used to predict the proportion of patients in a
given stage of intervention.
In addition, in~\cite{Liao2021} a Markov Decision Process (MDP) framework on
Discrete time Markov chain (DTMC) is developed to optimise medical equipment
repair and replacement decisions.
The model is used to determine the optimal repair and replacement decision
based on the product life cycle and status.
The authors in~\cite{Liao2021} also use a dataset of \(24,516\) repair and
maintenance records that reveal the most common reasons of fault and the
most economically viable repair options.
In~\cite{vile2017queueing} the optimal staffing problem is addressed for a
non-preemptive priority queue with two customer classes and a time-dependent
arrival rate.
The authors use mixed discrete-continuous time Markov chains (MDCTMCs) to
evaluate the behaviour of the system and generate the minimum staffing levels
required.
The applications of interest here were systems were the customers can be
categorised into priority classes, such as emergency departments and call
centres.


Another queueing theory technique that has been traditionally utilised to
represent healthcare systems is Discrete Event Simulation (DES).
DES is a technique that is used to model the behaviour of a system by
representing it as a set of events that occur over time.
More details on DES can be found in Section~\ref{sec:discrete_event_simulation}.

The authors of~\cite{Standfield2014165} compare Markov modelling with
DES to assess if any of the two may change some healthcare resource allocation
decisions.
The authors compare the two approaches in a systematic review and state that
DES is suitable for modelling systems with limited resources and are able
to better capture complexity and uncertainty in the system along with the
ability to capture individual patient histories.
On the other hand, some disadvantages of DES over Markov modelling is that
it is computationally more expensive, requires more data and is more
difficult to validate.
The authors conclude that DES may be preferred over Markov modelling when
individual patient history is an important driver of future events.
In~\cite{williams2020discrete} the authors use a discrete event simulation
model to determine the optimal number of critical care beds required for a
hospital.
Discrete event simulation is utilised to help resource planning and simulate
different what-if scenarios.
The authors in~\cite{Duguay2007311} use a discrete event simulation model to
model the emergency department of a hospital in Canada.
The aim of the paper was to reduce patient waiting times, improve the overall
service delivery and system throughput.
Additionally, the authors in~\cite{Getsios2010411} use a discrete event
simulation model to predict the progression of Alzheimer's disease through
correlated changes in cognition, behavioural disturbance and function.
Individuals in the models are assigned unique demographic and clinical
characteristics and were the severity of the disease was tracked throughout.
The simulation results suggest that donepezil leads to health benefits and cost
savings for patients with mild to moderate Alzheimer's disease and is even
more beneficial when patients are in the mild stages of the disease.



\section{Game theory and Queueing theory}
\label{sec:lit_review_game_and_queueing_theory}

A number of papers have been published that touch upon the use of 
queueing models together with game theoretic concepts.
In~\cite{FirmCompetition} the authors study a simultaneous price competition 
between two firms that are modelled as two distinct queueing systems with a 
fixed capacity and a combined arrival rate.
They calculate the Nash equilibrium both for identical and heterogeneous firms
and show that for the former a pure Nash equilibrium always exist and for the 
latter a unique equilibrium exists where only one firm operates.
The authors have also extended their model in~\cite{FirmCompetition2} by 
allowing the players (firms) to choose capacities. 
A main result from this paper was that when both firms operate independently as
a monopoly, the equilibria are socially optimal, but this is not the case when
the firms operate together.
Another extension of~\cite{FirmCompetition} was introduced 
in~\cite{FirmCompetitionExtension} where a long-run version of the competition 
was considered that also had capacity as a decision variable.
An additional paper that focuses on competition is~\cite{fan2009short} where
the authors created a competition between two sellers where seller 1 supplies 
a product instantly and seller 2 is modelled as a make-to-order M/M/1 queue.
The game that is played requires the two sellers to make a choice on the price 
of the product and then seller 2 to set a capacity that guarantees a maximum 
expected delay.
In our work, while giving some consideration to equilibrium behaviour,
similar to the work of~\cite{FirmCompetition, FirmCompetition2}, emergent
behaviour is more precisely addressed by considering learning algorithms like
asymmetric replicator dynamics~\cite{fudenberg1998theory}.
More details on learning algorithms can be found in
Section~\ref{sec:game_intro_learning_algorithms}.

Another specific part of our research, as described later in the thesis, is the 
construction of a queueing system with a tandem buffer and a single service
centre.
There are several examples from literature that touch upon queueing models
with tandem queues.
In~\cite{d2015pure} the authors explore threshold joining strategies in a 
Markov model that has two tandem queues.
Another example is the one described in~\cite{burnetas2013customer}
where they investigated a network of multiple tandem queues where customers 
decide which queue to attend before joining.
Similarly, in~\cite{bacsar2002stackelberg} the authors examine a network of 
\(N\) tandem M/M/1 queues and with multi-type customers. 
The customers in this paper react to a price \(p\) by picking demand rates that 
maximise utility.
In~\cite{veltman2005equilibrium} a profit maximisation problem is studied that
has two servers; an M/M/1 queue and a parking service providing complementary 
service while the customer is in the first service. 
The providers gain a reward when customers complete both services and no reward 
when they finish one of them.
One of the main conclusions of this study is that by increasing the general 
demand both providers lower their prices to compensate for the increase in wait.
The problem was later extended by~\cite{sun2009equilibrium} where they 
considered arrivals of batches that can share the parking service.
Finally,~\cite{afeche2007decentralized} examines a tandem network of two M/M/1 
queues that are ran by two different profit-maximising service providers.
The network receives three types of customers; those requiring both services, 
customers requiring the first service and customers requiring the second
service.
The authors showed that optimal prices also maximise social utility and that
removing two types of customers that don't need both services leads to higher 
profit and lower demand rate.
In our work, the concepts described in~\cite{d2015pure, burnetas2013customer,
bacsar2002stackelberg} are extended by introducing a threshold parameter that 
determines when individuals can progress from one queue to the other.

Additionally, in~\cite{hagtvedt2009cooperative} the authors explore combining
queueing theory, agent-based simulation and game theory to study the impact of
ambulance diversion.
They consider overcrowded emergency departments (ED) and the use of ambulance
diversion (AD) during which a hospital is not accepting patients by ambulance.
The formulated games are analysed to explore the potential of cooperation
in this setting.
The authors conclude that in such a setting cooperation is not something that
emerges naturally in the presence of strategic behaviour and propose a
centralised form of ambulance routing.
The lack of cooperation in healthcare settings is something that will be
further explored in the work of this thesis.
In particular Section~\ref{sec:game_theoretic_model} will describe a game
that is formulated to study the impact of ambulance blockage outside the
emergency department.
In~\cite{cachon2007obtaining} the authors study a queueing model in which two
strategic servers may choose their own capacities and service rates where
the faster a server works, the more cost it incurs.
The buyer chooses to allocate demand based on the performance of the servers
where faster servers are allocated more demand.
The authors investigate the trade-off between efficiency and incentives and
find that it is possible to design an allocation policy that is both efficient
and can also incentivise the servers to work quickly.
This paper shares some similarities with the work of this thesis.
In particular, this thesis formulates a game to investigate the trade-off
between ambulance blockage and overall efficiency of the ED which is a similar
concept to the one described in~\cite{cachon2007obtaining}.
In~\cite{kalai1992optimal} the authors study the behaviour of vendors in
competition.
Similar to the work of~\cite{cachon2007obtaining}, the authors consider a
queueing model where servers choose their own service rates at a cost.
The servers are also rewarded for each customer that they serve and
based on that cost a two-player game between the two servers is formulated.



\section{Game theory in Healthcare}
\label{sec:lit_review_game_theory_in_healthcare}

In this section a review of the literature is provided that is relevant to
game theoretic models used in healthcare systems.
Game theory is a mathematical tool that is used to model strategic interactions
between players in a system.
The players are assumed to be rational and make decisions based on their own
self-interest and the information they have about the other players' decisions
and the system's state.
A more formal definition of the game theory concepts used in this thesis is
given in Section~\ref{sec:game_theory_intro}.

In the above models, the players are attempting to increase their share of 
individuals choosing to queue.
In public healthcare type settings, this is not 
necessarily the case. 
Rational usage of public services will not necessarily lead to a socially
optimal outcome.
Rather, the overall service needs to be considered as players aim to minimise
their experienced congestion.
In~\cite{sadat2015can} a healthcare application was studied where patients 
could choose between two hospitals, where a utility function is derived that is
based on patients' perceived quality of life.
In~\cite{knight_public_services} the authors place the individuals' choices
between different public services  within the formulation of routing games and
measure inefficiencies using a 
concept known as the price of anarchy (PoA)~\cite{koutsoupias1999worst}.
They show that the price of anarchy increases with worth of service and that is
low for systems with insufficient capacities.
In~\cite{TwoTierHealthcareSystem} a two-tier healthcare system with a capacity
constrained is studied where patients can choose between two systems to receive
their service.
The first system is labelled as the free system (public government-funded
hospital) which offers service without seeking any profit and the second one is
the toll system (private hospital) that aims to maximise its own profit.
The authors, also compare the two-tier system with its one-tier equivalent,
where only the free system exists.
In~\cite{knight_measuring_poa} a normal form game is built that is informed by a 
two-dimensional Markov chain in order to model interactions between critical
care units.
In~\cite{EfficiencyQualityTradeOff} a queueing-game-theoretical model is
introduced where there are two types of service providers; a high quality
high-congested hospital and a low quality low-congested hospital.
The authors study a two-stage Stackelberg game where the government is the
\textit{leader} and the arriving patients are the \textit{followers}.
In~\cite{deo2011centralized} the authors study the network effect of ambulance 
diversion by proposing a non-cooperative game between two EDs that are modelled
as a queueing network.
Each ED's objective is to minimise its own waiting time and chooses a diversion
threshold based on the patients it has.
In equilibrium both EDs choose to divert ambulances in order to avoid getting
arrivals from the other ED.
In this thesis this concept is extended by allowing the ambulance service to 
decide how to distribute its patients among the two EDs.
The players of the game are both the hospitals and the customers of the
hospitals, as opposed to the previous models which are one or the other.
Thus, the novelty of our work is combining both these aspects.



\section{Behavioural Operational Research}
\label{sec:lit_review_behavioural_modelling}

Behavioural OR seeks to (i) advance our understanding of how behavioural
factors affect the conduct of, and interact with, model-based processes that
support problem solving and decision making~\cite{kunc2020review}, and (ii) to
leverage this understanding for improving
outcomes~\cite{hamalainen2013importance, franco2016engaging}.
Moreover, behavioural OR is a sensitive discipline and is subject to the
individual studying it, which means that it is interpreted differently from
researchers to researcher~\cite{hacking1983representing}.

\subsection{Agent-based modelling}
When a modeller looks into implementing behaviour into a certain model,
agent-based modelling is one of the most commonly used techniques.
Agent-based modelling is a micro-abstracted modelling technique that is
capable of modelling complex systems that are composed of many interacting
agents~\cite{greasley2016behavior, jackson2017agent}.
It is also capable of dealing with both deterministic and stochastic
problems.

In such models each simulated item is considered an agent.
An agent represents a person within a group of interest and is modelled in such
a way that it has its own perception of the system and the environment in
general.
Therefore, these autonomous agents may be able to make decisions based on their
own memory and experience.
Additionally they can, not only interact with the environment, but they can
also interact among themselves individually (or with the whole population) and
make certain choices based on these interactions.
Essentially in every time step all agents observe the environment and choose to
move based on that observation.
Some traditional examples of agent-based models are segregation models,
% <!-- alex ignore fire -->
predator-prey models, and forest fire models~\cite{jackson2017agent}.


\subsection{Agent-based modelling and healthcare}

There are numerous examples of simulation models that have been used in
literature to model healthcare systems but only a few of them are agent-based
models.
As stated in~\cite{EscuderoMarin20111239} healthcare systems are based on human
interactions which is what makes them so complex.
Agent-based simulation modelling is capable of capturing both human intention
and human interaction which is why it makes it one of the most suitable
candidate for modelling such healthcare systems.
This section provides a brief overview of the literature around agent-based
models applied to healthcare.

The authors of~\cite{hagtvedt2009cooperative}, that was also mentioned in
Section~\ref{sec:lit_review_game_and_queueing_theory}, also make use of
agent-based modelling to model the setting of ambulance diversion. 
They used an agent-based model to provide insights into how spatial structure,
number of hospitals and different policies contribute to cooperation in
avoiding ambulance diversion.
In essence, the agent-based model was used to assess whether partial diversion
of ambulances would be a viable option for the ambulance service where it
was shown at the end that it was not.
Furthermore, the authors of~\cite{Giesen2009} also make use of agent-based
modelling to model the different queuing strategies in the youth health care
setting.
The agent-based simulation model is parameterised with actual market data and
different queuing strategies are investigated.
The model is structured to incorporate additional complexities that a queueing
theoretic model would fail to capture and thus, the model is able to provide
insights on the queueing strategy decision that would not be possible
otherwise.

In~\cite{Stainsby2009536} the authors make use of agent-based simulation to
model the setting of hospital emergency departments.
The paper describes the system analysis and the preliminary model that was
developed for the simulation these emergency departments along with the
advantages of using agent-based modelling.
The authors claim that agent-based models could be a better fit to model
situations where the human aspect is important.
The human element is part of the system and thus, it should be included in the
model to describe the fundamental concepts of the system.
The authors of~\cite{Bilge2006699} present two ongoing projects of
agent-based models that are applied to the healthcare setting.
The first discussed project is an agent-based simulation model that is used to
model the long term monitoring of Chronic Obstructive Pulmonary Disease (COPD)
which is a major public health problem.
The second project is applying agent-based simulation to visualise and explore
informal social networks amongst staff at the Akdeniz University Hospital.




\subsection{Reinforcement learning}

Reinforcement learning is a machine learning technique that is used to
train agents to make decisions in an uncertain environment.
Reinforcement learning is formulated in such a way that the agent may choose
between a set of policies and aims to maximise the expected reward.
The agent is able to learn from its own experience and is able to make
decisions based on the reward it receives.
There are numerous applications of reinforcement learning in the literature
and it is used in many different fields such as robotics, finance, and
gaming~\cite{sutton2018reinforcement}.
This Section provides a brief overview of the literature around reinforcement
learning.

In~\cite{Iizuka20001075} the authors combine agent-based modelling with
reinforcement learning to model the setting of price negotiations.
The authors study the ability of agents to perform price negotiations and
propose a new model that is based on the ability of agents to distinguish
between different words.
The words correspond to the agent's demand and the agents use these words to
negotiate.
A reinforcement learning algorithm is used to train the agents to distinguish
between the words and use them to negotiate.
At the end it is shown that these words become meaningful in the process of
negotiations and certain strategies are learned by the reinforcement learning
algorithm.
In~\cite{Zou20047} the authors combine agent-based modelling with reinforcement
learning to model the electricity market.
An agent-based simulation model is developed to compare market characteristics
of different pricing methods.
The authors use reinforcement learning to train the generators to improve their
bidding strategies in a repeated bidding game where generators aim to maximise
their profit.
The authors in~\cite{Tellidou2006228} also make use of reinforcement learning
in the power market setting.
Specifically, they use Q-learning to train the suppliers' bidding strategies
based on maximising their profit and their utilisation rate.
Through Q-learning's exploitation and exploration trade-off the suppliers are
able to learn the most profitable action to take under different market
conditions.
At the end of the paper the authors discuss the outcomes of four test cases
with three suppliers under different demand values.

In~\cite{Takadama200326} the authors compare three different learning mechanisms
in a multi-agent based simulation and analyse the results in a bargaining game.
The authors used reinforcement learning to validate and verify the results of
the simulation.
The results show that the learning mechanisms that enable agents to acquire
their rational behaviours differ according to the knowledge representation of
the agents.
A similar idea is extended in~\cite{Takadama2007156} where a multi-agent based
simulation is introduced to explore agents who can reproduce human-like
behaviours in the repeated bargaining game.
The authors compare the results of Roth's learning agents and Q-learning agents
in the sequential bargaining game.
The authors conclude that reinforcement learning agents cannot learn consistent
behaviours in the repeated bargaining game while Q-learning agents can learn
such behaviour but cannot reproduce the decreasing trend found in subject
experiments.
The concepts from~\cite{Takadama200326,Takadama2007156} are extended
in~\cite{Takadama2008} where the authors explore agents that can reproduce
human-like behaviours and human-like thinking in the sequential bargaining
game.
The authors compare the results of Q-learning agents with different action
selection mechanisms and conclude that only Q-learning agents with Boltzmann
distribution selection can reproduce both human-like behaviours and thinking.

Although agent-based modelling has been somewhat used in the literature to
model healthcare systems, this has not been the case for reinforcement
learning.
In this thesis reinforcement learning is used to train the servers of a
queueing system to pick their service rates in order to maximise their
utility.

\section{Chapter summary}

This chapter has provided an overview of the literature around OR in healthcare
systems.
More specifically, Section~\ref{sec:lit_review_OR_healthcare} provides a review
of the literature of the techniques used in this thesis along with their
application to healthcare systems.
In particular the main techniques that has been reviewed are Markov modelling
and discrete event simulation.
Chapter~\ref{sec:queueing_section} of this thesis introduces a novel queueing
network that is modelled using both discrete event simulation and Markov
chains.
In Section~\ref{sec:queueing_ems_ed_application} the proposed model is applied
to a healthcare scenario to model the setting of an emergency department that
receives patients from an ambulance service.

In Sections~\ref{sec:lit_review_game_and_queueing_theory}
and~\ref{sec:lit_review_game_theory_in_healthcare} the literature around
the combination of queueing theory and game theory and the literature around
game theoretic techniques applied in healthcare is reviewed.
Chapter~\ref{sec:game_theoretic_model} of this thesis introduces a novel
game theoretic model that is applied to the queueing network model introduced
in Section~\ref{sec:queueing_section}.
The game theoretic model is used to model the scenario where two queueing
systems and a patient distribution service compete in a 3-player game to
maximise their own utilities.
Subsequently, Section~\ref{sec:game_ems_ed_application} describes how the
game theoretic model is applied to the healthcare scenario where the players
are an ambulance service, and two emergency departments.

Section~\ref{sec:lit_review_behavioural_modelling} of this chapter
provides a brief overview of the literature around behavioural modelling.
One of the main techniques that has been reviewed is agent-based modelling and
reinforcement learning.
These two techniques are used in Chapter~\ref{sec:agent_based_model} to
extend the model introduced in Section~\ref{sec:queueing_section} and
observe the model from a more behavioural perspective.
