\section{Server-dependent variation}\label{sec:server_dependent}

This section describes the creation of the server-dependent variation of the
queueing system described in Section~\ref{sec:queueing_section}.
The server-dependent model has a similar structure as the state-dependent model
described in Section~\ref{sec:state_dependent}.
In this variation each server has its own service rate.

The new service rate \(\mu\) that is used here is defined as follows:

\begin{equation}\label{eq:server_dependent_service_rate}
    \mu =
    \begin{cases}
        \mu_1, & \text{for server } 1 \\
        \mu_2, & \text{for server } 2 \\
        \quad \vdots & \qquad \quad \vdots \\
        \mu_C, & \text{for server } C
    \end{cases}
\end{equation}

Once again, the server-dependent variation of the queueing system was
implemented only on the DES version of the queue.
Consider an example of the queueing system described in
Section~\ref{sec:queueing_section} where the number of servers is set to
\(C = 4\), the threshold is set to \(T = 1\), node 1 capacity is \(N = 2\) and
node 2 capacity is \(M = 1\).
The value of the service rate \(\mu\) for the DES model can take the following
values:

\begin{equation*}
    \mu =
    \begin{cases}
        \mu_1, & \text{for server } 1 \\
        \mu_2, & \text{for server } 2 \\
        \mu_3, & \text{for server } 3 \\
        \mu_4, & \text{for server } 4
    \end{cases}
\end{equation*}



\subsection{Implementation}

The server-dependent variation of the queueing system is implemented in a
similar way as the state-dependent implementation.
Using the python library \texttt{ciw} the server-dependent service time
distribution is defined as a class that inherits from the
\texttt{ciw.dists.Distribution} class.
The class is defined as follows:

\begin{lstlisting}[style=pystyle]
>>> import random
>>> import ciw
>>> class ServerDependentExponential(
...     ciw.dists.Distribution
... ):
...     def __init__(self, rates):
...         self.rates = rates
... 
...     def sample(self, t=None, ind=None):
...         """
...         This method is used to sample the service time for an individual based
...         on the server that the individual is assigned to
...         """
...         server = ind.server.id_number
...         rate = self.rates[server]
...         return random.expovariate(rate)
    
\end{lstlisting}

Similar to the state-dependent implementation, the function
\texttt{simulate\_model} from Section~\ref{sec:discrete_event_simulation} is
used to simulate the server-dependent variation of the queueing system.
The main difference from the state-dependent case is that the service rate
\(\mu\) is now a dictionary with the server's id as the key and the service
rate of that server as the value.

\begin{lstlisting}[style=pystyle]
>>> import ambulance_game as abg
>>> import ciw
>>> import numpy as np
>>>
>>> lambda_1 = 1
>>> lambda_2 = 1
>>> num_of_servers = 4
>>> threshold = 1
>>> system_capacity = 8
>>> buffer_capacity = 2
>>> runtime = 1000
>>> seed_num = 0
>>>
>>> Q = abg.simulation.simulate_model(
...     lambda_1=lambda_1,
...     lambda_2=lambda_2,
...     mu={
...         1: 0.5,
...         2: 0.5,
...         3: 1.0,
...         4: 1.5,
...     },
...     num_of_servers=num_of_servers,
...     threshold=threshold,
...     seed_num=seed_num,
...     system_capacity=system_capacity,
...     buffer_capacity=buffer_capacity,
...     runtime=runtime,
... )
>>> mean_waiting_time = np.mean([w.waiting_time for w in Q.get_all_records()])
>>> np.round(mean_waiting_time, 8)
0.02668017

\end{lstlisting}

