\section{The agent-based model}\label{sec:agent_based_model}

Section~\ref{sec:state_server_dependent_model} describes an extension to the
queueing system described in Section~\ref{sec:queueing_section} that allows
each server to have their own service rate for every possible state of the
system.
In this section, an agent-based model is proposed where servers are considered
as agents that can make their own decisions.
The idea is that every agent in the system can choose their own service rate
based on some utility function that they aim to maximise.
This would mean that the servers would be able to choose and update their own
speed at which they serve individuals while the system is running.
Such decisions could be based on a number of factors, such as minimising the
number of customers in the system, minimising the proportion of patients lost
to the system, maximising their own idle time and so on.

\subsection{Utility functions}

Utility functions are a way of quantifying how ``happy'' an agent is with the
current condition of the system~\cite{fishburn1970utility, fishburn1968utility}.
Each agent in the system can have their own utility function that they aim to
maximise.
In a realistic scenario, these utility functions could be based in a number of
factors, where each agent can have a different weight for each factor.

At the end of the simulation run there are some key performance measures that
can be extracted to quantify the performance of the overall system and each
server individually.
Table~\ref{tab:server_agents_performance_measures} shows these measures that
will then be used to formulate the utility functions.

\begin{table}[htbp]
    \centering
    \caption{Performance measures that could affect each agent's utility}
    \label{tab:server_agents_performance_measures}
    \begin{tabular}{|c|c|}
        \hline
        \textbf{Notation} & \textbf{Description} \\
        \hline
        \hline
        \(I\) & Total number of individuals (both served and lost) \\
        \hline
        \(I_s\) & Number of served individuals \\
        \hline
        \(I_L\) & Number of individuals that are lost due to the system being
        full \\
        \hline
        \(I_s^{(k)}\) & All individuals served by server \(k\) \\
        \hline
        \(R\) & Overall runtime of the simulation \\
        \hline
        \(B^{(k)}\) & Busy time of server \(k\) \\
        \hline
        \(R - B^{(k)}\) & Idle time of server \(k\) \\
        \hline
        \(\bar{\mu}^{(k)}\) & Mean service rate of server \(k\) \\
        \hline
        \(\bar{m}^{(k)}\) & Mean service time of server \(k\) \\
        \hline
    \end{tabular}
\end{table}

Note that the difference between the mean service rate and the mean service
time is that the mean service rate is the average out of all the service rates
for every state \((u, v)\) with no particular weight given to any of them.
In contrast, the mean service time is the average of all the service times
that each server has experienced.
That means that if a particular state has not been visited by a server, then
the mean service time will not be affected by that state.

The above measures could be combined together in a number of ways to formulate
utility functions.
Some examples of utility functions that could be used are the following:

\begin{align}
    U_k^{(1)} &= e \, I_s^{(k)} + (1 - e) \, (R - B^{(k)})
    \label{eq:utility_1} \\
    U_k^{(2)} &= e \, \frac{I_s^{(k)}}{I_s} + (1 - e) \, \frac{R - B^{(k)}}{R}
    \label{eq:utility_2} \\
    U_k^{(3)} &= e \, \bar{m}^{(k)} + (1 - e) \, (R - B^{(k)})
    \label{eq:utility_3} \\
    U_k^{(4)} &= e \, \frac{1}{\bar{m}^{(k)}} + (1 - e) \, (R - B^{(k)})
    \label{eq:utility_4} \\
    U_k^{(5)} &= e \, \frac{I_s^{(k)}}{I_s} + (1 - e) \, \bar{m}^{(k)}
    \label{eq:utility_5} \\ 
    U_k^{(6)} &= e \, \frac{I_s^{(k)}}{I} + (1 - e) \, \frac{1}{\bar{m}^{(k)}}
    \label{eq:utility_6} \\
    U_k^{(7)} &= e \, \frac{I_s}{I} + (1 - e) \, \frac{R - B^{(k)}}{R}
    \label{eq:utility_7} \\
\end{align}

where \(U_k^{(i)}\) is utility \(i\) of server \(k\) and \(e\) is a parameter that
can be used to weight the importance of each measure.
For example, \(e = 0.5\) would mean that the two measures are equally important
for the agent.

\subsection{Case study}\label{sec:agent_based_case_study}

In this subsection, an empirical study is presented to show how the above
utility functions can be used to measure each server's ``happiness''.
The study builds upon the empirical study on the queueing system described
in~\cite{harper2020server}.
This subsection will use the same data set as in~\cite{harper2020server}, but
applied to the queueing system described in Section~\ref{sec:queueing_section}
and the agent-based model described in Section~\ref{sec:agent_based_model}.

In that study a set of data was collected from a large emergency department in
Wales, UK.
The data was collected over 6 months and contained information
on each patient that arrived at the emergency department.
Such information included the time of arrival, the time of service, the triage
category and the time of discharge.
The dataset consisted of \(4,832\) patients that were considered as ``urgent''.
In~\cite{harper2020server} it was shown that as the workload of the system
changed, the service times of the serves also changed.
In fact, low to moderate workload levels resulted in service times that were
lower than the service times experienced at high workload levels.

The system was modelled as a single-server queueing system with two service
speeds.
Arrivals follow a Poisson distribution with mean inter-arrival time
of \(92\) minutes, thus the arrival rate can be set to
\(\lambda = \frac{1}{92}\).
The service speed was partitioned into two distributions; one for low to
moderate workload levels and one for high workload levels.
The service speed for the low to moderate workload levels was found to follow
a lognormal distribution with a mean of \(86\) minutes and the service speed for
the high workload levels was found to follow a lognormal distribution
with a mean of \(62\) minutes.
Therefore, the service rates can be set to \(\mu_1 = \frac{1}{86}\) and
\(\mu_2 = \frac{1}{62}\).
It was observed that the slow service speed was used for when \(6\) or less
individuals were in the system and the fast service speed was used for when
\(7\) or more individuals were present. 

The above parameters were slightly modified so that they are applied to the
queueing system described in Section~\ref{sec:queueing_section}.
In essence, the queueing system is now a \(4\) server system and the arrival
rate is set to \(\lambda = 4 \times \frac{1}{92}\) and the two service speeds
stay the same.
In addition, the new arrival rate is now split into two distributions, one for
type 1 individuals and one for type 2 individuals.
Type 1 arrivals follow a Poisson distribution with mean inter-arrival time
of \(57.5\) minutes and type 2 arrivals follow a Poisson distribution with mean
inter-arrival time of \(38.3\) minutes.
Therefore, the arrival rates for type 1 and type 2 individuals are
\(\lambda_1 = \frac{1}{57.5}\) and \(\lambda_2 = \frac{1}{38.3}\) respectively.

Additionally, for this modified example the \(4\) servers fall into one of
three groups; experienced, moderate and intern.
In particular server \(1\) is an experienced server, server \(2\) and server
\(3\) are moderate servers and server \(4\) is an intern.
This means that server \(1\) has a slightly higher service rate than the other
servers and if they are available they will always be assigned the incoming
individual.
Servers \(2\) and \(3\) have the same service rate and if they are available
they may be assigned the incoming individual with an equal probability.
Finally, server \(4\) has the lowest service rate and they will only be
assigned the incoming individual if the other servers are unavailable.
More specifically, the service rates for the ``experienced'' server is
multiplied by a factor of \(1.2\) and the service rates for the ``intern''
server is multiplied by a factor of \(0.8\), while the ``moderate'' servers
stay unchanged.
This might not be a realistic example, but it is used here to demonstrate the
utilities in this agent-based model.

Because of the way the queueing system described in
Section~\ref{sec:queueing_section} is modelled, there are some additional
parameters that need to be considered.
These are the capacity of Node \(1\), the capacity of Node \(2\), and the
threshold.
The capacity of Node \(1\) is set to \(N = 35\), the capacity of Node \(2\) is
set to \(M = 20\) and the threshold is set to \(T = 10\).
Therefore the complete set of parameters for the queueing system are:

\begin{multicols}{2}
    \begin{itemize}
        \item \(\lambda_1 = \frac{1}{57.5}\)
        \item \(\lambda_2 = \frac{1}{38.3}\)
        \item \(C = 4\)
        \item \(\mu^{(1)} =
        \begin{cases}
            1.2 \times \frac{1}{86}, & \text{if } u + v < 7 \\
            1.2 \times \frac{1}{62}, & \text{if } u + v \geq 7    
        \end{cases}\)
        \item \(T = 10\)
        \item \(N = 35\)
        \item \(M = 20\)
        \item \(\mu^{(4)} =
        \begin{cases}
            0.8 \times \frac{1}{86}, & \text{if } u + v < 7 \\
            0.8 \times \frac{1}{62}, & \text{if } u + v \geq 7    
        \end{cases}\)
    \end{itemize}
\end{multicols}

\begin{center}
    \begin{minipage}{0.6\textwidth}
        \begin{itemize}
            \item \(\mu^{(i)} =
            \begin{cases}
                \frac{1}{86}, & \text{if } u + v < 7 \\
                \frac{1}{62}, & \text{if } u + v \geq 7    
            \end{cases} \quad \text{for } i \in \{2,3\}\)    
        \end{itemize}    
    \end{minipage}
\end{center}

Running the simulation for \(100,000\) time units each server's utilisation
(i.e. what percentage of time they were busy), the proportion of individuals
that they served and their mean service time were recorded.

\begin{table}[H]
    \caption{Each server's performance measure for a run of the simulation.}
    \label{tab:case_study_server_metrics}
    \small
    \begin{tabular}{|c|c|c|c|}
        \hline
        Server & Server utilisation & Proportion of individuals served &
        Mean service time \\
        \hline
        1 & 87.24 & 33.07 & 62.44 \\
        2 & 82.51 & 24.54 & 79.56 \\
        3 & 83.55 & 24.26 & 81.51 \\
        4 & 78.45 & 18.13 & 102.3 \\
        \hline
    \end{tabular}
\end{table}

These performance measures can now be used to populate the different utility
functions described in equations~\eqref{eq:utility_1}~-~\eqref{eq:utility_7}.
All utility functions are a weighted average between two performance measures
that are chosen to be the important factors for a server.
The weigh parameter \(e\) can take on any value between \(0\) and \(1\).
Tables~\ref{tab:case_study_utility_1_all_servers}~-~\ref{tab:case_study_utility_7_all_servers}
show the utility functions for each server for different values of \(e\).

\begin{table}[H]
    \caption{Utility function \(1\) (\(U_k^{(1)}\)) for the \(4\) servers and
    different values of \(e\)}
    \label{tab:case_study_utility_1_all_servers}
    \begin{adjustbox}{width=\columnwidth,center}
        \begin{tabular}{|c|c|c|c|c|c|c|c|c|c|c|c|}
            \hline
            Server & \(e = 0\) & \(e = 0.1\) & \(e = 0.2\) & \(e = 0.3\)
                   & \(e = 0.4\) & \(e = 0.5\) & \(e = 0.6\) & \(e = 0.7\)
                   & \(e = 0.8\) & \(e = 0.9\) & \(e = 1\) \\
            \hline
            1 & 17703 & 16101 &	14500 & 12899 & 11298 & 9697 & 8096 & 6494
                & 4893 & 3292 & 1691 \\
            2 & 20729 & 18754 &	16780 & 14805 & 12830 & 10855 & 8880 & 6905
                & 4931 & 2956 & 981 \\
            3 & 21270 & 19247 & 17224 & 15200 & 13177 & 11154 & 9131 & 7108
                & 5084 & 3061 & 1038 \\
            4 & 25599 & 23111 &	20624 & 18136 & 15648 & 13161 & 10673 & 8186
                & 5698 & 3211 & 723 \\
            \hline
        \end{tabular}
    \end{adjustbox}
\end{table}

\begin{table}[H]
    \caption{Utility function \(2\) (\(U_k^{(2)}\)) for the \(4\) servers and
    different values of \(e\)}
    \label{tab:case_study_utility_2_all_servers}
    \begin{adjustbox}{width=\columnwidth,center}
        \begin{tabular}{|c|c|c|c|c|c|c|c|c|c|c|c|}
            \hline
            Server & \(e = 0\) & \(e = 0.1\) & \(e = 0.2\) & \(e = 0.3\)
                   & \(e = 0.4\) & \(e = 0.5\) & \(e = 0.6\) & \(e = 0.7\)
                   & \(e = 0.8\) & \(e = 0.9\) & \(e = 1\) \\
            \hline
            1 & 0.128 & 0.148 & 0.168 & 0.189 & 0.209 & 0.229 & 0.25 & 0.27
            & 0.29 & 0.311 & 0.331 \\
            2 & 0.175 & 0.182 & 0.189 & 0.196 & 0.203 & 0.21 & 0.217 & 0.224
            & 0.232 & 0.239 & 0.246 \\
            3 & 0.165 & 0.172 & 0.18 & 0.188 & 0.196 & 0.204 & 0.212 & 0.219
            & 0.227 & 0.235 & 0.243 \\
            4 & 0.216 & 0.212 & 0.209 & 0.205 & 0.202 & 0.199 & 0.195 & 0.192
            & 0.188 & 0.185 & 0.182 \\
            \hline
        \end{tabular}
    \end{adjustbox}
\end{table}

\begin{table}[H]
    \caption{Utility function \(3\) (\(U_k^{(3)}\)) for the \(4\) servers and
    different values of \(e\)}
    \label{tab:case_study_utility_3_all_servers}
    \begin{adjustbox}{width=\columnwidth,center}
        \begin{tabular}{|c|c|c|c|c|c|c|c|c|c|c|c|}
            \hline
            Server & \(e = 0\) & \(e = 0.1\) & \(e = 0.2\) & \(e = 0.3\)
                   & \(e = 0.4\) & \(e = 0.5\) & \(e = 0.6\) & \(e = 0.7\)
                   & \(e = 0.8\) & \(e = 0.9\) & \(e = 1\) \\
            \hline
            1 & 0.1 & 6.4 & 12.6 & 18.8 & 25.1 & 31.3 & 37.5 & 43.7 & 50.0
            & 56.2 & 62.4 \\
            2 & 0.2 & 8.1 & 16.1 & 24.0 & 31.9 & 39.9 & 47.8 & 55.7 & 63.7
            & 71.6 & 79.6 \\
            3 & 0.2 & 8.3 & 16.4 & 24.6 & 32.7 & 40.8 & 49.0 & 57.1 & 65.2
            & 73.4 & 81.5 \\
            4 & 0.2 & 10.4 & 20.6 & 30.8 & 41.0 & 51.3 & 61.5 & 71.7 & 81.9
            & 92.1 & 102.3 \\
            \hline
        \end{tabular}
    \end{adjustbox}
\end{table}

\begin{table}[H]
    \caption{Utility function \(4\) (\(U_k^{(4)}\)) for the \(4\) servers and
    different values of \(e\)}
    \label{tab:case_study_utility_4_all_servers}
    \begin{adjustbox}{width=\columnwidth,center}
        \begin{tabular}{|c|c|c|c|c|c|c|c|c|c|c|c|}
            \hline
            Server & \(e = 0\) & \(e = 0.1\) & \(e = 0.2\) & \(e = 0.3\)
                   & \(e = 0.4\) & \(e = 0.5\) & \(e = 0.6\) & \(e = 0.7\)
                   & \(e = 0.8\) & \(e = 0.9\) & \(e = 1\) \\
            \hline
            1 & 0.128 & 0.117 & 0.105 & 0.094 & 0.083 & 0.072 & 0.061 & 0.05
            & 0.038 & 0.027 & 0.016 \\
            2 & 0.175 & 0.159 & 0.142 & 0.126 & 0.11 & 0.094 & 0.078 & 0.061
            & 0.045 & 0.029 & 0.013 \\
            3 & 0.165 & 0.149 & 0.134 & 0.119 & 0.104 & 0.088 & 0.073 & 0.058
            & 0.043 & 0.027 & 0.012 \\
            4 & 0.216 & 0.195 & 0.174 & 0.154 & 0.133 & 0.113 & 0.092 & 0.072
            & 0.051 & 0.03 & 0.01 \\            
            \hline
        \end{tabular}
    \end{adjustbox}
\end{table}

\begin{table}[H]
    \caption{Utility function \(5\) (\(U_k^{(5)}\)) for the \(4\) servers and
    different values of \(e\)}
    \label{tab:case_study_utility_5_all_servers}
    \begin{adjustbox}{width=\columnwidth,center}
        \begin{tabular}{|c|c|c|c|c|c|c|c|c|c|c|c|}
            \hline
            Server & \(e = 0\) & \(e = 0.1\) & \(e = 0.2\) & \(e = 0.3\)
                   & \(e = 0.4\) & \(e = 0.5\) & \(e = 0.6\) & \(e = 0.7\)
                   & \(e = 0.8\) & \(e = 0.9\) & \(e = 1\) \\
            \hline
            1 & 62.44 & 56.23 & 50.02 & 43.81 & 37.6 & 31.39 & 25.18 & 18.96 &
            12.75 & 6.54 & 0.33 \\
            2 & 79.56 & 71.62 & 63.69 & 55.76 & 47.83 & 39.9 & 31.97 & 24.04
            & 16.11 & 8.18 & 0.25 \\
            3 & 81.51 & 73.38 & 65.25 & 57.13 & 49.0 & 40.87 & 32.75 & 24.62
            & 16.5 & 8.37 & 0.24 \\
            4 & 102.3 & 92.08 & 81.87 & 71.66 & 61.45 & 51.24 & 41.03 & 30.82
            & 20.6 & 10.39 & 0.18 \\
            \hline
        \end{tabular}
    \end{adjustbox}
\end{table}

\begin{table}[H]
    \caption{Utility function \(6\) (\(U_k^{(6)}\)) for the \(4\) servers and
    different values of \(e\)}
    \label{tab:case_study_utility_6_all_servers}
    \begin{adjustbox}{width=\columnwidth,center}
        \begin{tabular}{|c|c|c|c|c|c|c|c|c|c|c|c|}
            \hline
            Server & \(e = 0\) & \(e = 0.1\) & \(e = 0.2\) & \(e = 0.3\)
                   & \(e = 0.4\) & \(e = 0.5\) & \(e = 0.6\) & \(e = 0.7\)
                   & \(e = 0.8\) & \(e = 0.9\) & \(e = 1\) \\
            \hline
            1 & 0.02 & 0.05 & 0.08 & 0.11 & 0.14 & 0.17 & 0.2 & 0.24
            & 0.27 & 0.3 & 0.33 \\ 
            2 & 0.01 & 0.04 & 0.06 & 0.08 & 0.11 & 0.13 & 0.15 & 0.18
            & 0.2 & 0.22 & 0.25 \\ 
            3 & 0.01 & 0.04 & 0.06 & 0.08 & 0.1 & 0.13 & 0.15 & 0.17
            & 0.2 & 0.22 & 0.24 \\ 
            4 & 0.01 & 0.03 & 0.04 & 0.06 & 0.08 & 0.1 & 0.11 & 0.13
            & 0.15 & 0.16 & 0.18 \\ 
            \hline
        \end{tabular}
    \end{adjustbox}
\end{table}

\begin{table}[H]
    \caption{Utility function \(7\) (\(U_k^{(7)}\)) for the \(4\) servers and
    different values of \(e\)}
    \label{tab:case_study_utility_7_all_servers}
    \begin{adjustbox}{width=\columnwidth,center}
        \begin{tabular}{|c|c|c|c|c|c|c|c|c|c|c|c|}
            \hline
            Server & \(e = 0\) & \(e = 0.1\) & \(e = 0.2\) & \(e = 0.3\)
                   & \(e = 0.4\) & \(e = 0.5\) & \(e = 0.6\) & \(e = 0.7\)
                   & \(e = 0.8\) & \(e = 0.9\) & \(e = 1\) \\
            \hline
            1 & 0.128 & 0.215 & 0.302 & 0.389 & 0.476 & 0.563 & 0.65 & 0.737
            & 0.824 & 0.911 & 0.999 \\ 
            2 & 0.175 & 0.257 & 0.34 & 0.422 & 0.504 & 0.587 & 0.669 & 0.751
            & 0.834 & 0.916 & 0.999 \\ 
            3 & 0.165 & 0.248 & 0.331 & 0.415 & 0.498 & 0.582 & 0.665 & 0.748
            & 0.832 & 0.915 & 0.999 \\ 
            4 & 0.216 & 0.294 & 0.372 & 0.45 & 0.529 & 0.607 & 0.685 & 0.764
            & 0.842 & 0.92 & 0.999 \\ 
            \hline
        \end{tabular}
    \end{adjustbox}
\end{table}

The above tables show the range of values of the utility functions can take
for the different servers and different values of \(e\).
Throughout the rest of this section, these utility functions will be used
as the key performance indicators that the servers will use to make decisions
about their own service speed.
