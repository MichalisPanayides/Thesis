\section{State-dependent variation}\label{sec:state_dependent}

This section focuses on creating a state-dependent variation of the queueing
system described in Section~\ref{sec:queueing_section}.
The state-dependent version is based on the idea that the speed of the servers
could be based on the number of individuals present in the system.

In the previous chapters of this thesis, the value of the service rate \(\mu\)
was set to a constant value.
For this variation of the model \(\mu\) will have a different value depending
on the number of individuals present in the two nodes of the queue.
In other words \(\mu\) will take a different value depending on the state
\(u, v\) of which the system is in.
A parametric service rate \(\mu = \mu_{(u,v)}\) for a given \((u, v) \in S\) is
defined as follows:

\begin{equation}\label{eq:state_dependent_service_rate}
    \mu =
    \begin{cases}
        \mu_{(0,0)}, & \text{if } (u, v) = (0, 0) \\
        \mu_{(0,1)}, & \text{if } (u, v) = (0, 1) \\
        \quad \vdots & \qquad \quad \vdots \\
        \mu_{(M,N)}, & \text{if } (u, v) = (M, N)
    \end{cases}
\end{equation}

Consider an example of the queueing system described in
Section~\ref{sec:queueing_section} where the number of servers is set to
\(C = 1\), the threshold is set to \(T = 1\), node 1 capacity is \(N = 2\) and
node 2 capacity is \(M = 1\).
Figure~\ref{fig:state_dependent_markov_example} shows the state-dependent
Markov chain model of the queueing system.

\begin{figure}[H]
    \centering
    \begin{tikzpicture}[-, node distance = 2cm, auto]
\node[state] (u0v0) {(0,0)};
\node[state, right=of u0v0] (u0v1) {(0,1)};
\draw[->](u0v0) edge[bend left] node {\( \Lambda \)} (u0v1);
\draw[->](u0v1) edge[bend left] node {\( \mu_{(0,1)} \)} (u0v0);
\node[state, below=of u0v1] (u1v1) {(1,1)};
\draw[->](u0v1) edge[bend left] node {\( \lambda_2 \)} (u1v1);
\draw[->](u1v1) edge[bend left] node {\( \mu_{(1,1)} \)} (u0v1);
\node[state, right=of u0v1] (u0v2) {(0,2)};
\draw[->](u0v1) edge[bend left] node {\( \lambda_1 \)} (u0v2);
\draw[->](u0v2) edge[bend left] node {\( \mu_{(0,2)} \)} (u0v1);
\node[state, right=of u1v1] (u1v2) {(1,2)};
\draw[->](u1v1) edge[bend left] node {\( \lambda_1 \)} (u1v2);
\draw[->](u1v2) edge[bend left] node {\( \mu_{(1,2)} \)} (u1v1);
\draw[->](u0v2) edge node {\( \lambda_2 \)} (u1v2);
\end{tikzpicture}
    \caption{Markov chain example with \(C=1, T=1, N=2, M=1\)}
    \label{fig:state_dependent_markov_example}
\end{figure}

Consider the following value of \(\mu\) for the Markov chain model shown in
Figure~\ref{fig:state_dependent_markov_example}.

\begin{equation*}
    \mu =
    \begin{cases}
        \mu_{(0, 0)}, \text{ if } (u, v) = (0, 0) \\
        \mu_{(0, 1)}, \text{ if } (u, v) = (0, 1) \\
        \mu_{(0, 2)}, \text{ if } (u, v) = (0, 2) \\
        \mu_{(1, 1)}, \text{ if } (u, v) = (1, 1) \\
        \mu_{(1, 2)}, \text{ if } (u, v) = (1, 2)
    \end{cases}
\end{equation*}



\subsection{Implementation}

The state-dependent variation of the queueing system was implemented using the
python library \texttt{ciw}~\cite{ciwpython} only on the Discrete Event
Simulation (DES) version of the queueing model described in
Section~\ref{sec:discrete_event_simulation}.
All the code used to implement the state-dependent variation of the queueing
system is archived at~\cite{ambulance_game_github_repo} and developed openly on
GitHub.
More details on the code can be found in
Appendix~\ref{appendix:ambulance_game}.
The \texttt{ciw} library is structured in a way that allows the user to create
their own class for the service time distribution.
This class must inherit from the \texttt{ciw.dists.Distribution} class and
implement the \texttt{\_\_init\_\_} and \texttt{sample} methods.
The code snippet in~\ref{lst:state_dependent_class} shows the implementation of
the state-dependent service time distribution class.

\begin{lstlisting}[
    style=pystyle,
    caption={State-dependent service time distribution class.},
    label={lst:state_dependent_class}
]
>>> import ciw
>>> class StateDependentExponential(
...     ciw.dists.Distribution
... ):
...     def __init__(self, rates):
...         self.rates = rates
... 
...     def sample(self, t=None, ind=None):
...         """
...         This method is used to sample the service time for an
...         individual based on the current state
...         """
...         state = (
...             len(ind.simulation.nodes[1].individuals[0]),
...             len(ind.simulation.nodes[2].individuals[0]),
...         )
...         is_invalid_state = (
...             state[0] > 0 and state[1] < ind.simulation.threshold
...         )
...         if is_invalid_state:
...             state = (state[0] - 1, state[1] + 1)
...         rate = self.rates[state]
...         return random.expovariate(rate)
    
\end{lstlisting}

The function \texttt{simulate\_model} described in
Section~\ref{sec:discrete_event_simulation} takes \texttt{mu} as one of its
arguments.
If \texttt{mu} is set to a dictionary, with keys the states of the system and
values the service rate for each state (as shown in code
snippet~\ref{lst:state_dependent_example}), the service
distribution that will be
used in \texttt{ciw} will be a \texttt{StateDependentExponential} object.
The following code shows the implementation of the
\texttt{simulate\_model} function for the state-dependent variation of the
queueing system.

\begin{lstlisting}[
    style=pystyle,
    caption={Example of the state-dependent variation of the queueing system.},
    label={lst:state_dependent_example},
]
>>> import ambulance_game as abg
>>> import ciw
>>> import numpy as np
>>>
>>> lambda_1 = 1
>>> lambda_2 = 1
>>> num_of_servers = 1
>>> threshold = 1
>>> system_capacity = 2
>>> buffer_capacity = 1
>>> runtime = 1000
>>> seed_num = 0
>>>
>>> Q = abg.simulation.simulate_model(
...     lambda_1=lambda_1,
...     lambda_2=lambda_2,
...     mu={(0, 0): 2, (0, 1): 3, (0, 2): 4, (1, 1): 4, (1, 2): 5},
...     num_of_servers=num_of_servers,
...     threshold=threshold,
...     seed_num=seed_num,
...     system_capacity=system_capacity,
...     buffer_capacity=buffer_capacity,
...     runtime=runtime,
... )
>>> mean_waiting_time = np.mean([w.waiting_time for w in Q.get_all_records()])
>>> np.round(mean_waiting_time, 8)
0.06028274

\end{lstlisting}
