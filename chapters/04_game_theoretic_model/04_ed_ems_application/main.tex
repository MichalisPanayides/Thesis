\section{ED-EMS application}\label{sec:game_ems_ed_application}

Similar to Section~\ref{sec:queueing_ems_ed_application}, the game theoretic
model can be applied to the same healthcare setting.
All concepts described in this Section can be mapped to some components of
either the ED or the EMS.

The EMS has to decide how to distribute its patients among the two EDs so that
the weighted combination of the ambulance blocking time and the percentage of
lost ambulances is minimised.
This can be illustrated by Figure~\ref{fig:diagram_of_game_theoretic_model}.
The interaction between the two
EDs is a normal form game that is then used to inform the decision of the EMS.
Note that the formulated game here assumes that prior to making a choice the
EMS knows the strategies that each ED is playing
(Figure~\ref{fig:imperfect_info_game}).
This corresponds to reacting to experienced delays.

The queueing systems of the hospitals are designed in such a way where they can
accept two types of individuals (Chapter~\ref{sec:queueing_section}).
Each hospital may then choose to block type 2 individuals
when the hospital reaches a certain capacity.
The strategy sets for each hospital is the set
\( \{T \in \mathbb{N} \;|\; 1 \leq T \leq N\} \) where \(N \in\{N_A, N_B\}\) are
the total capacities of hospitals \(A\) and \(B\).
The chosen actions from the strategy set are denoted as \(T_A, T_B\) and are
the \textit{thresholds}.

Both hospitals follow a queueing model with two waiting spaces for
individuals.
The first waiting space (i.e. the waiting space of the hospital) is where the
patients queue right before receiving
their service and has a queue capacity of \( N - C \), where \(N\) is the total
capacity of the hospital and \(C\) is the number of healthcare
professionals able to see them.
The second waiting space (i.e. the parking space for ambulances) is where
ambulances, that are sent from the
EMS distributor, stay until their patients are allowed to enter the hospital.
The parking space has a capacity of \(M\) and no servers.
This is shown diagrammatically in Figure~\ref{fig:diagram_of_queueing_system}.

Note here that both types of individuals can become lost to the system.
An individual allocated from the ambulance service becomes lost to the system
whenever
an arrival occurs and the parking space is at full capacity (\(M\)
ambulances already parked).
Similarly, type \(1\) individuals get lost whenever they arrive at the waiting
space of the hospital and it is at full capacity (\(N - C\) individuals already
waiting).
Numerical results on the ED-EMS game theoretic model are presented and
discussed in Chapter~\ref{sec:numerical_results}.

There are certain assumptions that are made in this application.
Firstly, it is assumed that the distance from any patient's location to any
hospital is not a factor that will affect the EMS's decision.
That means that under the scope of this application, the EMS does not have to
consider the closest hospital to the patient's location.
Secondly, it is assumed that a patient's timer (from the perspective of the ED)
does not start counting until the patient enters the hospital.
For instance, consider the case where a patient is sent from the EMS to hospital
\(A\) and is blocked in the parking space of hospital \(A\) for \(6\) hours.
The patient then proceeds to wait in the hospital for an additional \(2\) hours
and is then receiving their treatment for \(1\) hour.
The patient's total time in the hospital is assumed to be \(3\) hours (\(2+1\)).
Finally, the last assumption that is made is that arrival and service times are
exponentially distributed.
