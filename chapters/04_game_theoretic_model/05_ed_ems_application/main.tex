\section{ED-EMS application}\label{sec:game_ems_es_application}

The EMS has to decide how to distribute its patients among the two EDs so that
the weighted combination of the ambulance blocking time and the percentage of
lost ambulances is minimised.
This can be illustrated by figure \ref{fig:diagram_of_game_theoretic_model}.
The interaction between the two
EDs is a normal form game that is then used to inform the decision of the EMS.
Note that the formulated game here assumes that prior to making a choice the
EMS knows the strategies that each ED is playing (figure
\ref{fig:imperfect_info_game}).
This corresponds to reacting to experienced delays.

The queueing systems of the hospitals are designed in such a way where they can
accept two types of individuals (section \ref{sec:queueing_section}).
Each hospital may then choose to block type 2 individuals
when the hospital reaches a certain capacity.
The strategy sets for each hospital is the set
\( \{T \in \mathbb{N} \;|\; 1 \leq T \leq N\} \) where \(N \in\{N_A, N_B\}\) are
the total capacities of hospitals \(A\) and \(B\).
We denote the chosen actions from the strategy set as \(T_A, T_B\) and call
these \textit{threshold}s.

Both hospitals follow a queueing model with two waiting spaces for
individuals.
The first waiting space (i.e. the waiting space of the hospital) is where the
patients queue right before receiving
their service and has a queue capacity of \( N - C \), where \(N\) is the total
capacity of the hospital and \(C\) is the number of healthcare
professionals able to see them.
The second waiting space (i.e. the parking space for ambulances) is where
ambulances, that are sent from the
EMS distributor, stay until their patients are allowed to enter the hospital.
The parking space has a capacity of \(M\) and no servers.
This is shown diagrammatically in Figure~\ref{fig:diagram_of_queueing_system}.

Note here that both types of individuals can become lost to the system.
An individual allocated from the ambulance service becomes lost to the system
whenever
an arrival occurs and the parking space is at full capacity (\(M\)
ambulances already parked).
Similarly, type 1 individuals get lost whenever they arrive at the waiting
space of the hospital and it is at full capacity (\(N - C\) individuals already
waiting).
