\chapter*{Abstract}
\addcontentsline{toc}{chapter}{Abstract}

This thesis investigates the behavioural dynamics that emerge at the
interface of Emergency Departments (EDs) and the Emergency Medical Service
(EMS).
The main focus is on the impact that time-targets may have on staff
behaviour and patient well-being.
This research is structured into two main parts: the first part is the
development of a queueing theoretic representation of an ED and the second
part is the development of a game theoretic model between two EDs and the
EMS that distributes ambulance patients to them.

This thesis uses a variety of mathematical and computational fields such as
linear algebra, graph theory, optimisation, probability theory, game
theory, queueing theory, agent-based simulation and reinforcement learning.

The queueing model is developed using both a discrete event simulation
and a Markov chain approach.
The queueing network consists of two queueing nodes where there is some
strategic managerial behaviour that relates to how two types of individuals
are routed between the two nodes.
The first node acts as a buffer for one type of individuals before moving 
to the second node, while the second node consists of a waiting room and a
service centre.
Both approaches are used to obtain performance measures of the queueing
system and explicit formulas are derived for the mean waiting time, the
mean blocking time and the proportion of individuals within a given target
time.
In addition, some numeric results are presented that compare the
Markov chain and discrete event simulation approaches.

Consequently, this thesis describes the development and application of a
3-player game  theoretic model between two such queueing networks and a
service that distributes individuals to them.
In particular the game is then reduced to a 2-player normal-form game.
The resultant model is used to explore dynamics between all players.
A backwards induction technique is used to get the utilities of the
normal-form game between the two queueing systems.
The particular game is then applied to a healthcare scenario to capture the
emergent behaviour between the EMS and two EDs.
The results and outcomes that are produced by various instance of the game
are then analysed and discussed.
The learning algorithm replicator dynamics is used to explore the
evolutionary behaviours that emerge in the game.
In particular, the behaviour that naturally emerges from the game seems to
be one that causes more blockage and includes less cooperation.
Several ways to escape this learned inefficient behaviour are discussed.

Finally, the thesis explores an extension of the queueing theoretic model
that allows servers to choose their own service speed.
This is implemented using an agent-based simulation approach.
The agent-based model is then used in conjunction with a reinforcement
learning algorithm to explore the effect that the servers' behaviour has 
on the overall performance of the system.

